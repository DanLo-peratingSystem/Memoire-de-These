\ifdefined\included
\else
\documentclass[a4paper,11pt,twoside]{StyleThese}
\usepackage{iflang}
\usepackage{bibentry}



%\usepackage[sectionbib]{chapterbib}          % Cross-reference package (Natural BiB)
%\usepackage{natbib}                  % Put References at the end of each chapter
%\usepackage{bibunits}
% Do not put 'sectionbib' option here.
% Sectionbib option in 'natbib' will do.


\usepackage{fancyhdr}                    % Fancy Header and Footer

\usepackage[utf8]{inputenc}
\usepackage[T1]{fontenc}
\usepackage[french]{babel} %
\usepackage{lmodern} \normalfont %to load T1lmr.fd 
\DeclareFontShape{T1}{cmr}{b}{sc} { <-> ssub * cmr/bx/sc }{}
%\hyphenation{gar}

\usepackage{amsmath,amssymb}             % AMS Math
\usepackage{nicefrac}
\usepackage{siunitx}					%% Unites Math SI

\usepackage{blindtext}

\usepackage{datetime}

\usepackage{lipsum} 

\usepackage[inline]{enumitem}

\usepackage{hhline}
%\usepackage[left=1.5in,right=1.3in,top=1.1in,bottom=1.1in]{geometry}
\usepackage[left=1.5in,right=1.3in,top=1.1in,bottom=1.1in,includefoot,includehead,headheight=13.6pt]{geometry}

%%\renewcommand{\baselinestretch}{1.05}

%%%%%%%% Multi-figures avec sub-captions
\usepackage{caption}
\usepackage{subcaption}

% Table of contents for each chapter

\usepackage[nottoc, notlof, notlot]{tocbibind}
\usepackage[nohints]{minitoc}
\setcounter{minitocdepth}{2}
\mtcindent=15pt
% Use \minitoc where to put a table of contents

\usepackage{aecompl}

%% Package cosmetic meilleur layout du texte en jouant sur le spacing par caractères
\usepackage[activate={true,nocompatibility},final,tracking=true,kerning=true,factor=1100,stretch=10,shrink=10]{microtype}
\usepackage[absolute,overlay]{textpos} 
\setlength{\TPHorizModule}{\paperwidth}\setlength{\TPVertModule}{\paperheight}
\sloppy

%%%%%%%%%%% JOLIS TABLEAUX
\usepackage{tabularx}		%\usepackage{tabular}
\usepackage{longtable}
\usepackage{multirow}
\newcommand{\mc}{\multicolumn} 
\newcommand{\mr}[2]{\multirow{#1}{*}{#2}} 	\newcommand{\mrQ}{\multirow{-4}{*}}
\usepackage{booktabs}

\usepackage[usenames,dvipsnames]{xcolor} 

\makeatletter
\newcommand{\ccolor}[3][]{%
	\kern-\fboxsep
	\if\relax\detokenize{#1}\relax
	\expandafter\@firstoftwo
	\else
	\expandafter\@secondoftwo
	\fi
	{\colorbox{#2}}%
	{\colorbox[#1]{#2}}%
	{#3}\kern-\fboxsep
}
\makeatother

%%%%% Insertion graphiques format PGF
\usepackage{pgfplots}
\pgfplotsset{width=\linewidth, compat=1.16}%, compat=1.17}
\usepackage{adjustbox}          %%% PERMET DE LES RECADRER + FACILEMENT


%%%%%%%%%% Bullets de listes sans saut de ligne %%%%%%%%%%
\usepackage{xparse}

\ExplSyntaxOn%
\seq_new:N \l_local_enum_seq

\newcommand{\storethestuff}[1]{%
  \seq_set_from_clist:Nn \l_local_enum_seq {#1}%
}

\newcommand{\dotheenumstuff}{%
\int_zero:N \l_tmpa_int
\seq_map_inline:Nn \l_local_enum_seq {%
    \int_incr:N \l_tmpa_int% Increase the counter
    \item ##1
    % Check whether the list has reached the end -- if so, use '.' instead of ','
    %\int_compare:nNnTF 
    % { \l_tmpa_int } < {\seq_count:N \l_local_enum_seq} 
    % {,} {.}
  }
}
\ExplSyntaxOff

\NewDocumentCommand{\linebullets}{+m}{%
  \storethestuff{#1}%
  \begin{enumerate*}[label={\alph*)},font={\bfseries},itemjoin={{, }}]
    \dotheenumstuff%
  \end{enumerate*}
}

\newcommand{\cmnt}[1]{}  %%%%% AJOUT DE COMMENTAIRE MULTILIGNES


%%%%%%%%%% ECRITURE CARACTERES DANS UN CERCLE %%%%%%%%%%
%\def\circleTxt[#1]{\raisebox{.5pt}{\textcircled{\raisebox{-1pt}{#1}}}}
\newcommand{\ctxt}[1]{\raisebox{.5pt}{\textcircled{\raisebox{-1.2pt}{#1}}}}
% Glossary / list of abbreviations

\usepackage[intoc]{nomencl}
\IfLanguageName{english}{%
\renewcommand{\nomname}{Glossary}
}{ %
\renewcommand{\nomname}{Liste des Abréviations}
}

\makenomenclature

% My pdf code

\usepackage{ifpdf}

\ifpdf
  \usepackage[pdftex]{graphicx}
  \DeclareGraphicsExtensions{.pdf,PDF,.png,PNG,.jpg,JPG}
  \usepackage[pagebackref,hyperindex=true]{hyperref} %% use \autoref{} instead of Table~\ref{}.
  \usepackage{tikz}
  \usetikzlibrary{arrows,shapes,calc}
\else
  \usepackage{graphicx}
  \DeclareGraphicsExtensions{.ps,.eps}
  \usepackage[a4paper,dvipdfm,pagebackref,hyperindex=true]{hyperref}
\fi

\graphicspath{{.}{schemas/}{graphiques/}{tables/}}

%% nicer backref links. NOTE: The flag ThesisInEnglish is used to define the
% language in the back references. Read more about it in These.tex

\IfLanguageName{english}{
\renewcommand*{\backref}[1]{}
\renewcommand*{\backrefalt}[4]{%
\ifcase #1 %
(Not cited.)%
\or
(Cited in page~#2.)%
\else
(Cited in pages~#2.)%
\fi}
\renewcommand*{\backrefsep}{, }
\renewcommand*{\backreftwosep}{ and~}
\renewcommand*{\backreflastsep}{ and~}
}{
\renewcommand*{\backref}[1]{}
\renewcommand*{\backrefalt}[4]{%
\ifcase #1 %
(Non cité.)%
\or
(Cité en page~#2.)%
\else
(Cité en pages~#2.)%
\fi}
\renewcommand*{\backrefsep}{, }
\renewcommand*{\backreftwosep}{ et~}
\renewcommand*{\backreflastsep}{ et~}
}

% Links in pdf
\usepackage{color}
\definecolor{linkcol}{rgb}{0,0,0.4} 
\definecolor{citecol}{rgb}{0.5,0,0} 
\definecolor{linkcol}{rgb}{0,0,0} 
\definecolor{citecol}{rgb}{0,0,0}
% Change this to change the informations included in the pdf file

\hypersetup
{
bookmarksopen=true,
pdftitle="Prévention des fautes temporelles sur architectures multicœur pour les systèmes à criticité mixte",
pdfauthor="Daniel LOCHE", %auteur du document
pdfsubject="Thèse", %sujet du document
%pdftoolbar=false, %barre d'outils non visible
pdfmenubar=true, %barre de menu visible
pdfhighlight=/O, %effet d'un clic sur un lien hypertexte
colorlinks=true, %couleurs sur les liens hypertextes
pdfpagemode=UseNone, %aucun mode de page
%pdfpagelayout=DoublePage, %ouverture en simple page
pdffitwindow=true, %pages ouvertes entierement dans toute la fenetre
linkcolor=linkcol, %couleur des liens hypertextes internes
citecolor=citecol, %couleur des liens pour les citations
urlcolor=linkcol %couleur des liens pour les url
}

% definitions.
% -------------------

\setcounter{secnumdepth}{3}
\setcounter{tocdepth}{2}

% Some useful commands and shortcut for maths:  partial derivative and stuff

\newcommand{\pd}[2]{\frac{\partial #1}{\partial #2}}
\def\abs{\operatorname{abs}}
\def\argmax{\operatornamewithlimits{arg\,max}}
\def\argmin{\operatornamewithlimits{arg\,min}}
\def\diag{\operatorname{Diag}}
\newcommand{\eqRef}[1]{(\ref{#1})}
\newcommand{\nline}{\smallbreak\noindent}

\usepackage{rotating}                    % Sideways of figures & tables

% \usepackage{txfonts}                     % Public Times New Roman text & math font
  
%%% Fancy Header %%%%%%%%%%%%%%%%%%%%%%%%%%%%%%%%%%%%%%%%%%%%%%%%%%%%%%%%%%%%%%%%%%
% Fancy Header Style Options

\pagestyle{fancy}                       % Sets fancy header and footer
\fancyfoot{}                            % Delete current footer settings

%\renewcommand{\chaptermark}[1]{         % Lower Case Chapter marker style
%  \markboth{\chaptername\ \thechapter.\ #1}}{}} %

%\renewcommand{\sectionmark}[1]{         % Lower case Section marker style
%  \markright{\thesection.\ #1}}         %

\fancyhead[LE,RO]{\bfseries\thepage}    % Page number (boldface) in left on even
% pages and right on odd pages
\fancyhead[RE]{\bfseries\nouppercase{\leftmark}}      % Chapter in the right on even pages
\fancyhead[LO]{\bfseries\nouppercase{\rightmark}}     % Section in the left on odd pages

\let\headruleORIG\headrule
\renewcommand{\headrule}{\color{black} \headruleORIG}
\renewcommand{\headrulewidth}{1.0pt}
\usepackage{colortbl}
\arrayrulecolor{black}

\fancypagestyle{plain}{
  \fancyhead{}
  \fancyfoot{}
  \renewcommand{\headrulewidth}{0pt} %%%%%%%%%%%%%%%%%%%%%%%%%%%%%%%%%%%%%%%%%%%%%%%%%%%%%%%%%%%%%%%%%%%%%%%%%%%%%%%%%%%%%
}

%\usepackage{MyAlgorithm}
%\usepackage[noend]{MyAlgorithmic}
%\usepackage[ED=EDSYS-SystEmb, Ets=INP]{tlsflyleaf}

%%% Clear Header %%%%%%%%%%%%%%%%%%%%%%%%%%%%%%%%%%%%%%%%%%%%%%%%%%%%%%%%%%%%%%%%%%
% Clear Header Style on the Last Empty Odd pages
\makeatletter

\def\cleardoublepage{\clearpage\if@twoside \ifodd\c@page\else%
  \hbox{}%
  \thispagestyle{empty}%              % Empty header styles
  \newpage%
  \if@twocolumn\hbox{}\newpage\fi\fi\fi}

\makeatother
 
%%%%%%%%%%%%%%%%%%%%%%%%%%%%%%%%%%%%%%%%%%%%%%%%%%%%%%%%%%%%%%%%%%%%%%%%%%%%%%% 
% Prints your review date and 'Draft Version' (From Josullvn, CS, CMU)
\newcommand{\reviewtimetoday}[2]{\special{!userdict begin
    /bop-hook{gsave 20 710 translate 45 rotate 0.8 setgray
      /Times-Roman findfont 12 scalefont setfont 0 0   moveto (#1) show
      0 -12 moveto (#2) show grestore}def end}}
% You can turn on or off this option.
% \reviewtimetoday{\today}{Draft Version}
%%%%%%%%%%%%%%%%%%%%%%%%%%%%%%%%%%%%%%%%%%%%%%%%%%%%%%%%%%%%%%%%%%%%%%%%%%%%%%% 

\newenvironment{maxime}[1]
{
	\def\Arg{#1}
\vspace*{0cm}
\hfill
\begin{minipage}{0.6\textwidth}%
%\rule[0.5ex]{\textwidth}{0.1mm}\\%
\hrulefill $\:$ \\%$\:$ {\bf #1}\\
%\vspace*{-0.25cm}
\it 
}%
{%
	
\hrulefill $\:$ {\bf \Arg}
\vspace*{0.5cm}%
\end{minipage}
}

\let\minitocORIG\minitoc
\renewcommand{\minitoc}{\minitocORIG \vspace{1.5em}}

%\usepackage{slashbox}

\newenvironment{bulletList}%
{ \begin{list}%
	{$\bullet$}%
	{\setlength{\labelwidth}{25pt}%
	 \setlength{\leftmargin}{30pt}%
	 \setlength{\itemsep}{\parsep}}}%
{ \end{list} }


%%%%%%% Outils pour \comment \alert \add %%%%%
\usepackage{easyReview}
\usepackage{soulutf8} % for accented letters

\let\newalert\alert
\renewcommand{\alert}[1]{\textit{\newalert{#1}}}

%\usepackage[commandnameprefix=ifneeded]{changes} %% \chhighlight and \chcomment to avoid collision with easyReview
\renewcommand{\epsilon}{\varepsilon}

% centered page environment

\newenvironment{vcenterpage}
{\newpage\vspace*{\fill}\thispagestyle{empty}\renewcommand{\headrulewidth}{0pt}}
{\vspace*{\fill}}

\usepackage{tablefootnote}

%%%%%% MISE EN FORME CADRES DEFINITIONS/THEOREMES/LEMES %%%%%%%%%%
\usepackage{amsthm}  % for theoremstyle

\theoremstyle{plain} 
\newtheorem{theorem}{Théorème}[section]
\newtheorem{corollary}{Corolaire}[theorem]

%\theoremstyle{lemma}
%\newtheorem{lemma}[theorem]{Lemme}


\theoremstyle{definition}
\newtheorem{definition}[theorem]{Définition}


\cmnt{
	\usepackage{ntheorem} %\usepackage{amsthm}  % for theoremstyle
	%\usepackage{mdframed}
	\usepackage[most]{tcolorbox}
	
	\theoremstyle{plain} 
	\theoremindent20pt
	\theoremheaderfont{\normalfont\bfseries\hspace{-\theoremindent}}
	\newtheorem{theorem}{Théorème}[section]
	\newtheorem{corollary}{Corolaire}[theorem]
	
	\theoremstyle{plain}
	\newtheorem{lemma}[theorem]{Lemme}
	
	
	\tcolorboxenvironment{theorem}{
		blanker,
		breakable,
		before skip=\topsep,
		after skip=\topsep,
		borderline west={1pt}{10pt}{double, shorten <=12pt}
	}
	
	\theorembodyfont{\normalfont}
	\theoremindent20pt
	\theoremheaderfont{\normalfont\bfseries\hspace{-\theoremindent}}
	\newtheorem{definition}[theorem]{Définition}
	
	
	\tcolorboxenvironment{definition}{
		blanker,
		breakable,
		before skip=\topsep,
		after skip=\topsep,
		borderline west={1pt}{10pt}{shorten <=12pt}
	}
}

\cmnt{ 
	\begin{theorem}
		Ceci est un Théorème.
	\end{theorem} 
	
	\begin{corollary}
		Ceci est un Corollaire.
	\end{corollary}
	
	\begin{definition}
		Ceci est une Définition.
	\end{definition}
	
	\begin{lemma}
		Ceci est un Lemme.
	\end{lemma}
}

\def\UrlBigBreaks{\do\/\do-\do:}
\usepackage{url}

\sloppy



\begin{document}
\setcounter{chapter}{6} %% Numéro du chapitre précédent ;)
\dominitoc
\fi

\chapter*{Conclusion et Perspectives} \label{chap:Conclusion}
\chaptermark{Conclusion et Perspectives}
\addstarredchapter{Conclusion} %Sinon cela n'apparait pas dans la table des matières

\section*{Conclusion}
\markboth{Conclusion et Perspectives}{Conclusion}
\addcontentsline{toc}{section}{Conclusion}

Au fil de cette thèse, il nous a été donné de faire un état des lieux sur l'évolution des systèmes cyberphysiques, notamment dans le domaine de l'automobile. Ces évolutions se sont pour une grande part orientées vers le choix d'architectures matérielles multicœurs de plus en plus puissantes, mais aussi de plus en plus complexes. 
Avec la cohabitation de logiciels à différents niveaux de criticité sur ces plateformes, cela donne lieu à des enjeux émergents sur le respect des contraintes de sûreté de fonctionnement dans les systèmes temps-réel.

C'est donc en partant de ce constat que nous avons établi un état des lieux à la fois des enjeux et contraintes industrielles et des travaux afférents dans le domaine académique sur l'intégration de systèmes à criticité mixte.
Les enjeux sont multiples et les solutions pour y répondre sont multicritères. Nous avons identifié deux grands enjeux lors de notre constat. 
\begin{itemize}
	\item Tout d'abord, la nécessité de garanties sur les temps d'exécution des tâches, et notamment les tâches critiques, de façon à ne pas dépasser des échéances liées aux contraintes temps-réel.
	\item Ensuite, le besoin croissant d'exploiter au maximum les ressources de calcul à disposition, de façon à diminuer considérablement le nombre de processeurs nécessaires au sein des systèmes embarqués.\end{itemize}

La problématique principale pour répondre à ces deux enjeux réside dans leur antinomie. Les solutions dédiées à la garantie des contraintes temps-réel se font au détriment de l'optimisation de la puissance de calcul, tandis que l'agrégation de logiciels à criticité multiple provoque des problèmes d'interférences qui peuvent mener à des retards d'exécution indésirables. Ces problèmes sont intrinsèques à l'usage des calculateurs multicœurs, par la présente de ressources partagées tel que le cache, les bus de données ou encore les différentes entrées/sorties. Avec l'exécution concourante de tâches sur plusieurs cœurs, cet usage de ressources partagées entre les logiciels mène à des points de contentions qui augmentent les temps de réponses des tâches jusqu'au dépassement des échéances. 

\paragraph*{} Pour répondre à cette problématique, nous avons en premier lieu proposé une approche qui se focalise sur une vision fonctionnelle du système. De fait, au sein d'un calculateur multicœur, les tâches qui s'exécutent sont en partie interconnectées pour réaliser des fonctions spécifiques. Cela se traduit à l'échelle du logiciel par ce que l'on définit comme des chaînes de tâches. Nous proposons par conséquent un modèle de mécanisme de Surveillance et de Contrôle spécialisé dans la garantie de contraintes temporelles lors de l'exécution d'une chaîne de tâches. Le modèle proposé se veut adapté au domaine industriel par sa généricité avec des hypothèses de modèle qui prennent en compte les contraintes industrielles, tel que la présence de logiciel en boite noire.
Notre mécanisme repose sur la Surveillance de l'avancement d'exécution d'une chaîne de tâche. Il anticipe l'instant où les interférences dues aux tâches non critiques présentent un risque irréversible de dépassement de l'échéance d'exécution bout-en-bout. Le cas échéant, l'exécution des tâches non critiques est Contrôlé via une mise en pause momentanée pour prévenir toute interférence supplémentaire et donc garantir le respect de l'échéance bout-en-bout.

\paragraph*{} Afin de mettre en place ce mécanisme, nous avons proposé tout un processus de développement qui s'inscrit dans une démarche expérimentale. Cette approche permet la caractérisation nécessaire d'un système à criticité mixte de façon à obtenir une meilleure connaissance de son comportement et notamment vis-à-vis des interférences d'exécution. Dans un second temps, il indique aussi les grandes étapes à intégrer dans le cadre de l'implémentation du mécanisme de Surveillance et Contrôle sur un système embarqué. Cela inclut la détermination des paramètres de fonctionnement propres au mécanisme, mais aussi la phase de vérification analytique avec une mesure des performances suivant trois critères pertinents (Fiabilité, Performance et Qualité) ainsi que des possibilités d'ajustements.

Enfin, nous avons mis en application cette démarche expérimentale sur une plateforme dédiée. Dans cette optique, nous avons développé un framework expérimental complet. Ce dernier permet d'exécuter directement n'importe quel jeu de tâches sous forme de fonctions de librairies avec à notre mécanisme de Surveillance et de Contrôle. La plateforme matérielle et logicielle que nous proposons permet, par le biais de fichiers de configuration d'entrées, une forte adaptabilité de configuration et de modifications pour s'adapter selon 3 axes~: le support matériel et logiciel bas-niveau (ici, nous avons choisis Xenomai sur un multicœur Intel) ; les hypothèses du modèle de tâches et de chaîne de tâches ; et les données d'entrée (le jeu de tâches, les paramètres du mécanisme, la charge de stress du calculateur). Grâce à tout cet ensemble, nous présentons ainsi une base de plateforme expérimentale minimale de façon à caractériser pour un contexte spécifique le profil de temps d'exécution d'un logiciel, la place des interférences inter-logiciel au sein de celui-ci, et l'apport de notre mécanisme de Surveillance et de Contrôle sur le respect des échéances bout-en-bout. 

La mise en place de cette plateforme complète accompagnée des protocoles expérimentaux proposés a pu être éprouvé au travers de la constitution d'un cas de test fictif, par le biais des tâches du benchmark MiBench. Ce cas de test nous a permis d'appliquer la démarche dans sa totalité et d'illustrer les capacités de notre mécanisme dans un cas spécifique. Les résultats obtenus sont encourageants et permettent concrètement d'obtenir un système à criticité mixte avec un mécanisme de contrôle réactif qui offre aux tâches non critiques le temps nécessaires hors des instants d'exception où le respect d'échéance de la chaîne de tâches critiques doit continuer d'être garanti. 

\paragraph*{} Dans un contexte comme celui du domaine automobile, où les multiples fonctions embarquées se retrouvent agrégés au sein de calculateurs toujours plus puissants, la problématique de co-existence de logiciels à différents niveaux de criticité devient inévitable. Quand des systèmes d'aide à la conduite (ADAS) tel que le Système de Changement de Ligne ou l'Avertissement d'Angle-Mort risquent de coexister avec du logiciel non-critique comme le système radio ou la chauffe des sièges, le besoin de garanties d'exécution devient pressant. Dans l'ensemble, nous proposons une approche inédite pour le respect des échéances temporelles tout en conservant de bonnes performances sur l'exploitation des ressources de calcul. Notre solution de maîtrise des fautes temporelles dues aux interférences est réactive contrairement à ce qui se fait habituellement de façon statique dans les solutions industrielles. Nous tentons de palier aux risques d'interférences liées au partage de ressources de façon conservative, en limitant l'exécution des tâches non critiques uniquement en cas de nécessité absolue. 

Ce processus expérimental propose globalement une nouvelle approche aux problématiques émergentes des systèmes à criticité mixte, qui prend en compte au mieux les contraintes propres aux systèmes industriels. Il s'agit d'un premier pas vers des approches orientées sur les chaînes de tâches, avec des outils pour la caractérisation des systèmes industriels et la mise en œuvre de ce type de mécanismes de sûreté de fonctionnement somme toute très lié aux enjeux du temps-réel.


\section*{Perspectives}
\markboth{Conclusion et Perspectives}{Perspectives}
\addcontentsline{toc}{section}{Perspectives}

%%\subsection{Mode dégradé multi-niveau}
%%\subsection{mode dégradé par mécanismes de contrôle hardware}

L'approche qui a été développée dans cette thèse est particulièrement prometteuse. Elle ouvre des pistes de recherche axées sur l'exploitation de chaînes de tâches pour proposer un mécanisme réactif de garantie des contraintes temporelles. Cela constitue une base qui ne demande qu'à être étoffée et améliorée.

Les résultats expérimentaux que nous avons obtenus sont encourageants dans le sens où ils démontrent la capacité d'un tel mécanisme à offrir des garanties de respect d'échéance avec des compromis limités sur l'exécution des tâches non critiques. Cependant, ils sont spécifiques au système à criticité mixte auquel il est appliqué. Le diagnostic sur la présence d'interférences et leurs effets sur l'exécution des tâches critiques est hautement dépendant à la fois du support matériel et du comportement des tâches en elles-mêmes. L'implémentation d'un tel mécanisme est ainsi conditionné à la caractérisation du système dans lequel il s'inscrit.
 
En ce sens, la généricité de la solution et son applicabilité à un grand nombre d'architectures tant matérielles que logicielles est une force. En effet, bien que réfléchie en partant d'un contexte automobile, notre proposition a été élargie pour s'adapter à des cas industriels de systèmes à criticité mixte divers et variés. 

\paragraph*{} La première piste d'amélioration réside dans l'extension de la proposition actuelle de façon à gérer plus d'une unique chaîne de tâches critiques. Il s'agit de l'extension la plus directe à prendre en compte, qui apporte un grand avantage sur les possibilités d'utilisations du mécanisme de surveillance et contrôle. 

Considérant la proposition actuelle, l'enjeu à résoudre pour cela réside dans la capacité à obtenir des garanties de respect d'échéance dans un mode dégradé qui prend en compte plusieurs chaînes de tâches. Cela implique donc un plus grand nombre de tâches dans ce mode, et donc des interférences inter-chaînes. Cela ne pourrait se faire qu'à la condition de connaître avec précision ces interférences-là, et les prendre en compte dans les estimations de pire temps d'exécution restant de chacune des chaînes de tâches critiques.

\paragraph*{} Une seconde piste de développement repose sur l'extension du modèle de tâches à criticité duale que nous avons employé. Bien qu'un bon nombre de systèmes reposent uniquement sur deux niveaux de criticité, l’extension à un plus grand nombre pourrait correspondre à une plus grande variété de cas d'applications. Ainsi, des domaines comme le ferroviaire, l'avionique ou l'automobile emploient 5 niveaux de criticité qui ont une incidence sur les contraintes de développement du logiciel. Proposer plusieurs niveaux de criticité aurait donc le double avantage d'être plus facilement combinable avec d'autres travaux de recherche qui travaillent sur plus de niveaux, et de potentiellement approfondir les possibilités de modes dégradés intermédiaires. En effet, les tâches non critiques peuvent alors être subdivisés dans des nuances de niveaux de criticité qui peuvent ne pas être traitées de la même façon par le mécanisme de contrôle.

\paragraph*{} Pour finir, dans l'idée de combiner notre mécanisme avec d'autres solutions de contrôle de l'exécution des tâches, il serait possible de proposer des niveaux intermédiaires de modes dégradés. Mais cette fois-ci il ne s'agirait pas de jouer sur quelles tâches sont mises en pause en mode dégradé, mais plutôt de proposer des solutions de contrôle moins drastiques, qui n'empêchent pas l'exécution des tâches non critiques, mais limitent par d'autres biais leurs capacités de nuisance par interférences. L'un de ces moyens repose sur l'exploitation du complément à notre méthode logicielle réactive~: une solution matérielle statique. En effet, par la possibilité de mettre en place des mécanismes d'isolation spatiale des tâches en cours d'exécution, il serait possible de limiter les interférences sur les ressources partagées par la réservation d'une partie du cache aux tâches critiques. Une telle amélioration permettrait la mise en place de mécanismes préventifs progressifs contre les interférences entre les tâches. De façon cela limiterai au maximum la sous exploitation des ressources de calcul, tout en garantissant au mieux le respect des échéances temporelles sur les chaînes de tâches critiques. 

% proposée la rend indépendante du support matériel. Aussi le mécanisme de Surveillance et de Contrôle présente peu de contraintes logicielles. Il s'agit donc d'une proposition adaptable pour un
\ifdefined\included
\else
\bibliographystyle{StyleThese}
\bibliography{these}
\end{document}
\fi