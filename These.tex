% Choose the language of your thesis passing 'french' or 'english' as
% \documentclass option.
% Note1: The 'page de garde' will always be written in French.
% Note2: You will have an error if you change the language of the document and
%        compile it without cleaning the auxiliary files. Compiling it again
%        should solve the problem.
\documentclass[french, a4paper, 11pt, twoside, pdftex]{StyleThese}

\newcommand{\included}{}
%\overfullrule=.5cm %% GROS TRAIT NOIR quand ça dépasse de la marge !

\usepackage{iflang}
\usepackage{bibentry}



%\usepackage[sectionbib]{chapterbib}          % Cross-reference package (Natural BiB)
%\usepackage{natbib}                  % Put References at the end of each chapter
%\usepackage{bibunits}
% Do not put 'sectionbib' option here.
% Sectionbib option in 'natbib' will do.


\usepackage{fancyhdr}                    % Fancy Header and Footer

\usepackage[utf8]{inputenc}
\usepackage[T1]{fontenc}
\usepackage[french]{babel} %
\usepackage{lmodern} \normalfont %to load T1lmr.fd 
\DeclareFontShape{T1}{cmr}{b}{sc} { <-> ssub * cmr/bx/sc }{}
%\hyphenation{gar}

\usepackage{amsmath,amssymb}             % AMS Math
\usepackage{nicefrac}
\usepackage{siunitx}					%% Unites Math SI

\usepackage{blindtext}

\usepackage{datetime}

\usepackage{lipsum} 

\usepackage[inline]{enumitem}

\usepackage{hhline}
%\usepackage[left=1.5in,right=1.3in,top=1.1in,bottom=1.1in]{geometry}
\usepackage[left=1.5in,right=1.3in,top=1.1in,bottom=1.1in,includefoot,includehead,headheight=13.6pt]{geometry}

%%\renewcommand{\baselinestretch}{1.05}

%%%%%%%% Multi-figures avec sub-captions
\usepackage{caption}
\usepackage{subcaption}

% Table of contents for each chapter

\usepackage[nottoc, notlof, notlot]{tocbibind}
\usepackage[nohints]{minitoc}
\setcounter{minitocdepth}{2}
\mtcindent=15pt
% Use \minitoc where to put a table of contents

\usepackage{aecompl}

%% Package cosmetic meilleur layout du texte en jouant sur le spacing par caractères
\usepackage[activate={true,nocompatibility},final,tracking=true,kerning=true,factor=1100,stretch=10,shrink=10]{microtype}
\usepackage[absolute,overlay]{textpos} 
\setlength{\TPHorizModule}{\paperwidth}\setlength{\TPVertModule}{\paperheight}
\sloppy

%%%%%%%%%%% JOLIS TABLEAUX
\usepackage{tabularx}		%\usepackage{tabular}
\usepackage{multirow}
\newcommand{\mc}{\multicolumn} 
\newcommand{\mr}[2]{\multirow{#1}{*}{#2}} 	\newcommand{\mrQ}{\multirow{-4}{*}}
\usepackage{booktabs}

\usepackage[usenames,dvipsnames]{xcolor} 

\makeatletter
\newcommand{\ccolor}[3][]{%
	\kern-\fboxsep
	\if\relax\detokenize{#1}\relax
	\expandafter\@firstoftwo
	\else
	\expandafter\@secondoftwo
	\fi
	{\colorbox{#2}}%
	{\colorbox[#1]{#2}}%
	{#3}\kern-\fboxsep
}
\makeatother

%%%%% Insertion graphiques format PGF
\usepackage{pgfplots}
\pgfplotsset{width=\linewidth, compat=1.16}%, compat=1.17}
\usepackage{adjustbox}          %%% PERMET DE LES RECADRER + FACILEMENT


%%%%%%%%%% Bullets de listes sans saut de ligne %%%%%%%%%%
\usepackage{xparse}

\ExplSyntaxOn%
\seq_new:N \l_local_enum_seq

\newcommand{\storethestuff}[1]{%
  \seq_set_from_clist:Nn \l_local_enum_seq {#1}%
}

\newcommand{\dotheenumstuff}{%
\int_zero:N \l_tmpa_int
\seq_map_inline:Nn \l_local_enum_seq {%
    \int_incr:N \l_tmpa_int% Increase the counter
    \item ##1
    % Check whether the list has reached the end -- if so, use '.' instead of ','
    %\int_compare:nNnTF 
    % { \l_tmpa_int } < {\seq_count:N \l_local_enum_seq} 
    % {,} {.}
  }
}
\ExplSyntaxOff

\NewDocumentCommand{\linebullets}{+m}{%
  \storethestuff{#1}%
  \begin{enumerate*}[label={\alph*)},font={\bfseries},itemjoin={{, }}]
    \dotheenumstuff%
  \end{enumerate*}
}

\newcommand{\cmnt}[1]{}  %%%%% AJOUT DE COMMENTAIRE MULTILIGNES


%%%%%%%%%% ECRITURE CARACTERES DANS UN CERCLE %%%%%%%%%%
%\def\circleTxt[#1]{\raisebox{.5pt}{\textcircled{\raisebox{-1pt}{#1}}}}
\newcommand{\ctxt}[1]{\raisebox{.5pt}{\textcircled{\raisebox{-1.2pt}{#1}}}}
% Glossary / list of abbreviations

\usepackage[intoc]{nomencl}
\IfLanguageName{english}{%
\renewcommand{\nomname}{Glossary}
}{ %
\renewcommand{\nomname}{Liste des Abréviations}
}

\makenomenclature

% My pdf code

\usepackage{ifpdf}

\ifpdf
  \usepackage[pdftex]{graphicx}
  \DeclareGraphicsExtensions{.pdf,PDF,.png,PNG,.jpg,JPG}
  \usepackage[pagebackref,hyperindex=true]{hyperref} %% use \autoref{} instead of Table~\ref{}.
  \usepackage{tikz}
  \usetikzlibrary{arrows,shapes,calc}
\else
  \usepackage{graphicx}
  \DeclareGraphicsExtensions{.ps,.eps}
  \usepackage[a4paper,dvipdfm,pagebackref,hyperindex=true]{hyperref}
\fi

\graphicspath{{.}{schemas/}{graphiques/}{tables/}}

%% nicer backref links. NOTE: The flag ThesisInEnglish is used to define the
% language in the back references. Read more about it in These.tex

\IfLanguageName{english}{
\renewcommand*{\backref}[1]{}
\renewcommand*{\backrefalt}[4]{%
\ifcase #1 %
(Not cited.)%
\or
(Cited in page~#2.)%
\else
(Cited in pages~#2.)%
\fi}
\renewcommand*{\backrefsep}{, }
\renewcommand*{\backreftwosep}{ and~}
\renewcommand*{\backreflastsep}{ and~}
}{
\renewcommand*{\backref}[1]{}
\renewcommand*{\backrefalt}[4]{%
\ifcase #1 %
(Non cité.)%
\or
(Cité en page~#2.)%
\else
(Cité en pages~#2.)%
\fi}
\renewcommand*{\backrefsep}{, }
\renewcommand*{\backreftwosep}{ et~}
\renewcommand*{\backreflastsep}{ et~}
}

% Links in pdf
\usepackage{color}
\definecolor{linkcol}{rgb}{0,0,0.4} 
\definecolor{citecol}{rgb}{0.5,0,0} 
\definecolor{linkcol}{rgb}{0,0,0} 
\definecolor{citecol}{rgb}{0,0,0}
% Change this to change the informations included in the pdf file

\hypersetup
{
bookmarksopen=true,
pdftitle="Prévention des fautes temporelles sur architectures multicœur pour les systèmes à criticité mixte",
pdfauthor="Daniel LOCHE", %auteur du document
pdfsubject="Thèse", %sujet du document
%pdftoolbar=false, %barre d'outils non visible
pdfmenubar=true, %barre de menu visible
pdfhighlight=/O, %effet d'un clic sur un lien hypertexte
colorlinks=true, %couleurs sur les liens hypertextes
pdfpagemode=UseNone, %aucun mode de page
%pdfpagelayout=DoublePage, %ouverture en simple page
pdffitwindow=true, %pages ouvertes entierement dans toute la fenetre
linkcolor=linkcol, %couleur des liens hypertextes internes
citecolor=citecol, %couleur des liens pour les citations
urlcolor=linkcol %couleur des liens pour les url
}

% definitions.
% -------------------

\setcounter{secnumdepth}{3}
\setcounter{tocdepth}{2}

% Some useful commands and shortcut for maths:  partial derivative and stuff

\newcommand{\pd}[2]{\frac{\partial #1}{\partial #2}}
\def\abs{\operatorname{abs}}
\def\argmax{\operatornamewithlimits{arg\,max}}
\def\argmin{\operatornamewithlimits{arg\,min}}
\def\diag{\operatorname{Diag}}
\newcommand{\eqRef}[1]{(\ref{#1})}
\newcommand{\nline}{\smallbreak\noindent}

\usepackage{rotating}                    % Sideways of figures & tables

% \usepackage{txfonts}                     % Public Times New Roman text & math font
  
%%% Fancy Header %%%%%%%%%%%%%%%%%%%%%%%%%%%%%%%%%%%%%%%%%%%%%%%%%%%%%%%%%%%%%%%%%%
% Fancy Header Style Options

\pagestyle{fancy}                       % Sets fancy header and footer
\fancyfoot{}                            % Delete current footer settings

%\renewcommand{\chaptermark}[1]{         % Lower Case Chapter marker style
%  \markboth{\chaptername\ \thechapter.\ #1}}{}} %

%\renewcommand{\sectionmark}[1]{         % Lower case Section marker style
%  \markright{\thesection.\ #1}}         %

\fancyhead[LE,RO]{\bfseries\thepage}    % Page number (boldface) in left on even
% pages and right on odd pages
\fancyhead[RE]{\bfseries\nouppercase{\leftmark}}      % Chapter in the right on even pages
\fancyhead[LO]{\bfseries\nouppercase{\rightmark}}     % Section in the left on odd pages

\let\headruleORIG\headrule
\renewcommand{\headrule}{\color{black} \headruleORIG}
\renewcommand{\headrulewidth}{1.0pt}
\usepackage{colortbl}
\arrayrulecolor{black}

\fancypagestyle{plain}{
  \fancyhead{}
  \fancyfoot{}
  \renewcommand{\headrulewidth}{0pt} %%%%%%%%%%%%%%%%%%%%%%%%%%%%%%%%%%%%%%%%%%%%%%%%%%%%%%%%%%%%%%%%%%%%%%%%%%%%%%%%%%%%%
}

%\usepackage{MyAlgorithm}
%\usepackage[noend]{MyAlgorithmic}
%\usepackage[ED=EDSYS-SystEmb, Ets=INP]{tlsflyleaf}

%%% Clear Header %%%%%%%%%%%%%%%%%%%%%%%%%%%%%%%%%%%%%%%%%%%%%%%%%%%%%%%%%%%%%%%%%%
% Clear Header Style on the Last Empty Odd pages
\makeatletter

\def\cleardoublepage{\clearpage\if@twoside \ifodd\c@page\else%
  \hbox{}%
  \thispagestyle{empty}%              % Empty header styles
  \newpage%
  \if@twocolumn\hbox{}\newpage\fi\fi\fi}

\makeatother
 
%%%%%%%%%%%%%%%%%%%%%%%%%%%%%%%%%%%%%%%%%%%%%%%%%%%%%%%%%%%%%%%%%%%%%%%%%%%%%%% 
% Prints your review date and 'Draft Version' (From Josullvn, CS, CMU)
\newcommand{\reviewtimetoday}[2]{\special{!userdict begin
    /bop-hook{gsave 20 710 translate 45 rotate 0.8 setgray
      /Times-Roman findfont 12 scalefont setfont 0 0   moveto (#1) show
      0 -12 moveto (#2) show grestore}def end}}
% You can turn on or off this option.
% \reviewtimetoday{\today}{Draft Version}
%%%%%%%%%%%%%%%%%%%%%%%%%%%%%%%%%%%%%%%%%%%%%%%%%%%%%%%%%%%%%%%%%%%%%%%%%%%%%%% 

\newenvironment{maxime}[1]
{
	\def\Arg{#1}
\vspace*{0cm}
\hfill
\begin{minipage}{0.6\textwidth}%
%\rule[0.5ex]{\textwidth}{0.1mm}\\%
\hrulefill $\:$ \\%$\:$ {\bf #1}\\
%\vspace*{-0.25cm}
\it 
}%
{%
	
\hrulefill $\:$ {\bf \Arg}
\vspace*{0.5cm}%
\end{minipage}
}

\let\minitocORIG\minitoc
\renewcommand{\minitoc}{\minitocORIG \vspace{1.5em}}

%\usepackage{slashbox}

\newenvironment{bulletList}%
{ \begin{list}%
	{$\bullet$}%
	{\setlength{\labelwidth}{25pt}%
	 \setlength{\leftmargin}{30pt}%
	 \setlength{\itemsep}{\parsep}}}%
{ \end{list} }


%%%%%%% Outils pour \comment \alert \add %%%%%
\usepackage{easyReview}
\usepackage{soulutf8} % for accented letters

\let\newalert\alert
\renewcommand{\alert}[1]{\textit{\newalert{#1}}}

%\usepackage[commandnameprefix=ifneeded]{changes} %% \chhighlight and \chcomment to avoid collision with easyReview
\renewcommand{\epsilon}{\varepsilon}

% centered page environment

\newenvironment{vcenterpage}
{\newpage\vspace*{\fill}\thispagestyle{empty}\renewcommand{\headrulewidth}{0pt}}
{\vspace*{\fill}}

\usepackage{tablefootnote}

%%%%%% MISE EN FORME CADRES DEFINITIONS/THEOREMES/LEMES %%%%%%%%%%
\usepackage{amsthm}  % for theoremstyle

\theoremstyle{plain} 
\newtheorem{theorem}{Théorème}[section]
\newtheorem{corollary}{Corolaire}[theorem]

%\theoremstyle{lemma}
%\newtheorem{lemma}[theorem]{Lemme}


\theoremstyle{definition}
\newtheorem{definition}[theorem]{Définition}


\cmnt{
	\usepackage{ntheorem} %\usepackage{amsthm}  % for theoremstyle
	%\usepackage{mdframed}
	\usepackage[most]{tcolorbox}
	
	\theoremstyle{plain} 
	\theoremindent20pt
	\theoremheaderfont{\normalfont\bfseries\hspace{-\theoremindent}}
	\newtheorem{theorem}{Théorème}[section]
	\newtheorem{corollary}{Corolaire}[theorem]
	
	\theoremstyle{plain}
	\newtheorem{lemma}[theorem]{Lemme}
	
	
	\tcolorboxenvironment{theorem}{
		blanker,
		breakable,
		before skip=\topsep,
		after skip=\topsep,
		borderline west={1pt}{10pt}{double, shorten <=12pt}
	}
	
	\theorembodyfont{\normalfont}
	\theoremindent20pt
	\theoremheaderfont{\normalfont\bfseries\hspace{-\theoremindent}}
	\newtheorem{definition}[theorem]{Définition}
	
	
	\tcolorboxenvironment{definition}{
		blanker,
		breakable,
		before skip=\topsep,
		after skip=\topsep,
		borderline west={1pt}{10pt}{shorten <=12pt}
	}
}

\cmnt{ 
	\begin{theorem}
		Ceci est un Théorème.
	\end{theorem} 
	
	\begin{corollary}
		Ceci est un Corollaire.
	\end{corollary}
	
	\begin{definition}
		Ceci est une Définition.
	\end{definition}
	
	\begin{lemma}
		Ceci est un Lemme.
	\end{lemma}
}

\def\UrlBigBreaks{\do\/\do-\do:}
\usepackage{url}


\renewcommand\nomgroup[1]{%  % GROUPEMENT DES ÉLÉMENTS DE LA NOMENCLATURE. ON PEUT EN AJOUTER AUTANT QUE NÉCESSAIRE.
	\item[\bfseries
	\ifstrequal{#1}{D}{Abréviations}{%
		\ifstrequal{#1}{A}{Modèle de Tâches}{%
			\ifstrequal{#1}{B}{Modèle de Chaîne de tâches}{%
				\ifstrequal{#1}{C}{Mécanisme de Surveillance et Contrôle}{%
		}}}}%
		]
	}
\setlength{\nomlabelwidth}{2cm}			%% PARAMÉTRABLE POUR ADAPTER LE PLACEMENT DES COLONNES DE LA NOMENCLATURE, SELON LA LONGUEUR DE VOTRE ABRÉVIATION LA PLUS LONGUE

%% Copyright 2013 Tristan GREGOIRE
%% Copyright 2015 Yann BACHY
%
% This work may be distributed and/or modified under the
% conditions of the LaTeX Project Public License, either version 1.3
% of this license or (at your option) any later version.
% The latest version of this license is in
%   http://www.latex-project.org/lppl.txt
% and version 1.3 or later is part of all distributions of LaTeX
% version 2005/12/01 or later.
%
%
% This work has the LPPL maintenance status `maintained'.
% 
% The Current Maintainer of this work is T. GREGOIRE
%

%\documentclass{book}

% Loading the tlsflyleaf.sty package require some option to define the
% establishment name, the doctoral school and the PhD speciality.
% In that aim you have 2 key-value option:
%   - Ets=<value> : define the establishment name
%   - ED=<value>  : define the doctoral school and speciality
%   - ED2=<value> : define the second speciality ("double mention"). OPTIONAL.
% The full list of accepted values for each option could be find either
% in the documentation or in ED-list.txt and Ets-list.txt files provide with the package.
%\usepackage[ED=MITT - STICRT, Ets=INSA]{tlsflyleaf}
%\usepackage[ED=SDU2E-Ast, ED2=SDU2E-Eco, Ets=UT3]{tlsflyleaf}
\usepackage[ED=EDSYS-SystEmb, Ets=INP]{tlsflyleaf}

% ==================
% Setup basic string
% - PhD Title
% - author
% - defence date
% - laboratory
% - cotutelle
\title{Prévention des fautes temporelles sur architectures multicœur pour les systèmes à criticité mixte}
\author{Daniel LOCHE}
\defencedate{01/07/2022}
\lab{Laboratoire d'analyse et d'architecture des systèmes -- LAAS-CNRS}
%\cotutelle{}

% ==================
% Setup people like your boss, the jury team and the referees
% - First you need to define how number they will be in each category
%   It is done with the commands \nboss{n}, \nreferee{n} and \njudge{n}.
%   You can define more people in each category than the number given 
%   but only the first "\npeople" will be print.
% - Then use the command \makesomeone{<category>}{<number>}{<name>}{<status>}{<other>}
%   where:
%     <category> should be select in ['boss', 'referee', 'judge', 'invitee']
%     <number>   is the rank for printing the person. 
%                Only number <= "\npeople" will be printed
%     <name>     First name and last name of the people
%     <status>   Is (s)he a "charg\'e de recher" ou un "professeur d'universit\'e"...
%     <other>    What ever string you want to add (laboratory, jury member place...).
%% Boss
\nboss{2}
\makesomeone{boss}{2}{Michael LAUER}{}{}  % Sera affiche en second
\makesomeone{boss}{1}{Jean-Charles FABRE}{}{} % Sera afiche en premier
%% Referee
\nreferee{2}
\makesomeone{referee}{1}{Liliana CUCU-GROSJEAN}{Chargée de Recherche}{INRIA, Paris}
\makesomeone{referee}{2}{Emmanuel GROLLEAU}{Professeur}{ISAE-ENSMA, Poitiers}
%% Judges
\njudge{7}
%\cmnt
{
\makesomeone{judge}{3}{Claire PAGETTI}{Ingénieure de Recherche, \\ONERA}{Examinatrice}
\makesomeone{judge}{5}{Jean-Charles FABRE}{Professeur des Universités, \\Toulouse INP}{Directeur de Thèse}
\makesomeone{judge}{4}{Sébastien FAUCOU}{Maître de Conférences, \\Univ. de Nantes}{Examinateur}
\makesomeone{judge}{6}{Michael LAUER}{Maître de Conférences, \\Univ. Toulouse 3}{Co-directeur de Thèse}
\makesomeone{judge}{1}{Liliana CUCU-GROSJEAN}{Directrice de Recherche, \\INRIA}{Rapporteure}
\makesomeone{judge}{2}{Emmanuel GROLLEAU}{Professeur des Universités, \\ISAE-ENSMA}{Rapporteur}
}
\cmnt
{
\makesomeone{judge}{1}{Claire PAGETTI}{Ing. de Recherche\cmnt{, \\ONERA}}{Président du Jury}
\makesomeone{judge}{5}{Jean-Charles FABRE}{Professeur des Universités\cmnt{, \\Toulouse INP}}{Directeur de Thèse}
\makesomeone{judge}{2}{Sébastien FAUCOU}{Maître de Conférences\cmnt{, \\Univ. de Nantes}}{Examinateur}
\makesomeone{judge}{6}{Michael LAUER}{Maître de Conférences\cmnt{, \\Univ. Toulouse 3}}{Co-directeur de Thèse}
\makesomeone{judge}{4}{Liliana CUCU-GROSJEAN}{Directrice de Recherche\cmnt{, \\INRIA}}{Rapporteure}
\makesomeone{judge}{3}{Emmanuel GROLLEAU}{Professeur des Universités\cmnt{, \\ISAE-ENSMA}}{Rapporteur}
}
%% Invitees
%\ninvitee{1}
\makesomeone{judge}{7}{François GŒUSSE}{Ingénieur, Renault SWLabs}{Invité}

% ============================================================
% DOCUMENT
%\begin{document}
%    \makeflyleaf
%\end{document}


\begin{document}

\makeflyleaf
\cleardoublepage

\dominitoc

\pagenumbering{roman}

 \cleardoublepage
%%%%%%%%%%%%%%%%%%
 %%%%% MAXIME %%%%%
%\topskip-10cm
\vspace*{5cm}
% \begin{maxime}{Pierre-Henri Cami.}
%	Souvent une évolution est une révolution sans en avoir l'R.
%\end{maxime}
\epigraph{{\large Ceci est ma citation pour faire stylé}}
{{\smallbreak\large \textit{Moi-même}}}
%\vspace*{\fill}

\cleardoublepage

%Another use of \if toggle can be found at the end of this file.
\IfLanguageName{english}{%
\section*{Acknowledgments}
}{%
\section*{Préface}
\addstarredchapter{Préface}

	Remerciements, à écrire à la fin.

}


\tableofcontents
%\makenomenclature


 %%%%%%%%%%%%%%%%%%%%%%%%
{ %%%%% NOMENCLATURE %%%%%  [lettre, ordre]{contenu}
								% => lettre = groupe tel que défini plus haut.
								% => ordre = si différents, permet de forcer un ordre d'apparition (sinon, ordre alphébétique)
								% => contenu = bah ton contenu, andouille.
	\nomenclature[A, 01]{$\tau_i$}{Tâche, d'indice i}
	\nomenclature[A, 03]{$T_i$}{Période d'apparition d'une tâche}
	\nomenclature[A, 04]{$D_i$}{Date limite d'échéance à la terminaison d'une tâche}
	\nomenclature[A, 05]{$C_i$}{Temps d'exécution pire cas d'une tâche}
	\nomenclature[A, 06]{$L_i$}{Niveau de criticité d'une tâche}
	\nomenclature[A, 07]{$\tau_{i,j}$}{j\up{ème} occurrence d'exécution de la tâche $\tau_i$ à sa période $T_i$}
	\nomenclature[A, 08]{$a_{i,j}$}{Moment d'activation du job $\tau_{i,j}$}
	\nomenclature[A, 09]{$s_{i,j}$}{Début d'exécution du job $\tau_{i,j}$}
	\nomenclature[A, 10]{$e_{i,j}$}{Fin d'exécution (terminaison) du job $\tau_{i,j}$}
	\nomenclature[B, 20]{$succ(\tau_i)$}{Fonction qui permet de trouver le job successeur du job $\tau_i$ d'une chaîne de tâches}
	\nomenclature[B, 21]{$succ^p(\tau_i)$}{fonction itérative pour trouver le p\up{ème} successeur du job $\tau_i$}
	\nomenclature[B, 15]{$R_j$}{Temps de réponse bout-en-bout d'une chaîne de tâches}
	\nomenclature[B, 16]{$D_c$}{Délai limite d'échéance d'exécution bout-en-bout d'une chaîne de tâche}
	\nomenclature[B, 23]{$S(t)$}{Etat de la chaîne de tâches à l'instant $t$}
	\nomenclature[B, 24]{$ET(j,t)$}{j\up{ème} trace d'exécution de la chaîne de tâches, à un instant t}
	\nomenclature[B, 30]{$RT(t)$}{Temps de réponse partiel d'une chaîne de tâche active à l'instant t}
	\nomenclature[B, 31]{$rWCRT(t)$}{Pire Temps de Réponse Restant d'une chaîne de tâche}
	\nomenclature[C, 42]{$T_{CCC}$}{Période de fonctionnement du Core Control Component}
	\nomenclature[C, 43]{$W_{MAX}$}{Durée maximale garantie entre 2 points de surveillance}
	\nomenclature[C, 44]{$t_{SW}$}{Latence de changement de mode}
	\nomenclature[C, 40]{TWC}{Task-Wrapper Component : module d'encapsulation pour la surveillance des tâches}
	\nomenclature[C, 41]{CCC}{Core Control Component : module de décision de changement de mode}

}

\printnomenclature
\cmnt{	 %% IMPORTANT SI T'AS DES SOUCIS DE DÉCALLAGE ENTRE LES PLANS DANS CHAQUE CHAPITRE ET LE NUMÉRO DES CHAPITRES !!
% Use \mtcfixnomenclature below if you have a glossary (added with
% \printnomenclature above) and you're see a shift in the mini-table of
% contents at the begining of each chapter (example: no mini-toc in chapter 1;
% mini-toc of chapter 1 appearing in chapter 2; and so on).
%
% You should not use \mtcfixnomenclature if you have no glossary (that means,
% if you don't use \printnomenclature or if your glossary is empty).
}
\mtcfixnomenclature


\mainmatter

\ifdefined\included
\else
\documentclass[french, a4paper, 11pt, twoside, pdftex]{StyleThese}
\usepackage{iflang}
\usepackage{bibentry}



%\usepackage[sectionbib]{chapterbib}          % Cross-reference package (Natural BiB)
%\usepackage{natbib}                  % Put References at the end of each chapter
%\usepackage{bibunits}
% Do not put 'sectionbib' option here.
% Sectionbib option in 'natbib' will do.


\usepackage{fancyhdr}                    % Fancy Header and Footer

\usepackage[utf8]{inputenc}
\usepackage[T1]{fontenc}
\usepackage[french]{babel} %
\usepackage{lmodern} \normalfont %to load T1lmr.fd 
\DeclareFontShape{T1}{cmr}{b}{sc} { <-> ssub * cmr/bx/sc }{}
%\hyphenation{gar}

\usepackage{amsmath,amssymb}             % AMS Math
\usepackage{nicefrac}
\usepackage{siunitx}					%% Unites Math SI

\usepackage{blindtext}

\usepackage{datetime}

\usepackage{lipsum} 

\usepackage[inline]{enumitem}

\usepackage{hhline}
%\usepackage[left=1.5in,right=1.3in,top=1.1in,bottom=1.1in]{geometry}
\usepackage[left=1.5in,right=1.3in,top=1.1in,bottom=1.1in,includefoot,includehead,headheight=13.6pt]{geometry}

%%\renewcommand{\baselinestretch}{1.05}

%%%%%%%% Multi-figures avec sub-captions
\usepackage{caption}
\usepackage{subcaption}

% Table of contents for each chapter

\usepackage[nottoc, notlof, notlot]{tocbibind}
\usepackage[nohints]{minitoc}
\setcounter{minitocdepth}{2}
\mtcindent=15pt
% Use \minitoc where to put a table of contents

\usepackage{aecompl}

%% Package cosmetic meilleur layout du texte en jouant sur le spacing par caractères
\usepackage[activate={true,nocompatibility},final,tracking=true,kerning=true,factor=1100,stretch=10,shrink=10]{microtype}
\usepackage[absolute,overlay]{textpos} 
\setlength{\TPHorizModule}{\paperwidth}\setlength{\TPVertModule}{\paperheight}
\sloppy

%%%%%%%%%%% JOLIS TABLEAUX
\usepackage{tabularx}		%\usepackage{tabular}
\usepackage{multirow}
\newcommand{\mc}{\multicolumn} 
\newcommand{\mr}[2]{\multirow{#1}{*}{#2}} 	\newcommand{\mrQ}{\multirow{-4}{*}}
\usepackage{booktabs}

\usepackage[usenames,dvipsnames]{xcolor} 

\makeatletter
\newcommand{\ccolor}[3][]{%
	\kern-\fboxsep
	\if\relax\detokenize{#1}\relax
	\expandafter\@firstoftwo
	\else
	\expandafter\@secondoftwo
	\fi
	{\colorbox{#2}}%
	{\colorbox[#1]{#2}}%
	{#3}\kern-\fboxsep
}
\makeatother

%%%%% Insertion graphiques format PGF
\usepackage{pgfplots}
\pgfplotsset{width=\linewidth, compat=1.16}%, compat=1.17}
\usepackage{adjustbox}          %%% PERMET DE LES RECADRER + FACILEMENT


%%%%%%%%%% Bullets de listes sans saut de ligne %%%%%%%%%%
\usepackage{xparse}

\ExplSyntaxOn%
\seq_new:N \l_local_enum_seq

\newcommand{\storethestuff}[1]{%
  \seq_set_from_clist:Nn \l_local_enum_seq {#1}%
}

\newcommand{\dotheenumstuff}{%
\int_zero:N \l_tmpa_int
\seq_map_inline:Nn \l_local_enum_seq {%
    \int_incr:N \l_tmpa_int% Increase the counter
    \item ##1
    % Check whether the list has reached the end -- if so, use '.' instead of ','
    %\int_compare:nNnTF 
    % { \l_tmpa_int } < {\seq_count:N \l_local_enum_seq} 
    % {,} {.}
  }
}
\ExplSyntaxOff

\NewDocumentCommand{\linebullets}{+m}{%
  \storethestuff{#1}%
  \begin{enumerate*}[label={\alph*)},font={\bfseries},itemjoin={{, }}]
    \dotheenumstuff%
  \end{enumerate*}
}

\newcommand{\cmnt}[1]{}  %%%%% AJOUT DE COMMENTAIRE MULTILIGNES


%%%%%%%%%% ECRITURE CARACTERES DANS UN CERCLE %%%%%%%%%%
%\def\circleTxt[#1]{\raisebox{.5pt}{\textcircled{\raisebox{-1pt}{#1}}}}
\newcommand{\ctxt}[1]{\raisebox{.5pt}{\textcircled{\raisebox{-1.2pt}{#1}}}}
% Glossary / list of abbreviations

\usepackage[intoc]{nomencl}
\IfLanguageName{english}{%
\renewcommand{\nomname}{Glossary}
}{ %
\renewcommand{\nomname}{Liste des Abréviations}
}

\makenomenclature

% My pdf code

\usepackage{ifpdf}

\ifpdf
  \usepackage[pdftex]{graphicx}
  \DeclareGraphicsExtensions{.pdf,PDF,.png,PNG,.jpg,JPG}
  \usepackage[pagebackref,hyperindex=true]{hyperref} %% use \autoref{} instead of Table~\ref{}.
  \usepackage{tikz}
  \usetikzlibrary{arrows,shapes,calc}
\else
  \usepackage{graphicx}
  \DeclareGraphicsExtensions{.ps,.eps}
  \usepackage[a4paper,dvipdfm,pagebackref,hyperindex=true]{hyperref}
\fi

\graphicspath{{.}{schemas/}{graphiques/}{tables/}}

%% nicer backref links. NOTE: The flag ThesisInEnglish is used to define the
% language in the back references. Read more about it in These.tex

\IfLanguageName{english}{
\renewcommand*{\backref}[1]{}
\renewcommand*{\backrefalt}[4]{%
\ifcase #1 %
(Not cited.)%
\or
(Cited in page~#2.)%
\else
(Cited in pages~#2.)%
\fi}
\renewcommand*{\backrefsep}{, }
\renewcommand*{\backreftwosep}{ and~}
\renewcommand*{\backreflastsep}{ and~}
}{
\renewcommand*{\backref}[1]{}
\renewcommand*{\backrefalt}[4]{%
\ifcase #1 %
(Non cité.)%
\or
(Cité en page~#2.)%
\else
(Cité en pages~#2.)%
\fi}
\renewcommand*{\backrefsep}{, }
\renewcommand*{\backreftwosep}{ et~}
\renewcommand*{\backreflastsep}{ et~}
}

% Links in pdf
\usepackage{color}
\definecolor{linkcol}{rgb}{0,0,0.4} 
\definecolor{citecol}{rgb}{0.5,0,0} 
\definecolor{linkcol}{rgb}{0,0,0} 
\definecolor{citecol}{rgb}{0,0,0}
% Change this to change the informations included in the pdf file

\hypersetup
{
bookmarksopen=true,
pdftitle="Prévention des fautes temporelles sur architectures multicœur pour les systèmes à criticité mixte",
pdfauthor="Daniel LOCHE", %auteur du document
pdfsubject="Thèse", %sujet du document
%pdftoolbar=false, %barre d'outils non visible
pdfmenubar=true, %barre de menu visible
pdfhighlight=/O, %effet d'un clic sur un lien hypertexte
colorlinks=true, %couleurs sur les liens hypertextes
pdfpagemode=UseNone, %aucun mode de page
%pdfpagelayout=DoublePage, %ouverture en simple page
pdffitwindow=true, %pages ouvertes entierement dans toute la fenetre
linkcolor=linkcol, %couleur des liens hypertextes internes
citecolor=citecol, %couleur des liens pour les citations
urlcolor=linkcol %couleur des liens pour les url
}

% definitions.
% -------------------

\setcounter{secnumdepth}{3}
\setcounter{tocdepth}{2}

% Some useful commands and shortcut for maths:  partial derivative and stuff

\newcommand{\pd}[2]{\frac{\partial #1}{\partial #2}}
\def\abs{\operatorname{abs}}
\def\argmax{\operatornamewithlimits{arg\,max}}
\def\argmin{\operatornamewithlimits{arg\,min}}
\def\diag{\operatorname{Diag}}
\newcommand{\eqRef}[1]{(\ref{#1})}
\newcommand{\nline}{\smallbreak\noindent}

\usepackage{rotating}                    % Sideways of figures & tables

% \usepackage{txfonts}                     % Public Times New Roman text & math font
  
%%% Fancy Header %%%%%%%%%%%%%%%%%%%%%%%%%%%%%%%%%%%%%%%%%%%%%%%%%%%%%%%%%%%%%%%%%%
% Fancy Header Style Options

\pagestyle{fancy}                       % Sets fancy header and footer
\fancyfoot{}                            % Delete current footer settings

%\renewcommand{\chaptermark}[1]{         % Lower Case Chapter marker style
%  \markboth{\chaptername\ \thechapter.\ #1}}{}} %

%\renewcommand{\sectionmark}[1]{         % Lower case Section marker style
%  \markright{\thesection.\ #1}}         %

\fancyhead[LE,RO]{\bfseries\thepage}    % Page number (boldface) in left on even
% pages and right on odd pages
\fancyhead[RE]{\bfseries\nouppercase{\leftmark}}      % Chapter in the right on even pages
\fancyhead[LO]{\bfseries\nouppercase{\rightmark}}     % Section in the left on odd pages

\let\headruleORIG\headrule
\renewcommand{\headrule}{\color{black} \headruleORIG}
\renewcommand{\headrulewidth}{1.0pt}
\usepackage{colortbl}
\arrayrulecolor{black}

\fancypagestyle{plain}{
  \fancyhead{}
  \fancyfoot{}
  \renewcommand{\headrulewidth}{0pt} %%%%%%%%%%%%%%%%%%%%%%%%%%%%%%%%%%%%%%%%%%%%%%%%%%%%%%%%%%%%%%%%%%%%%%%%%%%%%%%%%%%%%
}

%\usepackage{MyAlgorithm}
%\usepackage[noend]{MyAlgorithmic}
%\usepackage[ED=EDSYS-SystEmb, Ets=INP]{tlsflyleaf}

%%% Clear Header %%%%%%%%%%%%%%%%%%%%%%%%%%%%%%%%%%%%%%%%%%%%%%%%%%%%%%%%%%%%%%%%%%
% Clear Header Style on the Last Empty Odd pages
\makeatletter

\def\cleardoublepage{\clearpage\if@twoside \ifodd\c@page\else%
  \hbox{}%
  \thispagestyle{empty}%              % Empty header styles
  \newpage%
  \if@twocolumn\hbox{}\newpage\fi\fi\fi}

\makeatother
 
%%%%%%%%%%%%%%%%%%%%%%%%%%%%%%%%%%%%%%%%%%%%%%%%%%%%%%%%%%%%%%%%%%%%%%%%%%%%%%% 
% Prints your review date and 'Draft Version' (From Josullvn, CS, CMU)
\newcommand{\reviewtimetoday}[2]{\special{!userdict begin
    /bop-hook{gsave 20 710 translate 45 rotate 0.8 setgray
      /Times-Roman findfont 12 scalefont setfont 0 0   moveto (#1) show
      0 -12 moveto (#2) show grestore}def end}}
% You can turn on or off this option.
% \reviewtimetoday{\today}{Draft Version}
%%%%%%%%%%%%%%%%%%%%%%%%%%%%%%%%%%%%%%%%%%%%%%%%%%%%%%%%%%%%%%%%%%%%%%%%%%%%%%% 

\newenvironment{maxime}[1]
{
	\def\Arg{#1}
\vspace*{0cm}
\hfill
\begin{minipage}{0.6\textwidth}%
%\rule[0.5ex]{\textwidth}{0.1mm}\\%
\hrulefill $\:$ \\%$\:$ {\bf #1}\\
%\vspace*{-0.25cm}
\it 
}%
{%
	
\hrulefill $\:$ {\bf \Arg}
\vspace*{0.5cm}%
\end{minipage}
}

\let\minitocORIG\minitoc
\renewcommand{\minitoc}{\minitocORIG \vspace{1.5em}}

%\usepackage{slashbox}

\newenvironment{bulletList}%
{ \begin{list}%
	{$\bullet$}%
	{\setlength{\labelwidth}{25pt}%
	 \setlength{\leftmargin}{30pt}%
	 \setlength{\itemsep}{\parsep}}}%
{ \end{list} }


%%%%%%% Outils pour \comment \alert \add %%%%%
\usepackage{easyReview}
\usepackage{soulutf8} % for accented letters

\let\newalert\alert
\renewcommand{\alert}[1]{\textit{\newalert{#1}}}

%\usepackage[commandnameprefix=ifneeded]{changes} %% \chhighlight and \chcomment to avoid collision with easyReview
\renewcommand{\epsilon}{\varepsilon}

% centered page environment

\newenvironment{vcenterpage}
{\newpage\vspace*{\fill}\thispagestyle{empty}\renewcommand{\headrulewidth}{0pt}}
{\vspace*{\fill}}

\usepackage{tablefootnote}

%%%%%% MISE EN FORME CADRES DEFINITIONS/THEOREMES/LEMES %%%%%%%%%%
\usepackage{amsthm}  % for theoremstyle

\theoremstyle{plain} 
\newtheorem{theorem}{Théorème}[section]
\newtheorem{corollary}{Corolaire}[theorem]

%\theoremstyle{lemma}
%\newtheorem{lemma}[theorem]{Lemme}


\theoremstyle{definition}
\newtheorem{definition}[theorem]{Définition}


\cmnt{
	\usepackage{ntheorem} %\usepackage{amsthm}  % for theoremstyle
	%\usepackage{mdframed}
	\usepackage[most]{tcolorbox}
	
	\theoremstyle{plain} 
	\theoremindent20pt
	\theoremheaderfont{\normalfont\bfseries\hspace{-\theoremindent}}
	\newtheorem{theorem}{Théorème}[section]
	\newtheorem{corollary}{Corolaire}[theorem]
	
	\theoremstyle{plain}
	\newtheorem{lemma}[theorem]{Lemme}
	
	
	\tcolorboxenvironment{theorem}{
		blanker,
		breakable,
		before skip=\topsep,
		after skip=\topsep,
		borderline west={1pt}{10pt}{double, shorten <=12pt}
	}
	
	\theorembodyfont{\normalfont}
	\theoremindent20pt
	\theoremheaderfont{\normalfont\bfseries\hspace{-\theoremindent}}
	\newtheorem{definition}[theorem]{Définition}
	
	
	\tcolorboxenvironment{definition}{
		blanker,
		breakable,
		before skip=\topsep,
		after skip=\topsep,
		borderline west={1pt}{10pt}{shorten <=12pt}
	}
}

\cmnt{ 
	\begin{theorem}
		Ceci est un Théorème.
	\end{theorem} 
	
	\begin{corollary}
		Ceci est un Corollaire.
	\end{corollary}
	
	\begin{definition}
		Ceci est une Définition.
	\end{definition}
	
	\begin{lemma}
		Ceci est un Lemme.
	\end{lemma}
}

\def\UrlBigBreaks{\do\/\do-\do:}
\usepackage{url}

\sloppy
\begin{document}
\fi


\chapter*{Introduction}
\addstarredchapter{Introduction} %Sinon cela n'apparait pas dans la table des matières

La complexité des systèmes cyberphysiques s’est accrue dramatiquement  ces dernières décennies. 

C'est ainsi que le domaine de l'automobile est successivement passé du tout mécanique a des architectures Électrique et Électronique (AEE) de plus en plus sophistiquées. Bien évidemment, cette tendance lourde s’appuyant sur les progrès des techniques numériques a permis de rendre aux clients des services plus avancés et pertinents qui ont gagné en intelligence. Cela s'est fait en s’appuyant tout particulièrement sur des aspects logiciels prépondérants en délaissant les anciens systèmes mécaniques ou électro-mécaniques.

Ces évolutions progressives dans la voiture ont mené à des ajouts de calculateurs ayant chacun son lot de fonctionnalités avancées, potentiellement accompagnées des capteurs (température de l'habitacle, présence sur les sièges...) mais aussi des actionneurs (système d'air conditionné, vitres, verrouillage centralisé...) nécessaires.
C'est de cette façon que l'architecture distribuée dans l'automobile s'est étoffée pour atteindre jusqu'à 70 calculateurs dans un même véhicule. À terme, cette approche ne semble plus soutenable au vu de la demande en fonctionnalités supplémentaires liées aux technologies émergentes : le véhicule autonome et connecté.

C’est pour cette raison que la tendance d'ajout de calculateurs à une architecture distribuée toujours plus complexe est en train de s’inverser. C'est substitué par l’émergence de calculateurs multicœurs puissants qui peuvent se substituer a nombre d’ECU élémentaires. L’architecture actuelle s’oriente donc vers des architectures fédérées mettant en jeu des processeurs sur lesquels la coexistence d’applications critiques et non-critiques (niveau d’ASIL). Ces systèmes à criticité multiple  induisent des problèmes de partage de ressources et de sûreté de fonctionnement.

\section*{Systèmes embarqués automobiles}
    \subsection*{Évolutions des systèmes embarqués}
        Système mécanique => système cyberphysique
    \subsection*{Architectures EE}
        => Augmentation complexité architecture EE
        => augmentation des besoins (puissance de calcul, ADAS, voiture connectée/autonome...)
\section*{Tendances et Contraintes actuelles}
    \subsection*{Tendances}
        => Nouvelles architectures EE fédérées, virtualisation + multi-coeurs
            Présentation des différents risques d'interférence multicoeur
        => Évolutivité (Adaptive AUTOSAR, car as a service)
    \subsection*{Contraintes et limitations}
        Difficulté de transition
        Complexité
        Coûts
\section*{Objectif(s), contribution et Problématique}
    Transition partie I - enjeux des fautes temporelles à cause des tendances

\ifdefined\included
\else
\bibliographystyle{StyleThese}
\bibliography{these}
\end{document}
\fi
\ifdefined\included
\else
\documentclass[french, a4paper, 11pt, twoside, pdftex]{StyleThese}
\usepackage{iflang}
\usepackage{bibentry}



%\usepackage[sectionbib]{chapterbib}          % Cross-reference package (Natural BiB)
%\usepackage{natbib}                  % Put References at the end of each chapter
%\usepackage{bibunits}
% Do not put 'sectionbib' option here.
% Sectionbib option in 'natbib' will do.


\usepackage{fancyhdr}                    % Fancy Header and Footer

\usepackage[utf8]{inputenc}
\usepackage[T1]{fontenc}
\usepackage[french]{babel} %
\usepackage{lmodern} \normalfont %to load T1lmr.fd 
\DeclareFontShape{T1}{cmr}{b}{sc} { <-> ssub * cmr/bx/sc }{}
%\hyphenation{gar}

\usepackage{amsmath,amssymb}             % AMS Math
\usepackage{nicefrac}
\usepackage{siunitx}					%% Unites Math SI

\usepackage{blindtext}

\usepackage{datetime}

\usepackage{lipsum} 

\usepackage[inline]{enumitem}

\usepackage{hhline}
%\usepackage[left=1.5in,right=1.3in,top=1.1in,bottom=1.1in]{geometry}
\usepackage[left=1.5in,right=1.3in,top=1.1in,bottom=1.1in,includefoot,includehead,headheight=13.6pt]{geometry}

%%\renewcommand{\baselinestretch}{1.05}

%%%%%%%% Multi-figures avec sub-captions
\usepackage{caption}
\usepackage{subcaption}

% Table of contents for each chapter

\usepackage[nottoc, notlof, notlot]{tocbibind}
\usepackage[nohints]{minitoc}
\setcounter{minitocdepth}{2}
\mtcindent=15pt
% Use \minitoc where to put a table of contents

\usepackage{aecompl}

%% Package cosmetic meilleur layout du texte en jouant sur le spacing par caractères
\usepackage[activate={true,nocompatibility},final,tracking=true,kerning=true,factor=1100,stretch=10,shrink=10]{microtype}
\usepackage[absolute,overlay]{textpos} 
\setlength{\TPHorizModule}{\paperwidth}\setlength{\TPVertModule}{\paperheight}
\sloppy

%%%%%%%%%%% JOLIS TABLEAUX
\usepackage{tabularx}		%\usepackage{tabular}
\usepackage{multirow}
\newcommand{\mc}{\multicolumn} 
\newcommand{\mr}[2]{\multirow{#1}{*}{#2}} 	\newcommand{\mrQ}{\multirow{-4}{*}}
\usepackage{booktabs}

\usepackage[usenames,dvipsnames]{xcolor} 

\makeatletter
\newcommand{\ccolor}[3][]{%
	\kern-\fboxsep
	\if\relax\detokenize{#1}\relax
	\expandafter\@firstoftwo
	\else
	\expandafter\@secondoftwo
	\fi
	{\colorbox{#2}}%
	{\colorbox[#1]{#2}}%
	{#3}\kern-\fboxsep
}
\makeatother

%%%%% Insertion graphiques format PGF
\usepackage{pgfplots}
\pgfplotsset{width=\linewidth, compat=1.16}%, compat=1.17}
\usepackage{adjustbox}          %%% PERMET DE LES RECADRER + FACILEMENT


%%%%%%%%%% Bullets de listes sans saut de ligne %%%%%%%%%%
\usepackage{xparse}

\ExplSyntaxOn%
\seq_new:N \l_local_enum_seq

\newcommand{\storethestuff}[1]{%
  \seq_set_from_clist:Nn \l_local_enum_seq {#1}%
}

\newcommand{\dotheenumstuff}{%
\int_zero:N \l_tmpa_int
\seq_map_inline:Nn \l_local_enum_seq {%
    \int_incr:N \l_tmpa_int% Increase the counter
    \item ##1
    % Check whether the list has reached the end -- if so, use '.' instead of ','
    %\int_compare:nNnTF 
    % { \l_tmpa_int } < {\seq_count:N \l_local_enum_seq} 
    % {,} {.}
  }
}
\ExplSyntaxOff

\NewDocumentCommand{\linebullets}{+m}{%
  \storethestuff{#1}%
  \begin{enumerate*}[label={\alph*)},font={\bfseries},itemjoin={{, }}]
    \dotheenumstuff%
  \end{enumerate*}
}

\newcommand{\cmnt}[1]{}  %%%%% AJOUT DE COMMENTAIRE MULTILIGNES


%%%%%%%%%% ECRITURE CARACTERES DANS UN CERCLE %%%%%%%%%%
%\def\circleTxt[#1]{\raisebox{.5pt}{\textcircled{\raisebox{-1pt}{#1}}}}
\newcommand{\ctxt}[1]{\raisebox{.5pt}{\textcircled{\raisebox{-1.2pt}{#1}}}}
% Glossary / list of abbreviations

\usepackage[intoc]{nomencl}
\IfLanguageName{english}{%
\renewcommand{\nomname}{Glossary}
}{ %
\renewcommand{\nomname}{Liste des Abréviations}
}

\makenomenclature

% My pdf code

\usepackage{ifpdf}

\ifpdf
  \usepackage[pdftex]{graphicx}
  \DeclareGraphicsExtensions{.pdf,PDF,.png,PNG,.jpg,JPG}
  \usepackage[pagebackref,hyperindex=true]{hyperref} %% use \autoref{} instead of Table~\ref{}.
  \usepackage{tikz}
  \usetikzlibrary{arrows,shapes,calc}
\else
  \usepackage{graphicx}
  \DeclareGraphicsExtensions{.ps,.eps}
  \usepackage[a4paper,dvipdfm,pagebackref,hyperindex=true]{hyperref}
\fi

\graphicspath{{.}{schemas/}{graphiques/}{tables/}}

%% nicer backref links. NOTE: The flag ThesisInEnglish is used to define the
% language in the back references. Read more about it in These.tex

\IfLanguageName{english}{
\renewcommand*{\backref}[1]{}
\renewcommand*{\backrefalt}[4]{%
\ifcase #1 %
(Not cited.)%
\or
(Cited in page~#2.)%
\else
(Cited in pages~#2.)%
\fi}
\renewcommand*{\backrefsep}{, }
\renewcommand*{\backreftwosep}{ and~}
\renewcommand*{\backreflastsep}{ and~}
}{
\renewcommand*{\backref}[1]{}
\renewcommand*{\backrefalt}[4]{%
\ifcase #1 %
(Non cité.)%
\or
(Cité en page~#2.)%
\else
(Cité en pages~#2.)%
\fi}
\renewcommand*{\backrefsep}{, }
\renewcommand*{\backreftwosep}{ et~}
\renewcommand*{\backreflastsep}{ et~}
}

% Links in pdf
\usepackage{color}
\definecolor{linkcol}{rgb}{0,0,0.4} 
\definecolor{citecol}{rgb}{0.5,0,0} 
\definecolor{linkcol}{rgb}{0,0,0} 
\definecolor{citecol}{rgb}{0,0,0}
% Change this to change the informations included in the pdf file

\hypersetup
{
bookmarksopen=true,
pdftitle="Prévention des fautes temporelles sur architectures multicœur pour les systèmes à criticité mixte",
pdfauthor="Daniel LOCHE", %auteur du document
pdfsubject="Thèse", %sujet du document
%pdftoolbar=false, %barre d'outils non visible
pdfmenubar=true, %barre de menu visible
pdfhighlight=/O, %effet d'un clic sur un lien hypertexte
colorlinks=true, %couleurs sur les liens hypertextes
pdfpagemode=UseNone, %aucun mode de page
%pdfpagelayout=DoublePage, %ouverture en simple page
pdffitwindow=true, %pages ouvertes entierement dans toute la fenetre
linkcolor=linkcol, %couleur des liens hypertextes internes
citecolor=citecol, %couleur des liens pour les citations
urlcolor=linkcol %couleur des liens pour les url
}

% definitions.
% -------------------

\setcounter{secnumdepth}{3}
\setcounter{tocdepth}{2}

% Some useful commands and shortcut for maths:  partial derivative and stuff

\newcommand{\pd}[2]{\frac{\partial #1}{\partial #2}}
\def\abs{\operatorname{abs}}
\def\argmax{\operatornamewithlimits{arg\,max}}
\def\argmin{\operatornamewithlimits{arg\,min}}
\def\diag{\operatorname{Diag}}
\newcommand{\eqRef}[1]{(\ref{#1})}
\newcommand{\nline}{\smallbreak\noindent}

\usepackage{rotating}                    % Sideways of figures & tables

% \usepackage{txfonts}                     % Public Times New Roman text & math font
  
%%% Fancy Header %%%%%%%%%%%%%%%%%%%%%%%%%%%%%%%%%%%%%%%%%%%%%%%%%%%%%%%%%%%%%%%%%%
% Fancy Header Style Options

\pagestyle{fancy}                       % Sets fancy header and footer
\fancyfoot{}                            % Delete current footer settings

%\renewcommand{\chaptermark}[1]{         % Lower Case Chapter marker style
%  \markboth{\chaptername\ \thechapter.\ #1}}{}} %

%\renewcommand{\sectionmark}[1]{         % Lower case Section marker style
%  \markright{\thesection.\ #1}}         %

\fancyhead[LE,RO]{\bfseries\thepage}    % Page number (boldface) in left on even
% pages and right on odd pages
\fancyhead[RE]{\bfseries\nouppercase{\leftmark}}      % Chapter in the right on even pages
\fancyhead[LO]{\bfseries\nouppercase{\rightmark}}     % Section in the left on odd pages

\let\headruleORIG\headrule
\renewcommand{\headrule}{\color{black} \headruleORIG}
\renewcommand{\headrulewidth}{1.0pt}
\usepackage{colortbl}
\arrayrulecolor{black}

\fancypagestyle{plain}{
  \fancyhead{}
  \fancyfoot{}
  \renewcommand{\headrulewidth}{0pt} %%%%%%%%%%%%%%%%%%%%%%%%%%%%%%%%%%%%%%%%%%%%%%%%%%%%%%%%%%%%%%%%%%%%%%%%%%%%%%%%%%%%%
}

%\usepackage{MyAlgorithm}
%\usepackage[noend]{MyAlgorithmic}
%\usepackage[ED=EDSYS-SystEmb, Ets=INP]{tlsflyleaf}

%%% Clear Header %%%%%%%%%%%%%%%%%%%%%%%%%%%%%%%%%%%%%%%%%%%%%%%%%%%%%%%%%%%%%%%%%%
% Clear Header Style on the Last Empty Odd pages
\makeatletter

\def\cleardoublepage{\clearpage\if@twoside \ifodd\c@page\else%
  \hbox{}%
  \thispagestyle{empty}%              % Empty header styles
  \newpage%
  \if@twocolumn\hbox{}\newpage\fi\fi\fi}

\makeatother
 
%%%%%%%%%%%%%%%%%%%%%%%%%%%%%%%%%%%%%%%%%%%%%%%%%%%%%%%%%%%%%%%%%%%%%%%%%%%%%%% 
% Prints your review date and 'Draft Version' (From Josullvn, CS, CMU)
\newcommand{\reviewtimetoday}[2]{\special{!userdict begin
    /bop-hook{gsave 20 710 translate 45 rotate 0.8 setgray
      /Times-Roman findfont 12 scalefont setfont 0 0   moveto (#1) show
      0 -12 moveto (#2) show grestore}def end}}
% You can turn on or off this option.
% \reviewtimetoday{\today}{Draft Version}
%%%%%%%%%%%%%%%%%%%%%%%%%%%%%%%%%%%%%%%%%%%%%%%%%%%%%%%%%%%%%%%%%%%%%%%%%%%%%%% 

\newenvironment{maxime}[1]
{
	\def\Arg{#1}
\vspace*{0cm}
\hfill
\begin{minipage}{0.6\textwidth}%
%\rule[0.5ex]{\textwidth}{0.1mm}\\%
\hrulefill $\:$ \\%$\:$ {\bf #1}\\
%\vspace*{-0.25cm}
\it 
}%
{%
	
\hrulefill $\:$ {\bf \Arg}
\vspace*{0.5cm}%
\end{minipage}
}

\let\minitocORIG\minitoc
\renewcommand{\minitoc}{\minitocORIG \vspace{1.5em}}

%\usepackage{slashbox}

\newenvironment{bulletList}%
{ \begin{list}%
	{$\bullet$}%
	{\setlength{\labelwidth}{25pt}%
	 \setlength{\leftmargin}{30pt}%
	 \setlength{\itemsep}{\parsep}}}%
{ \end{list} }


%%%%%%% Outils pour \comment \alert \add %%%%%
\usepackage{easyReview}
\usepackage{soulutf8} % for accented letters

\let\newalert\alert
\renewcommand{\alert}[1]{\textit{\newalert{#1}}}

%\usepackage[commandnameprefix=ifneeded]{changes} %% \chhighlight and \chcomment to avoid collision with easyReview
\renewcommand{\epsilon}{\varepsilon}

% centered page environment

\newenvironment{vcenterpage}
{\newpage\vspace*{\fill}\thispagestyle{empty}\renewcommand{\headrulewidth}{0pt}}
{\vspace*{\fill}}

\usepackage{tablefootnote}

%%%%%% MISE EN FORME CADRES DEFINITIONS/THEOREMES/LEMES %%%%%%%%%%
\usepackage{amsthm}  % for theoremstyle

\theoremstyle{plain} 
\newtheorem{theorem}{Théorème}[section]
\newtheorem{corollary}{Corolaire}[theorem]

%\theoremstyle{lemma}
%\newtheorem{lemma}[theorem]{Lemme}


\theoremstyle{definition}
\newtheorem{definition}[theorem]{Définition}


\cmnt{
	\usepackage{ntheorem} %\usepackage{amsthm}  % for theoremstyle
	%\usepackage{mdframed}
	\usepackage[most]{tcolorbox}
	
	\theoremstyle{plain} 
	\theoremindent20pt
	\theoremheaderfont{\normalfont\bfseries\hspace{-\theoremindent}}
	\newtheorem{theorem}{Théorème}[section]
	\newtheorem{corollary}{Corolaire}[theorem]
	
	\theoremstyle{plain}
	\newtheorem{lemma}[theorem]{Lemme}
	
	
	\tcolorboxenvironment{theorem}{
		blanker,
		breakable,
		before skip=\topsep,
		after skip=\topsep,
		borderline west={1pt}{10pt}{double, shorten <=12pt}
	}
	
	\theorembodyfont{\normalfont}
	\theoremindent20pt
	\theoremheaderfont{\normalfont\bfseries\hspace{-\theoremindent}}
	\newtheorem{definition}[theorem]{Définition}
	
	
	\tcolorboxenvironment{definition}{
		blanker,
		breakable,
		before skip=\topsep,
		after skip=\topsep,
		borderline west={1pt}{10pt}{shorten <=12pt}
	}
}

\cmnt{ 
	\begin{theorem}
		Ceci est un Théorème.
	\end{theorem} 
	
	\begin{corollary}
		Ceci est un Corollaire.
	\end{corollary}
	
	\begin{definition}
		Ceci est une Définition.
	\end{definition}
	
	\begin{lemma}
		Ceci est un Lemme.
	\end{lemma}
}

\def\UrlBigBreaks{\do\/\do-\do:}
\usepackage{url}

\sloppy
\begin{document}
\setcounter{chapter}{1}
\dominitoc
\faketableofcontents
\fi

\chapter{Enjeux des systèmes à criticité multiple sur processeurs multi-coeurs}
\minitoc

L'évolution des système cyber-physiques a progressivement déplacé la complexité des systèmes vers les aspects logiciels, au sein d'une architecture Électrique et Électronique (AEE) de plus en plus complexe. C'est ainsi que l'automobile est successivement passées du tout mécanique vers l'électrification 

\section{Présentation des architectures Hardware}
    \subsection{Mono/Multi/Many Cores et GPU}
    \subsection{Architectures mémoires, cas des multi-coeurs}
\section{Risques d'interférences}
\section{criticité mixte \& contraintes temporelles}
    \subsection{Enjeux des usages des multicoeurs avec contraintes temporelles}
    \subsection{Problématique :  criticité multiple : comment optimiser l'usage des ressources avec garanties temporelles}
transition - présentation contenu Etat de l'Art


\ifdefined\included
\else
\bibliographystyle{StyleThese}
\bibliography{these}
\end{document}
\fi

\ifdefined\included
\else
\documentclass[french, a4paper, 11pt, twoside, pdftex]{StyleThese}
\usepackage{iflang}
\usepackage{bibentry}



%\usepackage[sectionbib]{chapterbib}          % Cross-reference package (Natural BiB)
%\usepackage{natbib}                  % Put References at the end of each chapter
%\usepackage{bibunits}
% Do not put 'sectionbib' option here.
% Sectionbib option in 'natbib' will do.


\usepackage{fancyhdr}                    % Fancy Header and Footer

\usepackage[utf8]{inputenc}
\usepackage[T1]{fontenc}
\usepackage[french]{babel} %
\usepackage{lmodern} \normalfont %to load T1lmr.fd 
\DeclareFontShape{T1}{cmr}{b}{sc} { <-> ssub * cmr/bx/sc }{}
%\hyphenation{gar}

\usepackage{amsmath,amssymb}             % AMS Math
\usepackage{nicefrac}
\usepackage{siunitx}					%% Unites Math SI

\usepackage{blindtext}

\usepackage{datetime}

\usepackage{lipsum} 

\usepackage[inline]{enumitem}

\usepackage{hhline}
%\usepackage[left=1.5in,right=1.3in,top=1.1in,bottom=1.1in]{geometry}
\usepackage[left=1.5in,right=1.3in,top=1.1in,bottom=1.1in,includefoot,includehead,headheight=13.6pt]{geometry}

%%\renewcommand{\baselinestretch}{1.05}

%%%%%%%% Multi-figures avec sub-captions
\usepackage{caption}
\usepackage{subcaption}

% Table of contents for each chapter

\usepackage[nottoc, notlof, notlot]{tocbibind}
\usepackage[nohints]{minitoc}
\setcounter{minitocdepth}{2}
\mtcindent=15pt
% Use \minitoc where to put a table of contents

\usepackage{aecompl}

%% Package cosmetic meilleur layout du texte en jouant sur le spacing par caractères
\usepackage[activate={true,nocompatibility},final,tracking=true,kerning=true,factor=1100,stretch=10,shrink=10]{microtype}
\usepackage[absolute,overlay]{textpos} 
\setlength{\TPHorizModule}{\paperwidth}\setlength{\TPVertModule}{\paperheight}
\sloppy

%%%%%%%%%%% JOLIS TABLEAUX
\usepackage{tabularx}		%\usepackage{tabular}
\usepackage{multirow}
\newcommand{\mc}{\multicolumn} 
\newcommand{\mr}[2]{\multirow{#1}{*}{#2}} 	\newcommand{\mrQ}{\multirow{-4}{*}}
\usepackage{booktabs}

\usepackage[usenames,dvipsnames]{xcolor} 

\makeatletter
\newcommand{\ccolor}[3][]{%
	\kern-\fboxsep
	\if\relax\detokenize{#1}\relax
	\expandafter\@firstoftwo
	\else
	\expandafter\@secondoftwo
	\fi
	{\colorbox{#2}}%
	{\colorbox[#1]{#2}}%
	{#3}\kern-\fboxsep
}
\makeatother

%%%%% Insertion graphiques format PGF
\usepackage{pgfplots}
\pgfplotsset{width=\linewidth, compat=1.16}%, compat=1.17}
\usepackage{adjustbox}          %%% PERMET DE LES RECADRER + FACILEMENT


%%%%%%%%%% Bullets de listes sans saut de ligne %%%%%%%%%%
\usepackage{xparse}

\ExplSyntaxOn%
\seq_new:N \l_local_enum_seq

\newcommand{\storethestuff}[1]{%
  \seq_set_from_clist:Nn \l_local_enum_seq {#1}%
}

\newcommand{\dotheenumstuff}{%
\int_zero:N \l_tmpa_int
\seq_map_inline:Nn \l_local_enum_seq {%
    \int_incr:N \l_tmpa_int% Increase the counter
    \item ##1
    % Check whether the list has reached the end -- if so, use '.' instead of ','
    %\int_compare:nNnTF 
    % { \l_tmpa_int } < {\seq_count:N \l_local_enum_seq} 
    % {,} {.}
  }
}
\ExplSyntaxOff

\NewDocumentCommand{\linebullets}{+m}{%
  \storethestuff{#1}%
  \begin{enumerate*}[label={\alph*)},font={\bfseries},itemjoin={{, }}]
    \dotheenumstuff%
  \end{enumerate*}
}

\newcommand{\cmnt}[1]{}  %%%%% AJOUT DE COMMENTAIRE MULTILIGNES


%%%%%%%%%% ECRITURE CARACTERES DANS UN CERCLE %%%%%%%%%%
%\def\circleTxt[#1]{\raisebox{.5pt}{\textcircled{\raisebox{-1pt}{#1}}}}
\newcommand{\ctxt}[1]{\raisebox{.5pt}{\textcircled{\raisebox{-1.2pt}{#1}}}}
% Glossary / list of abbreviations

\usepackage[intoc]{nomencl}
\IfLanguageName{english}{%
\renewcommand{\nomname}{Glossary}
}{ %
\renewcommand{\nomname}{Liste des Abréviations}
}

\makenomenclature

% My pdf code

\usepackage{ifpdf}

\ifpdf
  \usepackage[pdftex]{graphicx}
  \DeclareGraphicsExtensions{.pdf,PDF,.png,PNG,.jpg,JPG}
  \usepackage[pagebackref,hyperindex=true]{hyperref} %% use \autoref{} instead of Table~\ref{}.
  \usepackage{tikz}
  \usetikzlibrary{arrows,shapes,calc}
\else
  \usepackage{graphicx}
  \DeclareGraphicsExtensions{.ps,.eps}
  \usepackage[a4paper,dvipdfm,pagebackref,hyperindex=true]{hyperref}
\fi

\graphicspath{{.}{schemas/}{graphiques/}{tables/}}

%% nicer backref links. NOTE: The flag ThesisInEnglish is used to define the
% language in the back references. Read more about it in These.tex

\IfLanguageName{english}{
\renewcommand*{\backref}[1]{}
\renewcommand*{\backrefalt}[4]{%
\ifcase #1 %
(Not cited.)%
\or
(Cited in page~#2.)%
\else
(Cited in pages~#2.)%
\fi}
\renewcommand*{\backrefsep}{, }
\renewcommand*{\backreftwosep}{ and~}
\renewcommand*{\backreflastsep}{ and~}
}{
\renewcommand*{\backref}[1]{}
\renewcommand*{\backrefalt}[4]{%
\ifcase #1 %
(Non cité.)%
\or
(Cité en page~#2.)%
\else
(Cité en pages~#2.)%
\fi}
\renewcommand*{\backrefsep}{, }
\renewcommand*{\backreftwosep}{ et~}
\renewcommand*{\backreflastsep}{ et~}
}

% Links in pdf
\usepackage{color}
\definecolor{linkcol}{rgb}{0,0,0.4} 
\definecolor{citecol}{rgb}{0.5,0,0} 
\definecolor{linkcol}{rgb}{0,0,0} 
\definecolor{citecol}{rgb}{0,0,0}
% Change this to change the informations included in the pdf file

\hypersetup
{
bookmarksopen=true,
pdftitle="Prévention des fautes temporelles sur architectures multicœur pour les systèmes à criticité mixte",
pdfauthor="Daniel LOCHE", %auteur du document
pdfsubject="Thèse", %sujet du document
%pdftoolbar=false, %barre d'outils non visible
pdfmenubar=true, %barre de menu visible
pdfhighlight=/O, %effet d'un clic sur un lien hypertexte
colorlinks=true, %couleurs sur les liens hypertextes
pdfpagemode=UseNone, %aucun mode de page
%pdfpagelayout=DoublePage, %ouverture en simple page
pdffitwindow=true, %pages ouvertes entierement dans toute la fenetre
linkcolor=linkcol, %couleur des liens hypertextes internes
citecolor=citecol, %couleur des liens pour les citations
urlcolor=linkcol %couleur des liens pour les url
}

% definitions.
% -------------------

\setcounter{secnumdepth}{3}
\setcounter{tocdepth}{2}

% Some useful commands and shortcut for maths:  partial derivative and stuff

\newcommand{\pd}[2]{\frac{\partial #1}{\partial #2}}
\def\abs{\operatorname{abs}}
\def\argmax{\operatornamewithlimits{arg\,max}}
\def\argmin{\operatornamewithlimits{arg\,min}}
\def\diag{\operatorname{Diag}}
\newcommand{\eqRef}[1]{(\ref{#1})}
\newcommand{\nline}{\smallbreak\noindent}

\usepackage{rotating}                    % Sideways of figures & tables

% \usepackage{txfonts}                     % Public Times New Roman text & math font
  
%%% Fancy Header %%%%%%%%%%%%%%%%%%%%%%%%%%%%%%%%%%%%%%%%%%%%%%%%%%%%%%%%%%%%%%%%%%
% Fancy Header Style Options

\pagestyle{fancy}                       % Sets fancy header and footer
\fancyfoot{}                            % Delete current footer settings

%\renewcommand{\chaptermark}[1]{         % Lower Case Chapter marker style
%  \markboth{\chaptername\ \thechapter.\ #1}}{}} %

%\renewcommand{\sectionmark}[1]{         % Lower case Section marker style
%  \markright{\thesection.\ #1}}         %

\fancyhead[LE,RO]{\bfseries\thepage}    % Page number (boldface) in left on even
% pages and right on odd pages
\fancyhead[RE]{\bfseries\nouppercase{\leftmark}}      % Chapter in the right on even pages
\fancyhead[LO]{\bfseries\nouppercase{\rightmark}}     % Section in the left on odd pages

\let\headruleORIG\headrule
\renewcommand{\headrule}{\color{black} \headruleORIG}
\renewcommand{\headrulewidth}{1.0pt}
\usepackage{colortbl}
\arrayrulecolor{black}

\fancypagestyle{plain}{
  \fancyhead{}
  \fancyfoot{}
  \renewcommand{\headrulewidth}{0pt} %%%%%%%%%%%%%%%%%%%%%%%%%%%%%%%%%%%%%%%%%%%%%%%%%%%%%%%%%%%%%%%%%%%%%%%%%%%%%%%%%%%%%
}

%\usepackage{MyAlgorithm}
%\usepackage[noend]{MyAlgorithmic}
%\usepackage[ED=EDSYS-SystEmb, Ets=INP]{tlsflyleaf}

%%% Clear Header %%%%%%%%%%%%%%%%%%%%%%%%%%%%%%%%%%%%%%%%%%%%%%%%%%%%%%%%%%%%%%%%%%
% Clear Header Style on the Last Empty Odd pages
\makeatletter

\def\cleardoublepage{\clearpage\if@twoside \ifodd\c@page\else%
  \hbox{}%
  \thispagestyle{empty}%              % Empty header styles
  \newpage%
  \if@twocolumn\hbox{}\newpage\fi\fi\fi}

\makeatother
 
%%%%%%%%%%%%%%%%%%%%%%%%%%%%%%%%%%%%%%%%%%%%%%%%%%%%%%%%%%%%%%%%%%%%%%%%%%%%%%% 
% Prints your review date and 'Draft Version' (From Josullvn, CS, CMU)
\newcommand{\reviewtimetoday}[2]{\special{!userdict begin
    /bop-hook{gsave 20 710 translate 45 rotate 0.8 setgray
      /Times-Roman findfont 12 scalefont setfont 0 0   moveto (#1) show
      0 -12 moveto (#2) show grestore}def end}}
% You can turn on or off this option.
% \reviewtimetoday{\today}{Draft Version}
%%%%%%%%%%%%%%%%%%%%%%%%%%%%%%%%%%%%%%%%%%%%%%%%%%%%%%%%%%%%%%%%%%%%%%%%%%%%%%% 

\newenvironment{maxime}[1]
{
	\def\Arg{#1}
\vspace*{0cm}
\hfill
\begin{minipage}{0.6\textwidth}%
%\rule[0.5ex]{\textwidth}{0.1mm}\\%
\hrulefill $\:$ \\%$\:$ {\bf #1}\\
%\vspace*{-0.25cm}
\it 
}%
{%
	
\hrulefill $\:$ {\bf \Arg}
\vspace*{0.5cm}%
\end{minipage}
}

\let\minitocORIG\minitoc
\renewcommand{\minitoc}{\minitocORIG \vspace{1.5em}}

%\usepackage{slashbox}

\newenvironment{bulletList}%
{ \begin{list}%
	{$\bullet$}%
	{\setlength{\labelwidth}{25pt}%
	 \setlength{\leftmargin}{30pt}%
	 \setlength{\itemsep}{\parsep}}}%
{ \end{list} }


%%%%%%% Outils pour \comment \alert \add %%%%%
\usepackage{easyReview}
\usepackage{soulutf8} % for accented letters

\let\newalert\alert
\renewcommand{\alert}[1]{\textit{\newalert{#1}}}

%\usepackage[commandnameprefix=ifneeded]{changes} %% \chhighlight and \chcomment to avoid collision with easyReview
\renewcommand{\epsilon}{\varepsilon}

% centered page environment

\newenvironment{vcenterpage}
{\newpage\vspace*{\fill}\thispagestyle{empty}\renewcommand{\headrulewidth}{0pt}}
{\vspace*{\fill}}

\usepackage{tablefootnote}

%%%%%% MISE EN FORME CADRES DEFINITIONS/THEOREMES/LEMES %%%%%%%%%%
\usepackage{amsthm}  % for theoremstyle

\theoremstyle{plain} 
\newtheorem{theorem}{Théorème}[section]
\newtheorem{corollary}{Corolaire}[theorem]

%\theoremstyle{lemma}
%\newtheorem{lemma}[theorem]{Lemme}


\theoremstyle{definition}
\newtheorem{definition}[theorem]{Définition}


\cmnt{
	\usepackage{ntheorem} %\usepackage{amsthm}  % for theoremstyle
	%\usepackage{mdframed}
	\usepackage[most]{tcolorbox}
	
	\theoremstyle{plain} 
	\theoremindent20pt
	\theoremheaderfont{\normalfont\bfseries\hspace{-\theoremindent}}
	\newtheorem{theorem}{Théorème}[section]
	\newtheorem{corollary}{Corolaire}[theorem]
	
	\theoremstyle{plain}
	\newtheorem{lemma}[theorem]{Lemme}
	
	
	\tcolorboxenvironment{theorem}{
		blanker,
		breakable,
		before skip=\topsep,
		after skip=\topsep,
		borderline west={1pt}{10pt}{double, shorten <=12pt}
	}
	
	\theorembodyfont{\normalfont}
	\theoremindent20pt
	\theoremheaderfont{\normalfont\bfseries\hspace{-\theoremindent}}
	\newtheorem{definition}[theorem]{Définition}
	
	
	\tcolorboxenvironment{definition}{
		blanker,
		breakable,
		before skip=\topsep,
		after skip=\topsep,
		borderline west={1pt}{10pt}{shorten <=12pt}
	}
}

\cmnt{ 
	\begin{theorem}
		Ceci est un Théorème.
	\end{theorem} 
	
	\begin{corollary}
		Ceci est un Corollaire.
	\end{corollary}
	
	\begin{definition}
		Ceci est une Définition.
	\end{definition}
	
	\begin{lemma}
		Ceci est un Lemme.
	\end{lemma}
}

\def\UrlBigBreaks{\do\/\do-\do:}
\usepackage{url}

\sloppy
\begin{document}
\setcounter{chapter}{2} %% Numéro du chapitre précédent ;)
\dominitoc
\faketableofcontents
\fi

\chapter{État de l'Art} \label{chap:2_StateofArt}
\minitoc

Dans le \hyperref[chap:1_EnjeuxIntro]{chapitre précédent}, nous avons pu identifier en détails les problématiques émergentes provoquées par l'utilisation de calculateurs multicœurs dans une optique d'agrégation de logiciels à criticités mixtes. Cet ensemble de logiciels aux usages divers, et donc aux exigences de sûreté de fonctionnement variées mène à de nombreux choix et compromis qui recouvrent beaucoup d'éléments spécifiques du système. Cela débute dès le choix de l'architecture matérielle dont les processeurs à utiliser et va jusqu'au choix de l'ordonnancement des tâches et la gestion des périphériques. Tous ces éléments vont jouer sur le bon fonctionnement de l'ensemble, et notamment sur la bonne exécution du logiciel pour réaliser ses fonctions.

Dans ce contexte, plusieurs enjeux se heurtent les uns aux autres. Nous avons d'une part le besoin grandissant à l'origine de cet usage des calculateurs multicœurs~: exploiter au maximum la puissance de calcul pour réduire le nombre de calculateurs et donc les coûts. Mais d'autre part, la cohabitation de logiciels à criticités multiples requiert des garanties de sûreté de fonctionnement dont des garanties de non-interférences entre les logiciels pour s'assurer notamment de l'absence de défaillances temporelles par dépassement d'échéances.

De par la multiplicité des choix et donc des solutions envisageables au développement et l'implémentation d'une part~; de par la multiplicité des objectifs à respecter dans la réalisation des systèmes d'autre part~; il existe une très forte diversité de combinaisons possibles pour mettre en place un système à criticité mixte sur une architecture multicœur.

Nous présentons dans ce chapitre une vue d'ensemble des différents études académiques propres aux problématiques des systèmes à criticité mixte en les classifiant par grands axes méthodologiques. Cela nous permettra d'avoir une vue d'ensemble de ce sujet de recherche qui est si vaste. À partir de cela, nous focalisons le point sur lequel se concentre nos travaux et comment nous nous positionnons dans ce champ des possibles.

\pagebreak
\section{Vue d'ensemble des systèmes à criticité mixte}
    \subsection{Entre garanties et optimisation}
    La maîtrise des processeurs multicœurs est un domaine de recherche très vaste qui se subdivise assez rapidement en deux objectifs principaux comme nous avons déjà pu le présenter. D'un côté, nous avons la recherche du maximum de performance à partir des ressources de calcul à disposition et notamment en profitant du parallélisme. De l'autre côté, nous avons l'analyse de la dimension temporelle pour la garantie de respect des exigences temporelles. Il s'avère que ces deux objectifs peuvent être antagonistes.
    
    Historiquement, cette division s'est naturellement répartie selon le domaine d'utilisation de l'électronique embarquée et des contraintes associées. L'optimisation des ressources étant en premier lieu associée avec des usages grand public tel que sur nos ordinateurs et autres smartphones, que l'on peut se permettre de redémarrer en cas de problèmes, et qui se doivent de s'adapter du mieux possible à la charge variable qu'on leur impose. Plus généralement, une très grande part des recherches pour le développement des systèmes d'exploitation comme Linux ou Android sont orientées vers un meilleur usage du hardware. Cela passe en premier lieu par l'ordonnancement avec le \textit{completely fair scheduling}~\cite{pabla_completely_2009}, \cite{pricopi_task_2014} par exemple, mais aussi par l'allocation des tâches et l'équilibrage de charge des cœurs~\cite{pathania_distributed_2016}. Ce besoin devient de plus en plus fort avec l'utilisation de services cloud décentralisés~\cite{walsh_utility_2004}. Il s'agit là d'un problème ancien et en constante évolution face à l'incessante croissance en puissance et en complexité des processeurs~\cite{lozi_linux_2016}. De plus, à cette problématique s'ajoutent des sous-enjeux qui deviennent de plus en plus importants tels que la gestion de la chauffe par répartition matérielle de la charge ou encore la gestion de la consommation énergétique~\cite{li_optimizing_2016}. De par ses caractéristiques de systèmes hautement évolutifs, dynamiques, au contact constant avec l'utilisateur et donc en constante évolution, j'aime à qualifier ces systèmes de \textit{systèmes flexibles}. Il s'agit d'avoir la plus grande adaptabilité possible pour satisfaire l'utilisateur, en garantissant au mieux une expérience utilisateur et une disponibilité satisfaisante, mais avec des exigences de fiabilité qui sont rarement le critère central de décision dans le processus de développement.
        
    De l'autre côté du spectre, nous avons les cas d'application industriels qui ont des besoins de fiabilité et de sûreté de fonctionnement à l'utilisation bien plus exigeants. C'est dans cette branche-là que se focalisent sans nul doute la plus grande part des recherches de par les enjeux financiers et technologiques qui sont impliqués. C'est d'autant plus vrai que les enjeux industriels rencontrent de plus en plus les besoins qui étaient jusqu'alors propres aux systèmes flexibles. Les besoins en performance s'accentuent, ainsi que les besoins en évolutivité de par la cohabitation grandissante entre les systèmes en contact avec l'utilisateur et les systèmes enfouis. L'automobile, l'avion ainsi que tous les systèmes embarqués qui s'interfacent de plus en plus avec des smartphones en sont des exemples flagrants. À l'aune de ces faits, ces systèmes ont pour caractéristique émergente d'héberger du logiciel à criticité mixte, avec d'une part des composants originairement rencontrés dans les systèmes flexibles et d'autre part les composants plus stables et sûrs de fonctionnement rencontrés dans les systèmes embarqués industriels. Dans ce contexte, des revues ont d'ores-et-déjà été réalisées sur le champ des propositions du monde de la recherche pour gérer des systèmes à criticité mixte. Il nous faut citer l'incontournable Review de Burns et Davis~\cite{burns_mixed_2022} qui présente de façon synthétique les différentes solutions à cette problématique. 
    
    \pagebreak
    
    L'objectif de cette étape est de nous situer dans cet univers tentaculaire des systèmes à criticité mixte. Nous proposons donc une grille de lecture par branches macroscopiques de recherche telle que décrite dans la~\autoref{fig:stateoftheartsituation}, dont nous venons de présenter le premier étage qui distingue deux objectifs fondamentaux qui sont les garanties temporelles et l'optimisation des performances. Ces deux principes à première vue opposés se doivent pourtant de plus en plus d'être conciliés comme on a pu le présenter au cours de notre constat \hyperref[chap:1_EnjeuxIntro]{au chapitre d'introduction}. 
    %%, nous mettons de côté les recherches orientées vers les calculateurs monocœurs pour nous focaliser sur l'usage émergent du matériel multicœur. 
    À cette fin, nous décidons d'orienter notre réflexion en premier lieu sur ce qu'il est possible de garantir sur les exigences temporelles. Il s'agit d'aborder la problématique sous un angle qui permettrait un large panoramique d'opportunités d'améliorations que ce soit pour y adjoindre des mécanismes de maximisation d'usage du processeur ou à l'inverse des outils complémentaires pour la maîtrise d'exécution des logiciels critiques. De cette façon, on serait en mesure de proposer un cadre qui apporte un certain nombre de garanties minimales, auquel il est possible d'adjoindre d'autres solutions complémentaires, notamment parmi celles que l'on présente ici. 
    
    \begin{figure}[ht]
    	\centering
    	\includegraphics[width=\linewidth]{schemas/StateOfTheArt_plan}
    	\captionsetup{justification=centering}
    	\caption[Classification État de l'Art Systèmes à criticité mixte sur multicœur]
    			{Vue d'ensemble des solutions pour systèmes à criticité mixte sur calculateur multicœur}
    	\label{fig:stateoftheartsituation}
    \end{figure}

    \subsection{Solutions Matérielles et Logicielles}
    
    \subsubsection{Solutions Matérielles}
    Parmi les axes existants pour maîtriser l'exécution de systèmes à criticité mixte, le premier choix repose dans la mise en place de mécanisme d'inhibition du problème initial. Il s'agit de gérer le problème à la source en prévenant purement et simplement toute possibilité d'interférence logicielle. Il s'agit probablement encore aujourd'hui de l'option la plus répandue dans l'industrie pour sa simplicité. Cela est dû à la simplicité du résultat obtenu notamment quand il s'agit de certifier le logiciel. En étant capable de prouver que 2 logiciels de niveau de criticité différents n'entreront jamais en concurrence pour l'utilisation du matériel, les critères de sûreté de fonctionnement pour la certification sont directement atteints.
    
    Il est possible de garantir l'absence totale d'interférences soit de façon logicielle par des mécanismes de contrôle de l'exécution, soit de façon matérielle, par construction. Ainsi, le choix fondamental de l'architecture matérielle peut directement répondre au problème. De fait si l'on est en capacité d'exécuter directement le logiciel sur un support qui, par construction, ne présente aucun risque d'interférences lié au partage de ressources, alors le problème est directement évité. C'est ce que tentent de proposer des travaux comme \cite{schoeberl_towards_2011}. L'autre possibilité repose tout simplement sur ce qui est réalisé aujourd'hui dans l'automobile par exemple, en répartissant le logiciel dans plusieurs processeurs, en utilisant donc des méthodes de communication et synchronisation inter-processeur dans une architecture distribuée.
    
    C'est aussi l'approche abordée avec les calculateurs scratchpad que nous avons présenté précédemment (c.f. \hyperref[Intro:multicoeurs]{chapitre d'Introduction}), basé sur une mémoire dédiée à chaque cœur et des séparations fortes pour prévenir toute contention. L'inconvénient fondamental de ces solutions réside dans leur immuabilité. Les perspectives d'évolution sont restreintes par le matériel, à moins d'être complètement remplacé. Par ailleurs, la conception et mise sur le marché de ce type de matériel spécialisé dépend du bon vouloir des fondeurs et pose en conséquence aussi des problèmes de coûts. Il existe par ailleurs des solutions matérielles plus fines comme avec l'ajout de modules hardware dédiés qui sont directement connectés au processeur pour accomplir des tâches spécifiques \cite{solet_hw-based_2018}. 
    
    Une analyse très complète du sous-domaine des solutions matérielles peut être consultée dans la thèse de A. Blin~\cite{blin_vers_2017}.
    
    
    \subsubsection{Solutions Logicielles}
    Inversement, il est possible de réponse aux problèmes d'interférences par des solutions logicielles. C'est dans ce cadre-ci qu'une très grande part des recherches sont effectuées. De fait un grand nombre de paramètres logiciels permettent d'influer sur les performances d'exécution de tâches concourantes. On pourra citer notamment les stratégies pour~: 
    \begin{itemize}
    	\item	l'ordonnancement des tâches,
    	\item	l'allocation des tâches sur les cœurs,
    	\item 	l'accès aux ressources partagées (caches, mémoire, bus d'accès...).
    \end{itemize}

	Tous ces éléments ayant des interactions fortes entre eux, les solutions logicielles proposent à la fois une très grande variété de choix et de possibilités d'aborder la question, mais cela pose par la même la question de la complexité de réalisation qui dépendra du niveau de focalisation sur un point précis ou à l'inverse de la manipulation de tous les aspects du système. Les solutions les plus globales se retrouvant notamment dans les solutions de virtualisation \cite{augier_real-time_2006} et plus généralement des systèmes d'exploitation dédiés temps-réel (RTOS -- Real-Time OS) comme PikeOS~\cite{kaiser_evolution_2007} ou encore OSEK/VDX~\cite{bechennec_trampoline_2006} qui est un standard ayant donné lieu à plusieurs implémentations. Ces deux exemples ont donné lieu respectivement à la norme ARINC 653 dans l'avionique, et AUTOSAR Classic dans l'automobile. Le cadre offert par ces propositions ne répond a priori pas directement aux problématiques émergentes d'interférences entre les logiciels. En revanche ils proposent tous les outils d'implémentation nécessaires. Ce sont donc des technologies probablement nécessaires, mais pas suffisantes, à l'implémentation et certification de systèmes temps-réels à criticité mixte qui requièrent la mise en place de mécanismes plus spécifiques. On pourra mentionner le cas de PikeOS qui a proposé récemment la mise en place d'une structure de la plateforme opérationnelle avec allocation de ressources temporelles (plages de temps fixes) et spatiale (plages d'utilisation de chaque ressource partagée) pour l'exécution des tâches critiques et non critiques de façon à prévenir toute exécution simultanée qui pourrait engendrer des interférences entre ces deux catégories de logiciels~\cite{sysgo_ag_arinc_2019}. Cela a permis en 2013 la certification au plus haut niveau pour le ferroviaire (SIL4 de la norme EN50128)  du framework temps-réel ainsi développé pour un processeur dual-cœur. Un autre exemple de solution industrielle temps-réel qui propose du partitionnement spatial et temporel est proposé par Krono-Safe, il s'agit d'Asterios~\cite{krono-asterios-2017}. 
		
	C'est dans ce cadre que bon nombre d'études analytiques du problème sont développées. En effet, un des ressorts principaux de la maîtrise de l'exécution du logiciel réside dans le maintien d'un mode de fonctionnement nominal associé à des estimations de pire temps d'exécution (\textit{WCET -- Worst Case Execution Time}). Il s'agit des fondamentaux d'analyse des systèmes à criticité mixte, tel que présenté par \cite{vestal_preemptive_2007}. Chaque niveau de criticité pouvant être associé avec différents degrés d'exigence sur l'estimation des WCET. Sachant que plus une estimation de WCET est précise et exhaustive dans ce qu'elle prend en compte, plus ce sera complexe et coûteux à réaliser. De fait, l'estimation précise et sûre du WCET d'une tâche s'exécutant sur un processeur multicœur reste aujourd'hui un problème ouvert, encore abordé dans le cadre d'outils d'estimation analytique comme \cite{kastner_timeweaver_2019}. De ces estimations peuvent découler diverses stratégies de contrôle de l'exécution du logiciel.
	
	\subsection{Contrôle Statique et Dynamique}
	Il est possible d'établir une dichotomie parmi les mécanismes de gestion des tâches dans les systèmes à criticité mixte entre les solutions \textit{statiques} d'une part et les solutions de contrôle \textit{dynamique} d'autre part. 
	
	\subsubsection{Contrôle statique}
	Dans le cas d'utilisation de solutions statiques, les décisions d'exécution des tâches sont prévues en amont, lors de la phase de développement, pour obtenir un modèle totalement prédictif et stable d'exécution. Ils se basent donc sur la combinaison des spécifications fonctionnelles et sur des estimations de temps d'exécution moyen et pire temps pour prédéterminer un ordonnancement et une allocation statique des ressources. Cela permet d'éviter par construction l'exécution concourante de logiciels qui pourraient partager des ressources
	
	Il s'agit de ce qui est le plus communément employé, avec des surestimations nécessaire des créneaux réservés à chaque tâche de façon à garantir en toute condition le respect des exigences temporelles en évitant toute exécution parallèle indésirable. Dans l'industrie les frameworks employés se basent sur ce type de méthodes. Il y a notamment Classic AUTOSAR \cite{autosar_timing_2016} dans l'automobile peut utiliser un ordonnancement fixe, ou encore dans l'avionique avec le standard ARINC 653 \cite{prisaznuk_arinc_2008}. Toujours dans l'automobile, le réseau de communication standardisé Flexray~\cite{makowitz_flexray-communication_2006} utilise aussi une méthode d'allocation statique de créneaux de communication pour obtenir des garanties fortes sur la latence d'émission/réception des messages. En contrepartie de la forte maîtrise de l'exécution du logiciel, ces méthodes ont l'inconvénient de sur-allouer les ressources nécessaires au fonctionnement du système, et donc d'aller à l'encontre de l'objectif de maximisation d'utilisation de ces dernières.
	
	En plus des stratégies d'ordonnancement statiques qui présentent une forme d'isolation temporelle entre les tâches, il est possible de proposer de l'isolation spatiale, par le partitionnement des ressources mémoire notamment. On peut citer par exemple le travail de \cite{mancuso_real-time_2013} qui propose une méthode de profilage hors-ligne des tâches pour déterminer les pages mémoires les plus utilisées. Cela permet de contrôler la position de ces données dans les caches partagés pour limiter les interférences associées. D'autres travaux se sont intéressés plus spécifiquement à la question de l'allocation des tâches en complément d'une isolation spatiale et temporelle pour optimiser la taille des fenêtres temporelles réservées pour chaque tâche \cite{tamas-selicean_task_2011}. Enfin, certains travaux ont combiné l'utilisation de méthodes d'ordonnancement statique des tâches par exclusion d'exécution simultanée des tâches de différents niveaux de criticité avec des techniques d'isolation matérielle suivant des méthodes analytiques d'identification des interférences pour proposer une approche générale qui a pu être mise en application sur un cas avionique réel. Cette approche hybride mi-statique mi-adaptative semble présenter des résultats encourageants \cite{giannopoulou_scheduling_2013}.
	
	L'association de partitionnement temporel et spatial des tâches avec des méthodes comme celles susmentionnées permet d'obtenir les exigences nécessaires pour la sûreté des systèmes embarqués tels que décrit dans les divers standards associés. Ils permettent en revanche peu d'évolutions sans avoir à refaire tout le travail d'intégration depuis le départ. 
	
	De fait, il a été admis qu'il semble difficile, sinon impossible, de trouver une méthode qui capture les problèmes de contention dans leur ensemble dans un système multicœur avec cache partagé~\cite{suhendra_exploring_2008}. Cela mène les solutions suscitées à devoir surestimer les réservations de ressources nécessaires à l'obtention des garanties temps-réel. Ce constat nous conforte dans l'idée qu'à défaut de pouvoir contrôler et empêcher tout risque de contention dans ce type d'architecture, il est préférable d'en surveiller et contrôler les conséquences pour prévenir toute conséquence catastrophique.
	
	Il est à noter que cela ne disqualifie pas pour autant ces méthodes, qui apportent des propriétés solides. Une solution globale consistera sans aucun doute en un juste milieu entre l'usage de mécanismes statiques d'allocation et d'isolation statique des logiciels avec des mécanismes réactifs d'adaptation pour permettre malgré tout une meilleure utilisation des ressources de calcul.
	
    
	\subsubsection{Contrôle Dynamique}
    
     Survient alors la possibilité d'employer des mécanismes de monitoring pendant l'exécution de façon à pouvoir réagir à des évolutions du système qui mènent à des défaillances temporelles tel qu'un dépassement d'échéance. L'intérêt de ce type de mécanisme est de privilégier l'adaptation aux conditions de fonctionnement. Cela peut permettre une meilleure robustesse à des légères variations dans les conditions d'usage. Ces aléas de fonctionnement peuvent être de nombreuses origines, à la fois externes avec des conditions de fonctionnement qui n'ont pas été prévues (interférences électro-magnétiques, comportement utilisateur inattendu...) ou bien internes (conjonction de plusieurs évènements logiciels simultanés qui donnent un comportement imprévu). Les mécanismes réactifs sont en revanche plus complexes à concevoir pour continuer à offrir les garanties désirées et apportent forcément leur lot d'incertitudes. 
     
     \begin{figure}[ht]
     	\centering
     	\includegraphics[width=0.7\linewidth]{schemas/MC2_scheduling_model}
     	\captionsetup{justification=centering} 
     	\caption[Politique d'ordonnancement avec MC\up{2} sur quadricœur~\cite{herman_rtos_2012}]{Allocation de conteneurs avec leurs politiques d'ordonnancement pour chaque niveau de criticité avec MC\up{2} sur un quadricœur~\cite{herman_rtos_2012}}
     	\label{fig:mc2schedulingmodel}
     \end{figure}
 
     Une des premières approches pour l'implémentation de systèmes à criticité mixte est le framework MC\up{2} \cite{anderson_multicore_2009} dont une implémentation est proposée par \cite{herman_rtos_2012}. Ce dernier propose un cadre d'implémentation pour un système à 5 niveaux de criticité\footnote{les niveaux de criticité 3 et 4 décrits ont été regroupés dans le schéma de Herman \& al.} qui disposent chacun d'un conteneur dédié ainsi que d'une politique d'ordonnancement associée tel qu'illustré dans la ~\autoref{fig:mc2schedulingmodel}. Le concept sous-jacent est de prioriser les niveaux de criticité par étage. Ainsi, la partition de plus haut niveau de criticité est exécutée en priorité, avec un ordonnancement statique selon une table préétablie (\textit{cyclic executive}). Quand aucune tâche de ce niveau de criticité n'est en attente d'exécution, les tâches du second niveau de criticité peuvent s'exécuter selon un ordonnancement pEDF (\textit{partitionned Earliest Deadline First}), c'est-à-dire priorité à l'échéance la plus proche, partitionné selon chaque cœur. De la même façon, les troisième et quatrième niveaux s'exécutent si aucune tâche des niveaux de criticité supérieurs ne sont en attente, cette fois-ci selon une politique g-EDF (\textit{global-EDF}). Et uniquement si tous ces niveaux de criticité se sont exécutés sans dépassement, alors le dernier niveau de criticité peut s'exécuter en \textit{best-effort}. 

     
     D'autres travaux ont pu montrer que du contrôle dynamique par construction de modèles prédictifs d'exécution présente aussi une bonne piste d'optimisation de l'exécution des applications sur multicœur en limitant les interférences. Des travaux comme ceux de \cite{kim_application_2019} ont pu étudier la question par l'utilisation d'un modèle par apprentissage automatique. 
     C'est par conséquent vers ce type de solutions réactives avec un contrôle dynamique de l'exécution de tâches que nous allons nous positionner. 

	\section{Mécanismes de contrôle réactif}
    Les mécanismes de contrôle réactif peuvent se focaliser sur différentes parties du système pour limiter les risques de fautes logicielles provoquées par l'exécution de logiciels concurrents. D'ailleurs rien n'empêcherait d'employer des mécanismes qui agiraient sur plusieurs composants à la fois. Les principaux éléments sur lesquels les mécanismes de contrôle agissent sont : 
     \begin{itemize}
     	\item	directement sur l'ordonnancement des tâches, par mise en pause ou changement de priorités des tâches durant le fonctionnement. Il s'agit alors d'un contrôle dynamique \textit{temporel} (c.-à-d. le \textit{moment} où s'exécutent les tâches est ajusté par rapport au fonctionnement nominal).
     	\item	sur l'utilisation du support d'exécution soit par limitation d'accès (budget d'utilisation), soit par priorisation de l'utilisation, soit par réservation d'espace dédié à certaines tâches pour les espaces mémoires notamment. Il s'agit alors d'un contrôle dynamique spatial (c.-à-d. la capacité d'accès au support d'exécution, et donc aux ressources partagées, est ajusté par rapport au fonctionnement nominal).
     \end{itemize}
 
 	\subsection{Contrôle réactif spatial}
 	Le contrôle réactif spatial repose sur la gestion des ressources partagées de façon à isoler les tâches de différents niveaux de criticité sur l'utilisation de ces dernières. Il peut s'agir d'allocation d'espace mémoire avec réservation de cache par exemple \cite{suhendra_exploring_2008}. D'autres travaux se focalisent plutôt sur les politiques d'accès aux ressources via les bus d'accès.
 	
 	En ce sens, des travaux comme ceux de \cite{blin_maximizing_2016} proposent un mécanisme de monitoring qui a été calibré hors-ligne, en amont, de façon à identifier une surcharge du système. Le cas échéant, un contrôle spatial est effectué  pour rendre l'utilisation du bus d'accès mémoire exclusif au cœur exécutant les tâches critiques pour garantir le respect des échéances sur ces dernières. La solution ici proposée est intéressante dans son approche, mais de par ses hypothèses implique une mise en pause de cœurs entiers du processeur, qui auront été alloués aux tâches non critiques. De même, \cite{yun_memory_2012} propose une régulation de la bande passante au niveau système pour chaque cœur.
 	
 	Il existe par ailleurs des mécanismes réactifs qui vont modifier l'allocation dynamique des tâches sur les cœurs pour équilibrer la charge. C'est le cas du modèle proposé dans \cite{xu_semi-partitioned_2019} qui suppose un changement de mode d'une partie des cœurs, permettant la migration de tâches non critique vers les cœurs qui sont toujours en fonctionnement nominal.
    
    \subsection{Contrôle réactif temporel}
    Nombreux sont les travaux qui réalisent du contrôle dynamique temporel, axé sur une adaptation de l'ordonnancement des tâches. Les ordonnancements dynamiques tels que les dérivés d'EDF~\cite{lelli_efficient_2011}, \cite{behera_schedulability_2012}, \cite{rodriguez_multicriteria_2013} permettent l'optimisation des ressources de calcul. La question demeure sur l'usage de stratégies d'ordonnancement globaux (l'exécution des tâches et leur allocation aux cœurs disponibles se fait à l'exécution) ou bien partitionnés (chaque cœur dispose de son propre ordonnanceur, les tâches sont allouées à chaque cœur à la conception). Ces deux grandes méthodes d'ordonnancement sont représentées dans la ~\autoref{fig:SchedulingModels}. Il existe par ailleurs des stratégies d'ordonnancement intermédiaires dites semi-partitionnées où certaines tâches sont partitionnées tandis que d'autres peuvent migrer librement entre les cœurs. Des comparaisons ont pu être étudiées entre ces types d'ordonnancement, par exemple par \cite{li_analysis_2014}. 
    %, mais n'offrent a priori pas de garanties strictes sur les contraintes temps réel.
    \begin{figure}[h!]
    	\centering
    	\begin{subfigure}{.45\textwidth} \centering
    		\includegraphics[width=\linewidth]{Ordonnancement_Global}
    		\caption{Modèle d'ordonnancement global}
    		\label{fig:globalScheduling}
    	\end{subfigure}
    	\begin{subfigure}{.45\textwidth} \centering
    		\includegraphics[width=\linewidth]{Ordonnancement_Partitionned}
    		\caption{Modèle d'ordonnancement partitionné}
    		\label{fig:partitionnedScheduling}
    	\end{subfigure}
    	\caption{Modèles d'ordonnancement}
    	\label{fig:SchedulingModels}
    \end{figure}
        
    Aussi, il est possible de disposer d'un ordonnancement des tâches différent pour plusieurs modes de fonctionnement du système. Ce dernier change donc de mode selon les besoins ou les risques. Un modèle spécifique de système à criticité mixte ainsi proposé est le modèle à \textit{criticité duale}. En d'autres termes, les tâches sont catégorisées selon deux niveaux de criticité différents. Selon les cas, on peut parler de criticité haute et basse, de tâches strictes et souples ou encore simplement de tâches critiques et non critiques. Les solutions concernées dans ces situations reposent alors sur un passage d'un mode de fonctionnement faiblement critique vers un mode de fonctionnement hautement critique où seul l'exécution des tâches de criticité haute est garantie. C'est le cas par exemple de la proposition d'amélioration de MC\up{2} avec un changement de mode dans \cite{chisholm_supporting_2017}. Dans une vision axée sur la sûreté de fonctionnement des systèmes, on peut considérer que le mode de fonctionnement faiblement critique est le comportement nominal attendu. À l'inverse, les modes plus critiques sont des modes dégradés progressifs dans lesquels certaines tâches peuvent avoir un fonctionnement non garanti, voire abandonné. Ces modes de fonctionnement ne doivent en conséquence pas constituer la règle, mais l'exception, envisagée pour les exigences de sûreté de fonctionnement. Les travaux de Trapp et al. ont eu l'occasion de s'intéresser à la pertinence d'utiliser de tels mécanismes de changement de modes \cite{trapp_runtime_2007}. Il s'agit d'un élément relativement important à mentionner dans le cadre d'une proposition de mécanisme qui apporte une garantie stricte dans des situations pire cas, en complément d'une infrastructure logicielle qui propose un cadre d'exécution de logiciel à criticité mixte avec une bonne exploitation des ressources.
    
    \begin{figure}[ht!]
    	\centering
    	\includegraphics[width=0.7\linewidth]{schemas/DYNASCORE_tasks_sched_example}
    	\caption[Principe de contrôle d'exécution DYNASCORE \cite{kritikakou_multiplexing_2016}]{Principe de contrôle d'exécution (non optimisé) dans \cite{kritikakou_multiplexing_2016}}
    	\label{fig:dynascoretasksschedexample}
    \end{figure}
    Dans ce sens, \cite{kritikakou_distributed_2014}, \cite{kritikakou_dynascore_2017} propose un mécanisme réactif qui se base sur deux niveaux de criticité avec le diagnostic de risque d'interférences des tâches à faible niveau de criticité sur les tâches critiques du fait du matériel partagé. La proposition consiste à surveiller l'exécution des tâches critiques de façon à identifier le moment où les interférences ne peuvent plus être tolérées, au risque de déclencher des dépassements d'échéances critiques. À cet instant-là, les tâches non-critiques sont stoppées pour prévenir le risque. La méthode de contrôle consiste à mesurer à des points de fonctionnement prédéfinis l'état d'avancement de l'exécution des tâches critiques, et de le comparer à des estimations de pire temps d'exécution restant à ce point de fonctionnement. Par calcul d'une condition de sûreté, le mécanisme de contrôle identifie le risque  pour passer en mode dégradé. Tel qu'illustré en ~\autoref{fig:dynascoretasksschedexample}, les tâches critiques $\tau_{C1}$, $\tau_{C2}$ sont exécutées sur les cœurs 1 et 2 tandis que les tâches non critiques $\tau_{1}$ [...] $\tau_{n}$ sont exécutées sur les autres cœurs. La vérification à intervalles réguliers permet d'identifier le point de décision au-delà duquel les tâches non critiques sont arrêtées temporairement pour garantir la terminaison des tâches critiques avant leur date butoir.
    

    
    Ces résultats ont été une grande source d'inspiration pour la suite de ces travaux dans le concept de surveillance de l'exécution des tâches critiques pour ne provoquer une transition dans un mode dégradé qu'en cas de dernier recours. Cela offre l'avantage d'être accessible quels que soient les autres mécanismes et méthodes employées par le système pour l'exécution des tâches, tout en offrant des garanties claires pour certaines d'entre elles. En revanche, on pourrait reprocher le fait que cette méthode requiert l'instrumentation des tâches au niveau de leur code source, ce qui n'est pas possible dans le cadre d'emploi de logiciels en boite noire.

   \section{Positionnement des travaux}
   
   Dans la vaste étendue des études sur les systèmes à criticité mixte et de sûreté de fonctionnement, nous avons proposé une classification macroscopique. Bien entendu, cette dichotomie itérative est arbitraire et certains types de mécanismes n'y sont par conséquent pas positionnables. Par exemple, il est possible de trouver des mécanismes matériels dynamiques qui se basent sur une reconfiguration matérielle en cas de faute temporelle, tel que dans \cite{lin_scheduling_2015} avec la prise en compte d'un calculateur primaire et secondaire pour le changement de mode. Ou encore le projet DREAMS~\cite{fohler_evaluation_2018} qui propose notamment une gestion dynamique des ressources en cas de surcharge du processeur. Ceci étant dit, la classification que nous proposons, agrégée dans la ~\autoref{fig:stateoftheartsituation} permet d'avoir une approche du domaine du plus global au plus spécifique par segmentation des propositions académiques en dichotomies successives. Il nous est ainsi possible de positionner notre axe d'approche dans cet ensemble et d'identifier les choix techniques et méthodologiques associés à chaque embranchement de cette vue d'ensemble.
   
   En effet, au regard des éléments précédemment mentionnés, il nous semble intéressant de trouver une solution logicielle qui soit \textit{flexible} et \textit{combinable} facilement avec d'autres techniques existantes, qui abordent potentiellement le problème sur un autre de ses paramètres. Plus spécifiquement, il devrait être possible de proposer des méthodes d'ordonnancement et d'allocation des tâches qui permet une bonne utilisation des ressources matérielles, tout en y adjoignant notre proposition qui offre des \textit{garanties sur les échéances temporelles}. 
   Ajoutez à cela la complexité d'identification et neutralisation des interférences matérielles, cela impose un \textit{mécanisme logiciel réactif} d'anticipation et mitigation des fautes qui influe sur l'exécution des tâches non critiques. 
   Aussi, il devient souhaitable de permettre un retour en fonctionnement nominal, dans les conditions où les risques de défaillances ont été écartés. 
   \\
   \smallbreak
   Par ailleurs, nous nous positionnons dans un contexte industriel où les évolutions logicielles sont constantes, avec des contraintes de coûts et de temps de développement non négligeables. Ce contexte implique aussi l'usage de logiciels en boite noire, où il n'est pas possible d'accéder et modifier le code source de façon systématique. Cela demande donc des solutions qui demandent un surcoût de développement limité en cas d'évolution (typiquement associé à le re-estimation analytique précise de pires temps d'exécution par exemple) et employable sans modification de code source.
   
   Avec la considération de tous ces éléments, notre objectif est~:
   \begin{itemize}
   	\item 	d'éviter les défaillantes temporelles par dépassement d'échéances critiques d'une part, 
   	\item   de permettre une bonne utilisation des ressources matérielles d'autre part.
   \end{itemize} 
	Ce qui peut se formaliser de la manière suivante~: 
	\begin{mdframed}[outerlinewidth=1.5pt,
	innerlinewidth=1.5pt,
	middlelinewidth=2pt,
	middlelinecolor=white,
	bottomline=false,topline=false,rightline=false]
	Comment prendre en compte les problèmes d’\textbf{interférences} sur calculateur multicœurs  pour garantir le \textbf{respect d’échéances} temporelles tout en \textbf{optimisant l’utilisation des ressources} de calcul ?
\end{mdframed}
	
	Pour répondre à cela, nous proposons un mécanisme de surveillance et de contrôle de l'exécution des tâches critiques, de façon à intervenir sur l'exécution des tâches non critiques uniquement en cas de nécessité. L'intention étant de prévenir de façon préventive et circonstancielle les interférences inter-tâches pour offrir des garanties temporelles aux tâches critiques. L'approche se voudra non intrusive sur le logiciel embarqué en abordant les tâches critiques d'un point de vue fonctionnel avec un mécanisme de contrôle bas niveau pour la surveillance et le contrôle de l'exécution.  

\ifdefined\included
\else
\bibliographystyle{StyleThese}
\bibliography{these}
\end{document}
\fi

\ifdefined\included
\else
\documentclass[french, a4paper, 11pt, twoside, pdftex]{StyleThese}
\usepackage{iflang}
\usepackage{bibentry}



%\usepackage[sectionbib]{chapterbib}          % Cross-reference package (Natural BiB)
%\usepackage{natbib}                  % Put References at the end of each chapter
%\usepackage{bibunits}
% Do not put 'sectionbib' option here.
% Sectionbib option in 'natbib' will do.


\usepackage{fancyhdr}                    % Fancy Header and Footer

\usepackage[utf8]{inputenc}
\usepackage[T1]{fontenc}
\usepackage[french]{babel} %
\usepackage{lmodern} \normalfont %to load T1lmr.fd 
\DeclareFontShape{T1}{cmr}{b}{sc} { <-> ssub * cmr/bx/sc }{}
%\hyphenation{gar}

\usepackage{amsmath,amssymb}             % AMS Math
\usepackage{nicefrac}
\usepackage{siunitx}					%% Unites Math SI

\usepackage{blindtext}

\usepackage{datetime}

\usepackage{lipsum} 

\usepackage[inline]{enumitem}

\usepackage{hhline}
%\usepackage[left=1.5in,right=1.3in,top=1.1in,bottom=1.1in]{geometry}
\usepackage[left=1.5in,right=1.3in,top=1.1in,bottom=1.1in,includefoot,includehead,headheight=13.6pt]{geometry}

%%\renewcommand{\baselinestretch}{1.05}

%%%%%%%% Multi-figures avec sub-captions
\usepackage{caption}
\usepackage{subcaption}

% Table of contents for each chapter

\usepackage[nottoc, notlof, notlot]{tocbibind}
\usepackage[nohints]{minitoc}
\setcounter{minitocdepth}{2}
\mtcindent=15pt
% Use \minitoc where to put a table of contents

\usepackage{aecompl}

%% Package cosmetic meilleur layout du texte en jouant sur le spacing par caractères
\usepackage[activate={true,nocompatibility},final,tracking=true,kerning=true,factor=1100,stretch=10,shrink=10]{microtype}
\usepackage[absolute,overlay]{textpos} 
\setlength{\TPHorizModule}{\paperwidth}\setlength{\TPVertModule}{\paperheight}
\sloppy

%%%%%%%%%%% JOLIS TABLEAUX
\usepackage{tabularx}		%\usepackage{tabular}
\usepackage{multirow}
\newcommand{\mc}{\multicolumn} 
\newcommand{\mr}[2]{\multirow{#1}{*}{#2}} 	\newcommand{\mrQ}{\multirow{-4}{*}}
\usepackage{booktabs}

\usepackage[usenames,dvipsnames]{xcolor} 

\makeatletter
\newcommand{\ccolor}[3][]{%
	\kern-\fboxsep
	\if\relax\detokenize{#1}\relax
	\expandafter\@firstoftwo
	\else
	\expandafter\@secondoftwo
	\fi
	{\colorbox{#2}}%
	{\colorbox[#1]{#2}}%
	{#3}\kern-\fboxsep
}
\makeatother

%%%%% Insertion graphiques format PGF
\usepackage{pgfplots}
\pgfplotsset{width=\linewidth, compat=1.16}%, compat=1.17}
\usepackage{adjustbox}          %%% PERMET DE LES RECADRER + FACILEMENT


%%%%%%%%%% Bullets de listes sans saut de ligne %%%%%%%%%%
\usepackage{xparse}

\ExplSyntaxOn%
\seq_new:N \l_local_enum_seq

\newcommand{\storethestuff}[1]{%
  \seq_set_from_clist:Nn \l_local_enum_seq {#1}%
}

\newcommand{\dotheenumstuff}{%
\int_zero:N \l_tmpa_int
\seq_map_inline:Nn \l_local_enum_seq {%
    \int_incr:N \l_tmpa_int% Increase the counter
    \item ##1
    % Check whether the list has reached the end -- if so, use '.' instead of ','
    %\int_compare:nNnTF 
    % { \l_tmpa_int } < {\seq_count:N \l_local_enum_seq} 
    % {,} {.}
  }
}
\ExplSyntaxOff

\NewDocumentCommand{\linebullets}{+m}{%
  \storethestuff{#1}%
  \begin{enumerate*}[label={\alph*)},font={\bfseries},itemjoin={{, }}]
    \dotheenumstuff%
  \end{enumerate*}
}

\newcommand{\cmnt}[1]{}  %%%%% AJOUT DE COMMENTAIRE MULTILIGNES


%%%%%%%%%% ECRITURE CARACTERES DANS UN CERCLE %%%%%%%%%%
%\def\circleTxt[#1]{\raisebox{.5pt}{\textcircled{\raisebox{-1pt}{#1}}}}
\newcommand{\ctxt}[1]{\raisebox{.5pt}{\textcircled{\raisebox{-1.2pt}{#1}}}}
% Glossary / list of abbreviations

\usepackage[intoc]{nomencl}
\IfLanguageName{english}{%
\renewcommand{\nomname}{Glossary}
}{ %
\renewcommand{\nomname}{Liste des Abréviations}
}

\makenomenclature

% My pdf code

\usepackage{ifpdf}

\ifpdf
  \usepackage[pdftex]{graphicx}
  \DeclareGraphicsExtensions{.pdf,PDF,.png,PNG,.jpg,JPG}
  \usepackage[pagebackref,hyperindex=true]{hyperref} %% use \autoref{} instead of Table~\ref{}.
  \usepackage{tikz}
  \usetikzlibrary{arrows,shapes,calc}
\else
  \usepackage{graphicx}
  \DeclareGraphicsExtensions{.ps,.eps}
  \usepackage[a4paper,dvipdfm,pagebackref,hyperindex=true]{hyperref}
\fi

\graphicspath{{.}{schemas/}{graphiques/}{tables/}}

%% nicer backref links. NOTE: The flag ThesisInEnglish is used to define the
% language in the back references. Read more about it in These.tex

\IfLanguageName{english}{
\renewcommand*{\backref}[1]{}
\renewcommand*{\backrefalt}[4]{%
\ifcase #1 %
(Not cited.)%
\or
(Cited in page~#2.)%
\else
(Cited in pages~#2.)%
\fi}
\renewcommand*{\backrefsep}{, }
\renewcommand*{\backreftwosep}{ and~}
\renewcommand*{\backreflastsep}{ and~}
}{
\renewcommand*{\backref}[1]{}
\renewcommand*{\backrefalt}[4]{%
\ifcase #1 %
(Non cité.)%
\or
(Cité en page~#2.)%
\else
(Cité en pages~#2.)%
\fi}
\renewcommand*{\backrefsep}{, }
\renewcommand*{\backreftwosep}{ et~}
\renewcommand*{\backreflastsep}{ et~}
}

% Links in pdf
\usepackage{color}
\definecolor{linkcol}{rgb}{0,0,0.4} 
\definecolor{citecol}{rgb}{0.5,0,0} 
\definecolor{linkcol}{rgb}{0,0,0} 
\definecolor{citecol}{rgb}{0,0,0}
% Change this to change the informations included in the pdf file

\hypersetup
{
bookmarksopen=true,
pdftitle="Prévention des fautes temporelles sur architectures multicœur pour les systèmes à criticité mixte",
pdfauthor="Daniel LOCHE", %auteur du document
pdfsubject="Thèse", %sujet du document
%pdftoolbar=false, %barre d'outils non visible
pdfmenubar=true, %barre de menu visible
pdfhighlight=/O, %effet d'un clic sur un lien hypertexte
colorlinks=true, %couleurs sur les liens hypertextes
pdfpagemode=UseNone, %aucun mode de page
%pdfpagelayout=DoublePage, %ouverture en simple page
pdffitwindow=true, %pages ouvertes entierement dans toute la fenetre
linkcolor=linkcol, %couleur des liens hypertextes internes
citecolor=citecol, %couleur des liens pour les citations
urlcolor=linkcol %couleur des liens pour les url
}

% definitions.
% -------------------

\setcounter{secnumdepth}{3}
\setcounter{tocdepth}{2}

% Some useful commands and shortcut for maths:  partial derivative and stuff

\newcommand{\pd}[2]{\frac{\partial #1}{\partial #2}}
\def\abs{\operatorname{abs}}
\def\argmax{\operatornamewithlimits{arg\,max}}
\def\argmin{\operatornamewithlimits{arg\,min}}
\def\diag{\operatorname{Diag}}
\newcommand{\eqRef}[1]{(\ref{#1})}
\newcommand{\nline}{\smallbreak\noindent}

\usepackage{rotating}                    % Sideways of figures & tables

% \usepackage{txfonts}                     % Public Times New Roman text & math font
  
%%% Fancy Header %%%%%%%%%%%%%%%%%%%%%%%%%%%%%%%%%%%%%%%%%%%%%%%%%%%%%%%%%%%%%%%%%%
% Fancy Header Style Options

\pagestyle{fancy}                       % Sets fancy header and footer
\fancyfoot{}                            % Delete current footer settings

%\renewcommand{\chaptermark}[1]{         % Lower Case Chapter marker style
%  \markboth{\chaptername\ \thechapter.\ #1}}{}} %

%\renewcommand{\sectionmark}[1]{         % Lower case Section marker style
%  \markright{\thesection.\ #1}}         %

\fancyhead[LE,RO]{\bfseries\thepage}    % Page number (boldface) in left on even
% pages and right on odd pages
\fancyhead[RE]{\bfseries\nouppercase{\leftmark}}      % Chapter in the right on even pages
\fancyhead[LO]{\bfseries\nouppercase{\rightmark}}     % Section in the left on odd pages

\let\headruleORIG\headrule
\renewcommand{\headrule}{\color{black} \headruleORIG}
\renewcommand{\headrulewidth}{1.0pt}
\usepackage{colortbl}
\arrayrulecolor{black}

\fancypagestyle{plain}{
  \fancyhead{}
  \fancyfoot{}
  \renewcommand{\headrulewidth}{0pt} %%%%%%%%%%%%%%%%%%%%%%%%%%%%%%%%%%%%%%%%%%%%%%%%%%%%%%%%%%%%%%%%%%%%%%%%%%%%%%%%%%%%%
}

%\usepackage{MyAlgorithm}
%\usepackage[noend]{MyAlgorithmic}
%\usepackage[ED=EDSYS-SystEmb, Ets=INP]{tlsflyleaf}

%%% Clear Header %%%%%%%%%%%%%%%%%%%%%%%%%%%%%%%%%%%%%%%%%%%%%%%%%%%%%%%%%%%%%%%%%%
% Clear Header Style on the Last Empty Odd pages
\makeatletter

\def\cleardoublepage{\clearpage\if@twoside \ifodd\c@page\else%
  \hbox{}%
  \thispagestyle{empty}%              % Empty header styles
  \newpage%
  \if@twocolumn\hbox{}\newpage\fi\fi\fi}

\makeatother
 
%%%%%%%%%%%%%%%%%%%%%%%%%%%%%%%%%%%%%%%%%%%%%%%%%%%%%%%%%%%%%%%%%%%%%%%%%%%%%%% 
% Prints your review date and 'Draft Version' (From Josullvn, CS, CMU)
\newcommand{\reviewtimetoday}[2]{\special{!userdict begin
    /bop-hook{gsave 20 710 translate 45 rotate 0.8 setgray
      /Times-Roman findfont 12 scalefont setfont 0 0   moveto (#1) show
      0 -12 moveto (#2) show grestore}def end}}
% You can turn on or off this option.
% \reviewtimetoday{\today}{Draft Version}
%%%%%%%%%%%%%%%%%%%%%%%%%%%%%%%%%%%%%%%%%%%%%%%%%%%%%%%%%%%%%%%%%%%%%%%%%%%%%%% 

\newenvironment{maxime}[1]
{
	\def\Arg{#1}
\vspace*{0cm}
\hfill
\begin{minipage}{0.6\textwidth}%
%\rule[0.5ex]{\textwidth}{0.1mm}\\%
\hrulefill $\:$ \\%$\:$ {\bf #1}\\
%\vspace*{-0.25cm}
\it 
}%
{%
	
\hrulefill $\:$ {\bf \Arg}
\vspace*{0.5cm}%
\end{minipage}
}

\let\minitocORIG\minitoc
\renewcommand{\minitoc}{\minitocORIG \vspace{1.5em}}

%\usepackage{slashbox}

\newenvironment{bulletList}%
{ \begin{list}%
	{$\bullet$}%
	{\setlength{\labelwidth}{25pt}%
	 \setlength{\leftmargin}{30pt}%
	 \setlength{\itemsep}{\parsep}}}%
{ \end{list} }


%%%%%%% Outils pour \comment \alert \add %%%%%
\usepackage{easyReview}
\usepackage{soulutf8} % for accented letters

\let\newalert\alert
\renewcommand{\alert}[1]{\textit{\newalert{#1}}}

%\usepackage[commandnameprefix=ifneeded]{changes} %% \chhighlight and \chcomment to avoid collision with easyReview
\renewcommand{\epsilon}{\varepsilon}

% centered page environment

\newenvironment{vcenterpage}
{\newpage\vspace*{\fill}\thispagestyle{empty}\renewcommand{\headrulewidth}{0pt}}
{\vspace*{\fill}}

\usepackage{tablefootnote}

%%%%%% MISE EN FORME CADRES DEFINITIONS/THEOREMES/LEMES %%%%%%%%%%
\usepackage{amsthm}  % for theoremstyle

\theoremstyle{plain} 
\newtheorem{theorem}{Théorème}[section]
\newtheorem{corollary}{Corolaire}[theorem]

%\theoremstyle{lemma}
%\newtheorem{lemma}[theorem]{Lemme}


\theoremstyle{definition}
\newtheorem{definition}[theorem]{Définition}


\cmnt{
	\usepackage{ntheorem} %\usepackage{amsthm}  % for theoremstyle
	%\usepackage{mdframed}
	\usepackage[most]{tcolorbox}
	
	\theoremstyle{plain} 
	\theoremindent20pt
	\theoremheaderfont{\normalfont\bfseries\hspace{-\theoremindent}}
	\newtheorem{theorem}{Théorème}[section]
	\newtheorem{corollary}{Corolaire}[theorem]
	
	\theoremstyle{plain}
	\newtheorem{lemma}[theorem]{Lemme}
	
	
	\tcolorboxenvironment{theorem}{
		blanker,
		breakable,
		before skip=\topsep,
		after skip=\topsep,
		borderline west={1pt}{10pt}{double, shorten <=12pt}
	}
	
	\theorembodyfont{\normalfont}
	\theoremindent20pt
	\theoremheaderfont{\normalfont\bfseries\hspace{-\theoremindent}}
	\newtheorem{definition}[theorem]{Définition}
	
	
	\tcolorboxenvironment{definition}{
		blanker,
		breakable,
		before skip=\topsep,
		after skip=\topsep,
		borderline west={1pt}{10pt}{shorten <=12pt}
	}
}

\cmnt{ 
	\begin{theorem}
		Ceci est un Théorème.
	\end{theorem} 
	
	\begin{corollary}
		Ceci est un Corollaire.
	\end{corollary}
	
	\begin{definition}
		Ceci est une Définition.
	\end{definition}
	
	\begin{lemma}
		Ceci est un Lemme.
	\end{lemma}
}

\def\UrlBigBreaks{\do\/\do-\do:}
\usepackage{url}

\sloppy
\begin{document}
\setcounter{chapter}{3} %% Numéro du chapitre précédent ;)
\dominitoc
\faketableofcontents
\fi

\chapter{Principe et architecture} \label{chap:3_PrincipeArchi}
\minitoc

Dans ce chapitre, nous allons regarder plus en détail les hypothèses que nous avons considérées pour proposer un mécanisme de Surveillance et de Contrôle. Le contexte industriel mentionné précédemment au~\autoref{chap:1_EnjeuxIntro} a grandement contribué aux spécifications de notre proposition. De fait et pour rappel, les objectifs principaux de ces travaux est de proposer un mécanisme logiciel qui permette dans le même temps à utiliser au maximum les ressources matérielles disponibles et avoir des garanties minimales sur les temps d'exécutions. Tout l'enjeu de cette proposition est d'arriver à une solution qui limite les coûts de développement notamment en étant compatible avec des composants logiciels \textit{legacy} et/ou en boîte noire qui ne permettent pas d'être modifié dans leur fonctionnement interne. Mais aussi en limitant les besoins de modification de l'architecture logicielle suite à des mises à jour ou l'ajout de fonctionnalités supplémentaires. Par ailleurs, on tentera de s'abstraire le plus possible du contexte automobile pour proposer une approche généraliste qui puisse s'adapter à différents domaines et selon le contexte. 

Nous verrons au sein de ce chapitre dans un premier temps les prérequis considérés ainsi que notre approche sur la modélisation du problème. Suite à cela nous verront dans un second temps la solution proposée qui est un mécanisme de sûreté de fonctionnement réactif de type Surveillance et Contrôle. Son objectif sera de prévenir les fautes temporelles par dépassement d'échéances sur les tâches critiques, via une approche basée sur des chaînes de tâches.
Pour finir, nous verrons dans quelle mesure notre proposition se positionne vis-à-vis des standards d'architectures existants dans différents domaines industriels dont l'automobile.

%The MCA role is first to monitor the state of a HI-criticality task chain to detect potential deadline miss. If such a potential fault is anticipated, then the MCA switches the system to HI-criticality mode, pausing all non essential workload (LO-criticality tasks), to prevent further interference on the HI-criticality tasks and allow a safe termination. To be efficient, the switch  must be triggered only when necessary (as a “mode switch procrastination”, as called in~\cite{hu_ffob_2019}). That is why we also focus on end-to-end deadline, rather than individual task deadlines, in order to avoid false-positive switching, meaning switching to HI-criticality mode although there is slack in the task chain. Indeed, with an end-to-end perspective, we can use the slack given by a task finishing early to compensate the lateness of an other task in the chain.

%In the following, we introduce the execution model considered in our work, then we describe the proposed MCA architecture to finally present in more details the principle of the anticipation mechanism.

\section{Un modèle basé sur des chaînes de tâches pour garantir les contraintes temporelles}

    %The model describes how HI-criticality tasks behave and are linked in a chain to produce a HI-criticality functionality. Note that this model is one among many usable with our approach. We choose this one for ease of presentation. 
    \comment{Je sais ce que j'ai loupé d'entrée de jeu !}{Il faut que je rajoute l'explication de la philosophie générale, i.e. taches critiques interférées par non critiques, passage en mode dégradé pour que les taches critiques se retrouvent en pseudo isolation sans interférences.}
    
    Afin d'étudier et développer notre mécanisme de gestion de fautes temporelles dans le cadre d'un système à criticité mixte (\textit{Mixed Criticality Systems}), nous avons besoin de formaliser la représentation des tâches qui seront à l'étude et leur modèle. Remarquons que le modèle ici proposé est relativement arbitraire et choisi essentiellement pour des raisons de commodité. De fait, on retiendra deux critères principaux pour guider le choix de notre modèle~: la simplicité d'implémentation et l'accessibilité à des suites logicielles qui peuvent servir de tâches pour simuler un système réel lors de nos tests. L'objectif est ainsi de trouver un juste milieu entre un modèle représentatif d'une réalité technique dans les milieux industriels d'une part et un modèle qui nous évite des surcoûts de développement pour obtenir une première preuve de concept fonctionnelle.
    
    
    Ce modèle doit décrire d'une part la méthode d'exécution des tâches, la façon d'interagir, entre-elles, notamment pour les tâches à haut niveau de criticité qui sont reliées sous la forme d'une chaîne pour réaliser une fonction critique. Il est à noter que le mécanisme de sûreté de fonctionnement que nous proposons par la suite est \textit{in fine} indépendant du modèle de tâche ici proposé. Il conviendra d'adapter au besoin la partie de Contrôle du mécanisme, de façon à ce qu'elle prenne en compte l'état d'exécution du système selon le modèle de tâche utilisé, s'il est différent de celui présenté ici. Typiquement la vérification des contraintes de précédence peut différer. On aura l'occasion d'aborder rapidement ces aspects par la suite, avec quelques exemples de modifications requises suivant des changements de ce modèle de tâche. \alert{Petite note pour pas oublier}
    
    
    \subsection{Notion de Criticité}
    	
    Les systèmes à criticité mixte ont mené à de nombreuses études pour gérer la répartition temporelle des tâches, et en même temps prendre en compte les conditions de partitionnement et le partage des ressources de façon à optimiser l'usage de ces dernières. La plupart des travaux dans ce domaine se basent sur le modèle d'un ensemble de tâches $\tau_i$ caractérisées par $ (T_i,D_i,C_i,L_i) $ : 
    \begin{itemize}
    	\item $T_i$ - la période d'apparition de la tâche
    	\item $D_i$ - la date limite d'échéance d'exécution (que l'on appelle communément \textit{deadline})
    	\item $C_i$ - Le Temps d'Exécution Pire Cas %(\emph{Worst Case Execution Time - WCET})
    	\item $L_i$ - Le Niveau de Criticité de la tâche
    \end{itemize}
     
    Avant toute chose, il est à noter que la notion de criticité des tâches que nous avons déjà mentionnée dès le chapitre d'introduction est polysémique. En effet, selon le domaine et la spécialisation des interlocuteurs, ce mot peut traduire des enjeux de sûreté de fonctionnement différents. Il convient donc de clarifier cette notion de criticité pour notre cas avant d'aller de l'avant. Ce problème de sens a été souligné notamment par~\cite{graydon_safety_2013}. Il existe deux grandes approches à la notion de criticité. 
    
    La première, d'un point de vue modélisation de modèle d'exécution temps-réel, se retrouve essentiellement dans les recherches académiques. La notion de criticité est reliée aux contraintes temporelles imposées sur les tâches. Cela se traduit par le niveau de fiabilité de l'estimation des pires temps d'exécution $C_i$~\cite{vestal_preemptive_2007}. 
    %Plus le niveau de criticité est élevé, plus l'estimation du WCET est prudente \alert{\textit{"conservative" en anglais}}. 
    Plus le niveau de criticité sera élevé, plus ce WCET sera estimé de façon fiable (l'on pourrait dire de façon "prudente") et avec des contraintes fortes. Autrement dit, un haut niveau de criticité requiert de prendre en compte les cas les plus extrêmes de retard d'exécution pour estimer le WCET et les temps d'exécutions estimés pour l'ordonnançabilité sont plus longs. Ce type de formulation a été étendue avec des modes de criticité différente. À chaque mode de criticité est associé un niveau de criticité des tâches, plus ou moins pessimiste.
    Tout l'intérêt de ces modèles  dans le cadre de systèmes à criticité mixte est de proposer des stratégies d'ordonnancement qui prennent en compte ces niveaux et ces modes de criticité pour permettre la priorisation d'exécution de certaines tâches plus critiques que d'autres, pour aller jusqu'à stopper des tâches moins critiques pour garantir l'exécution des tâches de criticité supérieure. Cette façon d'aborder la criticité est donc totalement orientée suivant des critères d'ordonnancement et de respect des échéances temporelles.
    
    On notera dès cette première formulation un mélange entre les niveaux de criticité des tâches et des modes de criticité. 
    

    Le second, d'un point de vue des standards de sûreté de fonctionnement, est largement exploité dans l'industrie. Dans ce cadre-là, la criticité détermine un niveau de confiance que l'on peut accorder à un composant tel que décrit précédemment en~\autoref{sec:SureteDeFonctionnement}. Le niveau de criticité est alors déterminé non pas au regard des contraintes temporelles, mais plutôt avec des analyses de sûreté (Analyses de Risques, FMEA, FTA...) pour obtenir une catégorisation tel que défini par les différentes normes du domaine (un niveau d'ASIL tel que défini dans ISO26262 pour l'automobile par exemple). Elle se définit alors en fonction de critères comme \linebullets{l'évaluation des conséquences en cas de défaillance, la probabilité de défaillance, les moyens disponibles pour compenser ou gérer la faute}si elle survient. Par conséquent, le niveau de criticité d'une application ne reflète alors pas nécessairement la sévérité ou les répercutions d'une faute. De fait, une application avec des chances de défaillance très faibles, mais où les répercussions sont très graves pourra être d'un niveau de criticité faible. Inversement, une application avec peu, voire aucune, conséquence en cas de faute mais un risque de défaillance très élevé peut avoir le même niveau de criticité. Avec ce type d'interprétation, le principe d'arrêt de tâches à niveau de criticité inférieur pour éviter les fautes temporelles sur des tâches plus critiques perds de son sens. Alors même que ce genre de solution est courant dans les mécanismes de sûreté de fonctionnement de systèmes à criticité mixte, mais avec la première définition suscitée de la criticité. De plus, les niveaux de criticité dans le domaine industriel impliquent souvent des propriétés de composabilité, qui permettent de combiner des composants à plus faible niveau de criticité pour obtenir l'équivalent d'un unique composant de niveau de criticité supérieure. Par exemple, si un capteur automobile requiert pour son usage un niveau d'ASIL A, il est possible d'utiliser conjointement deux capteurs avec un niveau d'ASIL B en équivalent. L'objectif des compositions étant de limiter les coûts, étant donné que plus le niveau d'ASIL est élevé, plus les exigences de développement et de vérification sont forts.
    
    Enfin, que ce soit dans un cas comme dans l'autre, il est question de \textbf{Modes} de fonctionnement. Mais là encore cela peut impliquer des sens différents. Il existe d'une part les \textbf{Modes de Service}, c'est-à-dire des modes de fonctionnement dont l'objectif est uniquement la reconfiguration pour la survie et le bon fonctionnement du système. Il y a ensuite les \textbf{Modes d'Opération} qui décrivent des modes de fonctionnement dans l'usage "normal" du système. Par exemple des modes "décollage", "vol de croisière" et atterrissage" pour un avion. Et enfin les \textbf{Modes d'ordonnancement} qui se focalise exclusivement sur la façon selon laquelle le processeur va exécuter (ou non) les tâches pour permettre le bon ordonnancement global du logiciel.
    
	Au regard de ces disparités, il convient donc de spécifier selon quel critère nous souhaitons définir les tâches \textit{critiques} pour lesquelles nous désirons conserver des garanties de temps d'exécution pire cas. L'objectif inclusif de ces travaux étant de limiter pour une application donnée les risque de dépassement d'échéances d'exécution du logiciel qui réalise le service essentiel du système, les définitions se veulent volontairement générales pour pouvoir s'adapter selon les besoins du cas d'étude.
	
	\begin{definition}\textbf{ -- Criticité} \\
		La criticité d'une tâche exécutée sur un processeur donné pour réaliser une fonctionnalité du-dit processeur se définit selon son \textbf{importante}. L'importance se mesure par les conséquences en cas de défaillance de cette fonctionnalité à la fois au regard de l'utilisateur (dangers) ainsi que l'importance de la fonctionnalité vis-à-vis des cas d'application essentiels du système. 
	\end{definition}

    Le système que nous allons étudier ici est dit à niveau de criticité dual. Il exécute un set de tâches logicielles (aussi appelée "charge utile") exécutées sur un support logiciel (classiquement, le système d'exploitation). Elles se répartissent entre les tâches à haute criticité d'une part ("tâches critiques"), et à faible criticité d'autre part (non critiques). 	Dans ce cadre particulier, il est alors possible de définir : 
    
    \begin{definition}\textbf{ -- Tâche critique} \\
    	Une tâche critique est une tâche au niveau d'importance élevée. On peut aussi parler communément de tâche "vitale", dans le sens où une défaillance de cette fonctionnalité aurait soit des conséquences sévères pour l'utilisateur, soit empêcherait le bon fonctionnement d'un des cas d'application essentiels du système.
    	Inversement, une tâche pour laquelle une faute provoquant un dépassement d'échéance n'aurait pas de conséquences sévères sur l'utilisateur ou ne préviendrait pas la réalisation des fonctionnalités essentielles peut se définir comme non critique. En d'autres termes, une tâche de faible importance (relativement aux tâches critiques).
    \end{definition}
    
    Enfin, en cohérence avec les modes de criticité mentionnés plus haut, nous définissons deux Modes de Service directement reliés à des Modes d'Ordonnancement pour contribuer à la sûreté de fonctionnement du système. D'une part le \textbf{Mode Nominal} où tout le système fonctionne normalement en l'absence de fautes. D'autre part, le \textbf{Mode Dégradé} dans lequel le système ne réalise pas la totalité des fonctionnalités de façon à pouvoir conserver des garanties d'exécution sur les tâches critiques et donc prévenir les fautes.
    

    \subsection{Modèle de Tâches et Chaînes de tâches}
    	Maintenant que nous avons clairement posé la notion de criticité sur laquelle nous nous focalisons, il est possible d'expliciter notre objectif pour répondre à la problématique.
    	
    	Pour rappel, nous souhaitons exploiter au maximum les ressources matérielles disponibles au sein d'un multicœur basé sur le cache.
    	Dans le même temps, la contrainte propre aux fonctionnalités critiques du système nous impose de conserver des garanties de sûreté de fonctionnement de façon à éviter toute défaillance catastrophique.
    	Pour répondre à cela, nous nous plaçons alors dans le cadre d'un système où les tâches sont divisées entre deux niveaux de criticité. D'une part les tâches non critiques qui sont potentiellement moins restreintes en terme d'usage de ressources (temps de calcul et ressources partagées).  D'autre part les tâches critiques suivant le critère d'importance susmentionné pour lesquelles on doit garantir une qualité de service. 
    	
    	\alert{Insérer petit schéma intermédiaire avec des tâches verts et rouges (critiques/non critiques) sur un multicœur ?}
    	
    	\subsubsection{Modèle de tâches}
		La plupart des hypothèses faites ici se focalisent sur les tâches critiques, tandis que la seule hypothèse forte sur les tâches non critique est la capacité à les stopper (soit un arrêt total, soit une mise en pause) et les relancer en cours d'exécution de façon à pouvoir déclencher un Mode Dégradé où il n'y a plus de tâches non critiques avec les risques d'interférences afférents envers les tâches critiques. Sous les systèmes type Unix, cela correspond typiquement à l'envoi d'un signal SIGSTOP et SIGCONT. Sans cette condition, les Modes de Service mentionnés ci-dessus ne sont pas exploitables pour notre besoin. 
		
		Chaque tâche critique $\tau_i$ est activée et exécutée suivant une période $T_i$. 
		À chaque période, le job $\tau_{i,j}$ correspond à la $j^{ieme}$ exécution de la tâche $\tau_i$. 
		On peut alors noter pour chaque job $\tau_{i,j}$ son moment d'activation $a_{i,j}$, son début d'exécution $s_{i,j}$ et sa terminaison $e_{i,j}$. 
		On considère qu'un job consomme toutes ses données d'entrée (inputs) au début de son exécution, s'exécute et fourni à la fin de son exécution les données de sortie. Les données d'entrée et de sortie des tâches sont stockées en espace mémoire partagé : la transmission des données d'une tâche à l'autre se fait de façon asynchrone.
		Cela nous mène à la question de l'interaction entre les tâches et notamment la façon de représenter la précédence.
		
		\alert{Citer différents modèles d'exécution de tâches existants ici ? c.f. ~\cite{friese_estimating_2018}}
    
    	\subsubsection{Chaînes de tâches}
	    La question de la dépendance entre les tâches est importante pour aborder le problème des contraintes temps-réel avec une vision plus macroscopique. En effet, dans le cadre de l'usage de tâches ayant des contraintes temporelles souples (c.f.~\autoref{sec:SystemesTempsReels} - Systèmes temps-réel), c'est uniquement avec une vision plus globale de l'exécution du système qu'il est possible de tirer au maximum parti des légers dépassements pour éviter dans la globalité d'avoir recours à des politiques d'exécution des tâches plus restrictives, et par conséquent qui sous-exploitent la puissance de calcul disponible.
	    Nous considérons ici la dépendance entre les tâches via les données partagées entre ces dernières selon un modèle type producteurs/consommateurs. Les tâches ont des relations de cause à effet et par conséquent, d'un point de vue strictement fonctionnel on peut décrire le système comme étant une accumulation de fonctionnalités réalisées par l'exécution de tâches successives. Cela permet alors d'introduire la notion de contrainte temporelle fonctionnelle, qui décrivent des contraintes d'exécution de chaînes de tâches bout-en-bout.
	    
	    
	    On représente une dépendance entre tâches sous la forme de chaînes de tâches, suivant le modèle $\tau_{1} \rightarrow \tau_2 \rightarrow \ldots \rightarrow \tau_n$. Dans un tel exemple, $\tau_1$ est la \textbf{tâche d'entrée} de la chaîne, tandis que $\tau_n$ est la \textbf{tâche de sortie} de la chaîne. Notons que ce modèle peut être étendu pour supporter des tâches représentées par un Diagramme Orienté Acyclique (\textit{Directed Acyclic Graph - DAG}) sans difficulté. Nous nous contentons ici de travailler avec des chaînes directes, sans divergences ou convergences dans le graphe. Nous aborderons la question ultérieurement. De fait, cet ajout de complexité dans le modèle de chaîne de tâche n'apporte rien sur les résultats ni sur la démarche et n'implique, au demeurant, pas de modifications sur la solution proposée. 

	    \alert{J'hésite à présenter ça dans "l'autre sens" : présenter un modèle de chaînes de tâches plus complet (avec divergences, convergences, etc.) et au final restreindre le modèle à des chaînes linéaires qu'au niveau du cas d'étude (chapitre 5). Option 2 (actuelle): en perspective de la thèse présenter les implications d'un modèle de chaînes plus complexe}
	    
	    \begin{figure}[ht]
	    	\centering
	    	\includegraphics[width=.8\linewidth]{TaskChain_chronogram.pdf}
	    	\caption{Exemple d'exécution d'une chaîne de tâches $\tau_1 \rightarrow \tau_2 \rightarrow \tau_3$}
	    	\label{fig:chain_chronogram}
	    \end{figure}
    
	    Dans ce contexte, on peut donc définir la relation entre une tâche $\tau_i$ et son successeur $\tau_{i+1}$. Pour produire la donnée de sortie du job $\tau_{i+1,k}$ de la tâche $\tau_{i+1}$, ce dernier consommes toutes les données d'entrée en attente provenant des jobs $\tau_{i,j}$. Les données en attente étant celles qui n'ont pas été consommé par le job précédent de $\tau_{i+1}$, i.e. $\tau_{i+1,k-1}$. On peut donc écrire que pour un $\{i,k\}$ donnés, sont consommés les données de tous les jobs $\tau_{i,j}$ ssi $j \textup{ tel que } e_{i,j} \leq s_{i+1, k}$ et $e_{i,j} > s_{i+1, k-1}$. Autrement dit, un job $\tau_{i,j}$ n'a un effet sur $\tau_{i+1,k}$ si et seulement si ce dernier est le premier job de $\tau_{i+1}$ exécuté après la terminaison de $\tau_{i,j}$.
	    
		Dans ces conditions, on nomme $\tau_{i+1,k}$ le \textbf{successeur} du job $\tau_{i,j}$. On note $succ()$ la fonction qui permet de trouver le successeur d'un job donné. Par extension, la fonction itérative $succ^{n-1}()$ permet de trouver le job de sortie d'une chaîne de tâche donnée, selon le job d'entrée. 
	    Pour illustrer cela, on peut prendre l'exemple d'une chaîne de trois tâches $\tau_1 \rightarrow \tau_2 \rightarrow \tau_3$, tel que représenté en~\autoref{fig:chain_chronogram}. On peut par exemple voir qu'une des exécutions de la chaîne de tâche, débutant par $\tau_{1,1}$, donne : $succ^{2}(\tau_{1,1}) = succ(succ(\tau_{1,1})) = succ(\tau_{2,2}) = \tau_{3,2}$. Mais aussi que le job $ \tau_{2,3} $ doit prendre en compte les valeurs de sorties de 2 jobs de la tâche $ \tau_1 $. %On a alors le temps de réponse de la-dite chaîne : $R_1 = e_{3,2} - s_{1,1} = 31 - 5 = 26$ unités de temps.
	    
  		Étant donné que les tâches peuvent être définies par des périodes d'activation différentes, cela signifie notamment que si une tâche $\tau_i$ est exécutée plus fréquemment que son successeur $succ(\tau_{i+1})$, alors il est possible  qu'un job $\tau_{i+1,j}$ soit le successeur de plusieurs jobs de la tâche $\tau_{i}$. Cette façon de considérer les choses permet une plus grande flexibilité de notre modèle pour s'adapter à un cas concret. En effet, de cette façon le modèle gère déjà un nombre assez significatif d'implémentations de tâches existantes~:
	\begin{itemize}
		\item   		les tâches qui ne considèrent que la donnée d'entrée la plus récente, les données précédentes étant considérées obsolètes. Dans ce cas-là, si les données de 2 jobs prédécesseurs sont consommés, en réalité la donnée du job le plus ancien des deux sera simplement ignorée.
		\item   		Les tâches avec une file d'attente en entrée : lorsque la tâche est exécutée elle consomme toutes les données d'entrée en attente.
		\item   		Pour les tâches où chaque exécution de la tâche ne prend en compte qu'une seule donnée de la file d'attente, ce n'est géré que partiellement. Cela peut être la donnée la plus ancienne (stratégie FIFO\footnote{FIFO : First-In First-Out, les données sont traitées de la première arrivée à la dernière.}) ou la plus récente (stratégie LIFO\footnote{LIFO : Last-In First-Out, les données sont traitées de la dernière (plus récente) arrivée à la plus ancienne.}). Si une tâche s'exécute plus lentement que sa prédécesseure, c'est le cas que nous ne gérons pas. Cependant, il est improbable, car cela implique que la quantité de données en file d'attente peut potentiellement exploser (or, la file d'attente ne peut être infinie). En revanche si les tâches ont toutes la même fréquence d'activation, ou si les tâches prédécesseures s'exécutent systématiquement à une plus faible fréquence que les tâches suivantes, alors on se retrouve finalement dans la même situation que le premier type de tâches citées. De fait cela implique qu'il n'y a en réalité jamais plusieurs données en attente à l'entrée en fonctionnement normal. 
	\end{itemize}
  		De façon succincte, quand on parle de "consommer" plusieurs jobs de la tâche précédente tel qu'on le présente ici, cela n'implique pas que toutes ces données seront prises en compte. Tout dépend du modèle de fonctionnement interne des tâches. Mais quoi qu'il en soit, cela permet de gérer un grand nombre de comportements classiques.
  		
  		Ce modèle présente en revanche un inconvénient qui est de ne pas considérer de potentiels retards entre le moment où une tâche délivre sa donnée de sortie et le moment où cette donnée est réellement disponible pour son successeur. Il faudrait pour cela ajouter une constante de latence correctement estimée selon le système à la condition de précédence de tâche.
	    %%i.e. un \textit{job} $k$ donné de la tâche $i+1$ consomme toutes les données des \textit{jobs} de la tâche $i$ qui se sont terminées avant le début de son exécution et qui n'ont pas été consommées par le job précédent $k-1$.


        \paragraph{Temps de réponse bout-en-bout}

    La notion de successeur permet de définir le temps de réponse bout-en-bout $R_j$ de la $ j^{ème} $ instance d'exécution d'une chaîne de tâche. Ainsi $ R_j $ désigne le temps écoulé entre l'activation du \textit{job} d'entrée $\tau_{1, j}$ de la chaîne, jusqu'à la terminaison du \textit{job} de sortie $\tau_{n, k} = succ^{n-1}(\tau_{1, j})$.
    On a alors $R_{j} = e_{n,k} - a_{1,j}$ que l'on peut retrouver dans l'exemple du~\autoref{fig:chain_chronogram}. Sur cet exemple, il est possible de reconstituer trois instances d'exécution de la chaîne de tâche avec les trois temps de réponse correspondants : $R_1$, $R_2$, $R_3$. 
    %Parmi ces instances, on remarque que 2 d'entre-elles sont très semblables, avec le même job de terminaison $ \tau{3,3} $.
    
	Intuitivement, l'échéance bout-en-bout représente alors la durée maximale acceptable pour qu'une donnée d'entrée de la chaîne ait un effet du côté de la sortie. Pour une fonctionnalité donnée on comprend bien que cette échéance doit être bornée, et qu'il faut donc des garanties pour que tout se passe bien bout-en-bout. À considérer l'échéance d'une chaîne de tâche $D$, pour éviter toute faute temporelle de non-respect d'échéance, il faut a minima respecter~:  $\max_{j \in \mathbb{N}}\{R_j\} \leq D$.
	
	\smallbreak 
	
	L'objectif à présent est de proposer une approche qui permette justement d'exploiter ces contraintes bout-en-bout, de façon à éviter les risques de fautes temporelles au niveau fonctionnel. Cela se traduit par la volonté de prévenir les risques de dépassement d'échéances, non pas au niveau de chaque tâche individuelle mais plutôt à un niveau d'observation au-dessus qui est en lien direct avec la représentation fonctionnelle.
	
    
\section{Mécanisme d'anticipation par Surveillance et Contrôle}
    \subsection{Méthode d'anticipation}
    
    Je propose donc un mécanisme basé sur la surveillance à l'exécution de l'avancement d'une chaîne de tâche. 
    Pour ce faire, on introduit les notions d'\textbf{État de Chaîne de Tâche} et de \textbf{Trace d'Exécution de Chaîne de Tâche}. Une chaîne de tâche donnée est associée à un État et plusieurs Traces d'Exécutions. Ces deux éléments évoluant au fil de l'exécution du système, on peut noter $S_C(t)$ l'État et $ET_C(i,t)$ une Trace d'Exécution donnée de la Chaîne de tâches.
    
    On peut alors définir pour une Chaîne de Tâche dont la tâche d'entrée est $\tau_{1}$, et la tâche de sortie $\tau_{n}$ : 
    \begin{definition} \label{def:TraceExecutionChaine}
    	Une \textbf{Trace d'Exécution} $ET_C(i)$ se définit par un job d'entrée ainsi que tous les \textit{successeurs} itératifs de ce job. \\
    	\begin{equation*}
    		S_C(i) = \{ \tau_{1,i} \:,\: succ(\tau_{1,i})\:, \;\dots\; ,\: succ^n(\tau_{1,i}) = \tau_{n,i} \}
    	\end{equation*} \\    	
    	À un instant $t$, une Trace d'Exécution est dite \emph{active} si son job de départ $\tau_{1,i}$ a déjà été activé, et que son successeur itératif correspondant à la tâche de sortie de la chaîne $\tau_{n,i}$ n'a pas encore été terminé. Autrement, elle est \textit{inactive}. En d'autres termes : \\
    	\begin{equation*}
    		TE_C(i,t)  \quad \textrm{est active ssi}\quad  a_{1,i} \leq t \quad \textrm{et} \quad e_{n,i} > t
    	\end{equation*} 
    \end{definition}
    
    %Our anticipation mechanism is based on the run-time monitoring of the task chain progress. 
    %To that end, we introduce the notions of \textbf{Task Chain State} and \textbf{Task Chain Execution Trace} (TCET). 
    %A TCET contains an entry task job and all the iterative successors of that job. 
    %At a time $t$ a TCET can be \emph{active}, if its entry task job has been activated and if its exit task job has not yet ended, or \emph{inactive} otherwise. 
    
    À un moment $t$ de l'exécution du système, il est possible de définir à partir de l'état des Traces d'exécution (actives ou inactives) l'État de la Chaîne de tâches : 
    \begin{definition}[Etat d'une Chaîne de Tâche]\label{def:EtatChaine}
    	L'État d'une Chaîne de Tâche définie à un instant $t$ l'état d'avancement de l'exécution de la chaîne de tâche qui est toujours en cours et a été activée la plus anciennement. En d'autres termes : 
    	\begin{equation*}
    		S(t)=\langle t_0, \tau_i\rangle
    	\end{equation*}
    Avec $t_0$ la plus ancienne activation parmi les $ET_C(i,t)$ et $\tau_i$ la prochaine tâche de cette trace d'exécution qui n'a pas encore été exécutée.
    \end{definition}
    De cette façon, l'État d'une chaîne de tâches indique quelles sont les tâches restantes à exécuter dans la chaîne à un instant donné et son temps de réponse partiel actuel que l'on notera $RT(t) = t - t_0$.
    
    Pour finir, au regard de l'État d'une chaîne de tâches, on peut s'intéresser au temps restant à la complétion de cette chaîne. Il est possible d'estimer le temps qu'il faudra pour aller jusqu'à exécuter la tâche de sortie de la Trace d'Exécution active observée. Et si de plus cette estimation est faite dans la même logique qu'une estimation de Temps d'Exécution Pire Cas, on obtient alors une estimation de \textbf{\emph}{Pire Temps de Réponse restant} $rWCRT(t)$ à l'instant $t$.
    
    Alors, en combinant l'État à un instant $t$ avec une estimation du Temps de Réponse Pire Cas restant, on peut donc estimer une borne haute garantie de temps de réponse de la chaîne de tâche. C'est là que l'on peut faire entrer en jeu un mécanisme d'anticipation. 
    On dispose pour une chaîne de tâches donnée de son temps de réponse partiel actuel ainsi qu'une estimation de temps de réponse restant en pire cas. Il est alors possible de déterminer s'il y a un risque de défaillance par dépassement de l'échéance bout-en-bout $D$.
    \begin{theorem}[Risque de dépassement d'échéance]
    	Si à un instant donné $t$, l'inéquation suivante est respectée, alors il y a risque de dépassement d'échéance.
    	\begin{equation*} 
    		RT(t) + rWCRT(t) \geq D_C
    	\end{equation*}
    \end{theorem}

	Pour illustrer cette logique, on peut voir sur le chronogramme~\ref{fig:chronogram_rWCRT_example} à nouveau un exemple avec une chaîne de tâche $\tau_1 \rightarrow \tau_2 \rightarrow \tau_3$. À l'instant $t=18$ indiqué il y a deux Traces d'Exécution actives (chaînes reliées par une flèche de succession). On a de ce constat l'État de la chaîne $S_C(t) = \langle \tau_0, \tau_2\rangle = \langle 5, \tau_{2} \rangle $.
	On en déduit $ RT(18) = t - t_0 = 18-4 = 14 $. Si l'on ajoute à cela une estimation du Temps de Réponse Pire Cas restant $rWCRT(\tau_i)$, qui est le temps estimé pour que $\rightarrow \tau_2 \rightarrow \tau_3$ soit exécuté selon les contraintes de précédence, alors on a l'estimation du Pire Temps de Réponse : $ RT(18) + rWCRT(18) = 33$ que l'on peut comparer à la date d'échéance $ D_c = 30 $. Dans cet exemple, il existe donc un risque de dépassement de l'échéance.
	Il est à noter aussi que dans cet exemple l'instant $t$ a été pris en plein pendant l'exécution de la tâche $ \tau_2 $. Ce qui est pris en considération comme si cette dernière n'était pas exécutée. Si l'on ne prend pas non plus en compte son exécution partielle, c'est parce que d'un point de vue externe à cette tâche, sans l'instrumentaliser il n'est pas possible de le savoir. Et de fait, l'une de nos contraintes étant d'être le moins intrusif possible sur le code, notamment pour les cas où certains logiciels ne sont pas modifiables (black-box ou legacy). 
	
    \begin{figure}[ht]
		\centering 
		\includegraphics[width=0.7\linewidth]{chronogram_rWCRT_example.pdf}
		\caption{Exemple de trace d'exécution et Etat de la Chaîne pour anticipation}
		\label{fig:chronogram_rWCRT_example}
	\end{figure}

	À présent, il faut nous souvenir de notre objectif dans tout cela. Pourquoi vouloir anticiper une défaillance de la sorte ? Pour rappel l'objectif est d'anticiper un risque de dépassement d'échéance à l'échelle d'une chaîne de tâches, de façon à pouvoir passer d'un Mode d'ordonnancement Nominal vers un mode Dégradé dans lequel on va prévenir la défaillance. Cette dernière a pour origine première les interférences matérielles qui augmentent les temps d'exécution des tâches. Aussi pour éviter davantage ce rallongement et donc conserver la garantie de terminaison de la chaîne de tâche avant l'échéance, on remonte à la source en prévenant temporairement tout risque d'interférences. Et cela est obtenu via un mode dégradé dans lequel la méthode consiste à stopper temporairement les tâches non critiques, cause des interférences.
	Il nous faut donc prendre en compte la méthode de passage en mode dégradé. 
	

    
    \subsection{Passage en Mode Dégradé}
    
        \paragraph{Estimation de Temps d'Exécution Pire Cas restant}
    Bien évidemment, l'estimation du Temps de Réponse Pire Cas restant est un élément clé de l'approche. Tout l'intérêt de cette méthode réside dans la capacité à passer dans un mode dégradé. En conséquence, ce que nous nous devons de garantir, c'est le non-dépassement d'échéance sachant qu'il est possible à tout moment de prendre la décision du passage en mode dégradé, dans lequel les tâches critiques n'étant plus sujettes à interférences externes, auront un Temps d'Exécution Pire Cas bien plus faible qu'en mode Nominal. 
    Cela implique directement que si le $rWCRT(\tau_i)$ que nous considérons est dans le contexte Dégradé, alors la détection d'un éventuel risque se fait de façon beaucoup plus permissive que si l'on considère directement les risques en mode Nominal.
    
    Il existe plusieurs méthodes à son obtention. 
    De façon théorique, il est possible d'exploiter les méthodes déjà existantes d'estimation de temps d'exécution pire cas, auxquelles il faut ajouter la prise en compte des temps d'activation des tâches. Ce type d'approche devient hautement dépendante du système étudié, que ce soit l'architecture matérielle, mais surtout la politique d'ordonnancement des tâches, le type de tâches (périodique, sporadique, interruption)... De façon générale, la complexité des approches théoriques n'est pas négligeable et, il faut l'admettre, hors de notre cadre d'expertise. C'est d'autant plus vrai dans un cas d'application sur processeur multicœur pour lequel il est facile de tomber dans des estimations trop pessimistes. Pour cette raison, nous avons décidé dans notre proposition d'avoir une approche plus expérimentale.
    
    %To help with the estimation of $rWCRT$, we assume that the HI-criticality task chain execute on a single core. To avoid interference between the MCA and the task chain we prevent the MCA to use the same core. Lo-criticality tasks can execute on any core as depicted on \autoref{fig:SoftwareArchitecture}. 
    
    Un avantage dont nous bénéficions ici, c'est que l'hypothèse de se ramener à un cas en isolation dans le mode dégradé limite grandement les risques de variations sur les temps d'exécutions des tâches critiques. C'est ce qui permet une plus grande certitude sur les estimations expérimentales, qui ne nécessitent alors plus de couvrir toute une combinatoire incluant les tâches non critiques. L'estimation est donc faite expérimentalement en exécutant le système avec un passage forcé en mode dégradé dans lequel on peut alors mesurer sans les tâches non-critiques les Temps de Réponse Pire Cas restants $rWCRT(\tau_i)$. On notera que pour une chaîne de N tâches, il y a N-1 rWCRT à estimer. Plus de détails sur le protocole adopté pour l'estimation seront abordés en~\autoref{chap:4_ProtocolExpe}. 
    
    Il reste à aborder à présent la transition en elle-même vers le mode dégradé. Cette phase est importante du fait qu'elle implique des délais supplémentaires qui devront être pris en compte dans l'anticipation. De cette façon, à considérer que l'on recalcule périodiquement le risque de dépassement d'échéance, il est possible de définir l'instant où l'on sait que l'attente d'une période supplémentaire va faire que même en mode dégradé, il ne sera plus possible de garantir le respect de l'échéance bout-en-bout. Par conséquent, il devient alors clair que cet instant-là devient le dernier moment auquel il faut nécessairement passer en mode dégradé pour justement conserver cette garantie.
        
	\paragraph{Transition de Mode}
    Ce changement de mode se fait en 3 étapes. Il faut en premier lieu bien entendu la détection du point de bascule auquel le risque de défaillance est détecté, c'est l'étape de décision. Une fois la décision prise, la seconde étape est d'activer le mécanisme de passage en mode dégradé. Dans notre cas, il s'agit d'envoyer un signal aux tâches non critiques de façon à mettre en pause leur exécution, c'est l'étape de contrôle. Enfin, le système de surveillance doit continuer d'observer l'État de la chaîne de tâche de façon à relancer les tâches non critiques une fois le risque passé. Il s'agit de l'étape de recouvrement.
    
    Ces trois étapes font ressortir un élément important pour l'anticipation qui n'a pas été pris en compte pour le moment, et il s'agit de la durée entre l'étape de décision et la fin de l'étape d'exécution. En effet, le temps pour que toutes les tâches non critiques soient mises en pause est non nul, et ce délai doit être pris en compte dans l'estimation de Temps d'Exécution Pire Cas restant dans l'optique où le pire cas en question est considéré dans le mode dégradé.
    
	En conclusion, en considérant les grandeurs suivantes : 
	\begin{itemize}
		\item 	$ W_{MAX} $ La durée maximum garantie entre 2 points de surveillance de l'Etat de la chaîne de tâche
		\item 	$ t_{SW} $ Le délai maximum nécessaire à la transition du mode nominal au mode dégradé
		\item 	$ rWCRT(\tau_i) $ pour chaque $\tau_i$ de la chaîne de tâche, les Temps de Réponses Pire Cas restant en mode dégradé
	\end{itemize}
	Il est alors possible de calculer la somme du temps de réponse partiel actuel avec ces trois métriques. Tant que cette somme reste inférieure à l'échéance bout-en-bout, alors on peut conserver le système en mode nominal. En revanche, à partir du moment où cela dépasse l'échéance, alors c'est le moment où il n'est plus sûr de rester en mode nominal, et il faut donc déclencher le mode dégradé pour garantir le respect de l'échéance. Cela correspond en conséquence à surveiller que l'inéquation suivante reste vraie pour savoir l'instant critique auquel il faut passer en mode dégradé~:
	\begin{equation} \label{eq:safe_cond}
		RT(t) + rWCRT(\tau_i) + W_{max} + t_{SW} \leq D
	\end{equation} 
	Cette équation est notamment adaptée des travaux de~\cite{kritikakou_run-time_2014}. Chaque point de surveillance de l'État de la chaîne de tâches est considéré comme étant temporellement sûr (au sens où il n'y a pas de risque de faute temporelle due au dépassement d'échéance) tant que cette inégalité est respectée.
    
    \begin{proof}
		En présumant que (\ref{eq:safe_cond}) est respectée, on peut montrer qu'il est sûr d'attendre le prochain point de surveillance pour décider de changer de mode. Soit $t_{next}$ le prochain instant de surveillance.
		Par définition, $t_{next} \leq t + W_{max}$ alors nécessairement $RT(t_{next}) \leq RT(t) + W_{max}$, par conséquent  \\
		$RT(t_{next}) + rWCRT(\tau_i) + t_{SW} \leq RT(t) + rWCRT(\tau_i) + W_{max} + t_{SW}$. \\
		Aussi, $rWCRT()$ ne peut que décroître avec le temps qui s'écoule, par conséquent
		\[ rWCRT(t_{next}) \leq rWCRT(\tau_i)	\]
		et 
\[ RT(t_{next}) + rWCRT(t_{next}) + t_{SW} \leq RT(t) + rWCRT(\tau_i) + W_{max} + t_{SW} \] 
		Étant donné que (\ref{eq:safe_cond}) est respecté, on a $RT(t_{next}) + rWCRT(t_{next}) + t_{SW} \leq D$.
		De ce fait, il sera sûr de passer en mode dégradé au prochain point de surveillance.  
    \end{proof}
    
    \begin{figure}[ht]
    	\centering
    	\includegraphics[width=0.7\linewidth]{schemas/chronogram_rWCRT_complet}
    	\caption{}
    	\label{fig:rwcrtchronogram}
    \end{figure}

    En adaptant ce qui a été présenté précédemment avec le risque de dépassement d'échéance, on peut retrouver en~\autoref{fig:rwcrtchronogram} l'ajout des composants qui permettent l'anticipation du changement de mode, au regard de la période de surveillance et du temps de changement.
	La méthode de réglage de la période de surveillance sera discuté dans le~\autoref{chap:4_ProtocolExpe}.
	
	\section{Architecture Logicielle}
	
	Maintenant que nous avons présenté toute la logique derrière le mécanisme de surveillance et de contrôle, nous allons présenter plus en détail l'architecture logicielle nécessaire à son implémentation.
	
	Pour résumer la situation, nous sommes dans le cas d'un système implémenté sur un calculateur multicœur, dans lequel nous avons distingué une chaîne de tâche critique des autres tâches considérées non-critiques. Il faut bien entendu ajouter à cela un Agent de Surveillance et de Contrôle pour assurer la fonctionnalité de mitigation des fautes temporelles. Ce dernier se destine à être exécuté en couche bas niveau, au même niveau que la politique d'ordonnancement du système. Dans le cadre de la suite de nos expérimentations, pour simplifier l'implémentation, nous avons considéré l'isolation de l'Agent de Surveillance et Contrôle sur un cœur du processeur. L'ensemble de ces éléments se résume en~\autoref{fig:SoftwareArchitecture}.

\begin{figure}[ht]
            \centering
            \includegraphics[width=.7\linewidth]{ArchitectureLogicielle.pdf}
            \caption{Monitoring \& Control Agent basic concept} \label{fig:SoftwareArchitecture}
\end{figure}

	L'Agent de Surveillance et Contrôle peut se décomposer en deux éléments distincts. D'un part un \emph{Task Wrapper Component} (TWC) et de l'autre le \emph{Core Control Component} (CCC). Le TWC se destine à récupérer toutes les informations nécessaires à la surveillance et la mise à jour de l'État de la Chaîne de tâche, tandis que le CCC doit prendre en compte ces informations, de façon à réaliser la prise de décision du passage en mode dégradé au regard de l'inéquation~\ref{eq:safe_cond}.
	
	
        \begin{figure}[ht]
            \centering
            \includegraphics[width=0.7\linewidth]{ArchitectureFramework.pdf}
            \caption{Monitoring \& Control Agent Architecture\label{fig:architecture}}
        \end{figure}
        
        
        \subsection{Task Wrapper Component (TWC)} 
        
        L'objectif du Task Wrapper Component est de gérer l'aspect Surveillance du mécanisme. Il englobe l'instrumentalisation du logiciel de façon non intrusive pour obtenir les informations nécessaires à la Surveillance. C'est une différence assez importante avec bon nombre de travaux qui se basent sur la surveillance de l'exécution de tâches. Souvent ces dernières reposent sur une instrumentation logicielle au niveau du code des tâches. Cela offre un suivi plus fin, mais avec un coût de développement logiciel supplémentaire pour chaque tâche. De plus c'est par définition incompatible avec du logiciel fourni en boite noire.
        
        Le mécanisme de Surveillance et Contrôle nécessite \linebullets{une connaissance des date d'activation et de terminaison des jobs, ainsi que la connaissance de la structure de la chaîne de tâche}. Concernant le premier point, il faut donc ajouter des blocs de code exécutés directement avant le début d'exécution des tâches ainsi que juste avant leur terminaison. En récupérant les \textit{timestamps}\footnote{timestamp~: en informatique, le temps se décompte par un compteur a une fréquence donnée. Un timestamp correspond à un temps suivant ce compteur} d'exécution de ces blocs de code, il est ainsi possible de connaître précisément les dates de début $s_{i,j}$ et de fin $e_{i,j}$ de chaque job. Ils ont vocation à être envoyés au Core Control Component pour mettre à jour l'État de la chaîne de tâches ainsi que le suivi des Traces d'Exécution actives.
        
        Par ailleurs, une telle instrumentalisation permet d'ajouter une couche de sécurité supplémentaire. En effet, le changement de mode se fait initialement par envoi d'un signal pour stopper les tâches non critiques. Ceci étant dit, pour une meilleure réactivité et ajouter de la redondance dans l'arrêt des tâches, il est aussi possible d'utiliser le bloc logiciel exécuté avant l'exécution des tâches non critiques pour vérifier si la tâche en question a effectivement l'autorisation de s'exécuter ou bien si on est en mode dégradé, doit être stoppée. 
        
        En somme, le Task Wrapper Component doit encapsuler les tâches pour ajouter deux wrappers : un "Before" et un "After" qui vont avoir deux rôles : 
        \begin{itemize}
        	\item envoyer au Core Control Component les timestamps d'exécution associés à chaque job de tâche critique (dates de début et de fin),
        	\item ajouter de la redondance pour prévenir l'exécution des tâches non-critiques en mode dégradé et potentiellement accélérer le changement de mode.
        \end{itemize}
		On remarquera qu'il n'est pas utile d'exécuter un wrapper "After" à la suite des tâches non critiques étant donné que nous ne monitorons pas leur exécution dans le cadre du mécanisme.

        \subsection{Core Control Component (CCC)}
        
        Le Core Control Component gère donc l'aspect "Contrôle" du mécanisme. Il doit être exécuté périodiquement, tous les $T_{CCC}$ unités de temps. À chaque exécution, son rôle est de récupérer toutes les données de timestamps du TWC. En ayant connaissance de la structure de chaîne de tâche qui est exécutée, il peut alors reconstruire les Traces d'Exécution telles que définies précédemment (Définition~\ref{def:TraceExecutionChaine}). 
        
        À partir des Traces actives, l'État de la chaîne de tâche $S(t)$ est alors mis à jour. Il s'agit de calculer successivement : 
        \begin{itemize}
        	\item $RT(t)$, la durée à partir de la date de départ de la tâche d'entrée,
        	\item $rWCRT(\tau_i(t))$, le Temps de Réponse Pire Cas restant à prendre en compte selon l'état de la trace d'exécution active.
        \end{itemize} 
	    En y ajoutant les constantes $W_{MAX}$ et $t_{SW}$, l'inégalité~\ref{eq:safe_cond} est re-calculée.
        Dans le cas où elle est toujours respectée, rien ne se produit et le CCC peut attendre la prochaine période de vérification. Dans le cas où elle devient fausse, alors la procédure de changement de mode est enclenchée. %D'une part le TWC reçoit l'information de ce changement en mode dégradé, et d'autre part le signal est envoyé aux tâches non critiques pour être mises en pause.
                
        Il pouvait y avoir deux choix possibles d'activation du Core Control Component. Soit en le déclenchant de façon asynchrone, à chaque fois que la TWC reçoit une nouvelle donnée de timestamp, soit en le déclenchant de façon périodique indépendamment de l'exécution des tâches. Notre choix s'est donc porté sur la seconde option pour une raison fondamentale. En effet, bien qu'un déclenchement asynchrone soit plus simple à implémenter d'un point de vue technique, on perd la maîtrise de l'utilisation processeur de notre Agent de Surveillance et Contrôle, ce qui peut être dommageable pour la maîtrise de l'ordonnancement du système. De plus, dans le cas où une défaillance sur la réception des timestamps surviendrait, le CCC pourrait se retrouver dans une attente indéfinie de données, ne mettant alors pas à jour l'État de la chaîne de tâches... et laissant donc passer les risques de faute temporelle. Par conséquent, avec un déclenchement périodique, on obtient une décorrélation du reste du système, et donc quoi qu'il arrive il sera possible à intervalle fixe de déterminer s'il y a un risque. En effet, même en l'absence de nouveaux messages la valeur de $ET(t)$ continue d'augmenter. 
        
        
		\subsection{Définition des constantes de Contrôle}
        Il est important de fixer correctement les paramètres constants du CCC, qui sont $t_{SW}$ and $W_{MAX}$.
        Si ces paramètres sont sous estimés, typiquement dans le cas où le temps de passage en mode dégradé est plus long, alors l'inégalité~\ref{eq:safe_cond} pourra devenir tout bonnement fausse. 
        Concernant la durée maximale entre 2 mises à jour de l'État de la chaîne de tâche $W_{MAX}$, la valeur est assez simple à obtenir. Du fait de la périodicité de la tâche ainsi que son très haut niveau de priorité, associé à une demande en ressource très réduit, il suffit de prendre la durée de la période fixée $T_{CCC}$. Là encore on réalise qu'avec une activation sporadique à l'arrivée des messages, cette durée aurait été plus complexe à fixer. 
        
        Cependant, cela repose donc sur le choix sur la fréquence d'exécution du CCC. Ce choix peut avoir plusieurs conséquences et va donc être sujet à de potentiels compromis. D'une part, un $T_{CCC}$ trop petit signifie une plus forte utilisation des ressources de calcul, ce qui va directement à l'encontre de notre objectif d'optimiser la puissance de calcul. C'est d'ailleurs une problématique courante des mécanismes de Surveillance, d'avoir un impact négatif sur l'usage des ressources du processeur et s'avérer contre-productif. \\
        A l'inverse, plus la période va être longue, plus le mécanisme d'anticipation sera sensible et anticipera des risques de plus en plus improbables. De fait, plus la prochaine date de vérification de l'État de la chaîne est éloigné dans le temps, plus le moindre glissement de temps d'exécution des tâches critiques va laisser penser à un risque de dépassement d'échéance. Cela fait alors augmenter le taux de faux positifs, c'est-à-dire le taux de déclenchements du passage en mode dégradé qui n'étaient pas nécessaires. Par ailleurs, il faut que le Core Control Component puisse soutenir le débit d'arrivée de données d'exécution des tâches critiques, car il n'est pas possible de maintenir en liste d'attente une quantité infinie de messages. \\
        En conclusion, la valeur de $T_{CCC}$ doit être choisie au regard d'une estimation du débit d'exécution des jobs, donc dépendant du nombre de tâches et de leur période d'exécution. 
        

        \cmnt{
            One should note that what makes such approach possible is the evolution of the $rWCRT$ at run-time and as the $S(t)$ evolves. It would not be possible to apply such approach when it comes to monitor \& control individual tasks to guarantee their individual deadlines. For individual deadlines, our method would fit only if we are able to monitor tasks timing state ``inside" the tasks execution, i.e. instrumenting the tasks source code to add internal checkpoints. Such approach on individual tasks would discard by definition the use of black box software assumption for instance, and otherwise would need much higher refresh rate frequencies in order to follow individual tasks execution timing state. Such solution is presented for individual tasks in~\cite{kritikakou_dynascore_2017}.
            }
            
\section{Application au domaine  automobile (diag. fonctionnel, SWC, etc)}
    \subsection{Concept Description}
        Our approach presents a software execution \emph{Monitoring and Control Agent (MCA)} to guarantee end-to-end deadline constraints.
        %Our challenge is to investigate that an appropriate combination of general-purpose scheduling policies, tasks allocation to cores, freezing strategies of non-critical tasks when critical tasks need more computing resources is a solution to comply with end-to-end real-time constraints. %In practice deadline specifications are often arbitrary and approximate due to system complexity. 
        We focus on the respect of end-to-end constraints of tasks chains, not individual tasks constraints. The idea behind this is to offer more ``flexibility'' on tasks scheduling for guaranteeing mandatory task chains constraints if we control only end-to-end constraints instead of every critical task timing constraint. By doing so, we gain "flexibility" as we allow some parts of the chain to be behind time as they can be compensated before the end of the chain without any external action. The MCA monitors at run-time the execution time of critical tasks and anticipate when the end-to-end deadlines may be compromised to stop non-critical tasks when needed in order to avoid such risk. The anticipation is based on the estimation of remaining WCET. Finally, when the critical task chain recovers from the potential risk, the non-critical tasks can resume their execution to get back to a nominal state.
        
        We define a \textit{degraded mode}, opposed to the \textit{nominal mode} of execution. In nominal mode, critical and non-critical tasks are executed normally. In Degraded mode, non-critical tasks are not executed, to prevent further interferences on critical tasks. The degraded mode implies simpler WCET estimations because we eliminate the disturbances from non-critical tasks; such WCET will be lower than in a nominal mode. It is probably less pessimistic as we eliminate memory interferences, non-critical tasks scheduling and possible common resources (drivers for instance) usage. The main disturbances remaining will be only between the tasks from the chain. Consequently, our anticipation mechanism will be based on reduced estimation of WCET (compared to nominal mode), to activate degraded mode only as a last resort.
        \smallbreak
        To reach degraded mode, MCA role is to pause/stop non-critical tasks execution. This control is triggered by an anticipation algorithm. To be efficient, this algorithm should trigger the control at the latest possible time while guaranteeing real-time end-to-end constraints.
        
        %Individual deadline could be compromised however the goal is to respect task chains deadlines.
        %Individual WCET are also useful, but if they are not available, approximations extracted from behavioral models are enough, for example with an equal distribution of the global end-to-end time value as described in [ref].  
        \begin{figure}[h]
            \centering
            \includegraphics[width=\linewidth]{SchemaChaineFonctionnelle}
            \caption{Functional Architecture definition} \label{fig:funcArch}
        \end{figure}
        \smallbreak
        \subsubsection{Functional Specification} 
            A critical task chain must describe the implementation of a system functionality from its triggering to its consequence. This would stick most of the time with a computing chain going from a sensor measure to an actuator command. First idea would be to stick with safety criticality levels (ASIL D to ASIL A and QM, for automotive applications), but we quickly notice that there is no direct link between this classification and critical tasks chains. A safety critical task is not necessarily defined from its timing constraints. The only possible conclusion here is that a critical task chain only includes non-QM tasks. 
            \smallbreak
            We propose here a definition based around high-level specifications as represented in figure~\ref{fig:funcArch}. The global system is defined as a set of features\footnote{Features: all the services the system must provide. e.g: Lane Support System (LSS) is a feature.}. Every feature gathers a set of functionalities that are translated into Use Cases\footnote{e.g: Lane Departure Warning \& Lane Keeping Assist are part of the use cases of LSS feature.}. A Use Case defines a feature behavior for a given context and inputs (and the consequent outputs). Finally, those are translated into functional chains representing different functions and their interactions needed for the realization of the Use Case. 
            
            If we combine this information with a severity classification in case of failure of the use cases, it is possible to define critical chains as functional chains with a high severity risk. This is one possible criterion allowing an easy separation between a critical functional chain and the others. It could be adapted during the design phase, depending on the functional chains allocated to the processor. 
            \smallbreak
            Such information allows to define the software components involved in the critical task chain. All the software components used to realize a critical functional chain form a critical task chain at an OS point of view. At this point, it is possible to define the task chain end-to-end deadline, following the severity temporal risk in case of failure. Such deadline should be at minimum the sum of individual tasks deadline, but could probably be higher, depending on the global system and the task chain function. Our objective is to guarantee such critical task chain end-to-end execution time on the multicore.
\ifdefined\included
\else
\bibliographystyle{StyleThese}
\bibliography{these}
\end{document}
\fi

\ifdefined\included
\else
\documentclass[french, a4paper, 11pt, twoside, pdftex]{StyleThese}
\usepackage{iflang}
\usepackage{bibentry}



%\usepackage[sectionbib]{chapterbib}          % Cross-reference package (Natural BiB)
%\usepackage{natbib}                  % Put References at the end of each chapter
%\usepackage{bibunits}
% Do not put 'sectionbib' option here.
% Sectionbib option in 'natbib' will do.


\usepackage{fancyhdr}                    % Fancy Header and Footer

\usepackage[utf8]{inputenc}
\usepackage[T1]{fontenc}
\usepackage[french]{babel} %
\usepackage{lmodern} \normalfont %to load T1lmr.fd 
\DeclareFontShape{T1}{cmr}{b}{sc} { <-> ssub * cmr/bx/sc }{}
%\hyphenation{gar}

\usepackage{amsmath,amssymb}             % AMS Math
\usepackage{nicefrac}
\usepackage{siunitx}					%% Unites Math SI

\usepackage{blindtext}

\usepackage{datetime}

\usepackage{lipsum} 

\usepackage[inline]{enumitem}

\usepackage{hhline}
%\usepackage[left=1.5in,right=1.3in,top=1.1in,bottom=1.1in]{geometry}
\usepackage[left=1.5in,right=1.3in,top=1.1in,bottom=1.1in,includefoot,includehead,headheight=13.6pt]{geometry}

%%\renewcommand{\baselinestretch}{1.05}

%%%%%%%% Multi-figures avec sub-captions
\usepackage{caption}
\usepackage{subcaption}

% Table of contents for each chapter

\usepackage[nottoc, notlof, notlot]{tocbibind}
\usepackage[nohints]{minitoc}
\setcounter{minitocdepth}{2}
\mtcindent=15pt
% Use \minitoc where to put a table of contents

\usepackage{aecompl}

%% Package cosmetic meilleur layout du texte en jouant sur le spacing par caractères
\usepackage[activate={true,nocompatibility},final,tracking=true,kerning=true,factor=1100,stretch=10,shrink=10]{microtype}
\usepackage[absolute,overlay]{textpos} 
\setlength{\TPHorizModule}{\paperwidth}\setlength{\TPVertModule}{\paperheight}
\sloppy

%%%%%%%%%%% JOLIS TABLEAUX
\usepackage{tabularx}		%\usepackage{tabular}
\usepackage{multirow}
\newcommand{\mc}{\multicolumn} 
\newcommand{\mr}[2]{\multirow{#1}{*}{#2}} 	\newcommand{\mrQ}{\multirow{-4}{*}}
\usepackage{booktabs}

\usepackage[usenames,dvipsnames]{xcolor} 

\makeatletter
\newcommand{\ccolor}[3][]{%
	\kern-\fboxsep
	\if\relax\detokenize{#1}\relax
	\expandafter\@firstoftwo
	\else
	\expandafter\@secondoftwo
	\fi
	{\colorbox{#2}}%
	{\colorbox[#1]{#2}}%
	{#3}\kern-\fboxsep
}
\makeatother

%%%%% Insertion graphiques format PGF
\usepackage{pgfplots}
\pgfplotsset{width=\linewidth, compat=1.16}%, compat=1.17}
\usepackage{adjustbox}          %%% PERMET DE LES RECADRER + FACILEMENT


%%%%%%%%%% Bullets de listes sans saut de ligne %%%%%%%%%%
\usepackage{xparse}

\ExplSyntaxOn%
\seq_new:N \l_local_enum_seq

\newcommand{\storethestuff}[1]{%
  \seq_set_from_clist:Nn \l_local_enum_seq {#1}%
}

\newcommand{\dotheenumstuff}{%
\int_zero:N \l_tmpa_int
\seq_map_inline:Nn \l_local_enum_seq {%
    \int_incr:N \l_tmpa_int% Increase the counter
    \item ##1
    % Check whether the list has reached the end -- if so, use '.' instead of ','
    %\int_compare:nNnTF 
    % { \l_tmpa_int } < {\seq_count:N \l_local_enum_seq} 
    % {,} {.}
  }
}
\ExplSyntaxOff

\NewDocumentCommand{\linebullets}{+m}{%
  \storethestuff{#1}%
  \begin{enumerate*}[label={\alph*)},font={\bfseries},itemjoin={{, }}]
    \dotheenumstuff%
  \end{enumerate*}
}

\newcommand{\cmnt}[1]{}  %%%%% AJOUT DE COMMENTAIRE MULTILIGNES


%%%%%%%%%% ECRITURE CARACTERES DANS UN CERCLE %%%%%%%%%%
%\def\circleTxt[#1]{\raisebox{.5pt}{\textcircled{\raisebox{-1pt}{#1}}}}
\newcommand{\ctxt}[1]{\raisebox{.5pt}{\textcircled{\raisebox{-1.2pt}{#1}}}}
% Glossary / list of abbreviations

\usepackage[intoc]{nomencl}
\IfLanguageName{english}{%
\renewcommand{\nomname}{Glossary}
}{ %
\renewcommand{\nomname}{Liste des Abréviations}
}

\makenomenclature

% My pdf code

\usepackage{ifpdf}

\ifpdf
  \usepackage[pdftex]{graphicx}
  \DeclareGraphicsExtensions{.pdf,PDF,.png,PNG,.jpg,JPG}
  \usepackage[pagebackref,hyperindex=true]{hyperref} %% use \autoref{} instead of Table~\ref{}.
  \usepackage{tikz}
  \usetikzlibrary{arrows,shapes,calc}
\else
  \usepackage{graphicx}
  \DeclareGraphicsExtensions{.ps,.eps}
  \usepackage[a4paper,dvipdfm,pagebackref,hyperindex=true]{hyperref}
\fi

\graphicspath{{.}{schemas/}{graphiques/}{tables/}}

%% nicer backref links. NOTE: The flag ThesisInEnglish is used to define the
% language in the back references. Read more about it in These.tex

\IfLanguageName{english}{
\renewcommand*{\backref}[1]{}
\renewcommand*{\backrefalt}[4]{%
\ifcase #1 %
(Not cited.)%
\or
(Cited in page~#2.)%
\else
(Cited in pages~#2.)%
\fi}
\renewcommand*{\backrefsep}{, }
\renewcommand*{\backreftwosep}{ and~}
\renewcommand*{\backreflastsep}{ and~}
}{
\renewcommand*{\backref}[1]{}
\renewcommand*{\backrefalt}[4]{%
\ifcase #1 %
(Non cité.)%
\or
(Cité en page~#2.)%
\else
(Cité en pages~#2.)%
\fi}
\renewcommand*{\backrefsep}{, }
\renewcommand*{\backreftwosep}{ et~}
\renewcommand*{\backreflastsep}{ et~}
}

% Links in pdf
\usepackage{color}
\definecolor{linkcol}{rgb}{0,0,0.4} 
\definecolor{citecol}{rgb}{0.5,0,0} 
\definecolor{linkcol}{rgb}{0,0,0} 
\definecolor{citecol}{rgb}{0,0,0}
% Change this to change the informations included in the pdf file

\hypersetup
{
bookmarksopen=true,
pdftitle="Prévention des fautes temporelles sur architectures multicœur pour les systèmes à criticité mixte",
pdfauthor="Daniel LOCHE", %auteur du document
pdfsubject="Thèse", %sujet du document
%pdftoolbar=false, %barre d'outils non visible
pdfmenubar=true, %barre de menu visible
pdfhighlight=/O, %effet d'un clic sur un lien hypertexte
colorlinks=true, %couleurs sur les liens hypertextes
pdfpagemode=UseNone, %aucun mode de page
%pdfpagelayout=DoublePage, %ouverture en simple page
pdffitwindow=true, %pages ouvertes entierement dans toute la fenetre
linkcolor=linkcol, %couleur des liens hypertextes internes
citecolor=citecol, %couleur des liens pour les citations
urlcolor=linkcol %couleur des liens pour les url
}

% definitions.
% -------------------

\setcounter{secnumdepth}{3}
\setcounter{tocdepth}{2}

% Some useful commands and shortcut for maths:  partial derivative and stuff

\newcommand{\pd}[2]{\frac{\partial #1}{\partial #2}}
\def\abs{\operatorname{abs}}
\def\argmax{\operatornamewithlimits{arg\,max}}
\def\argmin{\operatornamewithlimits{arg\,min}}
\def\diag{\operatorname{Diag}}
\newcommand{\eqRef}[1]{(\ref{#1})}
\newcommand{\nline}{\smallbreak\noindent}

\usepackage{rotating}                    % Sideways of figures & tables

% \usepackage{txfonts}                     % Public Times New Roman text & math font
  
%%% Fancy Header %%%%%%%%%%%%%%%%%%%%%%%%%%%%%%%%%%%%%%%%%%%%%%%%%%%%%%%%%%%%%%%%%%
% Fancy Header Style Options

\pagestyle{fancy}                       % Sets fancy header and footer
\fancyfoot{}                            % Delete current footer settings

%\renewcommand{\chaptermark}[1]{         % Lower Case Chapter marker style
%  \markboth{\chaptername\ \thechapter.\ #1}}{}} %

%\renewcommand{\sectionmark}[1]{         % Lower case Section marker style
%  \markright{\thesection.\ #1}}         %

\fancyhead[LE,RO]{\bfseries\thepage}    % Page number (boldface) in left on even
% pages and right on odd pages
\fancyhead[RE]{\bfseries\nouppercase{\leftmark}}      % Chapter in the right on even pages
\fancyhead[LO]{\bfseries\nouppercase{\rightmark}}     % Section in the left on odd pages

\let\headruleORIG\headrule
\renewcommand{\headrule}{\color{black} \headruleORIG}
\renewcommand{\headrulewidth}{1.0pt}
\usepackage{colortbl}
\arrayrulecolor{black}

\fancypagestyle{plain}{
  \fancyhead{}
  \fancyfoot{}
  \renewcommand{\headrulewidth}{0pt} %%%%%%%%%%%%%%%%%%%%%%%%%%%%%%%%%%%%%%%%%%%%%%%%%%%%%%%%%%%%%%%%%%%%%%%%%%%%%%%%%%%%%
}

%\usepackage{MyAlgorithm}
%\usepackage[noend]{MyAlgorithmic}
%\usepackage[ED=EDSYS-SystEmb, Ets=INP]{tlsflyleaf}

%%% Clear Header %%%%%%%%%%%%%%%%%%%%%%%%%%%%%%%%%%%%%%%%%%%%%%%%%%%%%%%%%%%%%%%%%%
% Clear Header Style on the Last Empty Odd pages
\makeatletter

\def\cleardoublepage{\clearpage\if@twoside \ifodd\c@page\else%
  \hbox{}%
  \thispagestyle{empty}%              % Empty header styles
  \newpage%
  \if@twocolumn\hbox{}\newpage\fi\fi\fi}

\makeatother
 
%%%%%%%%%%%%%%%%%%%%%%%%%%%%%%%%%%%%%%%%%%%%%%%%%%%%%%%%%%%%%%%%%%%%%%%%%%%%%%% 
% Prints your review date and 'Draft Version' (From Josullvn, CS, CMU)
\newcommand{\reviewtimetoday}[2]{\special{!userdict begin
    /bop-hook{gsave 20 710 translate 45 rotate 0.8 setgray
      /Times-Roman findfont 12 scalefont setfont 0 0   moveto (#1) show
      0 -12 moveto (#2) show grestore}def end}}
% You can turn on or off this option.
% \reviewtimetoday{\today}{Draft Version}
%%%%%%%%%%%%%%%%%%%%%%%%%%%%%%%%%%%%%%%%%%%%%%%%%%%%%%%%%%%%%%%%%%%%%%%%%%%%%%% 

\newenvironment{maxime}[1]
{
	\def\Arg{#1}
\vspace*{0cm}
\hfill
\begin{minipage}{0.6\textwidth}%
%\rule[0.5ex]{\textwidth}{0.1mm}\\%
\hrulefill $\:$ \\%$\:$ {\bf #1}\\
%\vspace*{-0.25cm}
\it 
}%
{%
	
\hrulefill $\:$ {\bf \Arg}
\vspace*{0.5cm}%
\end{minipage}
}

\let\minitocORIG\minitoc
\renewcommand{\minitoc}{\minitocORIG \vspace{1.5em}}

%\usepackage{slashbox}

\newenvironment{bulletList}%
{ \begin{list}%
	{$\bullet$}%
	{\setlength{\labelwidth}{25pt}%
	 \setlength{\leftmargin}{30pt}%
	 \setlength{\itemsep}{\parsep}}}%
{ \end{list} }


%%%%%%% Outils pour \comment \alert \add %%%%%
\usepackage{easyReview}
\usepackage{soulutf8} % for accented letters

\let\newalert\alert
\renewcommand{\alert}[1]{\textit{\newalert{#1}}}

%\usepackage[commandnameprefix=ifneeded]{changes} %% \chhighlight and \chcomment to avoid collision with easyReview
\renewcommand{\epsilon}{\varepsilon}

% centered page environment

\newenvironment{vcenterpage}
{\newpage\vspace*{\fill}\thispagestyle{empty}\renewcommand{\headrulewidth}{0pt}}
{\vspace*{\fill}}

\usepackage{tablefootnote}

%%%%%% MISE EN FORME CADRES DEFINITIONS/THEOREMES/LEMES %%%%%%%%%%
\usepackage{amsthm}  % for theoremstyle

\theoremstyle{plain} 
\newtheorem{theorem}{Théorème}[section]
\newtheorem{corollary}{Corolaire}[theorem]

%\theoremstyle{lemma}
%\newtheorem{lemma}[theorem]{Lemme}


\theoremstyle{definition}
\newtheorem{definition}[theorem]{Définition}


\cmnt{
	\usepackage{ntheorem} %\usepackage{amsthm}  % for theoremstyle
	%\usepackage{mdframed}
	\usepackage[most]{tcolorbox}
	
	\theoremstyle{plain} 
	\theoremindent20pt
	\theoremheaderfont{\normalfont\bfseries\hspace{-\theoremindent}}
	\newtheorem{theorem}{Théorème}[section]
	\newtheorem{corollary}{Corolaire}[theorem]
	
	\theoremstyle{plain}
	\newtheorem{lemma}[theorem]{Lemme}
	
	
	\tcolorboxenvironment{theorem}{
		blanker,
		breakable,
		before skip=\topsep,
		after skip=\topsep,
		borderline west={1pt}{10pt}{double, shorten <=12pt}
	}
	
	\theorembodyfont{\normalfont}
	\theoremindent20pt
	\theoremheaderfont{\normalfont\bfseries\hspace{-\theoremindent}}
	\newtheorem{definition}[theorem]{Définition}
	
	
	\tcolorboxenvironment{definition}{
		blanker,
		breakable,
		before skip=\topsep,
		after skip=\topsep,
		borderline west={1pt}{10pt}{shorten <=12pt}
	}
}

\cmnt{ 
	\begin{theorem}
		Ceci est un Théorème.
	\end{theorem} 
	
	\begin{corollary}
		Ceci est un Corollaire.
	\end{corollary}
	
	\begin{definition}
		Ceci est une Définition.
	\end{definition}
	
	\begin{lemma}
		Ceci est un Lemme.
	\end{lemma}
}

\def\UrlBigBreaks{\do\/\do-\do:}
\usepackage{url}

\sloppy
\begin{document}
\setcounter{chapter}{3} %% Numéro du chapitre précédent ;)
\dominitoc
\faketableofcontents
\fi

\chapter{Protocole et démarche expérimentale}
\minitoc

%%%% CONTENU %%%%
\section{Principe Général et Objectifs}
        We present in this section the experimental protocol proposed to characterise the system tasks (the ``workload") and calibrate the Monitoring and Control agent. The experimental protocol is divided in 7 steps separated in 3 phases~: \begin{enumerate*} \item Design phase, \item Calibration phase, and \item Run-time validation phase\end{enumerate*} as resumed in \autoref{tab:ExpeResume}.
        
        \begin{table}
            \centering
            \caption{Experimental Flowchart}
            \label{tab:ExpeResume}
            \includegraphics[width=\linewidth]{TableauRecapExperimentations.pdf}
        \end{table}
       
        The experimental steps are incremental following two inputs, final step being the complete system with task chain monitoring and control. The first input is the functional load under test that is executed on the real-time framework. It can be either~: 
        \begin{enumerate}
            \item single task~: a specific task from the workload is executed~;
            \item task chain~: a specific task chain, made of multiple tasks, is executed and monitored~;
            \item Task Chain with Monitoring \& Control mechanism.
        \end{enumerate}
        
        
        Second input is the system load executed along with the functional load to influence its execution. It can be either~: 
        \begin{enumerate}
            \item none, to test an isolated functional load~;
            \item forced stress~: strong cache, memory and CPU stress from linux \texttt{Stress-ng} tool set~;
            \item real-time tasks~: the LO-criticality tasks of the workload are executed.
        \end{enumerate} 
        \smallbreak
        Those inputs results in a two entry table as shown in \autoref{tab:ExpeResume} with each box corresponding to a step in the protocol.

\section{Phase de Design}
            This phase is needed if the workload involved don't come with a detailed specification, including their behavior and execution times. This is the case of our experiments as we select tasks from an already existing benchmark and we have no information about tasks execution times or even their compatibility with our real-time environment. Thus this phase is to characterise the available task set and define the workload specifications. It will be split into the HI-criticality task chain and LO-criticality tasks with their characteristics (min/avg/max execution time, periodicity...). This phase is defined by steps \circleTxt[1], \circleTxt[2], \circleTxt[3] in \autoref{tab:ExpeResume}.
            
    \subsection{Profil des tâches en isolation}
                    First objective is to get a global idea of tasks execution time profiles. One experiment is made per task, the task being executed individually with the framework. The task is called periodically with a given input, and task response times are logged.
                    
    \subsection{Profil des tâches avec stress imposé}
                    We add to the precedent step an artificial system load to cause high stress on cache, memory, I/O and computing use while the tasks are executed one by one. The output is a table with a profile for each task made of the min/average/max execution times and system metrics (system calls, context switches, scheduling interrupts, eventual period misses...). Such profile allow to categorise the tasks following their sensitivity to interferences compared to previous step \circleTxt[1]. This allows to define which tasks can be used for the HI-criticality task chain or as stressing LO-criticality tasks but also discard any task that would not fit our needs.
                    
    \subsection{Chaine de tâches avec système complet sans Contrôle}
                	Previous step classified the task set between HI and LO-criticality tasks. We define on this step the specific task chain and LO tasks that will be studied next and verify the pertinence of such choice. We check the workload schedulability in the soft real-time sense (i.e. schedulable if deadlines tardiness are bounded by a reasonably small constant). We also measure the task chain response time profile under ``realistic" conditions without the Control mechanism enabled. Expected result is a schedulable system with reference task chain response times with interferences.
                	
                	
\section{Phase de Calibration}
            This phase is mandatory to configure the Control mechanism to the software and hardware specificities and lower false-positive rate. It is made of steps \circleTxt[4], \circleTxt[5], \circleTxt[6] in orange boxes of \autoref{tab:ExpeResume}. Configuration includes task chain worst-case response time and intermediary response times in isolation. Performance optimisation consist in tweaking the switch time $t_{sw}$ and anticipation execution frequency $W_{max}$ constants, in the objective of lowering false-positive anticipation rates.
            
    \subsection{Chaine de tâches avec stress forcé}
                    The task chain is then tested under a worst-case scenario. It is executed with the artificial system load, to stress as much as possible the task chain similarly to step \circleTxt[2]. We get a baseline of the worst-case chain response time. This value is important because if the end-to-end deadline is always greater than the worst-case response time observed then the mechanism would be of no use (i.e. deadline never broken from temporal faults). This step gives a quantification of the task chain sensitivity to interferences and thus indicates the pertinence of using a Monitoring and Control Agent to manage them.
                    
    \subsection{Chaine de tâche en isolation}
                    The objective is to calibrate Control mechanism parameters~: $rWCRT(\tau_i)$, Core Control Component period ($T_{CCC}$) and switch time ($t_{sw}$) to degraded mode. The task chain is executed alone with the MCA but with the Control mode switch disabled. We log every chain intermediary and end-to-end response times. The result gives the data of all the remaining response times obtained during the test. We set the $rWCRT(\tau_i)$ parameters as an upper limits of the remaining response times registered.
                    
    \subsection{Chaine de tâche avec mécanisme de Contrôle}
                	Finally, the Control mechanism is enabled, with the parameters set on previous step. As this step does not include the LO tasks that bring interferences to the task chain, the Core Control Component should not trigger any switch to degraded mode. This step is important for the final analysis as it already points out the base false positive rate obtained with chosen parameters. A qualitative MCA should have the least degraded mode switch possible. Otherwise it could mean that either the CCC parameters are not ideally set (typically $W_{max}$), or the expected timing delays caused from interferences are too close to the usual timing variation of the task chain execution even in isolation. In other words, the Control Component is not able to differentiate response time variations due to temporal faults from ones due to nominal execution time variations. Another possibility is the end-to-end deadline requirement is too close to the nominal end-to-end response time in isolation. %If this has no effect then it could just mean that regarding step \circleTxt[4], the task chain is not subject to timing faults.
                	
                	
\section{Phase de Validation en exécution}
    \subsection{Chaine de tâches avec système complet et mécanisme de Contrôle}
                    The validation phase implies a last step (\circleTxt[7] in green box of \autoref{tab:ExpeResume}), which is with the whole final system being executed~: HI task chain and LO tasks with the MCA enabled. The objective is to collect the concluding information on the Monitoring and Control Agent behavior to measure the 3 quantification criteria (efficiency, performance and quality) of the solution explained in \autoref{MCAcriteria}. We also use the data from steps \circleTxt[3] and \circleTxt[6] as a reference for the conclusions.
\ifdefined\included
\else
\bibliographystyle{StyleThese}
\bibliography{these}
\end{document}
\fi


\ifdefined\included
\else
\documentclass[a4paper,11pt,twoside]{StyleThese}
\usepackage{iflang}
\usepackage{bibentry}



%\usepackage[sectionbib]{chapterbib}          % Cross-reference package (Natural BiB)
%\usepackage{natbib}                  % Put References at the end of each chapter
%\usepackage{bibunits}
% Do not put 'sectionbib' option here.
% Sectionbib option in 'natbib' will do.


\usepackage{fancyhdr}                    % Fancy Header and Footer

\usepackage[utf8]{inputenc}
\usepackage[T1]{fontenc}
\usepackage[french]{babel} %
\usepackage{lmodern} \normalfont %to load T1lmr.fd 
\DeclareFontShape{T1}{cmr}{b}{sc} { <-> ssub * cmr/bx/sc }{}
%\hyphenation{gar}

\usepackage{amsmath,amssymb}             % AMS Math
\usepackage{nicefrac}
\usepackage{siunitx}					%% Unites Math SI

\usepackage{blindtext}

\usepackage{datetime}

\usepackage{lipsum} 

\usepackage[inline]{enumitem}

\usepackage{hhline}
%\usepackage[left=1.5in,right=1.3in,top=1.1in,bottom=1.1in]{geometry}
\usepackage[left=1.5in,right=1.3in,top=1.1in,bottom=1.1in,includefoot,includehead,headheight=13.6pt]{geometry}

%%\renewcommand{\baselinestretch}{1.05}

%%%%%%%% Multi-figures avec sub-captions
\usepackage{caption}
\usepackage{subcaption}

% Table of contents for each chapter

\usepackage[nottoc, notlof, notlot]{tocbibind}
\usepackage[nohints]{minitoc}
\setcounter{minitocdepth}{2}
\mtcindent=15pt
% Use \minitoc where to put a table of contents

\usepackage{aecompl}

%% Package cosmetic meilleur layout du texte en jouant sur le spacing par caractères
\usepackage[activate={true,nocompatibility},final,tracking=true,kerning=true,factor=1100,stretch=10,shrink=10]{microtype}
\usepackage[absolute,overlay]{textpos} 
\setlength{\TPHorizModule}{\paperwidth}\setlength{\TPVertModule}{\paperheight}
\sloppy

%%%%%%%%%%% JOLIS TABLEAUX
\usepackage{tabularx}		%\usepackage{tabular}
\usepackage{multirow}
\newcommand{\mc}{\multicolumn} 
\newcommand{\mr}[2]{\multirow{#1}{*}{#2}} 	\newcommand{\mrQ}{\multirow{-4}{*}}
\usepackage{booktabs}

\usepackage[usenames,dvipsnames]{xcolor} 

\makeatletter
\newcommand{\ccolor}[3][]{%
	\kern-\fboxsep
	\if\relax\detokenize{#1}\relax
	\expandafter\@firstoftwo
	\else
	\expandafter\@secondoftwo
	\fi
	{\colorbox{#2}}%
	{\colorbox[#1]{#2}}%
	{#3}\kern-\fboxsep
}
\makeatother

%%%%% Insertion graphiques format PGF
\usepackage{pgfplots}
\pgfplotsset{width=\linewidth, compat=1.16}%, compat=1.17}
\usepackage{adjustbox}          %%% PERMET DE LES RECADRER + FACILEMENT


%%%%%%%%%% Bullets de listes sans saut de ligne %%%%%%%%%%
\usepackage{xparse}

\ExplSyntaxOn%
\seq_new:N \l_local_enum_seq

\newcommand{\storethestuff}[1]{%
  \seq_set_from_clist:Nn \l_local_enum_seq {#1}%
}

\newcommand{\dotheenumstuff}{%
\int_zero:N \l_tmpa_int
\seq_map_inline:Nn \l_local_enum_seq {%
    \int_incr:N \l_tmpa_int% Increase the counter
    \item ##1
    % Check whether the list has reached the end -- if so, use '.' instead of ','
    %\int_compare:nNnTF 
    % { \l_tmpa_int } < {\seq_count:N \l_local_enum_seq} 
    % {,} {.}
  }
}
\ExplSyntaxOff

\NewDocumentCommand{\linebullets}{+m}{%
  \storethestuff{#1}%
  \begin{enumerate*}[label={\alph*)},font={\bfseries},itemjoin={{, }}]
    \dotheenumstuff%
  \end{enumerate*}
}

\newcommand{\cmnt}[1]{}  %%%%% AJOUT DE COMMENTAIRE MULTILIGNES


%%%%%%%%%% ECRITURE CARACTERES DANS UN CERCLE %%%%%%%%%%
%\def\circleTxt[#1]{\raisebox{.5pt}{\textcircled{\raisebox{-1pt}{#1}}}}
\newcommand{\ctxt}[1]{\raisebox{.5pt}{\textcircled{\raisebox{-1.2pt}{#1}}}}
% Glossary / list of abbreviations

\usepackage[intoc]{nomencl}
\IfLanguageName{english}{%
\renewcommand{\nomname}{Glossary}
}{ %
\renewcommand{\nomname}{Liste des Abréviations}
}

\makenomenclature

% My pdf code

\usepackage{ifpdf}

\ifpdf
  \usepackage[pdftex]{graphicx}
  \DeclareGraphicsExtensions{.pdf,PDF,.png,PNG,.jpg,JPG}
  \usepackage[pagebackref,hyperindex=true]{hyperref} %% use \autoref{} instead of Table~\ref{}.
  \usepackage{tikz}
  \usetikzlibrary{arrows,shapes,calc}
\else
  \usepackage{graphicx}
  \DeclareGraphicsExtensions{.ps,.eps}
  \usepackage[a4paper,dvipdfm,pagebackref,hyperindex=true]{hyperref}
\fi

\graphicspath{{.}{schemas/}{graphiques/}{tables/}}

%% nicer backref links. NOTE: The flag ThesisInEnglish is used to define the
% language in the back references. Read more about it in These.tex

\IfLanguageName{english}{
\renewcommand*{\backref}[1]{}
\renewcommand*{\backrefalt}[4]{%
\ifcase #1 %
(Not cited.)%
\or
(Cited in page~#2.)%
\else
(Cited in pages~#2.)%
\fi}
\renewcommand*{\backrefsep}{, }
\renewcommand*{\backreftwosep}{ and~}
\renewcommand*{\backreflastsep}{ and~}
}{
\renewcommand*{\backref}[1]{}
\renewcommand*{\backrefalt}[4]{%
\ifcase #1 %
(Non cité.)%
\or
(Cité en page~#2.)%
\else
(Cité en pages~#2.)%
\fi}
\renewcommand*{\backrefsep}{, }
\renewcommand*{\backreftwosep}{ et~}
\renewcommand*{\backreflastsep}{ et~}
}

% Links in pdf
\usepackage{color}
\definecolor{linkcol}{rgb}{0,0,0.4} 
\definecolor{citecol}{rgb}{0.5,0,0} 
\definecolor{linkcol}{rgb}{0,0,0} 
\definecolor{citecol}{rgb}{0,0,0}
% Change this to change the informations included in the pdf file

\hypersetup
{
bookmarksopen=true,
pdftitle="Prévention des fautes temporelles sur architectures multicœur pour les systèmes à criticité mixte",
pdfauthor="Daniel LOCHE", %auteur du document
pdfsubject="Thèse", %sujet du document
%pdftoolbar=false, %barre d'outils non visible
pdfmenubar=true, %barre de menu visible
pdfhighlight=/O, %effet d'un clic sur un lien hypertexte
colorlinks=true, %couleurs sur les liens hypertextes
pdfpagemode=UseNone, %aucun mode de page
%pdfpagelayout=DoublePage, %ouverture en simple page
pdffitwindow=true, %pages ouvertes entierement dans toute la fenetre
linkcolor=linkcol, %couleur des liens hypertextes internes
citecolor=citecol, %couleur des liens pour les citations
urlcolor=linkcol %couleur des liens pour les url
}

% definitions.
% -------------------

\setcounter{secnumdepth}{3}
\setcounter{tocdepth}{2}

% Some useful commands and shortcut for maths:  partial derivative and stuff

\newcommand{\pd}[2]{\frac{\partial #1}{\partial #2}}
\def\abs{\operatorname{abs}}
\def\argmax{\operatornamewithlimits{arg\,max}}
\def\argmin{\operatornamewithlimits{arg\,min}}
\def\diag{\operatorname{Diag}}
\newcommand{\eqRef}[1]{(\ref{#1})}
\newcommand{\nline}{\smallbreak\noindent}

\usepackage{rotating}                    % Sideways of figures & tables

% \usepackage{txfonts}                     % Public Times New Roman text & math font
  
%%% Fancy Header %%%%%%%%%%%%%%%%%%%%%%%%%%%%%%%%%%%%%%%%%%%%%%%%%%%%%%%%%%%%%%%%%%
% Fancy Header Style Options

\pagestyle{fancy}                       % Sets fancy header and footer
\fancyfoot{}                            % Delete current footer settings

%\renewcommand{\chaptermark}[1]{         % Lower Case Chapter marker style
%  \markboth{\chaptername\ \thechapter.\ #1}}{}} %

%\renewcommand{\sectionmark}[1]{         % Lower case Section marker style
%  \markright{\thesection.\ #1}}         %

\fancyhead[LE,RO]{\bfseries\thepage}    % Page number (boldface) in left on even
% pages and right on odd pages
\fancyhead[RE]{\bfseries\nouppercase{\leftmark}}      % Chapter in the right on even pages
\fancyhead[LO]{\bfseries\nouppercase{\rightmark}}     % Section in the left on odd pages

\let\headruleORIG\headrule
\renewcommand{\headrule}{\color{black} \headruleORIG}
\renewcommand{\headrulewidth}{1.0pt}
\usepackage{colortbl}
\arrayrulecolor{black}

\fancypagestyle{plain}{
  \fancyhead{}
  \fancyfoot{}
  \renewcommand{\headrulewidth}{0pt} %%%%%%%%%%%%%%%%%%%%%%%%%%%%%%%%%%%%%%%%%%%%%%%%%%%%%%%%%%%%%%%%%%%%%%%%%%%%%%%%%%%%%
}

%\usepackage{MyAlgorithm}
%\usepackage[noend]{MyAlgorithmic}
%\usepackage[ED=EDSYS-SystEmb, Ets=INP]{tlsflyleaf}

%%% Clear Header %%%%%%%%%%%%%%%%%%%%%%%%%%%%%%%%%%%%%%%%%%%%%%%%%%%%%%%%%%%%%%%%%%
% Clear Header Style on the Last Empty Odd pages
\makeatletter

\def\cleardoublepage{\clearpage\if@twoside \ifodd\c@page\else%
  \hbox{}%
  \thispagestyle{empty}%              % Empty header styles
  \newpage%
  \if@twocolumn\hbox{}\newpage\fi\fi\fi}

\makeatother
 
%%%%%%%%%%%%%%%%%%%%%%%%%%%%%%%%%%%%%%%%%%%%%%%%%%%%%%%%%%%%%%%%%%%%%%%%%%%%%%% 
% Prints your review date and 'Draft Version' (From Josullvn, CS, CMU)
\newcommand{\reviewtimetoday}[2]{\special{!userdict begin
    /bop-hook{gsave 20 710 translate 45 rotate 0.8 setgray
      /Times-Roman findfont 12 scalefont setfont 0 0   moveto (#1) show
      0 -12 moveto (#2) show grestore}def end}}
% You can turn on or off this option.
% \reviewtimetoday{\today}{Draft Version}
%%%%%%%%%%%%%%%%%%%%%%%%%%%%%%%%%%%%%%%%%%%%%%%%%%%%%%%%%%%%%%%%%%%%%%%%%%%%%%% 

\newenvironment{maxime}[1]
{
	\def\Arg{#1}
\vspace*{0cm}
\hfill
\begin{minipage}{0.6\textwidth}%
%\rule[0.5ex]{\textwidth}{0.1mm}\\%
\hrulefill $\:$ \\%$\:$ {\bf #1}\\
%\vspace*{-0.25cm}
\it 
}%
{%
	
\hrulefill $\:$ {\bf \Arg}
\vspace*{0.5cm}%
\end{minipage}
}

\let\minitocORIG\minitoc
\renewcommand{\minitoc}{\minitocORIG \vspace{1.5em}}

%\usepackage{slashbox}

\newenvironment{bulletList}%
{ \begin{list}%
	{$\bullet$}%
	{\setlength{\labelwidth}{25pt}%
	 \setlength{\leftmargin}{30pt}%
	 \setlength{\itemsep}{\parsep}}}%
{ \end{list} }


%%%%%%% Outils pour \comment \alert \add %%%%%
\usepackage{easyReview}
\usepackage{soulutf8} % for accented letters

\let\newalert\alert
\renewcommand{\alert}[1]{\textit{\newalert{#1}}}

%\usepackage[commandnameprefix=ifneeded]{changes} %% \chhighlight and \chcomment to avoid collision with easyReview
\renewcommand{\epsilon}{\varepsilon}

% centered page environment

\newenvironment{vcenterpage}
{\newpage\vspace*{\fill}\thispagestyle{empty}\renewcommand{\headrulewidth}{0pt}}
{\vspace*{\fill}}

\usepackage{tablefootnote}

%%%%%% MISE EN FORME CADRES DEFINITIONS/THEOREMES/LEMES %%%%%%%%%%
\usepackage{amsthm}  % for theoremstyle

\theoremstyle{plain} 
\newtheorem{theorem}{Théorème}[section]
\newtheorem{corollary}{Corolaire}[theorem]

%\theoremstyle{lemma}
%\newtheorem{lemma}[theorem]{Lemme}


\theoremstyle{definition}
\newtheorem{definition}[theorem]{Définition}


\cmnt{
	\usepackage{ntheorem} %\usepackage{amsthm}  % for theoremstyle
	%\usepackage{mdframed}
	\usepackage[most]{tcolorbox}
	
	\theoremstyle{plain} 
	\theoremindent20pt
	\theoremheaderfont{\normalfont\bfseries\hspace{-\theoremindent}}
	\newtheorem{theorem}{Théorème}[section]
	\newtheorem{corollary}{Corolaire}[theorem]
	
	\theoremstyle{plain}
	\newtheorem{lemma}[theorem]{Lemme}
	
	
	\tcolorboxenvironment{theorem}{
		blanker,
		breakable,
		before skip=\topsep,
		after skip=\topsep,
		borderline west={1pt}{10pt}{double, shorten <=12pt}
	}
	
	\theorembodyfont{\normalfont}
	\theoremindent20pt
	\theoremheaderfont{\normalfont\bfseries\hspace{-\theoremindent}}
	\newtheorem{definition}[theorem]{Définition}
	
	
	\tcolorboxenvironment{definition}{
		blanker,
		breakable,
		before skip=\topsep,
		after skip=\topsep,
		borderline west={1pt}{10pt}{shorten <=12pt}
	}
}

\cmnt{ 
	\begin{theorem}
		Ceci est un Théorème.
	\end{theorem} 
	
	\begin{corollary}
		Ceci est un Corollaire.
	\end{corollary}
	
	\begin{definition}
		Ceci est une Définition.
	\end{definition}
	
	\begin{lemma}
		Ceci est un Lemme.
	\end{lemma}
}

\def\UrlBigBreaks{\do\/\do-\do:}
\usepackage{url}

\sloppy



\begin{document}
\setcounter{chapter}{6} %% Numéro du chapitre précédent ;)
\dominitoc
\fi

\chapter*{Conclusion et Perspectives} \label{chap:Conclusion}
\chaptermark{Conclusion et Perspectives}
\addstarredchapter{Conclusion} %Sinon cela n'apparait pas dans la table des matières

\section*{Conclusion}
\markboth{Conclusion et Perspectives}{Conclusion}
\addcontentsline{toc}{section}{Conclusion}

Au fil de cette thèse, il nous a été donné de faire un état des lieux sur l'évolution des systèmes cyberphysiques, notamment dans le domaine de l'automobile. Ces évolutions se sont pour une grande part orientées vers le choix d'architectures matérielles multicœurs de plus en plus puissantes, mais aussi de plus en plus complexes. 
Avec la cohabitation de logiciels à différents niveaux de criticité sur ces plateformes, cela donne lieu à des enjeux émergents sur le respect des contraintes de sûreté de fonctionnement dans les systèmes temps-réel.

C'est donc en partant de ce constat que nous avons établi un état des lieux à la fois des enjeux et contraintes industrielles et des travaux afférents dans le domaine académique sur l'intégration de systèmes à criticité mixte.
Les enjeux sont multiples et les solutions pour y répondre sont multicritères. Nous avons identifié deux grands enjeux lors de notre constat. 
\begin{itemize}
	\item Tout d'abord, la nécessité de garanties sur les temps d'exécution des tâches, et notamment les tâches critiques, de façon à ne pas dépasser des échéances liées aux contraintes temps-réel.
	\item Ensuite, le besoin croissant d'exploiter au maximum les ressources de calcul à disposition, de façon à diminuer considérablement le nombre de processeurs nécessaires au sein des systèmes embarqués.\end{itemize}

La problématique principale pour répondre à ces deux enjeux réside dans leur antinomie. Les solutions dédiées à la garantie des contraintes temps-réel se font au détriment de l'optimisation de la puissance de calcul, tandis que l'agrégation de logiciels à criticité multiple provoque des problèmes d'interférences qui peuvent mener à des retards d'exécution indésirables. Ces problèmes sont intrinsèques à l'usage des calculateurs multicœurs, par la présente de ressources partagées tel que le cache, les bus de données ou encore les différentes entrées/sorties. Avec l'exécution concourante de tâches sur plusieurs cœurs, cet usage de ressources partagées entre les logiciels mène à des points de contentions qui augmentent les temps de réponses des tâches jusqu'au dépassement des échéances. 

\paragraph*{} Pour répondre à cette problématique, nous avons en premier lieu proposé une approche qui se focalise sur une vision fonctionnelle du système. De fait, au sein d'un calculateur multicœur, les tâches qui s'exécutent sont en partie interconnectées pour réaliser des fonctions spécifiques. Cela se traduit à l'échelle du logiciel par ce que l'on définit comme des chaînes de tâches. Nous proposons par conséquent un modèle de mécanisme de Surveillance et de Contrôle spécialisé dans la garantie de contraintes temporelles lors de l'exécution d'une chaîne de tâches. Le modèle proposé se veut adapté au domaine industriel par sa généricité avec des hypothèses de modèle qui prennent en compte les contraintes industrielles, tel que la présence de logiciel en boite noire.
Notre mécanisme repose sur la Surveillance de l'avancement d'exécution d'une chaîne de tâche. Il anticipe l'instant où les interférences dues aux tâches non critiques présentent un risque irréversible de dépassement de l'échéance d'exécution bout-en-bout. Le cas échéant, l'exécution des tâches non critiques est Contrôlé via une mise en pause momentanée pour prévenir toute interférence supplémentaire et donc garantir le respect de l'échéance bout-en-bout.

\paragraph*{} Afin de mettre en place ce mécanisme, nous avons proposé tout un processus de développement qui s'inscrit dans une démarche expérimentale. Cette approche permet la caractérisation nécessaire d'un système à criticité mixte de façon à obtenir une meilleure connaissance de son comportement et notamment vis-à-vis des interférences d'exécution. Dans un second temps, il indique aussi les grandes étapes à intégrer dans le cadre de l'implémentation du mécanisme de Surveillance et Contrôle sur un système embarqué. Cela inclut la détermination des paramètres de fonctionnement propres au mécanisme, mais aussi la phase de vérification analytique avec une mesure des performances suivant trois critères pertinents (Fiabilité, Performance et Qualité) ainsi que des possibilités d'ajustements.

Enfin, nous avons mis en application cette démarche expérimentale sur une plateforme dédiée. Dans cette optique, nous avons développé un framework expérimental complet. Ce dernier permet d'exécuter directement n'importe quel jeu de tâches sous forme de fonctions de librairies avec à notre mécanisme de Surveillance et de Contrôle. La plateforme matérielle et logicielle que nous proposons permet, par le biais de fichiers de configuration d'entrées, une forte adaptabilité de configuration et de modifications pour s'adapter selon 3 axes~: le support matériel et logiciel bas-niveau (ici, nous avons choisis Xenomai sur un multicœur Intel) ; les hypothèses du modèle de tâches et de chaîne de tâches ; et les données d'entrée (le jeu de tâches, les paramètres du mécanisme, la charge de stress du calculateur). Grâce à tout cet ensemble, nous présentons ainsi une base de plateforme expérimentale minimale de façon à caractériser pour un contexte spécifique le profil de temps d'exécution d'un logiciel, la place des interférences inter-logiciel au sein de celui-ci, et l'apport de notre mécanisme de Surveillance et de Contrôle sur le respect des échéances bout-en-bout. 

La mise en place de cette plateforme complète accompagnée des protocoles expérimentaux proposés a pu être éprouvé au travers de la constitution d'un cas de test fictif, par le biais des tâches du benchmark MiBench. Ce cas de test nous a permis d'appliquer la démarche dans sa totalité et d'illustrer les capacités de notre mécanisme dans un cas spécifique. Les résultats obtenus sont encourageants et permettent concrètement d'obtenir un système à criticité mixte avec un mécanisme de contrôle réactif qui offre aux tâches non critiques le temps nécessaires hors des instants d'exception où le respect d'échéance de la chaîne de tâches critiques doit continuer d'être garanti. 

\paragraph*{} Dans un contexte comme celui du domaine automobile, où les multiples fonctions embarquées se retrouvent agrégés au sein de calculateurs toujours plus puissants, la problématique de co-existence de logiciels à différents niveaux de criticité devient inévitable. Quand des systèmes d'aide à la conduite (ADAS) tel que le Système de Changement de Ligne ou l'Avertissement d'Angle-Mort risquent de coexister avec du logiciel non-critique comme le système radio ou la chauffe des sièges, le besoin de garanties d'exécution devient pressant. Dans l'ensemble, nous proposons une approche inédite pour le respect des échéances temporelles tout en conservant de bonnes performances sur l'exploitation des ressources de calcul. Notre solution de maîtrise des fautes temporelles dues aux interférences est réactive contrairement à ce qui se fait habituellement de façon statique dans les solutions industrielles. Nous tentons de palier aux risques d'interférences liées au partage de ressources de façon conservative, en limitant l'exécution des tâches non critiques uniquement en cas de nécessité absolue. 

Ce processus expérimental propose globalement une nouvelle approche aux problématiques émergentes des systèmes à criticité mixte, qui prend en compte au mieux les contraintes propres aux systèmes industriels. Il s'agit d'un premier pas vers des approches orientées sur les chaînes de tâches, avec des outils pour la caractérisation des systèmes industriels et la mise en œuvre de ce type de mécanismes de sûreté de fonctionnement somme toute très lié aux enjeux du temps-réel.


\section*{Perspectives}
\markboth{Conclusion et Perspectives}{Perspectives}
\addcontentsline{toc}{section}{Perspectives}

%%\subsection{Mode dégradé multi-niveau}
%%\subsection{mode dégradé par mécanismes de contrôle hardware}

L'approche qui a été développée dans cette thèse est particulièrement prometteuse. Elle ouvre des pistes de recherche axées sur l'exploitation de chaînes de tâches pour proposer un mécanisme réactif de garantie des contraintes temporelles. Cela constitue une base qui ne demande qu'à être étoffée et améliorée.

Les résultats expérimentaux que nous avons obtenus sont encourageants dans le sens où ils démontrent la capacité d'un tel mécanisme à offrir des garanties de respect d'échéance avec des compromis limités sur l'exécution des tâches non critiques. Cependant, ils sont spécifiques au système à criticité mixte auquel il est appliqué. Le diagnostic sur la présence d'interférences et leurs effets sur l'exécution des tâches critiques est hautement dépendant à la fois du support matériel et du comportement des tâches en elles-mêmes. L'implémentation d'un tel mécanisme est ainsi conditionné à la caractérisation du système dans lequel il s'inscrit.
 
En ce sens, la généricité de la solution et son applicabilité à un grand nombre d'architectures tant matérielles que logicielles est une force. En effet, bien que réfléchie en partant d'un contexte automobile, notre proposition a été élargie pour s'adapter à des cas industriels de systèmes à criticité mixte divers et variés. 

\paragraph*{} La première piste d'amélioration réside dans l'extension de la proposition actuelle de façon à gérer plus d'une unique chaîne de tâches critiques. Il s'agit de l'extension la plus directe à prendre en compte, qui apporte un grand avantage sur les possibilités d'utilisations du mécanisme de surveillance et contrôle. 

Considérant la proposition actuelle, l'enjeu à résoudre pour cela réside dans la capacité à obtenir des garanties de respect d'échéance dans un mode dégradé qui prend en compte plusieurs chaînes de tâches. Cela implique donc un plus grand nombre de tâches dans ce mode, et donc des interférences inter-chaînes. Cela ne pourrait se faire qu'à la condition de connaître avec précision ces interférences-là, et les prendre en compte dans les estimations de pire temps d'exécution restant de chacune des chaînes de tâches critiques.

\paragraph*{} Une seconde piste de développement repose sur l'extension du modèle de tâches à criticité duale que nous avons employé. Bien qu'un bon nombre de systèmes reposent uniquement sur deux niveaux de criticité, l’extension à un plus grand nombre pourrait correspondre à une plus grande variété de cas d'applications. Ainsi, des domaines comme le ferroviaire, l'avionique ou l'automobile emploient 5 niveaux de criticité qui ont une incidence sur les contraintes de développement du logiciel. Proposer plusieurs niveaux de criticité aurait donc le double avantage d'être plus facilement combinable avec d'autres travaux de recherche qui travaillent sur plus de niveaux, et de potentiellement approfondir les possibilités de modes dégradés intermédiaires. En effet, les tâches non critiques peuvent alors être subdivisés dans des nuances de niveaux de criticité qui peuvent ne pas être traitées de la même façon par le mécanisme de contrôle.

\paragraph*{} Pour finir, dans l'idée de combiner notre mécanisme avec d'autres solutions de contrôle de l'exécution des tâches, il serait possible de proposer des niveaux intermédiaires de modes dégradés. Mais cette fois-ci il ne s'agirait pas de jouer sur quelles tâches sont mises en pause en mode dégradé, mais plutôt de proposer des solutions de contrôle moins drastiques, qui n'empêchent pas l'exécution des tâches non critiques, mais limitent par d'autres biais leurs capacités de nuisance par interférences. L'un de ces moyens repose sur l'exploitation du complément à notre méthode logicielle réactive~: une solution matérielle statique. En effet, par la possibilité de mettre en place des mécanismes d'isolation spatiale des tâches en cours d'exécution, il serait possible de limiter les interférences sur les ressources partagées par la réservation d'une partie du cache aux tâches critiques. Une telle amélioration permettrait la mise en place de mécanismes préventifs progressifs contre les interférences entre les tâches. De façon cela limiterai au maximum la sous exploitation des ressources de calcul, tout en garantissant au mieux le respect des échéances temporelles sur les chaînes de tâches critiques. 

% proposée la rend indépendante du support matériel. Aussi le mécanisme de Surveillance et de Contrôle présente peu de contraintes logicielles. Il s'agit donc d'une proposition adaptable pour un
\ifdefined\included
\else
\bibliographystyle{StyleThese}
\bibliography{these}
\end{document}
\fi


%%%%%%%%%%%%%%%%%%%%%%%
 %%%% PUBLICATIONS %%%%%
\chapter*{Publications}
\addstarredchapter{Publications}
Les travaux présentés dans ce manuscrit ont été effectués à Toulouse au Laboratoire d’Analyse et d’Architecture des Systèmes (LAAS) du Centre National de la Recherche Scientifique (CNRS) au sein de l'équipe Tolérance aux Fautes et Sûreté de Fonctionnement. Ils ont été réalisés dans le cadre d'une collaboration CIFRE avec le Technocentre Renault, Guyancourt et plus spécifiquement l'équipe SW-T. À l'heure à laquelle ces lignes sont écrites, cela a donné lieu aux publications suivantes~:

\begingroup\raggedright
\begin{description}
	\item	D. Loche, M. Lauer, M. Roy, et J.-C. Fabre, « \textit{Mixed Critical Automotive Embedded Applications on Multicores: A Safe Scheduling Approach for Dependability} ». Dans 5th International Workshop on Critical Automotive Applications: Robustness \& Safety (CARS), 2019, Naples, Italy
	\smallbreak
	\item	D. Loche, M. Lauer, M. Roy, et J.-C. Fabre, « \textit{Safe Scheduling on Multicores: an approach leveraging multi-criticality and end-to-end deadlines} ». Dans 10th European Congress on Embedded Real Time Software and Systems (ERTS 2020), 2020, Toulouse, France.
	\smallbreak
	\item	D. Loche, A. Generes, M. Lauer, et J.-C. Fabre, « \textit{Run-time Monitoring and Control for Temporal Fault Prevention in Mixed-criticality Systems} ». Dans European Dependable Computing Conference (EDCC 2021), Intel; Fraunhofer IKS; LAAS, 2021, Munich (virtual), Germany. pp.53-60. (doi~: 10.1109/EDCC53658.2021.00015)
\end{description}
\endgroup
\vspace*{\fill}

\cmnt
{ %% Tentative list Publications - foiré... du coup j'ai fais à la main !
	\begin{bibunit}
		\renewcommand{\bibsection}{\large \textbf{\begin{center}
					Publications
		\end{center}}}
		\makeatletter
		\renewcommand\@biblabel[1]{}
		\makeatother
		\def\bibfont{\small}

		%list all the references in the publications bibtex file
		\nocite{*}
		\putbib[MyPublications]
	\end{bibunit}
}

%\appendix
%\chapter*{Résultats Expérimentaux}\label{chap:annexe1}
\cleardoublepage

\newgeometry{top=3.3cm,bottom=3cm}      %% SI VOS ANNEXES SONT ÉNORMES ET QUE VOUS VOULEZ GRAPILLER SUR LA MARGE, C'EST ICI.
%% PENSER A FINIR LES ANNEXES PAR UN "/restoregeometry" DE FAÇON À CONSERVER LA MARGE D'ORIGINE SUR LE RESTE DU MANUSCRIPT

\section*{ANNEXE 1 -- Caractérisation des tâches MiBench} \label{Annexe1}
\markboth{ANNEXE}{Annexe}  %% A modifier : nom des entêtes pair/impair des pages de l'annexe.
\addstarredchapter{Annexe}
\setcounter{chapter}{0}  %% reset du compteur pour pas que les annexes se basent sur l'incrémentation des chapitres du manuscript.
\setcounter{table}{0}
\renewcommand{\thetable}{A\arabic{table}} %% Nouvelle numérotation pour les tableaux idem, pour pas continuer à la suite de numérotation du manuscrit.
 \begin{table}[ht!]
	\centering
	\caption{Caractérisation des tâches MiBench - 1200 runs par tâches} 	\label{tab:Phase1-2-results}
	\begin{adjustwidth}{-40pt}{0pt}
		\begin{tabular}{@{}lrrrrrrl@{}}
			\toprule
			         & \multicolumn{3}{c}{Tâche en isolation}  & \multicolumn{3}{c}{Tâche + interférences} &  \\
			\cmidrule(lr){2-4} \cmidrule(lr){5-7}
			Tâche	 & MIN (ms)  & AVG (ms)  & MAX (ms)  & MIN (ms)  & AVG (ms)  & MAX (ms)  & Classification     \\
			\midrule
			fft-S               & 1,7E+00   & 1,7E+00   & 1,7E+00   & 1,7E+00   & 1,7E+00   & 1,9E+00   & CLEAN    \\
			fft-inv-S           & 3,5E+00   & 3,5E+00   & 3,5E+00   & 3,5E+00   & 3,5E+00   & 3,7E+00   & CLEAN    \\
			basicmath-S         & 3,8E+00   & 3,8E+00   & 4,0E+00   & 3,8E+00   & 3,9E+00   & 4,3E+00   & CLEAN    \\
			fft-L               & 7,1E+00   & 7,1E+00   & 7,2E+00   & 6,7E+00   & 7,2E+00   & 7,7E+00   & CLEAN    \\
			rijndael-E-L        & 127,7E+00 & 152,3E+00 & 5,8E+03   & 1,2E+03   & 1,5E+03   & 1,7E+03   & REJECT   \\
			\midrule
			\textbf{Médianes} 	& \multicolumn{3}{c}{7,83 ms}	& \multicolumn{3}{c}{9,64 ms}	& 	\\
			\bottomrule
		\end{tabular}
	\end{adjustwidth}
\end{table}

\restoregeometry


%%%%%%%%%%%%%%%%%%%%%%%%
 %%%% BIBLIOGRAPHIE %%%%%
\begingroup\raggedright        	%% Je préfère enlever l'ajustement du texte (plutôt qu'aligné à gauche) dans la biblio, c'est + lisible.
\bibliographystyle{StyleThese} 	% ou StyleTheseWithEtAl  // plain
\bibliography{these}			%% Penser à bien nommer son fichier these.bib pour la bibliographie.
\endgroup


\cleardoublepage

%%%%%%%%%%%%%%%%%%%%%%%%%%%%%
 %%%%% RESUMÉ - ABSTRACT %%%%%
 \newgeometry{top=3.6cm,bottom=3.6cm}

\begin{vcenterpage}
\noindent\rule[2pt]{\textwidth}{0.5pt}
\\ %%%%% RESUME EN FRANCAIS %%%%%
	{\large\textbf{Résumé :} \addstarredchapter{Résumé}

		Ceci est un résumé de votre thèse en Français, en moins de 1000 caractères.
	}
\smallbreak
{\large\textbf{Mots clés~:} mots-clés, de la thèse, séparés, par une virgule}
\\
\noindent\rule[2pt]{\textwidth}{0.5pt}
\end{vcenterpage}
%\newpage
\begin{vcenterpage}
\noindent\rule[2pt]{\textwidth}{0.5pt}
\section*{Thesis title in english}
%\\ %%%%% RESUME EN ANGLAIS %%%%%
	{\large\textbf{Abstract:}

		That's your english version of the abstract, in less than 1000 caracters.
	}
	\smallbreak
	{\large\textbf{Keywords:} english, keywords}
\\ %\Large\textbf{\textsc{Preventing temporal faults on multicore architectures for mixed-criticality systems}}
\noindent\rule[2pt]{\textwidth}{0.5pt}
\end{vcenterpage}

\restoregeometry
\end{document}
