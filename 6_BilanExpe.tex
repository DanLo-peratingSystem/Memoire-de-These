\ifdefined\included
\else
\documentclass[french, a4paper, 11pt, twoside, pdftex]{StyleThese}
\usepackage{iflang}
\usepackage{bibentry}



%\usepackage[sectionbib]{chapterbib}          % Cross-reference package (Natural BiB)
%\usepackage{natbib}                  % Put References at the end of each chapter
%\usepackage{bibunits}
% Do not put 'sectionbib' option here.
% Sectionbib option in 'natbib' will do.


\usepackage{fancyhdr}                    % Fancy Header and Footer

\usepackage[utf8]{inputenc}
\usepackage[T1]{fontenc}
\usepackage[french]{babel} %
\usepackage{lmodern} \normalfont %to load T1lmr.fd 
\DeclareFontShape{T1}{cmr}{b}{sc} { <-> ssub * cmr/bx/sc }{}
%\hyphenation{gar}

\usepackage{amsmath,amssymb}             % AMS Math
\usepackage{nicefrac}
\usepackage{siunitx}					%% Unites Math SI

\usepackage{blindtext}

\usepackage{datetime}

\usepackage{lipsum} 

\usepackage[inline]{enumitem}

\usepackage{hhline}
%\usepackage[left=1.5in,right=1.3in,top=1.1in,bottom=1.1in]{geometry}
\usepackage[left=1.5in,right=1.3in,top=1.1in,bottom=1.1in,includefoot,includehead,headheight=13.6pt]{geometry}

%%\renewcommand{\baselinestretch}{1.05}

%%%%%%%% Multi-figures avec sub-captions
\usepackage{caption}
\usepackage{subcaption}

% Table of contents for each chapter

\usepackage[nottoc, notlof, notlot]{tocbibind}
\usepackage[nohints]{minitoc}
\setcounter{minitocdepth}{2}
\mtcindent=15pt
% Use \minitoc where to put a table of contents

\usepackage{aecompl}

%% Package cosmetic meilleur layout du texte en jouant sur le spacing par caractères
\usepackage[activate={true,nocompatibility},final,tracking=true,kerning=true,factor=1100,stretch=10,shrink=10]{microtype}
\usepackage[absolute,overlay]{textpos} 
\setlength{\TPHorizModule}{\paperwidth}\setlength{\TPVertModule}{\paperheight}
\sloppy

%%%%%%%%%%% JOLIS TABLEAUX
\usepackage{tabularx}		%\usepackage{tabular}
\usepackage{multirow}
\newcommand{\mc}{\multicolumn} 
\newcommand{\mr}[2]{\multirow{#1}{*}{#2}} 	\newcommand{\mrQ}{\multirow{-4}{*}}
\usepackage{booktabs}

\usepackage[usenames,dvipsnames]{xcolor} 

\makeatletter
\newcommand{\ccolor}[3][]{%
	\kern-\fboxsep
	\if\relax\detokenize{#1}\relax
	\expandafter\@firstoftwo
	\else
	\expandafter\@secondoftwo
	\fi
	{\colorbox{#2}}%
	{\colorbox[#1]{#2}}%
	{#3}\kern-\fboxsep
}
\makeatother

%%%%% Insertion graphiques format PGF
\usepackage{pgfplots}
\pgfplotsset{width=\linewidth, compat=1.16}%, compat=1.17}
\usepackage{adjustbox}          %%% PERMET DE LES RECADRER + FACILEMENT


%%%%%%%%%% Bullets de listes sans saut de ligne %%%%%%%%%%
\usepackage{xparse}

\ExplSyntaxOn%
\seq_new:N \l_local_enum_seq

\newcommand{\storethestuff}[1]{%
  \seq_set_from_clist:Nn \l_local_enum_seq {#1}%
}

\newcommand{\dotheenumstuff}{%
\int_zero:N \l_tmpa_int
\seq_map_inline:Nn \l_local_enum_seq {%
    \int_incr:N \l_tmpa_int% Increase the counter
    \item ##1
    % Check whether the list has reached the end -- if so, use '.' instead of ','
    %\int_compare:nNnTF 
    % { \l_tmpa_int } < {\seq_count:N \l_local_enum_seq} 
    % {,} {.}
  }
}
\ExplSyntaxOff

\NewDocumentCommand{\linebullets}{+m}{%
  \storethestuff{#1}%
  \begin{enumerate*}[label={\alph*)},font={\bfseries},itemjoin={{, }}]
    \dotheenumstuff%
  \end{enumerate*}
}

\newcommand{\cmnt}[1]{}  %%%%% AJOUT DE COMMENTAIRE MULTILIGNES


%%%%%%%%%% ECRITURE CARACTERES DANS UN CERCLE %%%%%%%%%%
%\def\circleTxt[#1]{\raisebox{.5pt}{\textcircled{\raisebox{-1pt}{#1}}}}
\newcommand{\ctxt}[1]{\raisebox{.5pt}{\textcircled{\raisebox{-1.2pt}{#1}}}}
% Glossary / list of abbreviations

\usepackage[intoc]{nomencl}
\IfLanguageName{english}{%
\renewcommand{\nomname}{Glossary}
}{ %
\renewcommand{\nomname}{Liste des Abréviations}
}

\makenomenclature

% My pdf code

\usepackage{ifpdf}

\ifpdf
  \usepackage[pdftex]{graphicx}
  \DeclareGraphicsExtensions{.pdf,PDF,.png,PNG,.jpg,JPG}
  \usepackage[pagebackref,hyperindex=true]{hyperref} %% use \autoref{} instead of Table~\ref{}.
  \usepackage{tikz}
  \usetikzlibrary{arrows,shapes,calc}
\else
  \usepackage{graphicx}
  \DeclareGraphicsExtensions{.ps,.eps}
  \usepackage[a4paper,dvipdfm,pagebackref,hyperindex=true]{hyperref}
\fi

\graphicspath{{.}{schemas/}{graphiques/}{tables/}}

%% nicer backref links. NOTE: The flag ThesisInEnglish is used to define the
% language in the back references. Read more about it in These.tex

\IfLanguageName{english}{
\renewcommand*{\backref}[1]{}
\renewcommand*{\backrefalt}[4]{%
\ifcase #1 %
(Not cited.)%
\or
(Cited in page~#2.)%
\else
(Cited in pages~#2.)%
\fi}
\renewcommand*{\backrefsep}{, }
\renewcommand*{\backreftwosep}{ and~}
\renewcommand*{\backreflastsep}{ and~}
}{
\renewcommand*{\backref}[1]{}
\renewcommand*{\backrefalt}[4]{%
\ifcase #1 %
(Non cité.)%
\or
(Cité en page~#2.)%
\else
(Cité en pages~#2.)%
\fi}
\renewcommand*{\backrefsep}{, }
\renewcommand*{\backreftwosep}{ et~}
\renewcommand*{\backreflastsep}{ et~}
}

% Links in pdf
\usepackage{color}
\definecolor{linkcol}{rgb}{0,0,0.4} 
\definecolor{citecol}{rgb}{0.5,0,0} 
\definecolor{linkcol}{rgb}{0,0,0} 
\definecolor{citecol}{rgb}{0,0,0}
% Change this to change the informations included in the pdf file

\hypersetup
{
bookmarksopen=true,
pdftitle="Prévention des fautes temporelles sur architectures multicœur pour les systèmes à criticité mixte",
pdfauthor="Daniel LOCHE", %auteur du document
pdfsubject="Thèse", %sujet du document
%pdftoolbar=false, %barre d'outils non visible
pdfmenubar=true, %barre de menu visible
pdfhighlight=/O, %effet d'un clic sur un lien hypertexte
colorlinks=true, %couleurs sur les liens hypertextes
pdfpagemode=UseNone, %aucun mode de page
%pdfpagelayout=DoublePage, %ouverture en simple page
pdffitwindow=true, %pages ouvertes entierement dans toute la fenetre
linkcolor=linkcol, %couleur des liens hypertextes internes
citecolor=citecol, %couleur des liens pour les citations
urlcolor=linkcol %couleur des liens pour les url
}

% definitions.
% -------------------

\setcounter{secnumdepth}{3}
\setcounter{tocdepth}{2}

% Some useful commands and shortcut for maths:  partial derivative and stuff

\newcommand{\pd}[2]{\frac{\partial #1}{\partial #2}}
\def\abs{\operatorname{abs}}
\def\argmax{\operatornamewithlimits{arg\,max}}
\def\argmin{\operatornamewithlimits{arg\,min}}
\def\diag{\operatorname{Diag}}
\newcommand{\eqRef}[1]{(\ref{#1})}
\newcommand{\nline}{\smallbreak\noindent}

\usepackage{rotating}                    % Sideways of figures & tables

% \usepackage{txfonts}                     % Public Times New Roman text & math font
  
%%% Fancy Header %%%%%%%%%%%%%%%%%%%%%%%%%%%%%%%%%%%%%%%%%%%%%%%%%%%%%%%%%%%%%%%%%%
% Fancy Header Style Options

\pagestyle{fancy}                       % Sets fancy header and footer
\fancyfoot{}                            % Delete current footer settings

%\renewcommand{\chaptermark}[1]{         % Lower Case Chapter marker style
%  \markboth{\chaptername\ \thechapter.\ #1}}{}} %

%\renewcommand{\sectionmark}[1]{         % Lower case Section marker style
%  \markright{\thesection.\ #1}}         %

\fancyhead[LE,RO]{\bfseries\thepage}    % Page number (boldface) in left on even
% pages and right on odd pages
\fancyhead[RE]{\bfseries\nouppercase{\leftmark}}      % Chapter in the right on even pages
\fancyhead[LO]{\bfseries\nouppercase{\rightmark}}     % Section in the left on odd pages

\let\headruleORIG\headrule
\renewcommand{\headrule}{\color{black} \headruleORIG}
\renewcommand{\headrulewidth}{1.0pt}
\usepackage{colortbl}
\arrayrulecolor{black}

\fancypagestyle{plain}{
  \fancyhead{}
  \fancyfoot{}
  \renewcommand{\headrulewidth}{0pt} %%%%%%%%%%%%%%%%%%%%%%%%%%%%%%%%%%%%%%%%%%%%%%%%%%%%%%%%%%%%%%%%%%%%%%%%%%%%%%%%%%%%%
}

%\usepackage{MyAlgorithm}
%\usepackage[noend]{MyAlgorithmic}
%\usepackage[ED=EDSYS-SystEmb, Ets=INP]{tlsflyleaf}

%%% Clear Header %%%%%%%%%%%%%%%%%%%%%%%%%%%%%%%%%%%%%%%%%%%%%%%%%%%%%%%%%%%%%%%%%%
% Clear Header Style on the Last Empty Odd pages
\makeatletter

\def\cleardoublepage{\clearpage\if@twoside \ifodd\c@page\else%
  \hbox{}%
  \thispagestyle{empty}%              % Empty header styles
  \newpage%
  \if@twocolumn\hbox{}\newpage\fi\fi\fi}

\makeatother
 
%%%%%%%%%%%%%%%%%%%%%%%%%%%%%%%%%%%%%%%%%%%%%%%%%%%%%%%%%%%%%%%%%%%%%%%%%%%%%%% 
% Prints your review date and 'Draft Version' (From Josullvn, CS, CMU)
\newcommand{\reviewtimetoday}[2]{\special{!userdict begin
    /bop-hook{gsave 20 710 translate 45 rotate 0.8 setgray
      /Times-Roman findfont 12 scalefont setfont 0 0   moveto (#1) show
      0 -12 moveto (#2) show grestore}def end}}
% You can turn on or off this option.
% \reviewtimetoday{\today}{Draft Version}
%%%%%%%%%%%%%%%%%%%%%%%%%%%%%%%%%%%%%%%%%%%%%%%%%%%%%%%%%%%%%%%%%%%%%%%%%%%%%%% 

\newenvironment{maxime}[1]
{
	\def\Arg{#1}
\vspace*{0cm}
\hfill
\begin{minipage}{0.6\textwidth}%
%\rule[0.5ex]{\textwidth}{0.1mm}\\%
\hrulefill $\:$ \\%$\:$ {\bf #1}\\
%\vspace*{-0.25cm}
\it 
}%
{%
	
\hrulefill $\:$ {\bf \Arg}
\vspace*{0.5cm}%
\end{minipage}
}

\let\minitocORIG\minitoc
\renewcommand{\minitoc}{\minitocORIG \vspace{1.5em}}

%\usepackage{slashbox}

\newenvironment{bulletList}%
{ \begin{list}%
	{$\bullet$}%
	{\setlength{\labelwidth}{25pt}%
	 \setlength{\leftmargin}{30pt}%
	 \setlength{\itemsep}{\parsep}}}%
{ \end{list} }


%%%%%%% Outils pour \comment \alert \add %%%%%
\usepackage{easyReview}
\usepackage{soulutf8} % for accented letters

\let\newalert\alert
\renewcommand{\alert}[1]{\textit{\newalert{#1}}}

%\usepackage[commandnameprefix=ifneeded]{changes} %% \chhighlight and \chcomment to avoid collision with easyReview
\renewcommand{\epsilon}{\varepsilon}

% centered page environment

\newenvironment{vcenterpage}
{\newpage\vspace*{\fill}\thispagestyle{empty}\renewcommand{\headrulewidth}{0pt}}
{\vspace*{\fill}}

\usepackage{tablefootnote}

%%%%%% MISE EN FORME CADRES DEFINITIONS/THEOREMES/LEMES %%%%%%%%%%
\usepackage{amsthm}  % for theoremstyle

\theoremstyle{plain} 
\newtheorem{theorem}{Théorème}[section]
\newtheorem{corollary}{Corolaire}[theorem]

%\theoremstyle{lemma}
%\newtheorem{lemma}[theorem]{Lemme}


\theoremstyle{definition}
\newtheorem{definition}[theorem]{Définition}


\cmnt{
	\usepackage{ntheorem} %\usepackage{amsthm}  % for theoremstyle
	%\usepackage{mdframed}
	\usepackage[most]{tcolorbox}
	
	\theoremstyle{plain} 
	\theoremindent20pt
	\theoremheaderfont{\normalfont\bfseries\hspace{-\theoremindent}}
	\newtheorem{theorem}{Théorème}[section]
	\newtheorem{corollary}{Corolaire}[theorem]
	
	\theoremstyle{plain}
	\newtheorem{lemma}[theorem]{Lemme}
	
	
	\tcolorboxenvironment{theorem}{
		blanker,
		breakable,
		before skip=\topsep,
		after skip=\topsep,
		borderline west={1pt}{10pt}{double, shorten <=12pt}
	}
	
	\theorembodyfont{\normalfont}
	\theoremindent20pt
	\theoremheaderfont{\normalfont\bfseries\hspace{-\theoremindent}}
	\newtheorem{definition}[theorem]{Définition}
	
	
	\tcolorboxenvironment{definition}{
		blanker,
		breakable,
		before skip=\topsep,
		after skip=\topsep,
		borderline west={1pt}{10pt}{shorten <=12pt}
	}
}

\cmnt{ 
	\begin{theorem}
		Ceci est un Théorème.
	\end{theorem} 
	
	\begin{corollary}
		Ceci est un Corollaire.
	\end{corollary}
	
	\begin{definition}
		Ceci est une Définition.
	\end{definition}
	
	\begin{lemma}
		Ceci est un Lemme.
	\end{lemma}
}

\def\UrlBigBreaks{\do\/\do-\do:}
\usepackage{url}

\sloppy
\begin{document}
\setcounter{chapter}{6} %% Numéro du chapitre précédent ;)
\dominitoc
\faketableofcontents
\fi

\chapter{Mise en Application expérimentale}

\minitoc

%%%% CONTENU %%%%
      
    \section{Application à MiBench du Protocole}
            Using Mibench as a workload had advantages but also drawbacks. It allows to get specific tasks with a defined and already studied behavior \cmnt{(cache use, computing resources needs, I/O...) }but we are dependent on the way they are initially programmed. They might not completely fit our needs to simulate embedded applications or have incompatibilities with the chosen real-time environment. First step in using this benchmark is to check those criteria to select precisely the tasks from MiBench we use.
        \subsection{Phase de Design}
            \subsubsection{Profil des tâches en isolation}
            
                    We need to establish the execution time profile of each task of the bench. As a result some tasks will be removed from the tests, either due to execution time magnitude differences or inconsistent behaviors between experiments. 
                    %%because the adaptation to Xenomai did not succeeded.
                    Accordingly, we measure on each experiment the min, max and median execution times, but also some system counters as the Xenomai mode switches and the amount of linux system calls. Without interferences, the execution time characteristics should have low  variations.% Task adapted for real-time context should have almost no mode switches and system calls.
                    We see in \autoref{tab:xenoIsol} a sample of the tasks characteristics collected, for 3 different profiles.
            
                    \begin{table}[ht]
                        \centering
                        \caption{Tasks profiles in \textit{Xenomai} environment}
                        \begin{tabular}{@{}lrrcrr@{}}
                        \toprule
                        Task & \multicolumn{2}{c}{execution times (ms)} & \phantom{} & \multicolumn{2}{c}{System Counters} \\ % pas sûr
                        \cmidrule{2-3} \cmidrule{5-6}
                                    &   Median      &   Max    &&   Mode Switch & Sys. Call     \\
                        \midrule
                        Patricia    &   0.026       &   0.099       &&  10051        & 10338    \\
                        FFT         &   7.36        &   7.39        &&  58           & 2343      \\
                        rijndaelE   &   140,11      &   141.81      &&  158          & 446       \\
                        \bottomrule
                        \end{tabular}
                                \label{tab:xenoIsol}
                    \end{table}
                    \smallbreak
                
                    With such data, we identified the majority execution time range in MiBench task set around 10ms (from 2-3ms to 20-30 ms) and the basic system calls and mode switch amounts due to initialisation phase (respectively 58 mode switches and $\approx$hundreds of system calls).
            
                    %We want to use the information on the runtime of the task to understand which task becomes unusable.  \textit{Patricia}, for example, has 70 µs for runtime in \textit{Xenomai} environment. 
                    %Compare to Patricia in Linux environment, we couldn't manage to change its code for \textit{Xenomai}.
                    %However, \textit{FFT (L)} has 7.2 ms of runtime, which are representative of it behaviour in both environment.
                    %Compare to the \textit{FFT\_L} task which has 50 syscalls and 7.2 ms, the system calls \textit{Patricia} but don't run anything. We consider that \textit{Patricia} is aberrant and doesn't represent it normal behaviour. % before modification
                    Consequently, we discard tasks out of the execution time magnitude like \textit{adpcmCaudio\_L } with an average execution time of 432 ms.
                    By the end of step \circleTxt[1], we retained 34 tasks~: Bitcount\_L,       Bitcount\_S, Basicmath\_S, Basicmath\_L, Dijkstra\_L, Dijkstra\_S, Fft\_inv\_L, Fft\_inv\_S, Fft\_L, Fft\_S, GsmToast\_L, GsmToast\_S, GsmUToast\_L, GsmUToast\_S, RijndaelE\_S, RijndaelD\_S, Sha\_L, Sha\_S, Stringsearch\_L, Stringsearch\_S, AdpcmCaudio\_L, AdpcmCaudio\_S, AdpcmDaudio\_L, AdpcmDaudio\_S, Cjpeg\_L, Cjpeg\_S, Djpeg\_L, Djpeg\_S, Susan\_L\_corners, Susan\_S\_corners, Susan\_L\_edges, Susan\_S\_edges, Susan\_L\_smooth, Susan\_S\_smooth.  



            \subsubsection{Profil des tâches avec stress imposé}
                        We add stress on cache level and communication bus from previous step experiments.  
                        The objective is to discriminate our tasks in two groups depending on their reaction under stress. If it increases execution time too significantly (more than x10 from average time in isolation) it means the tested task is not suited for the tested environment and suffers not only from interferences but also from LO-criticality tasks preemption. A significant increase in mode switches also indicates such behavior. The tasks that do not pass correctly this test will be either ignored or used LO-criticality stress tasks. 
                        %Some examples can be seen in \autoref{fig:UnusableTasks}. 
                        Tasks without an exploding execution time or huge increase of mode switches will be used to generate the HI-criticality task chain. 
                        Execution time profiles of task used for this purpose are in \autoref{tab:Stress}. We finally retained 22 tasks at the end of step~\circleTxt[2].
                        %Bitcount\_L, %Bitcount\_S, Basicmath\_S, Dijkstra\_L, Dijkstra\_S, Fft\_inv\_L, Fft\_inv\_S, Fft\_L, Fft\_S, GsmToast\_L, GsmUToast\_L, RijndaelE\_S, Sha\_L, Sha\_S, Stringsearch\_L, Stringsearch\_S, AdpcmCaudio\_L, AdpcmDaudio\_L, Cjpeg\_S, Susan\_S\_corners, Susan\_S\_edges, Susan\_S\_smooth.
                        %Execution time profiles of task used for this purpose are in \autoref{fig:usableTasks}. 
                        \begin{table}[ht]
                            \centering
                            \caption{Tasks profiles in \textit{Xenomai} environment}
                            \label{tab:Stress}
                        \begin{tabular}{@{}lrrcrr@{}}  %% r = right | l = left | c = center
                        \toprule
                        Task & \multicolumn{2}{c}{execution times isolated} & \phantom{abc}& \multicolumn{2}{c}{execution times stressed} \\
                        \cmidrule{2-3} \cmidrule{5-6} 
                                    &   Median (ms) &   Max (ms)    &&   Median (ms) &   Max (ms)     \\
                        \midrule
                        djpeg       &  1.97     &  2.28      &&  19.91       & 211.53    \\
                        rjindaelD   &   8.80    &  9.77      && 35.02        & 526.33    \\
                        FFT         &   1.85    &  1.86      &&  2.03        & 14.8     \\
                        FFT$^{-1}$  &   3.56    &   3.57     &&  4.05        & 19.74    \\
                        bitcount    &   8.36    &  9.52      &&  9.98        & 45.18    \\
                        \hline
                        \end{tabular}
                                \label{tab:xenoIsolMinMax}
                    \end{table}
                    
            \subsubsection{Chaine de tâches avec système complet sans Contrôle}
                        At this point, we defined our task set, composed of the LO-criticality tasks used as ``real" stress and the task chain made of 5 tasks~: 
                        \centerline{ $ FFT \rightarrow Bitcount  \rightarrow Basicmath  \rightarrow FFT^{-1} \rightarrow sha $.}
                        We need to verify the validity of our choice in term of schedulability and effectiveness of the LO-criticality tasks as interferences. Executing the whole task set together allow to verify both for this step~\circleTxt[3].
                         
                         %%We set the WCET with table phase 2 of our tasks in the task chains. We have~: % Formule
                    
                        %mettre une seule image (premiere en end to end) expliquer les differences (que c'est un chaines de taches) 
                        %reprend phase new 3 
                        %%% Creator: Matplotlib, PGF backend
%%
%% To include the figure in your LaTeX document, write
%%   \input{<filename>.pgf}
%%
%% Make sure the required packages are loaded in your preamble
%%   \usepackage{pgf}
%%
%% and, on pdftex
%%   \usepackage[utf8]{inputenc}\DeclareUnicodeCharacter{2212}{-}
%%
%% or, on luatex and xetex
%%   \usepackage{unicode-math}
%%
%% Figures using additional raster images can only be included by \input if
%% they are in the same directory as the main LaTeX file. For loading figures
%% from other directories you can use the `import` package
%%   \usepackage{import}
%%
%% and then include the figures with
%%   \import{<path to file>}{<filename>.pgf}
%%
%% Matplotlib used the following preamble
%%
\begingroup%
\makeatletter%
\begin{pgfpicture}%
\pgfpathrectangle{\pgfpointorigin}{\pgfqpoint{15.000000in}{5.000000in}}%
\pgfusepath{use as bounding box, clip}%
\begin{pgfscope}%
\pgfsetbuttcap%
\pgfsetmiterjoin%
\definecolor{currentfill}{rgb}{1.000000,1.000000,1.000000}%
\pgfsetfillcolor{currentfill}%
\pgfsetlinewidth{0.000000pt}%
\definecolor{currentstroke}{rgb}{1.000000,1.000000,1.000000}%
\pgfsetstrokecolor{currentstroke}%
\pgfsetdash{}{0pt}%
\pgfpathmoveto{\pgfqpoint{0.000000in}{0.000000in}}%
\pgfpathlineto{\pgfqpoint{15.000000in}{0.000000in}}%
\pgfpathlineto{\pgfqpoint{15.000000in}{5.000000in}}%
\pgfpathlineto{\pgfqpoint{0.000000in}{5.000000in}}%
\pgfpathclose%
\pgfusepath{fill}%
\end{pgfscope}%
\begin{pgfscope}%
\pgfsetbuttcap%
\pgfsetmiterjoin%
\definecolor{currentfill}{rgb}{1.000000,1.000000,1.000000}%
\pgfsetfillcolor{currentfill}%
\pgfsetlinewidth{0.000000pt}%
\definecolor{currentstroke}{rgb}{0.000000,0.000000,0.000000}%
\pgfsetstrokecolor{currentstroke}%
\pgfsetstrokeopacity{0.000000}%
\pgfsetdash{}{0pt}%
\pgfpathmoveto{\pgfqpoint{1.875000in}{0.550000in}}%
\pgfpathlineto{\pgfqpoint{4.402174in}{0.550000in}}%
\pgfpathlineto{\pgfqpoint{4.402174in}{4.400000in}}%
\pgfpathlineto{\pgfqpoint{1.875000in}{4.400000in}}%
\pgfpathclose%
\pgfusepath{fill}%
\end{pgfscope}%
\begin{pgfscope}%
\definecolor{textcolor}{rgb}{0.150000,0.150000,0.150000}%
\pgfsetstrokecolor{textcolor}%
\pgfsetfillcolor{textcolor}%
\pgftext[x=3.138587in,y=0.418056in,,top]{\color{textcolor}\sffamily\fontsize{11.000000}{13.200000}\selectfont End to end}%
\end{pgfscope}%
\begin{pgfscope}%
\pgfpathrectangle{\pgfqpoint{1.875000in}{0.550000in}}{\pgfqpoint{2.527174in}{3.850000in}}%
\pgfusepath{clip}%
\pgfsetroundcap%
\pgfsetroundjoin%
\pgfsetlinewidth{1.003750pt}%
\definecolor{currentstroke}{rgb}{0.800000,0.800000,0.800000}%
\pgfsetstrokecolor{currentstroke}%
\pgfsetdash{}{0pt}%
\pgfpathmoveto{\pgfqpoint{1.875000in}{0.681818in}}%
\pgfpathlineto{\pgfqpoint{4.402174in}{0.681818in}}%
\pgfusepath{stroke}%
\end{pgfscope}%
\begin{pgfscope}%
\definecolor{textcolor}{rgb}{0.150000,0.150000,0.150000}%
\pgfsetstrokecolor{textcolor}%
\pgfsetfillcolor{textcolor}%
\pgftext[x=1.667014in, y=0.629011in, left, base]{\color{textcolor}\sffamily\fontsize{11.000000}{13.200000}\selectfont \(\displaystyle {0}\)}%
\end{pgfscope}%
\begin{pgfscope}%
\pgfpathrectangle{\pgfqpoint{1.875000in}{0.550000in}}{\pgfqpoint{2.527174in}{3.850000in}}%
\pgfusepath{clip}%
\pgfsetroundcap%
\pgfsetroundjoin%
\pgfsetlinewidth{1.003750pt}%
\definecolor{currentstroke}{rgb}{0.800000,0.800000,0.800000}%
\pgfsetstrokecolor{currentstroke}%
\pgfsetdash{}{0pt}%
\pgfpathmoveto{\pgfqpoint{1.875000in}{1.285136in}}%
\pgfpathlineto{\pgfqpoint{4.402174in}{1.285136in}}%
\pgfusepath{stroke}%
\end{pgfscope}%
\begin{pgfscope}%
\definecolor{textcolor}{rgb}{0.150000,0.150000,0.150000}%
\pgfsetstrokecolor{textcolor}%
\pgfsetfillcolor{textcolor}%
\pgftext[x=1.590972in, y=1.232330in, left, base]{\color{textcolor}\sffamily\fontsize{11.000000}{13.200000}\selectfont \(\displaystyle {50}\)}%
\end{pgfscope}%
\begin{pgfscope}%
\pgfpathrectangle{\pgfqpoint{1.875000in}{0.550000in}}{\pgfqpoint{2.527174in}{3.850000in}}%
\pgfusepath{clip}%
\pgfsetroundcap%
\pgfsetroundjoin%
\pgfsetlinewidth{1.003750pt}%
\definecolor{currentstroke}{rgb}{0.800000,0.800000,0.800000}%
\pgfsetstrokecolor{currentstroke}%
\pgfsetdash{}{0pt}%
\pgfpathmoveto{\pgfqpoint{1.875000in}{1.888455in}}%
\pgfpathlineto{\pgfqpoint{4.402174in}{1.888455in}}%
\pgfusepath{stroke}%
\end{pgfscope}%
\begin{pgfscope}%
\definecolor{textcolor}{rgb}{0.150000,0.150000,0.150000}%
\pgfsetstrokecolor{textcolor}%
\pgfsetfillcolor{textcolor}%
\pgftext[x=1.514930in, y=1.835648in, left, base]{\color{textcolor}\sffamily\fontsize{11.000000}{13.200000}\selectfont \(\displaystyle {100}\)}%
\end{pgfscope}%
\begin{pgfscope}%
\pgfpathrectangle{\pgfqpoint{1.875000in}{0.550000in}}{\pgfqpoint{2.527174in}{3.850000in}}%
\pgfusepath{clip}%
\pgfsetroundcap%
\pgfsetroundjoin%
\pgfsetlinewidth{1.003750pt}%
\definecolor{currentstroke}{rgb}{0.800000,0.800000,0.800000}%
\pgfsetstrokecolor{currentstroke}%
\pgfsetdash{}{0pt}%
\pgfpathmoveto{\pgfqpoint{1.875000in}{2.491773in}}%
\pgfpathlineto{\pgfqpoint{4.402174in}{2.491773in}}%
\pgfusepath{stroke}%
\end{pgfscope}%
\begin{pgfscope}%
\definecolor{textcolor}{rgb}{0.150000,0.150000,0.150000}%
\pgfsetstrokecolor{textcolor}%
\pgfsetfillcolor{textcolor}%
\pgftext[x=1.514930in, y=2.438967in, left, base]{\color{textcolor}\sffamily\fontsize{11.000000}{13.200000}\selectfont \(\displaystyle {150}\)}%
\end{pgfscope}%
\begin{pgfscope}%
\pgfpathrectangle{\pgfqpoint{1.875000in}{0.550000in}}{\pgfqpoint{2.527174in}{3.850000in}}%
\pgfusepath{clip}%
\pgfsetroundcap%
\pgfsetroundjoin%
\pgfsetlinewidth{1.003750pt}%
\definecolor{currentstroke}{rgb}{0.800000,0.800000,0.800000}%
\pgfsetstrokecolor{currentstroke}%
\pgfsetdash{}{0pt}%
\pgfpathmoveto{\pgfqpoint{1.875000in}{3.095092in}}%
\pgfpathlineto{\pgfqpoint{4.402174in}{3.095092in}}%
\pgfusepath{stroke}%
\end{pgfscope}%
\begin{pgfscope}%
\definecolor{textcolor}{rgb}{0.150000,0.150000,0.150000}%
\pgfsetstrokecolor{textcolor}%
\pgfsetfillcolor{textcolor}%
\pgftext[x=1.514930in, y=3.042285in, left, base]{\color{textcolor}\sffamily\fontsize{11.000000}{13.200000}\selectfont \(\displaystyle {200}\)}%
\end{pgfscope}%
\begin{pgfscope}%
\pgfpathrectangle{\pgfqpoint{1.875000in}{0.550000in}}{\pgfqpoint{2.527174in}{3.850000in}}%
\pgfusepath{clip}%
\pgfsetroundcap%
\pgfsetroundjoin%
\pgfsetlinewidth{1.003750pt}%
\definecolor{currentstroke}{rgb}{0.800000,0.800000,0.800000}%
\pgfsetstrokecolor{currentstroke}%
\pgfsetdash{}{0pt}%
\pgfpathmoveto{\pgfqpoint{1.875000in}{3.698410in}}%
\pgfpathlineto{\pgfqpoint{4.402174in}{3.698410in}}%
\pgfusepath{stroke}%
\end{pgfscope}%
\begin{pgfscope}%
\definecolor{textcolor}{rgb}{0.150000,0.150000,0.150000}%
\pgfsetstrokecolor{textcolor}%
\pgfsetfillcolor{textcolor}%
\pgftext[x=1.514930in, y=3.645604in, left, base]{\color{textcolor}\sffamily\fontsize{11.000000}{13.200000}\selectfont \(\displaystyle {250}\)}%
\end{pgfscope}%
\begin{pgfscope}%
\pgfpathrectangle{\pgfqpoint{1.875000in}{0.550000in}}{\pgfqpoint{2.527174in}{3.850000in}}%
\pgfusepath{clip}%
\pgfsetroundcap%
\pgfsetroundjoin%
\pgfsetlinewidth{1.003750pt}%
\definecolor{currentstroke}{rgb}{0.800000,0.800000,0.800000}%
\pgfsetstrokecolor{currentstroke}%
\pgfsetdash{}{0pt}%
\pgfpathmoveto{\pgfqpoint{1.875000in}{4.301729in}}%
\pgfpathlineto{\pgfqpoint{4.402174in}{4.301729in}}%
\pgfusepath{stroke}%
\end{pgfscope}%
\begin{pgfscope}%
\definecolor{textcolor}{rgb}{0.150000,0.150000,0.150000}%
\pgfsetstrokecolor{textcolor}%
\pgfsetfillcolor{textcolor}%
\pgftext[x=1.514930in, y=4.248922in, left, base]{\color{textcolor}\sffamily\fontsize{11.000000}{13.200000}\selectfont \(\displaystyle {300}\)}%
\end{pgfscope}%
\begin{pgfscope}%
\definecolor{textcolor}{rgb}{0.150000,0.150000,0.150000}%
\pgfsetstrokecolor{textcolor}%
\pgfsetfillcolor{textcolor}%
\pgftext[x=1.459375in,y=2.475000in,,bottom,rotate=90.000000]{\color{textcolor}\sffamily\fontsize{12.000000}{14.400000}\selectfont Time (ms)}%
\end{pgfscope}%
\begin{pgfscope}%
\pgfpathrectangle{\pgfqpoint{1.875000in}{0.550000in}}{\pgfqpoint{2.527174in}{3.850000in}}%
\pgfusepath{clip}%
\pgfsetbuttcap%
\pgfsetroundjoin%
\definecolor{currentfill}{rgb}{0.347059,0.458824,0.641176}%
\pgfsetfillcolor{currentfill}%
\pgfsetlinewidth{1.505625pt}%
\definecolor{currentstroke}{rgb}{0.298039,0.298039,0.298039}%
\pgfsetstrokecolor{currentstroke}%
\pgfsetdash{}{0pt}%
\pgfsys@defobject{currentmarker}{\pgfqpoint{2.127717in}{1.015224in}}{\pgfqpoint{3.138587in}{1.282763in}}{%
\pgfpathmoveto{\pgfqpoint{3.138587in}{1.015224in}}%
\pgfpathlineto{\pgfqpoint{3.137302in}{1.015224in}}%
\pgfpathlineto{\pgfqpoint{3.136874in}{1.017927in}}%
\pgfpathlineto{\pgfqpoint{3.136778in}{1.020629in}}%
\pgfpathlineto{\pgfqpoint{3.135111in}{1.023332in}}%
\pgfpathlineto{\pgfqpoint{3.118343in}{1.026034in}}%
\pgfpathlineto{\pgfqpoint{3.022351in}{1.028736in}}%
\pgfpathlineto{\pgfqpoint{2.723061in}{1.031439in}}%
\pgfpathlineto{\pgfqpoint{2.280756in}{1.034141in}}%
\pgfpathlineto{\pgfqpoint{2.127717in}{1.036844in}}%
\pgfpathlineto{\pgfqpoint{2.458250in}{1.039546in}}%
\pgfpathlineto{\pgfqpoint{2.875882in}{1.042249in}}%
\pgfpathlineto{\pgfqpoint{3.079806in}{1.044951in}}%
\pgfpathlineto{\pgfqpoint{3.130779in}{1.047653in}}%
\pgfpathlineto{\pgfqpoint{3.137936in}{1.050356in}}%
\pgfpathlineto{\pgfqpoint{3.138549in}{1.053058in}}%
\pgfpathlineto{\pgfqpoint{3.138585in}{1.055761in}}%
\pgfpathlineto{\pgfqpoint{3.138587in}{1.058463in}}%
\pgfpathlineto{\pgfqpoint{3.138587in}{1.061165in}}%
\pgfpathlineto{\pgfqpoint{3.138587in}{1.063868in}}%
\pgfpathlineto{\pgfqpoint{3.138587in}{1.066570in}}%
\pgfpathlineto{\pgfqpoint{3.138587in}{1.069273in}}%
\pgfpathlineto{\pgfqpoint{3.138587in}{1.071975in}}%
\pgfpathlineto{\pgfqpoint{3.138587in}{1.074677in}}%
\pgfpathlineto{\pgfqpoint{3.138587in}{1.077380in}}%
\pgfpathlineto{\pgfqpoint{3.138587in}{1.080082in}}%
\pgfpathlineto{\pgfqpoint{3.138587in}{1.082785in}}%
\pgfpathlineto{\pgfqpoint{3.138587in}{1.085487in}}%
\pgfpathlineto{\pgfqpoint{3.138587in}{1.088190in}}%
\pgfpathlineto{\pgfqpoint{3.138587in}{1.090892in}}%
\pgfpathlineto{\pgfqpoint{3.138587in}{1.093594in}}%
\pgfpathlineto{\pgfqpoint{3.138587in}{1.096297in}}%
\pgfpathlineto{\pgfqpoint{3.138587in}{1.098999in}}%
\pgfpathlineto{\pgfqpoint{3.138587in}{1.101702in}}%
\pgfpathlineto{\pgfqpoint{3.138587in}{1.104404in}}%
\pgfpathlineto{\pgfqpoint{3.138587in}{1.107106in}}%
\pgfpathlineto{\pgfqpoint{3.138587in}{1.109809in}}%
\pgfpathlineto{\pgfqpoint{3.138587in}{1.112511in}}%
\pgfpathlineto{\pgfqpoint{3.138587in}{1.115214in}}%
\pgfpathlineto{\pgfqpoint{3.138587in}{1.117916in}}%
\pgfpathlineto{\pgfqpoint{3.138587in}{1.120618in}}%
\pgfpathlineto{\pgfqpoint{3.138587in}{1.123321in}}%
\pgfpathlineto{\pgfqpoint{3.138587in}{1.126023in}}%
\pgfpathlineto{\pgfqpoint{3.138587in}{1.128726in}}%
\pgfpathlineto{\pgfqpoint{3.138587in}{1.131428in}}%
\pgfpathlineto{\pgfqpoint{3.138587in}{1.134131in}}%
\pgfpathlineto{\pgfqpoint{3.138587in}{1.136833in}}%
\pgfpathlineto{\pgfqpoint{3.138587in}{1.139535in}}%
\pgfpathlineto{\pgfqpoint{3.138587in}{1.142238in}}%
\pgfpathlineto{\pgfqpoint{3.138587in}{1.144940in}}%
\pgfpathlineto{\pgfqpoint{3.138587in}{1.147643in}}%
\pgfpathlineto{\pgfqpoint{3.138587in}{1.150345in}}%
\pgfpathlineto{\pgfqpoint{3.138587in}{1.153047in}}%
\pgfpathlineto{\pgfqpoint{3.138587in}{1.155750in}}%
\pgfpathlineto{\pgfqpoint{3.138587in}{1.158452in}}%
\pgfpathlineto{\pgfqpoint{3.138587in}{1.161155in}}%
\pgfpathlineto{\pgfqpoint{3.138587in}{1.163857in}}%
\pgfpathlineto{\pgfqpoint{3.138587in}{1.166559in}}%
\pgfpathlineto{\pgfqpoint{3.138587in}{1.169262in}}%
\pgfpathlineto{\pgfqpoint{3.138587in}{1.171964in}}%
\pgfpathlineto{\pgfqpoint{3.138587in}{1.174667in}}%
\pgfpathlineto{\pgfqpoint{3.138587in}{1.177369in}}%
\pgfpathlineto{\pgfqpoint{3.138587in}{1.180072in}}%
\pgfpathlineto{\pgfqpoint{3.138587in}{1.182774in}}%
\pgfpathlineto{\pgfqpoint{3.138587in}{1.185476in}}%
\pgfpathlineto{\pgfqpoint{3.138587in}{1.188179in}}%
\pgfpathlineto{\pgfqpoint{3.138587in}{1.190881in}}%
\pgfpathlineto{\pgfqpoint{3.138587in}{1.193584in}}%
\pgfpathlineto{\pgfqpoint{3.138587in}{1.196286in}}%
\pgfpathlineto{\pgfqpoint{3.138587in}{1.198988in}}%
\pgfpathlineto{\pgfqpoint{3.138587in}{1.201691in}}%
\pgfpathlineto{\pgfqpoint{3.138587in}{1.204393in}}%
\pgfpathlineto{\pgfqpoint{3.138587in}{1.207096in}}%
\pgfpathlineto{\pgfqpoint{3.138587in}{1.209798in}}%
\pgfpathlineto{\pgfqpoint{3.138587in}{1.212500in}}%
\pgfpathlineto{\pgfqpoint{3.138587in}{1.215203in}}%
\pgfpathlineto{\pgfqpoint{3.138587in}{1.217905in}}%
\pgfpathlineto{\pgfqpoint{3.138587in}{1.220608in}}%
\pgfpathlineto{\pgfqpoint{3.138587in}{1.223310in}}%
\pgfpathlineto{\pgfqpoint{3.138587in}{1.226013in}}%
\pgfpathlineto{\pgfqpoint{3.138587in}{1.228715in}}%
\pgfpathlineto{\pgfqpoint{3.138587in}{1.231417in}}%
\pgfpathlineto{\pgfqpoint{3.138587in}{1.234120in}}%
\pgfpathlineto{\pgfqpoint{3.138587in}{1.236822in}}%
\pgfpathlineto{\pgfqpoint{3.138587in}{1.239525in}}%
\pgfpathlineto{\pgfqpoint{3.138587in}{1.242227in}}%
\pgfpathlineto{\pgfqpoint{3.138587in}{1.244929in}}%
\pgfpathlineto{\pgfqpoint{3.138587in}{1.247632in}}%
\pgfpathlineto{\pgfqpoint{3.138587in}{1.250334in}}%
\pgfpathlineto{\pgfqpoint{3.138587in}{1.253037in}}%
\pgfpathlineto{\pgfqpoint{3.138587in}{1.255739in}}%
\pgfpathlineto{\pgfqpoint{3.138587in}{1.258441in}}%
\pgfpathlineto{\pgfqpoint{3.138587in}{1.261144in}}%
\pgfpathlineto{\pgfqpoint{3.138582in}{1.263846in}}%
\pgfpathlineto{\pgfqpoint{3.138527in}{1.266549in}}%
\pgfpathlineto{\pgfqpoint{3.138161in}{1.269251in}}%
\pgfpathlineto{\pgfqpoint{3.136831in}{1.271954in}}%
\pgfpathlineto{\pgfqpoint{3.134390in}{1.274656in}}%
\pgfpathlineto{\pgfqpoint{3.132758in}{1.277358in}}%
\pgfpathlineto{\pgfqpoint{3.133837in}{1.280061in}}%
\pgfpathlineto{\pgfqpoint{3.136253in}{1.282763in}}%
\pgfpathlineto{\pgfqpoint{3.138587in}{1.282763in}}%
\pgfpathlineto{\pgfqpoint{3.138587in}{1.282763in}}%
\pgfpathlineto{\pgfqpoint{3.138587in}{1.280061in}}%
\pgfpathlineto{\pgfqpoint{3.138587in}{1.277358in}}%
\pgfpathlineto{\pgfqpoint{3.138587in}{1.274656in}}%
\pgfpathlineto{\pgfqpoint{3.138587in}{1.271954in}}%
\pgfpathlineto{\pgfqpoint{3.138587in}{1.269251in}}%
\pgfpathlineto{\pgfqpoint{3.138587in}{1.266549in}}%
\pgfpathlineto{\pgfqpoint{3.138587in}{1.263846in}}%
\pgfpathlineto{\pgfqpoint{3.138587in}{1.261144in}}%
\pgfpathlineto{\pgfqpoint{3.138587in}{1.258441in}}%
\pgfpathlineto{\pgfqpoint{3.138587in}{1.255739in}}%
\pgfpathlineto{\pgfqpoint{3.138587in}{1.253037in}}%
\pgfpathlineto{\pgfqpoint{3.138587in}{1.250334in}}%
\pgfpathlineto{\pgfqpoint{3.138587in}{1.247632in}}%
\pgfpathlineto{\pgfqpoint{3.138587in}{1.244929in}}%
\pgfpathlineto{\pgfqpoint{3.138587in}{1.242227in}}%
\pgfpathlineto{\pgfqpoint{3.138587in}{1.239525in}}%
\pgfpathlineto{\pgfqpoint{3.138587in}{1.236822in}}%
\pgfpathlineto{\pgfqpoint{3.138587in}{1.234120in}}%
\pgfpathlineto{\pgfqpoint{3.138587in}{1.231417in}}%
\pgfpathlineto{\pgfqpoint{3.138587in}{1.228715in}}%
\pgfpathlineto{\pgfqpoint{3.138587in}{1.226013in}}%
\pgfpathlineto{\pgfqpoint{3.138587in}{1.223310in}}%
\pgfpathlineto{\pgfqpoint{3.138587in}{1.220608in}}%
\pgfpathlineto{\pgfqpoint{3.138587in}{1.217905in}}%
\pgfpathlineto{\pgfqpoint{3.138587in}{1.215203in}}%
\pgfpathlineto{\pgfqpoint{3.138587in}{1.212500in}}%
\pgfpathlineto{\pgfqpoint{3.138587in}{1.209798in}}%
\pgfpathlineto{\pgfqpoint{3.138587in}{1.207096in}}%
\pgfpathlineto{\pgfqpoint{3.138587in}{1.204393in}}%
\pgfpathlineto{\pgfqpoint{3.138587in}{1.201691in}}%
\pgfpathlineto{\pgfqpoint{3.138587in}{1.198988in}}%
\pgfpathlineto{\pgfqpoint{3.138587in}{1.196286in}}%
\pgfpathlineto{\pgfqpoint{3.138587in}{1.193584in}}%
\pgfpathlineto{\pgfqpoint{3.138587in}{1.190881in}}%
\pgfpathlineto{\pgfqpoint{3.138587in}{1.188179in}}%
\pgfpathlineto{\pgfqpoint{3.138587in}{1.185476in}}%
\pgfpathlineto{\pgfqpoint{3.138587in}{1.182774in}}%
\pgfpathlineto{\pgfqpoint{3.138587in}{1.180072in}}%
\pgfpathlineto{\pgfqpoint{3.138587in}{1.177369in}}%
\pgfpathlineto{\pgfqpoint{3.138587in}{1.174667in}}%
\pgfpathlineto{\pgfqpoint{3.138587in}{1.171964in}}%
\pgfpathlineto{\pgfqpoint{3.138587in}{1.169262in}}%
\pgfpathlineto{\pgfqpoint{3.138587in}{1.166559in}}%
\pgfpathlineto{\pgfqpoint{3.138587in}{1.163857in}}%
\pgfpathlineto{\pgfqpoint{3.138587in}{1.161155in}}%
\pgfpathlineto{\pgfqpoint{3.138587in}{1.158452in}}%
\pgfpathlineto{\pgfqpoint{3.138587in}{1.155750in}}%
\pgfpathlineto{\pgfqpoint{3.138587in}{1.153047in}}%
\pgfpathlineto{\pgfqpoint{3.138587in}{1.150345in}}%
\pgfpathlineto{\pgfqpoint{3.138587in}{1.147643in}}%
\pgfpathlineto{\pgfqpoint{3.138587in}{1.144940in}}%
\pgfpathlineto{\pgfqpoint{3.138587in}{1.142238in}}%
\pgfpathlineto{\pgfqpoint{3.138587in}{1.139535in}}%
\pgfpathlineto{\pgfqpoint{3.138587in}{1.136833in}}%
\pgfpathlineto{\pgfqpoint{3.138587in}{1.134131in}}%
\pgfpathlineto{\pgfqpoint{3.138587in}{1.131428in}}%
\pgfpathlineto{\pgfqpoint{3.138587in}{1.128726in}}%
\pgfpathlineto{\pgfqpoint{3.138587in}{1.126023in}}%
\pgfpathlineto{\pgfqpoint{3.138587in}{1.123321in}}%
\pgfpathlineto{\pgfqpoint{3.138587in}{1.120618in}}%
\pgfpathlineto{\pgfqpoint{3.138587in}{1.117916in}}%
\pgfpathlineto{\pgfqpoint{3.138587in}{1.115214in}}%
\pgfpathlineto{\pgfqpoint{3.138587in}{1.112511in}}%
\pgfpathlineto{\pgfqpoint{3.138587in}{1.109809in}}%
\pgfpathlineto{\pgfqpoint{3.138587in}{1.107106in}}%
\pgfpathlineto{\pgfqpoint{3.138587in}{1.104404in}}%
\pgfpathlineto{\pgfqpoint{3.138587in}{1.101702in}}%
\pgfpathlineto{\pgfqpoint{3.138587in}{1.098999in}}%
\pgfpathlineto{\pgfqpoint{3.138587in}{1.096297in}}%
\pgfpathlineto{\pgfqpoint{3.138587in}{1.093594in}}%
\pgfpathlineto{\pgfqpoint{3.138587in}{1.090892in}}%
\pgfpathlineto{\pgfqpoint{3.138587in}{1.088190in}}%
\pgfpathlineto{\pgfqpoint{3.138587in}{1.085487in}}%
\pgfpathlineto{\pgfqpoint{3.138587in}{1.082785in}}%
\pgfpathlineto{\pgfqpoint{3.138587in}{1.080082in}}%
\pgfpathlineto{\pgfqpoint{3.138587in}{1.077380in}}%
\pgfpathlineto{\pgfqpoint{3.138587in}{1.074677in}}%
\pgfpathlineto{\pgfqpoint{3.138587in}{1.071975in}}%
\pgfpathlineto{\pgfqpoint{3.138587in}{1.069273in}}%
\pgfpathlineto{\pgfqpoint{3.138587in}{1.066570in}}%
\pgfpathlineto{\pgfqpoint{3.138587in}{1.063868in}}%
\pgfpathlineto{\pgfqpoint{3.138587in}{1.061165in}}%
\pgfpathlineto{\pgfqpoint{3.138587in}{1.058463in}}%
\pgfpathlineto{\pgfqpoint{3.138587in}{1.055761in}}%
\pgfpathlineto{\pgfqpoint{3.138587in}{1.053058in}}%
\pgfpathlineto{\pgfqpoint{3.138587in}{1.050356in}}%
\pgfpathlineto{\pgfqpoint{3.138587in}{1.047653in}}%
\pgfpathlineto{\pgfqpoint{3.138587in}{1.044951in}}%
\pgfpathlineto{\pgfqpoint{3.138587in}{1.042249in}}%
\pgfpathlineto{\pgfqpoint{3.138587in}{1.039546in}}%
\pgfpathlineto{\pgfqpoint{3.138587in}{1.036844in}}%
\pgfpathlineto{\pgfqpoint{3.138587in}{1.034141in}}%
\pgfpathlineto{\pgfqpoint{3.138587in}{1.031439in}}%
\pgfpathlineto{\pgfqpoint{3.138587in}{1.028736in}}%
\pgfpathlineto{\pgfqpoint{3.138587in}{1.026034in}}%
\pgfpathlineto{\pgfqpoint{3.138587in}{1.023332in}}%
\pgfpathlineto{\pgfqpoint{3.138587in}{1.020629in}}%
\pgfpathlineto{\pgfqpoint{3.138587in}{1.017927in}}%
\pgfpathlineto{\pgfqpoint{3.138587in}{1.015224in}}%
\pgfpathclose%
\pgfusepath{stroke,fill}%
}%
\begin{pgfscope}%
\pgfsys@transformshift{0.000000in}{0.000000in}%
\pgfsys@useobject{currentmarker}{}%
\end{pgfscope}%
\end{pgfscope}%
\begin{pgfscope}%
\pgfpathrectangle{\pgfqpoint{1.875000in}{0.550000in}}{\pgfqpoint{2.527174in}{3.850000in}}%
\pgfusepath{clip}%
\pgfsetbuttcap%
\pgfsetroundjoin%
\definecolor{currentfill}{rgb}{0.798529,0.536765,0.389706}%
\pgfsetfillcolor{currentfill}%
\pgfsetlinewidth{1.505625pt}%
\definecolor{currentstroke}{rgb}{0.298039,0.298039,0.298039}%
\pgfsetstrokecolor{currentstroke}%
\pgfsetdash{}{0pt}%
\pgfsys@defobject{currentmarker}{\pgfqpoint{3.138587in}{1.036574in}}{\pgfqpoint{3.694368in}{4.225000in}}{%
\pgfpathmoveto{\pgfqpoint{3.197626in}{1.036574in}}%
\pgfpathlineto{\pgfqpoint{3.138587in}{1.036574in}}%
\pgfpathlineto{\pgfqpoint{3.138587in}{1.068780in}}%
\pgfpathlineto{\pgfqpoint{3.138587in}{1.100986in}}%
\pgfpathlineto{\pgfqpoint{3.138587in}{1.133193in}}%
\pgfpathlineto{\pgfqpoint{3.138587in}{1.165399in}}%
\pgfpathlineto{\pgfqpoint{3.138587in}{1.197605in}}%
\pgfpathlineto{\pgfqpoint{3.138587in}{1.229812in}}%
\pgfpathlineto{\pgfqpoint{3.138587in}{1.262018in}}%
\pgfpathlineto{\pgfqpoint{3.138587in}{1.294224in}}%
\pgfpathlineto{\pgfqpoint{3.138587in}{1.326431in}}%
\pgfpathlineto{\pgfqpoint{3.138587in}{1.358637in}}%
\pgfpathlineto{\pgfqpoint{3.138587in}{1.390843in}}%
\pgfpathlineto{\pgfqpoint{3.138587in}{1.423050in}}%
\pgfpathlineto{\pgfqpoint{3.138587in}{1.455256in}}%
\pgfpathlineto{\pgfqpoint{3.138587in}{1.487462in}}%
\pgfpathlineto{\pgfqpoint{3.138587in}{1.519669in}}%
\pgfpathlineto{\pgfqpoint{3.138587in}{1.551875in}}%
\pgfpathlineto{\pgfqpoint{3.138587in}{1.584081in}}%
\pgfpathlineto{\pgfqpoint{3.138587in}{1.616287in}}%
\pgfpathlineto{\pgfqpoint{3.138587in}{1.648494in}}%
\pgfpathlineto{\pgfqpoint{3.138587in}{1.680700in}}%
\pgfpathlineto{\pgfqpoint{3.138587in}{1.712906in}}%
\pgfpathlineto{\pgfqpoint{3.138587in}{1.745113in}}%
\pgfpathlineto{\pgfqpoint{3.138587in}{1.777319in}}%
\pgfpathlineto{\pgfqpoint{3.138587in}{1.809525in}}%
\pgfpathlineto{\pgfqpoint{3.138587in}{1.841732in}}%
\pgfpathlineto{\pgfqpoint{3.138587in}{1.873938in}}%
\pgfpathlineto{\pgfqpoint{3.138587in}{1.906144in}}%
\pgfpathlineto{\pgfqpoint{3.138587in}{1.938351in}}%
\pgfpathlineto{\pgfqpoint{3.138587in}{1.970557in}}%
\pgfpathlineto{\pgfqpoint{3.138587in}{2.002763in}}%
\pgfpathlineto{\pgfqpoint{3.138587in}{2.034970in}}%
\pgfpathlineto{\pgfqpoint{3.138587in}{2.067176in}}%
\pgfpathlineto{\pgfqpoint{3.138587in}{2.099382in}}%
\pgfpathlineto{\pgfqpoint{3.138587in}{2.131589in}}%
\pgfpathlineto{\pgfqpoint{3.138587in}{2.163795in}}%
\pgfpathlineto{\pgfqpoint{3.138587in}{2.196001in}}%
\pgfpathlineto{\pgfqpoint{3.138587in}{2.228208in}}%
\pgfpathlineto{\pgfqpoint{3.138587in}{2.260414in}}%
\pgfpathlineto{\pgfqpoint{3.138587in}{2.292620in}}%
\pgfpathlineto{\pgfqpoint{3.138587in}{2.324827in}}%
\pgfpathlineto{\pgfqpoint{3.138587in}{2.357033in}}%
\pgfpathlineto{\pgfqpoint{3.138587in}{2.389239in}}%
\pgfpathlineto{\pgfqpoint{3.138587in}{2.421446in}}%
\pgfpathlineto{\pgfqpoint{3.138587in}{2.453652in}}%
\pgfpathlineto{\pgfqpoint{3.138587in}{2.485858in}}%
\pgfpathlineto{\pgfqpoint{3.138587in}{2.518065in}}%
\pgfpathlineto{\pgfqpoint{3.138587in}{2.550271in}}%
\pgfpathlineto{\pgfqpoint{3.138587in}{2.582477in}}%
\pgfpathlineto{\pgfqpoint{3.138587in}{2.614684in}}%
\pgfpathlineto{\pgfqpoint{3.138587in}{2.646890in}}%
\pgfpathlineto{\pgfqpoint{3.138587in}{2.679096in}}%
\pgfpathlineto{\pgfqpoint{3.138587in}{2.711303in}}%
\pgfpathlineto{\pgfqpoint{3.138587in}{2.743509in}}%
\pgfpathlineto{\pgfqpoint{3.138587in}{2.775715in}}%
\pgfpathlineto{\pgfqpoint{3.138587in}{2.807922in}}%
\pgfpathlineto{\pgfqpoint{3.138587in}{2.840128in}}%
\pgfpathlineto{\pgfqpoint{3.138587in}{2.872334in}}%
\pgfpathlineto{\pgfqpoint{3.138587in}{2.904541in}}%
\pgfpathlineto{\pgfqpoint{3.138587in}{2.936747in}}%
\pgfpathlineto{\pgfqpoint{3.138587in}{2.968953in}}%
\pgfpathlineto{\pgfqpoint{3.138587in}{3.001160in}}%
\pgfpathlineto{\pgfqpoint{3.138587in}{3.033366in}}%
\pgfpathlineto{\pgfqpoint{3.138587in}{3.065572in}}%
\pgfpathlineto{\pgfqpoint{3.138587in}{3.097779in}}%
\pgfpathlineto{\pgfqpoint{3.138587in}{3.129985in}}%
\pgfpathlineto{\pgfqpoint{3.138587in}{3.162191in}}%
\pgfpathlineto{\pgfqpoint{3.138587in}{3.194398in}}%
\pgfpathlineto{\pgfqpoint{3.138587in}{3.226604in}}%
\pgfpathlineto{\pgfqpoint{3.138587in}{3.258810in}}%
\pgfpathlineto{\pgfqpoint{3.138587in}{3.291017in}}%
\pgfpathlineto{\pgfqpoint{3.138587in}{3.323223in}}%
\pgfpathlineto{\pgfqpoint{3.138587in}{3.355429in}}%
\pgfpathlineto{\pgfqpoint{3.138587in}{3.387635in}}%
\pgfpathlineto{\pgfqpoint{3.138587in}{3.419842in}}%
\pgfpathlineto{\pgfqpoint{3.138587in}{3.452048in}}%
\pgfpathlineto{\pgfqpoint{3.138587in}{3.484254in}}%
\pgfpathlineto{\pgfqpoint{3.138587in}{3.516461in}}%
\pgfpathlineto{\pgfqpoint{3.138587in}{3.548667in}}%
\pgfpathlineto{\pgfqpoint{3.138587in}{3.580873in}}%
\pgfpathlineto{\pgfqpoint{3.138587in}{3.613080in}}%
\pgfpathlineto{\pgfqpoint{3.138587in}{3.645286in}}%
\pgfpathlineto{\pgfqpoint{3.138587in}{3.677492in}}%
\pgfpathlineto{\pgfqpoint{3.138587in}{3.709699in}}%
\pgfpathlineto{\pgfqpoint{3.138587in}{3.741905in}}%
\pgfpathlineto{\pgfqpoint{3.138587in}{3.774111in}}%
\pgfpathlineto{\pgfqpoint{3.138587in}{3.806318in}}%
\pgfpathlineto{\pgfqpoint{3.138587in}{3.838524in}}%
\pgfpathlineto{\pgfqpoint{3.138587in}{3.870730in}}%
\pgfpathlineto{\pgfqpoint{3.138587in}{3.902937in}}%
\pgfpathlineto{\pgfqpoint{3.138587in}{3.935143in}}%
\pgfpathlineto{\pgfqpoint{3.138587in}{3.967349in}}%
\pgfpathlineto{\pgfqpoint{3.138587in}{3.999556in}}%
\pgfpathlineto{\pgfqpoint{3.138587in}{4.031762in}}%
\pgfpathlineto{\pgfqpoint{3.138587in}{4.063968in}}%
\pgfpathlineto{\pgfqpoint{3.138587in}{4.096175in}}%
\pgfpathlineto{\pgfqpoint{3.138587in}{4.128381in}}%
\pgfpathlineto{\pgfqpoint{3.138587in}{4.160587in}}%
\pgfpathlineto{\pgfqpoint{3.138587in}{4.192794in}}%
\pgfpathlineto{\pgfqpoint{3.138587in}{4.225000in}}%
\pgfpathlineto{\pgfqpoint{3.139553in}{4.225000in}}%
\pgfpathlineto{\pgfqpoint{3.139553in}{4.225000in}}%
\pgfpathlineto{\pgfqpoint{3.139508in}{4.192794in}}%
\pgfpathlineto{\pgfqpoint{3.139384in}{4.160587in}}%
\pgfpathlineto{\pgfqpoint{3.139215in}{4.128381in}}%
\pgfpathlineto{\pgfqpoint{3.139037in}{4.096175in}}%
\pgfpathlineto{\pgfqpoint{3.138883in}{4.063968in}}%
\pgfpathlineto{\pgfqpoint{3.138773in}{4.031762in}}%
\pgfpathlineto{\pgfqpoint{3.138712in}{3.999556in}}%
\pgfpathlineto{\pgfqpoint{3.138702in}{3.967349in}}%
\pgfpathlineto{\pgfqpoint{3.138743in}{3.935143in}}%
\pgfpathlineto{\pgfqpoint{3.138836in}{3.902937in}}%
\pgfpathlineto{\pgfqpoint{3.138975in}{3.870730in}}%
\pgfpathlineto{\pgfqpoint{3.139150in}{3.838524in}}%
\pgfpathlineto{\pgfqpoint{3.139335in}{3.806318in}}%
\pgfpathlineto{\pgfqpoint{3.139501in}{3.774111in}}%
\pgfpathlineto{\pgfqpoint{3.139627in}{3.741905in}}%
\pgfpathlineto{\pgfqpoint{3.139717in}{3.709699in}}%
\pgfpathlineto{\pgfqpoint{3.139817in}{3.677492in}}%
\pgfpathlineto{\pgfqpoint{3.140004in}{3.645286in}}%
\pgfpathlineto{\pgfqpoint{3.140377in}{3.613080in}}%
\pgfpathlineto{\pgfqpoint{3.141028in}{3.580873in}}%
\pgfpathlineto{\pgfqpoint{3.142021in}{3.548667in}}%
\pgfpathlineto{\pgfqpoint{3.143390in}{3.516461in}}%
\pgfpathlineto{\pgfqpoint{3.145149in}{3.484254in}}%
\pgfpathlineto{\pgfqpoint{3.147313in}{3.452048in}}%
\pgfpathlineto{\pgfqpoint{3.149890in}{3.419842in}}%
\pgfpathlineto{\pgfqpoint{3.152859in}{3.387635in}}%
\pgfpathlineto{\pgfqpoint{3.156133in}{3.355429in}}%
\pgfpathlineto{\pgfqpoint{3.159551in}{3.323223in}}%
\pgfpathlineto{\pgfqpoint{3.162917in}{3.291017in}}%
\pgfpathlineto{\pgfqpoint{3.166087in}{3.258810in}}%
\pgfpathlineto{\pgfqpoint{3.169045in}{3.226604in}}%
\pgfpathlineto{\pgfqpoint{3.171938in}{3.194398in}}%
\pgfpathlineto{\pgfqpoint{3.175028in}{3.162191in}}%
\pgfpathlineto{\pgfqpoint{3.178595in}{3.129985in}}%
\pgfpathlineto{\pgfqpoint{3.182830in}{3.097779in}}%
\pgfpathlineto{\pgfqpoint{3.187760in}{3.065572in}}%
\pgfpathlineto{\pgfqpoint{3.193239in}{3.033366in}}%
\pgfpathlineto{\pgfqpoint{3.199029in}{3.001160in}}%
\pgfpathlineto{\pgfqpoint{3.204926in}{2.968953in}}%
\pgfpathlineto{\pgfqpoint{3.210882in}{2.936747in}}%
\pgfpathlineto{\pgfqpoint{3.217041in}{2.904541in}}%
\pgfpathlineto{\pgfqpoint{3.223645in}{2.872334in}}%
\pgfpathlineto{\pgfqpoint{3.230840in}{2.840128in}}%
\pgfpathlineto{\pgfqpoint{3.238501in}{2.807922in}}%
\pgfpathlineto{\pgfqpoint{3.246174in}{2.775715in}}%
\pgfpathlineto{\pgfqpoint{3.253231in}{2.743509in}}%
\pgfpathlineto{\pgfqpoint{3.259158in}{2.711303in}}%
\pgfpathlineto{\pgfqpoint{3.263863in}{2.679096in}}%
\pgfpathlineto{\pgfqpoint{3.267766in}{2.646890in}}%
\pgfpathlineto{\pgfqpoint{3.271608in}{2.614684in}}%
\pgfpathlineto{\pgfqpoint{3.276042in}{2.582477in}}%
\pgfpathlineto{\pgfqpoint{3.281266in}{2.550271in}}%
\pgfpathlineto{\pgfqpoint{3.286970in}{2.518065in}}%
\pgfpathlineto{\pgfqpoint{3.292699in}{2.485858in}}%
\pgfpathlineto{\pgfqpoint{3.298421in}{2.453652in}}%
\pgfpathlineto{\pgfqpoint{3.304887in}{2.421446in}}%
\pgfpathlineto{\pgfqpoint{3.313443in}{2.389239in}}%
\pgfpathlineto{\pgfqpoint{3.325276in}{2.357033in}}%
\pgfpathlineto{\pgfqpoint{3.340492in}{2.324827in}}%
\pgfpathlineto{\pgfqpoint{3.357692in}{2.292620in}}%
\pgfpathlineto{\pgfqpoint{3.374469in}{2.260414in}}%
\pgfpathlineto{\pgfqpoint{3.388731in}{2.228208in}}%
\pgfpathlineto{\pgfqpoint{3.400076in}{2.196001in}}%
\pgfpathlineto{\pgfqpoint{3.410252in}{2.163795in}}%
\pgfpathlineto{\pgfqpoint{3.422150in}{2.131589in}}%
\pgfpathlineto{\pgfqpoint{3.437800in}{2.099382in}}%
\pgfpathlineto{\pgfqpoint{3.456659in}{2.067176in}}%
\pgfpathlineto{\pgfqpoint{3.475547in}{2.034970in}}%
\pgfpathlineto{\pgfqpoint{3.490606in}{2.002763in}}%
\pgfpathlineto{\pgfqpoint{3.500185in}{1.970557in}}%
\pgfpathlineto{\pgfqpoint{3.506653in}{1.938351in}}%
\pgfpathlineto{\pgfqpoint{3.515574in}{1.906144in}}%
\pgfpathlineto{\pgfqpoint{3.532365in}{1.873938in}}%
\pgfpathlineto{\pgfqpoint{3.558436in}{1.841732in}}%
\pgfpathlineto{\pgfqpoint{3.589548in}{1.809525in}}%
\pgfpathlineto{\pgfqpoint{3.617858in}{1.777319in}}%
\pgfpathlineto{\pgfqpoint{3.636683in}{1.745113in}}%
\pgfpathlineto{\pgfqpoint{3.644863in}{1.712906in}}%
\pgfpathlineto{\pgfqpoint{3.647590in}{1.680700in}}%
\pgfpathlineto{\pgfqpoint{3.652756in}{1.648494in}}%
\pgfpathlineto{\pgfqpoint{3.665072in}{1.616287in}}%
\pgfpathlineto{\pgfqpoint{3.681976in}{1.584081in}}%
\pgfpathlineto{\pgfqpoint{3.694368in}{1.551875in}}%
\pgfpathlineto{\pgfqpoint{3.691989in}{1.519669in}}%
\pgfpathlineto{\pgfqpoint{3.669956in}{1.487462in}}%
\pgfpathlineto{\pgfqpoint{3.632083in}{1.455256in}}%
\pgfpathlineto{\pgfqpoint{3.588761in}{1.423050in}}%
\pgfpathlineto{\pgfqpoint{3.550831in}{1.390843in}}%
\pgfpathlineto{\pgfqpoint{3.523471in}{1.358637in}}%
\pgfpathlineto{\pgfqpoint{3.503898in}{1.326431in}}%
\pgfpathlineto{\pgfqpoint{3.483978in}{1.294224in}}%
\pgfpathlineto{\pgfqpoint{3.455709in}{1.262018in}}%
\pgfpathlineto{\pgfqpoint{3.415952in}{1.229812in}}%
\pgfpathlineto{\pgfqpoint{3.367771in}{1.197605in}}%
\pgfpathlineto{\pgfqpoint{3.318163in}{1.165399in}}%
\pgfpathlineto{\pgfqpoint{3.274141in}{1.133193in}}%
\pgfpathlineto{\pgfqpoint{3.239693in}{1.100986in}}%
\pgfpathlineto{\pgfqpoint{3.214990in}{1.068780in}}%
\pgfpathlineto{\pgfqpoint{3.197626in}{1.036574in}}%
\pgfpathclose%
\pgfusepath{stroke,fill}%
}%
\begin{pgfscope}%
\pgfsys@transformshift{0.000000in}{0.000000in}%
\pgfsys@useobject{currentmarker}{}%
\end{pgfscope}%
\end{pgfscope}%
\begin{pgfscope}%
\pgfpathrectangle{\pgfqpoint{1.875000in}{0.550000in}}{\pgfqpoint{2.527174in}{3.850000in}}%
\pgfusepath{clip}%
\pgfsetbuttcap%
\pgfsetmiterjoin%
\definecolor{currentfill}{rgb}{0.347059,0.458824,0.641176}%
\pgfsetfillcolor{currentfill}%
\pgfsetlinewidth{0.752812pt}%
\definecolor{currentstroke}{rgb}{0.298039,0.298039,0.298039}%
\pgfsetstrokecolor{currentstroke}%
\pgfsetdash{}{0pt}%
\pgfpathmoveto{\pgfqpoint{3.138587in}{0.681818in}}%
\pgfpathlineto{\pgfqpoint{3.138587in}{0.681818in}}%
\pgfpathlineto{\pgfqpoint{3.138587in}{0.681818in}}%
\pgfpathlineto{\pgfqpoint{3.138587in}{0.681818in}}%
\pgfpathclose%
\pgfusepath{stroke,fill}%
\end{pgfscope}%
\begin{pgfscope}%
\pgfpathrectangle{\pgfqpoint{1.875000in}{0.550000in}}{\pgfqpoint{2.527174in}{3.850000in}}%
\pgfusepath{clip}%
\pgfsetbuttcap%
\pgfsetmiterjoin%
\definecolor{currentfill}{rgb}{0.798529,0.536765,0.389706}%
\pgfsetfillcolor{currentfill}%
\pgfsetlinewidth{0.752812pt}%
\definecolor{currentstroke}{rgb}{0.298039,0.298039,0.298039}%
\pgfsetstrokecolor{currentstroke}%
\pgfsetdash{}{0pt}%
\pgfpathmoveto{\pgfqpoint{3.138587in}{0.681818in}}%
\pgfpathlineto{\pgfqpoint{3.138587in}{0.681818in}}%
\pgfpathlineto{\pgfqpoint{3.138587in}{0.681818in}}%
\pgfpathlineto{\pgfqpoint{3.138587in}{0.681818in}}%
\pgfpathclose%
\pgfusepath{stroke,fill}%
\end{pgfscope}%
\begin{pgfscope}%
\pgfpathrectangle{\pgfqpoint{1.875000in}{0.550000in}}{\pgfqpoint{2.527174in}{3.850000in}}%
\pgfusepath{clip}%
\pgfsetroundcap%
\pgfsetroundjoin%
\pgfsetlinewidth{1.505625pt}%
\definecolor{currentstroke}{rgb}{0.298039,0.298039,0.298039}%
\pgfsetstrokecolor{currentstroke}%
\pgfsetdash{}{0pt}%
\pgfpathmoveto{\pgfqpoint{3.138587in}{1.015224in}}%
\pgfpathlineto{\pgfqpoint{3.138587in}{2.252299in}}%
\pgfusepath{stroke}%
\end{pgfscope}%
\begin{pgfscope}%
\pgfpathrectangle{\pgfqpoint{1.875000in}{0.550000in}}{\pgfqpoint{2.527174in}{3.850000in}}%
\pgfusepath{clip}%
\pgfsetroundcap%
\pgfsetroundjoin%
\pgfsetlinewidth{4.516875pt}%
\definecolor{currentstroke}{rgb}{0.298039,0.298039,0.298039}%
\pgfsetstrokecolor{currentstroke}%
\pgfsetdash{}{0pt}%
\pgfpathmoveto{\pgfqpoint{3.138587in}{1.035910in}}%
\pgfpathlineto{\pgfqpoint{3.138587in}{1.522524in}}%
\pgfusepath{stroke}%
\end{pgfscope}%
\begin{pgfscope}%
\pgfsetrectcap%
\pgfsetmiterjoin%
\pgfsetlinewidth{1.254687pt}%
\definecolor{currentstroke}{rgb}{0.800000,0.800000,0.800000}%
\pgfsetstrokecolor{currentstroke}%
\pgfsetdash{}{0pt}%
\pgfpathmoveto{\pgfqpoint{1.875000in}{0.550000in}}%
\pgfpathlineto{\pgfqpoint{1.875000in}{4.400000in}}%
\pgfusepath{stroke}%
\end{pgfscope}%
\begin{pgfscope}%
\pgfsetrectcap%
\pgfsetmiterjoin%
\pgfsetlinewidth{1.254687pt}%
\definecolor{currentstroke}{rgb}{0.800000,0.800000,0.800000}%
\pgfsetstrokecolor{currentstroke}%
\pgfsetdash{}{0pt}%
\pgfpathmoveto{\pgfqpoint{4.402174in}{0.550000in}}%
\pgfpathlineto{\pgfqpoint{4.402174in}{4.400000in}}%
\pgfusepath{stroke}%
\end{pgfscope}%
\begin{pgfscope}%
\pgfsetrectcap%
\pgfsetmiterjoin%
\pgfsetlinewidth{1.254687pt}%
\definecolor{currentstroke}{rgb}{0.800000,0.800000,0.800000}%
\pgfsetstrokecolor{currentstroke}%
\pgfsetdash{}{0pt}%
\pgfpathmoveto{\pgfqpoint{1.875000in}{0.550000in}}%
\pgfpathlineto{\pgfqpoint{4.402174in}{0.550000in}}%
\pgfusepath{stroke}%
\end{pgfscope}%
\begin{pgfscope}%
\pgfsetrectcap%
\pgfsetmiterjoin%
\pgfsetlinewidth{1.254687pt}%
\definecolor{currentstroke}{rgb}{0.800000,0.800000,0.800000}%
\pgfsetstrokecolor{currentstroke}%
\pgfsetdash{}{0pt}%
\pgfpathmoveto{\pgfqpoint{1.875000in}{4.400000in}}%
\pgfpathlineto{\pgfqpoint{4.402174in}{4.400000in}}%
\pgfusepath{stroke}%
\end{pgfscope}%
\begin{pgfscope}%
\pgfpathrectangle{\pgfqpoint{1.875000in}{0.550000in}}{\pgfqpoint{2.527174in}{3.850000in}}%
\pgfusepath{clip}%
\pgfsetbuttcap%
\pgfsetroundjoin%
\definecolor{currentfill}{rgb}{1.000000,1.000000,1.000000}%
\pgfsetfillcolor{currentfill}%
\pgfsetlinewidth{1.003750pt}%
\definecolor{currentstroke}{rgb}{0.298039,0.298039,0.298039}%
\pgfsetstrokecolor{currentstroke}%
\pgfsetdash{}{0pt}%
\pgfsys@defobject{currentmarker}{\pgfqpoint{-0.020833in}{-0.020833in}}{\pgfqpoint{0.020833in}{0.020833in}}{%
\pgfpathmoveto{\pgfqpoint{0.000000in}{-0.020833in}}%
\pgfpathcurveto{\pgfqpoint{0.005525in}{-0.020833in}}{\pgfqpoint{0.010825in}{-0.018638in}}{\pgfqpoint{0.014731in}{-0.014731in}}%
\pgfpathcurveto{\pgfqpoint{0.018638in}{-0.010825in}}{\pgfqpoint{0.020833in}{-0.005525in}}{\pgfqpoint{0.020833in}{0.000000in}}%
\pgfpathcurveto{\pgfqpoint{0.020833in}{0.005525in}}{\pgfqpoint{0.018638in}{0.010825in}}{\pgfqpoint{0.014731in}{0.014731in}}%
\pgfpathcurveto{\pgfqpoint{0.010825in}{0.018638in}}{\pgfqpoint{0.005525in}{0.020833in}}{\pgfqpoint{0.000000in}{0.020833in}}%
\pgfpathcurveto{\pgfqpoint{-0.005525in}{0.020833in}}{\pgfqpoint{-0.010825in}{0.018638in}}{\pgfqpoint{-0.014731in}{0.014731in}}%
\pgfpathcurveto{\pgfqpoint{-0.018638in}{0.010825in}}{\pgfqpoint{-0.020833in}{0.005525in}}{\pgfqpoint{-0.020833in}{0.000000in}}%
\pgfpathcurveto{\pgfqpoint{-0.020833in}{-0.005525in}}{\pgfqpoint{-0.018638in}{-0.010825in}}{\pgfqpoint{-0.014731in}{-0.014731in}}%
\pgfpathcurveto{\pgfqpoint{-0.010825in}{-0.018638in}}{\pgfqpoint{-0.005525in}{-0.020833in}}{\pgfqpoint{0.000000in}{-0.020833in}}%
\pgfpathclose%
\pgfusepath{stroke,fill}%
}%
\begin{pgfscope}%
\pgfsys@transformshift{3.138587in}{1.037132in}%
\pgfsys@useobject{currentmarker}{}%
\end{pgfscope}%
\end{pgfscope}%
\begin{pgfscope}%
\pgfsetbuttcap%
\pgfsetmiterjoin%
\definecolor{currentfill}{rgb}{1.000000,1.000000,1.000000}%
\pgfsetfillcolor{currentfill}%
\pgfsetfillopacity{0.800000}%
\pgfsetlinewidth{1.003750pt}%
\definecolor{currentstroke}{rgb}{0.800000,0.800000,0.800000}%
\pgfsetstrokecolor{currentstroke}%
\pgfsetstrokeopacity{0.800000}%
\pgfsetdash{}{0pt}%
\pgfpathmoveto{\pgfqpoint{3.430779in}{3.627431in}}%
\pgfpathlineto{\pgfqpoint{4.295229in}{3.627431in}}%
\pgfpathquadraticcurveto{\pgfqpoint{4.325785in}{3.627431in}}{\pgfqpoint{4.325785in}{3.657987in}}%
\pgfpathlineto{\pgfqpoint{4.325785in}{4.293056in}}%
\pgfpathquadraticcurveto{\pgfqpoint{4.325785in}{4.323611in}}{\pgfqpoint{4.295229in}{4.323611in}}%
\pgfpathlineto{\pgfqpoint{3.430779in}{4.323611in}}%
\pgfpathquadraticcurveto{\pgfqpoint{3.400223in}{4.323611in}}{\pgfqpoint{3.400223in}{4.293056in}}%
\pgfpathlineto{\pgfqpoint{3.400223in}{3.657987in}}%
\pgfpathquadraticcurveto{\pgfqpoint{3.400223in}{3.627431in}}{\pgfqpoint{3.430779in}{3.627431in}}%
\pgfpathclose%
\pgfusepath{stroke,fill}%
\end{pgfscope}%
\begin{pgfscope}%
\definecolor{textcolor}{rgb}{0.150000,0.150000,0.150000}%
\pgfsetstrokecolor{textcolor}%
\pgfsetfillcolor{textcolor}%
\pgftext[x=3.585616in,y=4.146760in,left,base]{\color{textcolor}\sffamily\fontsize{12.000000}{14.400000}\selectfont Legende}%
\end{pgfscope}%
\begin{pgfscope}%
\pgfsetbuttcap%
\pgfsetmiterjoin%
\definecolor{currentfill}{rgb}{0.347059,0.458824,0.641176}%
\pgfsetfillcolor{currentfill}%
\pgfsetlinewidth{0.752812pt}%
\definecolor{currentstroke}{rgb}{0.298039,0.298039,0.298039}%
\pgfsetstrokecolor{currentstroke}%
\pgfsetdash{}{0pt}%
\pgfpathmoveto{\pgfqpoint{3.461334in}{3.931019in}}%
\pgfpathlineto{\pgfqpoint{3.766890in}{3.931019in}}%
\pgfpathlineto{\pgfqpoint{3.766890in}{4.037964in}}%
\pgfpathlineto{\pgfqpoint{3.461334in}{4.037964in}}%
\pgfpathclose%
\pgfusepath{stroke,fill}%
\end{pgfscope}%
\begin{pgfscope}%
\definecolor{textcolor}{rgb}{0.150000,0.150000,0.150000}%
\pgfsetstrokecolor{textcolor}%
\pgfsetfillcolor{textcolor}%
\pgftext[x=3.889112in,y=3.931019in,left,base]{\color{textcolor}\sffamily\fontsize{11.000000}{13.200000}\selectfont Isolé}%
\end{pgfscope}%
\begin{pgfscope}%
\pgfsetbuttcap%
\pgfsetmiterjoin%
\definecolor{currentfill}{rgb}{0.798529,0.536765,0.389706}%
\pgfsetfillcolor{currentfill}%
\pgfsetlinewidth{0.752812pt}%
\definecolor{currentstroke}{rgb}{0.298039,0.298039,0.298039}%
\pgfsetstrokecolor{currentstroke}%
\pgfsetdash{}{0pt}%
\pgfpathmoveto{\pgfqpoint{3.461334in}{3.718114in}}%
\pgfpathlineto{\pgfqpoint{3.766890in}{3.718114in}}%
\pgfpathlineto{\pgfqpoint{3.766890in}{3.825058in}}%
\pgfpathlineto{\pgfqpoint{3.461334in}{3.825058in}}%
\pgfpathclose%
\pgfusepath{stroke,fill}%
\end{pgfscope}%
\begin{pgfscope}%
\definecolor{textcolor}{rgb}{0.150000,0.150000,0.150000}%
\pgfsetstrokecolor{textcolor}%
\pgfsetfillcolor{textcolor}%
\pgftext[x=3.889112in,y=3.718114in,left,base]{\color{textcolor}\sffamily\fontsize{11.000000}{13.200000}\selectfont Stressé}%
\end{pgfscope}%
\begin{pgfscope}%
\pgfsetbuttcap%
\pgfsetmiterjoin%
\definecolor{currentfill}{rgb}{1.000000,1.000000,1.000000}%
\pgfsetfillcolor{currentfill}%
\pgfsetlinewidth{0.000000pt}%
\definecolor{currentstroke}{rgb}{0.000000,0.000000,0.000000}%
\pgfsetstrokecolor{currentstroke}%
\pgfsetstrokeopacity{0.000000}%
\pgfsetdash{}{0pt}%
\pgfpathmoveto{\pgfqpoint{4.907609in}{0.550000in}}%
\pgfpathlineto{\pgfqpoint{7.434783in}{0.550000in}}%
\pgfpathlineto{\pgfqpoint{7.434783in}{4.400000in}}%
\pgfpathlineto{\pgfqpoint{4.907609in}{4.400000in}}%
\pgfpathclose%
\pgfusepath{fill}%
\end{pgfscope}%
\begin{pgfscope}%
\definecolor{textcolor}{rgb}{0.150000,0.150000,0.150000}%
\pgfsetstrokecolor{textcolor}%
\pgfsetfillcolor{textcolor}%
\pgftext[x=6.171196in,y=0.418056in,,top]{\color{textcolor}\sffamily\fontsize{11.000000}{13.200000}\selectfont From t1}%
\end{pgfscope}%
\begin{pgfscope}%
\pgfpathrectangle{\pgfqpoint{4.907609in}{0.550000in}}{\pgfqpoint{2.527174in}{3.850000in}}%
\pgfusepath{clip}%
\pgfsetroundcap%
\pgfsetroundjoin%
\pgfsetlinewidth{1.003750pt}%
\definecolor{currentstroke}{rgb}{0.800000,0.800000,0.800000}%
\pgfsetstrokecolor{currentstroke}%
\pgfsetdash{}{0pt}%
\pgfpathmoveto{\pgfqpoint{4.907609in}{0.681818in}}%
\pgfpathlineto{\pgfqpoint{7.434783in}{0.681818in}}%
\pgfusepath{stroke}%
\end{pgfscope}%
\begin{pgfscope}%
\pgfpathrectangle{\pgfqpoint{4.907609in}{0.550000in}}{\pgfqpoint{2.527174in}{3.850000in}}%
\pgfusepath{clip}%
\pgfsetroundcap%
\pgfsetroundjoin%
\pgfsetlinewidth{1.003750pt}%
\definecolor{currentstroke}{rgb}{0.800000,0.800000,0.800000}%
\pgfsetstrokecolor{currentstroke}%
\pgfsetdash{}{0pt}%
\pgfpathmoveto{\pgfqpoint{4.907609in}{1.285136in}}%
\pgfpathlineto{\pgfqpoint{7.434783in}{1.285136in}}%
\pgfusepath{stroke}%
\end{pgfscope}%
\begin{pgfscope}%
\pgfpathrectangle{\pgfqpoint{4.907609in}{0.550000in}}{\pgfqpoint{2.527174in}{3.850000in}}%
\pgfusepath{clip}%
\pgfsetroundcap%
\pgfsetroundjoin%
\pgfsetlinewidth{1.003750pt}%
\definecolor{currentstroke}{rgb}{0.800000,0.800000,0.800000}%
\pgfsetstrokecolor{currentstroke}%
\pgfsetdash{}{0pt}%
\pgfpathmoveto{\pgfqpoint{4.907609in}{1.888455in}}%
\pgfpathlineto{\pgfqpoint{7.434783in}{1.888455in}}%
\pgfusepath{stroke}%
\end{pgfscope}%
\begin{pgfscope}%
\pgfpathrectangle{\pgfqpoint{4.907609in}{0.550000in}}{\pgfqpoint{2.527174in}{3.850000in}}%
\pgfusepath{clip}%
\pgfsetroundcap%
\pgfsetroundjoin%
\pgfsetlinewidth{1.003750pt}%
\definecolor{currentstroke}{rgb}{0.800000,0.800000,0.800000}%
\pgfsetstrokecolor{currentstroke}%
\pgfsetdash{}{0pt}%
\pgfpathmoveto{\pgfqpoint{4.907609in}{2.491773in}}%
\pgfpathlineto{\pgfqpoint{7.434783in}{2.491773in}}%
\pgfusepath{stroke}%
\end{pgfscope}%
\begin{pgfscope}%
\pgfpathrectangle{\pgfqpoint{4.907609in}{0.550000in}}{\pgfqpoint{2.527174in}{3.850000in}}%
\pgfusepath{clip}%
\pgfsetroundcap%
\pgfsetroundjoin%
\pgfsetlinewidth{1.003750pt}%
\definecolor{currentstroke}{rgb}{0.800000,0.800000,0.800000}%
\pgfsetstrokecolor{currentstroke}%
\pgfsetdash{}{0pt}%
\pgfpathmoveto{\pgfqpoint{4.907609in}{3.095092in}}%
\pgfpathlineto{\pgfqpoint{7.434783in}{3.095092in}}%
\pgfusepath{stroke}%
\end{pgfscope}%
\begin{pgfscope}%
\pgfpathrectangle{\pgfqpoint{4.907609in}{0.550000in}}{\pgfqpoint{2.527174in}{3.850000in}}%
\pgfusepath{clip}%
\pgfsetroundcap%
\pgfsetroundjoin%
\pgfsetlinewidth{1.003750pt}%
\definecolor{currentstroke}{rgb}{0.800000,0.800000,0.800000}%
\pgfsetstrokecolor{currentstroke}%
\pgfsetdash{}{0pt}%
\pgfpathmoveto{\pgfqpoint{4.907609in}{3.698410in}}%
\pgfpathlineto{\pgfqpoint{7.434783in}{3.698410in}}%
\pgfusepath{stroke}%
\end{pgfscope}%
\begin{pgfscope}%
\pgfpathrectangle{\pgfqpoint{4.907609in}{0.550000in}}{\pgfqpoint{2.527174in}{3.850000in}}%
\pgfusepath{clip}%
\pgfsetroundcap%
\pgfsetroundjoin%
\pgfsetlinewidth{1.003750pt}%
\definecolor{currentstroke}{rgb}{0.800000,0.800000,0.800000}%
\pgfsetstrokecolor{currentstroke}%
\pgfsetdash{}{0pt}%
\pgfpathmoveto{\pgfqpoint{4.907609in}{4.301729in}}%
\pgfpathlineto{\pgfqpoint{7.434783in}{4.301729in}}%
\pgfusepath{stroke}%
\end{pgfscope}%
\begin{pgfscope}%
\definecolor{textcolor}{rgb}{0.150000,0.150000,0.150000}%
\pgfsetstrokecolor{textcolor}%
\pgfsetfillcolor{textcolor}%
\pgftext[x=4.852053in,y=2.475000in,,bottom,rotate=90.000000]{\color{textcolor}\sffamily\fontsize{12.000000}{14.400000}\selectfont \ }%
\end{pgfscope}%
\begin{pgfscope}%
\pgfpathrectangle{\pgfqpoint{4.907609in}{0.550000in}}{\pgfqpoint{2.527174in}{3.850000in}}%
\pgfusepath{clip}%
\pgfsetbuttcap%
\pgfsetroundjoin%
\definecolor{currentfill}{rgb}{0.347059,0.458824,0.641176}%
\pgfsetfillcolor{currentfill}%
\pgfsetlinewidth{1.505625pt}%
\definecolor{currentstroke}{rgb}{0.298039,0.298039,0.298039}%
\pgfsetstrokecolor{currentstroke}%
\pgfsetdash{}{0pt}%
\pgfsys@defobject{currentmarker}{\pgfqpoint{5.160326in}{0.904602in}}{\pgfqpoint{6.171196in}{1.252162in}}{%
\pgfpathmoveto{\pgfqpoint{6.171196in}{0.904602in}}%
\pgfpathlineto{\pgfqpoint{6.170911in}{0.904602in}}%
\pgfpathlineto{\pgfqpoint{6.171099in}{0.908113in}}%
\pgfpathlineto{\pgfqpoint{6.170149in}{0.911624in}}%
\pgfpathlineto{\pgfqpoint{6.168975in}{0.915135in}}%
\pgfpathlineto{\pgfqpoint{6.168431in}{0.918645in}}%
\pgfpathlineto{\pgfqpoint{6.163126in}{0.922156in}}%
\pgfpathlineto{\pgfqpoint{6.168217in}{0.925667in}}%
\pgfpathlineto{\pgfqpoint{6.170595in}{0.929177in}}%
\pgfpathlineto{\pgfqpoint{6.171132in}{0.932688in}}%
\pgfpathlineto{\pgfqpoint{6.171195in}{0.936199in}}%
\pgfpathlineto{\pgfqpoint{6.171196in}{0.939709in}}%
\pgfpathlineto{\pgfqpoint{6.171196in}{0.943220in}}%
\pgfpathlineto{\pgfqpoint{6.171196in}{0.946731in}}%
\pgfpathlineto{\pgfqpoint{6.171196in}{0.950242in}}%
\pgfpathlineto{\pgfqpoint{6.171196in}{0.953752in}}%
\pgfpathlineto{\pgfqpoint{6.171196in}{0.957263in}}%
\pgfpathlineto{\pgfqpoint{6.171196in}{0.960774in}}%
\pgfpathlineto{\pgfqpoint{6.171196in}{0.964284in}}%
\pgfpathlineto{\pgfqpoint{6.171196in}{0.967795in}}%
\pgfpathlineto{\pgfqpoint{6.171196in}{0.971306in}}%
\pgfpathlineto{\pgfqpoint{6.171196in}{0.974816in}}%
\pgfpathlineto{\pgfqpoint{6.171196in}{0.978327in}}%
\pgfpathlineto{\pgfqpoint{6.171196in}{0.981838in}}%
\pgfpathlineto{\pgfqpoint{6.171195in}{0.985349in}}%
\pgfpathlineto{\pgfqpoint{6.171148in}{0.988859in}}%
\pgfpathlineto{\pgfqpoint{6.170527in}{0.992370in}}%
\pgfpathlineto{\pgfqpoint{6.170016in}{0.995881in}}%
\pgfpathlineto{\pgfqpoint{6.170433in}{0.999391in}}%
\pgfpathlineto{\pgfqpoint{6.169945in}{1.002902in}}%
\pgfpathlineto{\pgfqpoint{6.164576in}{1.006413in}}%
\pgfpathlineto{\pgfqpoint{5.956039in}{1.009924in}}%
\pgfpathlineto{\pgfqpoint{5.160326in}{1.013434in}}%
\pgfpathlineto{\pgfqpoint{5.784125in}{1.016945in}}%
\pgfpathlineto{\pgfqpoint{6.157128in}{1.020456in}}%
\pgfpathlineto{\pgfqpoint{6.170955in}{1.023966in}}%
\pgfpathlineto{\pgfqpoint{6.171194in}{1.027477in}}%
\pgfpathlineto{\pgfqpoint{6.171196in}{1.030988in}}%
\pgfpathlineto{\pgfqpoint{6.171196in}{1.034498in}}%
\pgfpathlineto{\pgfqpoint{6.171196in}{1.038009in}}%
\pgfpathlineto{\pgfqpoint{6.171196in}{1.041520in}}%
\pgfpathlineto{\pgfqpoint{6.171196in}{1.045031in}}%
\pgfpathlineto{\pgfqpoint{6.171196in}{1.048541in}}%
\pgfpathlineto{\pgfqpoint{6.171196in}{1.052052in}}%
\pgfpathlineto{\pgfqpoint{6.171196in}{1.055563in}}%
\pgfpathlineto{\pgfqpoint{6.171196in}{1.059073in}}%
\pgfpathlineto{\pgfqpoint{6.171196in}{1.062584in}}%
\pgfpathlineto{\pgfqpoint{6.171196in}{1.066095in}}%
\pgfpathlineto{\pgfqpoint{6.171196in}{1.069605in}}%
\pgfpathlineto{\pgfqpoint{6.171196in}{1.073116in}}%
\pgfpathlineto{\pgfqpoint{6.171196in}{1.076627in}}%
\pgfpathlineto{\pgfqpoint{6.171196in}{1.080138in}}%
\pgfpathlineto{\pgfqpoint{6.171196in}{1.083648in}}%
\pgfpathlineto{\pgfqpoint{6.171196in}{1.087159in}}%
\pgfpathlineto{\pgfqpoint{6.171196in}{1.090670in}}%
\pgfpathlineto{\pgfqpoint{6.171196in}{1.094180in}}%
\pgfpathlineto{\pgfqpoint{6.171196in}{1.097691in}}%
\pgfpathlineto{\pgfqpoint{6.171196in}{1.101202in}}%
\pgfpathlineto{\pgfqpoint{6.171196in}{1.104712in}}%
\pgfpathlineto{\pgfqpoint{6.171196in}{1.108223in}}%
\pgfpathlineto{\pgfqpoint{6.171196in}{1.111734in}}%
\pgfpathlineto{\pgfqpoint{6.171196in}{1.115245in}}%
\pgfpathlineto{\pgfqpoint{6.171196in}{1.118755in}}%
\pgfpathlineto{\pgfqpoint{6.171196in}{1.122266in}}%
\pgfpathlineto{\pgfqpoint{6.171196in}{1.125777in}}%
\pgfpathlineto{\pgfqpoint{6.171196in}{1.129287in}}%
\pgfpathlineto{\pgfqpoint{6.171196in}{1.132798in}}%
\pgfpathlineto{\pgfqpoint{6.171196in}{1.136309in}}%
\pgfpathlineto{\pgfqpoint{6.171196in}{1.139819in}}%
\pgfpathlineto{\pgfqpoint{6.171196in}{1.143330in}}%
\pgfpathlineto{\pgfqpoint{6.171196in}{1.146841in}}%
\pgfpathlineto{\pgfqpoint{6.171196in}{1.150352in}}%
\pgfpathlineto{\pgfqpoint{6.171196in}{1.153862in}}%
\pgfpathlineto{\pgfqpoint{6.171196in}{1.157373in}}%
\pgfpathlineto{\pgfqpoint{6.171196in}{1.160884in}}%
\pgfpathlineto{\pgfqpoint{6.171196in}{1.164394in}}%
\pgfpathlineto{\pgfqpoint{6.171196in}{1.167905in}}%
\pgfpathlineto{\pgfqpoint{6.171196in}{1.171416in}}%
\pgfpathlineto{\pgfqpoint{6.171196in}{1.174926in}}%
\pgfpathlineto{\pgfqpoint{6.171196in}{1.178437in}}%
\pgfpathlineto{\pgfqpoint{6.171196in}{1.181948in}}%
\pgfpathlineto{\pgfqpoint{6.171196in}{1.185459in}}%
\pgfpathlineto{\pgfqpoint{6.171196in}{1.188969in}}%
\pgfpathlineto{\pgfqpoint{6.171196in}{1.192480in}}%
\pgfpathlineto{\pgfqpoint{6.171196in}{1.195991in}}%
\pgfpathlineto{\pgfqpoint{6.171196in}{1.199501in}}%
\pgfpathlineto{\pgfqpoint{6.171196in}{1.203012in}}%
\pgfpathlineto{\pgfqpoint{6.171196in}{1.206523in}}%
\pgfpathlineto{\pgfqpoint{6.171196in}{1.210033in}}%
\pgfpathlineto{\pgfqpoint{6.171196in}{1.213544in}}%
\pgfpathlineto{\pgfqpoint{6.171196in}{1.217055in}}%
\pgfpathlineto{\pgfqpoint{6.171196in}{1.220566in}}%
\pgfpathlineto{\pgfqpoint{6.171196in}{1.224076in}}%
\pgfpathlineto{\pgfqpoint{6.171196in}{1.227587in}}%
\pgfpathlineto{\pgfqpoint{6.171196in}{1.231098in}}%
\pgfpathlineto{\pgfqpoint{6.171196in}{1.234608in}}%
\pgfpathlineto{\pgfqpoint{6.171196in}{1.238119in}}%
\pgfpathlineto{\pgfqpoint{6.171196in}{1.241630in}}%
\pgfpathlineto{\pgfqpoint{6.171195in}{1.245141in}}%
\pgfpathlineto{\pgfqpoint{6.171136in}{1.248651in}}%
\pgfpathlineto{\pgfqpoint{6.170911in}{1.252162in}}%
\pgfpathlineto{\pgfqpoint{6.171196in}{1.252162in}}%
\pgfpathlineto{\pgfqpoint{6.171196in}{1.252162in}}%
\pgfpathlineto{\pgfqpoint{6.171196in}{1.248651in}}%
\pgfpathlineto{\pgfqpoint{6.171196in}{1.245141in}}%
\pgfpathlineto{\pgfqpoint{6.171196in}{1.241630in}}%
\pgfpathlineto{\pgfqpoint{6.171196in}{1.238119in}}%
\pgfpathlineto{\pgfqpoint{6.171196in}{1.234608in}}%
\pgfpathlineto{\pgfqpoint{6.171196in}{1.231098in}}%
\pgfpathlineto{\pgfqpoint{6.171196in}{1.227587in}}%
\pgfpathlineto{\pgfqpoint{6.171196in}{1.224076in}}%
\pgfpathlineto{\pgfqpoint{6.171196in}{1.220566in}}%
\pgfpathlineto{\pgfqpoint{6.171196in}{1.217055in}}%
\pgfpathlineto{\pgfqpoint{6.171196in}{1.213544in}}%
\pgfpathlineto{\pgfqpoint{6.171196in}{1.210033in}}%
\pgfpathlineto{\pgfqpoint{6.171196in}{1.206523in}}%
\pgfpathlineto{\pgfqpoint{6.171196in}{1.203012in}}%
\pgfpathlineto{\pgfqpoint{6.171196in}{1.199501in}}%
\pgfpathlineto{\pgfqpoint{6.171196in}{1.195991in}}%
\pgfpathlineto{\pgfqpoint{6.171196in}{1.192480in}}%
\pgfpathlineto{\pgfqpoint{6.171196in}{1.188969in}}%
\pgfpathlineto{\pgfqpoint{6.171196in}{1.185459in}}%
\pgfpathlineto{\pgfqpoint{6.171196in}{1.181948in}}%
\pgfpathlineto{\pgfqpoint{6.171196in}{1.178437in}}%
\pgfpathlineto{\pgfqpoint{6.171196in}{1.174926in}}%
\pgfpathlineto{\pgfqpoint{6.171196in}{1.171416in}}%
\pgfpathlineto{\pgfqpoint{6.171196in}{1.167905in}}%
\pgfpathlineto{\pgfqpoint{6.171196in}{1.164394in}}%
\pgfpathlineto{\pgfqpoint{6.171196in}{1.160884in}}%
\pgfpathlineto{\pgfqpoint{6.171196in}{1.157373in}}%
\pgfpathlineto{\pgfqpoint{6.171196in}{1.153862in}}%
\pgfpathlineto{\pgfqpoint{6.171196in}{1.150352in}}%
\pgfpathlineto{\pgfqpoint{6.171196in}{1.146841in}}%
\pgfpathlineto{\pgfqpoint{6.171196in}{1.143330in}}%
\pgfpathlineto{\pgfqpoint{6.171196in}{1.139819in}}%
\pgfpathlineto{\pgfqpoint{6.171196in}{1.136309in}}%
\pgfpathlineto{\pgfqpoint{6.171196in}{1.132798in}}%
\pgfpathlineto{\pgfqpoint{6.171196in}{1.129287in}}%
\pgfpathlineto{\pgfqpoint{6.171196in}{1.125777in}}%
\pgfpathlineto{\pgfqpoint{6.171196in}{1.122266in}}%
\pgfpathlineto{\pgfqpoint{6.171196in}{1.118755in}}%
\pgfpathlineto{\pgfqpoint{6.171196in}{1.115245in}}%
\pgfpathlineto{\pgfqpoint{6.171196in}{1.111734in}}%
\pgfpathlineto{\pgfqpoint{6.171196in}{1.108223in}}%
\pgfpathlineto{\pgfqpoint{6.171196in}{1.104712in}}%
\pgfpathlineto{\pgfqpoint{6.171196in}{1.101202in}}%
\pgfpathlineto{\pgfqpoint{6.171196in}{1.097691in}}%
\pgfpathlineto{\pgfqpoint{6.171196in}{1.094180in}}%
\pgfpathlineto{\pgfqpoint{6.171196in}{1.090670in}}%
\pgfpathlineto{\pgfqpoint{6.171196in}{1.087159in}}%
\pgfpathlineto{\pgfqpoint{6.171196in}{1.083648in}}%
\pgfpathlineto{\pgfqpoint{6.171196in}{1.080138in}}%
\pgfpathlineto{\pgfqpoint{6.171196in}{1.076627in}}%
\pgfpathlineto{\pgfqpoint{6.171196in}{1.073116in}}%
\pgfpathlineto{\pgfqpoint{6.171196in}{1.069605in}}%
\pgfpathlineto{\pgfqpoint{6.171196in}{1.066095in}}%
\pgfpathlineto{\pgfqpoint{6.171196in}{1.062584in}}%
\pgfpathlineto{\pgfqpoint{6.171196in}{1.059073in}}%
\pgfpathlineto{\pgfqpoint{6.171196in}{1.055563in}}%
\pgfpathlineto{\pgfqpoint{6.171196in}{1.052052in}}%
\pgfpathlineto{\pgfqpoint{6.171196in}{1.048541in}}%
\pgfpathlineto{\pgfqpoint{6.171196in}{1.045031in}}%
\pgfpathlineto{\pgfqpoint{6.171196in}{1.041520in}}%
\pgfpathlineto{\pgfqpoint{6.171196in}{1.038009in}}%
\pgfpathlineto{\pgfqpoint{6.171196in}{1.034498in}}%
\pgfpathlineto{\pgfqpoint{6.171196in}{1.030988in}}%
\pgfpathlineto{\pgfqpoint{6.171196in}{1.027477in}}%
\pgfpathlineto{\pgfqpoint{6.171196in}{1.023966in}}%
\pgfpathlineto{\pgfqpoint{6.171196in}{1.020456in}}%
\pgfpathlineto{\pgfqpoint{6.171196in}{1.016945in}}%
\pgfpathlineto{\pgfqpoint{6.171196in}{1.013434in}}%
\pgfpathlineto{\pgfqpoint{6.171196in}{1.009924in}}%
\pgfpathlineto{\pgfqpoint{6.171196in}{1.006413in}}%
\pgfpathlineto{\pgfqpoint{6.171196in}{1.002902in}}%
\pgfpathlineto{\pgfqpoint{6.171196in}{0.999391in}}%
\pgfpathlineto{\pgfqpoint{6.171196in}{0.995881in}}%
\pgfpathlineto{\pgfqpoint{6.171196in}{0.992370in}}%
\pgfpathlineto{\pgfqpoint{6.171196in}{0.988859in}}%
\pgfpathlineto{\pgfqpoint{6.171196in}{0.985349in}}%
\pgfpathlineto{\pgfqpoint{6.171196in}{0.981838in}}%
\pgfpathlineto{\pgfqpoint{6.171196in}{0.978327in}}%
\pgfpathlineto{\pgfqpoint{6.171196in}{0.974816in}}%
\pgfpathlineto{\pgfqpoint{6.171196in}{0.971306in}}%
\pgfpathlineto{\pgfqpoint{6.171196in}{0.967795in}}%
\pgfpathlineto{\pgfqpoint{6.171196in}{0.964284in}}%
\pgfpathlineto{\pgfqpoint{6.171196in}{0.960774in}}%
\pgfpathlineto{\pgfqpoint{6.171196in}{0.957263in}}%
\pgfpathlineto{\pgfqpoint{6.171196in}{0.953752in}}%
\pgfpathlineto{\pgfqpoint{6.171196in}{0.950242in}}%
\pgfpathlineto{\pgfqpoint{6.171196in}{0.946731in}}%
\pgfpathlineto{\pgfqpoint{6.171196in}{0.943220in}}%
\pgfpathlineto{\pgfqpoint{6.171196in}{0.939709in}}%
\pgfpathlineto{\pgfqpoint{6.171196in}{0.936199in}}%
\pgfpathlineto{\pgfqpoint{6.171196in}{0.932688in}}%
\pgfpathlineto{\pgfqpoint{6.171196in}{0.929177in}}%
\pgfpathlineto{\pgfqpoint{6.171196in}{0.925667in}}%
\pgfpathlineto{\pgfqpoint{6.171196in}{0.922156in}}%
\pgfpathlineto{\pgfqpoint{6.171196in}{0.918645in}}%
\pgfpathlineto{\pgfqpoint{6.171196in}{0.915135in}}%
\pgfpathlineto{\pgfqpoint{6.171196in}{0.911624in}}%
\pgfpathlineto{\pgfqpoint{6.171196in}{0.908113in}}%
\pgfpathlineto{\pgfqpoint{6.171196in}{0.904602in}}%
\pgfpathclose%
\pgfusepath{stroke,fill}%
}%
\begin{pgfscope}%
\pgfsys@transformshift{0.000000in}{0.000000in}%
\pgfsys@useobject{currentmarker}{}%
\end{pgfscope}%
\end{pgfscope}%
\begin{pgfscope}%
\pgfpathrectangle{\pgfqpoint{4.907609in}{0.550000in}}{\pgfqpoint{2.527174in}{3.850000in}}%
\pgfusepath{clip}%
\pgfsetbuttcap%
\pgfsetroundjoin%
\definecolor{currentfill}{rgb}{0.798529,0.536765,0.389706}%
\pgfsetfillcolor{currentfill}%
\pgfsetlinewidth{1.505625pt}%
\definecolor{currentstroke}{rgb}{0.298039,0.298039,0.298039}%
\pgfsetstrokecolor{currentstroke}%
\pgfsetdash{}{0pt}%
\pgfsys@defobject{currentmarker}{\pgfqpoint{6.171196in}{0.904914in}}{\pgfqpoint{6.726976in}{2.973529in}}{%
\pgfpathmoveto{\pgfqpoint{6.262066in}{0.904914in}}%
\pgfpathlineto{\pgfqpoint{6.171196in}{0.904914in}}%
\pgfpathlineto{\pgfqpoint{6.171196in}{0.925809in}}%
\pgfpathlineto{\pgfqpoint{6.171196in}{0.946704in}}%
\pgfpathlineto{\pgfqpoint{6.171196in}{0.967600in}}%
\pgfpathlineto{\pgfqpoint{6.171196in}{0.988495in}}%
\pgfpathlineto{\pgfqpoint{6.171196in}{1.009390in}}%
\pgfpathlineto{\pgfqpoint{6.171196in}{1.030285in}}%
\pgfpathlineto{\pgfqpoint{6.171196in}{1.051180in}}%
\pgfpathlineto{\pgfqpoint{6.171196in}{1.072075in}}%
\pgfpathlineto{\pgfqpoint{6.171196in}{1.092970in}}%
\pgfpathlineto{\pgfqpoint{6.171196in}{1.113865in}}%
\pgfpathlineto{\pgfqpoint{6.171196in}{1.134760in}}%
\pgfpathlineto{\pgfqpoint{6.171196in}{1.155655in}}%
\pgfpathlineto{\pgfqpoint{6.171196in}{1.176551in}}%
\pgfpathlineto{\pgfqpoint{6.171196in}{1.197446in}}%
\pgfpathlineto{\pgfqpoint{6.171196in}{1.218341in}}%
\pgfpathlineto{\pgfqpoint{6.171196in}{1.239236in}}%
\pgfpathlineto{\pgfqpoint{6.171196in}{1.260131in}}%
\pgfpathlineto{\pgfqpoint{6.171196in}{1.281026in}}%
\pgfpathlineto{\pgfqpoint{6.171196in}{1.301921in}}%
\pgfpathlineto{\pgfqpoint{6.171196in}{1.322816in}}%
\pgfpathlineto{\pgfqpoint{6.171196in}{1.343711in}}%
\pgfpathlineto{\pgfqpoint{6.171196in}{1.364606in}}%
\pgfpathlineto{\pgfqpoint{6.171196in}{1.385501in}}%
\pgfpathlineto{\pgfqpoint{6.171196in}{1.406397in}}%
\pgfpathlineto{\pgfqpoint{6.171196in}{1.427292in}}%
\pgfpathlineto{\pgfqpoint{6.171196in}{1.448187in}}%
\pgfpathlineto{\pgfqpoint{6.171196in}{1.469082in}}%
\pgfpathlineto{\pgfqpoint{6.171196in}{1.489977in}}%
\pgfpathlineto{\pgfqpoint{6.171196in}{1.510872in}}%
\pgfpathlineto{\pgfqpoint{6.171196in}{1.531767in}}%
\pgfpathlineto{\pgfqpoint{6.171196in}{1.552662in}}%
\pgfpathlineto{\pgfqpoint{6.171196in}{1.573557in}}%
\pgfpathlineto{\pgfqpoint{6.171196in}{1.594452in}}%
\pgfpathlineto{\pgfqpoint{6.171196in}{1.615348in}}%
\pgfpathlineto{\pgfqpoint{6.171196in}{1.636243in}}%
\pgfpathlineto{\pgfqpoint{6.171196in}{1.657138in}}%
\pgfpathlineto{\pgfqpoint{6.171196in}{1.678033in}}%
\pgfpathlineto{\pgfqpoint{6.171196in}{1.698928in}}%
\pgfpathlineto{\pgfqpoint{6.171196in}{1.719823in}}%
\pgfpathlineto{\pgfqpoint{6.171196in}{1.740718in}}%
\pgfpathlineto{\pgfqpoint{6.171196in}{1.761613in}}%
\pgfpathlineto{\pgfqpoint{6.171196in}{1.782508in}}%
\pgfpathlineto{\pgfqpoint{6.171196in}{1.803403in}}%
\pgfpathlineto{\pgfqpoint{6.171196in}{1.824299in}}%
\pgfpathlineto{\pgfqpoint{6.171196in}{1.845194in}}%
\pgfpathlineto{\pgfqpoint{6.171196in}{1.866089in}}%
\pgfpathlineto{\pgfqpoint{6.171196in}{1.886984in}}%
\pgfpathlineto{\pgfqpoint{6.171196in}{1.907879in}}%
\pgfpathlineto{\pgfqpoint{6.171196in}{1.928774in}}%
\pgfpathlineto{\pgfqpoint{6.171196in}{1.949669in}}%
\pgfpathlineto{\pgfqpoint{6.171196in}{1.970564in}}%
\pgfpathlineto{\pgfqpoint{6.171196in}{1.991459in}}%
\pgfpathlineto{\pgfqpoint{6.171196in}{2.012354in}}%
\pgfpathlineto{\pgfqpoint{6.171196in}{2.033250in}}%
\pgfpathlineto{\pgfqpoint{6.171196in}{2.054145in}}%
\pgfpathlineto{\pgfqpoint{6.171196in}{2.075040in}}%
\pgfpathlineto{\pgfqpoint{6.171196in}{2.095935in}}%
\pgfpathlineto{\pgfqpoint{6.171196in}{2.116830in}}%
\pgfpathlineto{\pgfqpoint{6.171196in}{2.137725in}}%
\pgfpathlineto{\pgfqpoint{6.171196in}{2.158620in}}%
\pgfpathlineto{\pgfqpoint{6.171196in}{2.179515in}}%
\pgfpathlineto{\pgfqpoint{6.171196in}{2.200410in}}%
\pgfpathlineto{\pgfqpoint{6.171196in}{2.221305in}}%
\pgfpathlineto{\pgfqpoint{6.171196in}{2.242200in}}%
\pgfpathlineto{\pgfqpoint{6.171196in}{2.263096in}}%
\pgfpathlineto{\pgfqpoint{6.171196in}{2.283991in}}%
\pgfpathlineto{\pgfqpoint{6.171196in}{2.304886in}}%
\pgfpathlineto{\pgfqpoint{6.171196in}{2.325781in}}%
\pgfpathlineto{\pgfqpoint{6.171196in}{2.346676in}}%
\pgfpathlineto{\pgfqpoint{6.171196in}{2.367571in}}%
\pgfpathlineto{\pgfqpoint{6.171196in}{2.388466in}}%
\pgfpathlineto{\pgfqpoint{6.171196in}{2.409361in}}%
\pgfpathlineto{\pgfqpoint{6.171196in}{2.430256in}}%
\pgfpathlineto{\pgfqpoint{6.171196in}{2.451151in}}%
\pgfpathlineto{\pgfqpoint{6.171196in}{2.472047in}}%
\pgfpathlineto{\pgfqpoint{6.171196in}{2.492942in}}%
\pgfpathlineto{\pgfqpoint{6.171196in}{2.513837in}}%
\pgfpathlineto{\pgfqpoint{6.171196in}{2.534732in}}%
\pgfpathlineto{\pgfqpoint{6.171196in}{2.555627in}}%
\pgfpathlineto{\pgfqpoint{6.171196in}{2.576522in}}%
\pgfpathlineto{\pgfqpoint{6.171196in}{2.597417in}}%
\pgfpathlineto{\pgfqpoint{6.171196in}{2.618312in}}%
\pgfpathlineto{\pgfqpoint{6.171196in}{2.639207in}}%
\pgfpathlineto{\pgfqpoint{6.171196in}{2.660102in}}%
\pgfpathlineto{\pgfqpoint{6.171196in}{2.680998in}}%
\pgfpathlineto{\pgfqpoint{6.171196in}{2.701893in}}%
\pgfpathlineto{\pgfqpoint{6.171196in}{2.722788in}}%
\pgfpathlineto{\pgfqpoint{6.171196in}{2.743683in}}%
\pgfpathlineto{\pgfqpoint{6.171196in}{2.764578in}}%
\pgfpathlineto{\pgfqpoint{6.171196in}{2.785473in}}%
\pgfpathlineto{\pgfqpoint{6.171196in}{2.806368in}}%
\pgfpathlineto{\pgfqpoint{6.171196in}{2.827263in}}%
\pgfpathlineto{\pgfqpoint{6.171196in}{2.848158in}}%
\pgfpathlineto{\pgfqpoint{6.171196in}{2.869053in}}%
\pgfpathlineto{\pgfqpoint{6.171196in}{2.889949in}}%
\pgfpathlineto{\pgfqpoint{6.171196in}{2.910844in}}%
\pgfpathlineto{\pgfqpoint{6.171196in}{2.931739in}}%
\pgfpathlineto{\pgfqpoint{6.171196in}{2.952634in}}%
\pgfpathlineto{\pgfqpoint{6.171196in}{2.973529in}}%
\pgfpathlineto{\pgfqpoint{6.174912in}{2.973529in}}%
\pgfpathlineto{\pgfqpoint{6.174912in}{2.973529in}}%
\pgfpathlineto{\pgfqpoint{6.175391in}{2.952634in}}%
\pgfpathlineto{\pgfqpoint{6.175675in}{2.931739in}}%
\pgfpathlineto{\pgfqpoint{6.175736in}{2.910844in}}%
\pgfpathlineto{\pgfqpoint{6.175597in}{2.889949in}}%
\pgfpathlineto{\pgfqpoint{6.175339in}{2.869053in}}%
\pgfpathlineto{\pgfqpoint{6.175080in}{2.848158in}}%
\pgfpathlineto{\pgfqpoint{6.174959in}{2.827263in}}%
\pgfpathlineto{\pgfqpoint{6.175095in}{2.806368in}}%
\pgfpathlineto{\pgfqpoint{6.175564in}{2.785473in}}%
\pgfpathlineto{\pgfqpoint{6.176378in}{2.764578in}}%
\pgfpathlineto{\pgfqpoint{6.177491in}{2.743683in}}%
\pgfpathlineto{\pgfqpoint{6.178814in}{2.722788in}}%
\pgfpathlineto{\pgfqpoint{6.180253in}{2.701893in}}%
\pgfpathlineto{\pgfqpoint{6.181743in}{2.680998in}}%
\pgfpathlineto{\pgfqpoint{6.183269in}{2.660102in}}%
\pgfpathlineto{\pgfqpoint{6.184873in}{2.639207in}}%
\pgfpathlineto{\pgfqpoint{6.186623in}{2.618312in}}%
\pgfpathlineto{\pgfqpoint{6.188576in}{2.597417in}}%
\pgfpathlineto{\pgfqpoint{6.190735in}{2.576522in}}%
\pgfpathlineto{\pgfqpoint{6.193036in}{2.555627in}}%
\pgfpathlineto{\pgfqpoint{6.195363in}{2.534732in}}%
\pgfpathlineto{\pgfqpoint{6.197598in}{2.513837in}}%
\pgfpathlineto{\pgfqpoint{6.199682in}{2.492942in}}%
\pgfpathlineto{\pgfqpoint{6.201653in}{2.472047in}}%
\pgfpathlineto{\pgfqpoint{6.203656in}{2.451151in}}%
\pgfpathlineto{\pgfqpoint{6.205912in}{2.430256in}}%
\pgfpathlineto{\pgfqpoint{6.208660in}{2.409361in}}%
\pgfpathlineto{\pgfqpoint{6.212098in}{2.388466in}}%
\pgfpathlineto{\pgfqpoint{6.216326in}{2.367571in}}%
\pgfpathlineto{\pgfqpoint{6.221318in}{2.346676in}}%
\pgfpathlineto{\pgfqpoint{6.226917in}{2.325781in}}%
\pgfpathlineto{\pgfqpoint{6.232853in}{2.304886in}}%
\pgfpathlineto{\pgfqpoint{6.238790in}{2.283991in}}%
\pgfpathlineto{\pgfqpoint{6.244378in}{2.263096in}}%
\pgfpathlineto{\pgfqpoint{6.249313in}{2.242200in}}%
\pgfpathlineto{\pgfqpoint{6.253415in}{2.221305in}}%
\pgfpathlineto{\pgfqpoint{6.256693in}{2.200410in}}%
\pgfpathlineto{\pgfqpoint{6.259385in}{2.179515in}}%
\pgfpathlineto{\pgfqpoint{6.261962in}{2.158620in}}%
\pgfpathlineto{\pgfqpoint{6.265068in}{2.137725in}}%
\pgfpathlineto{\pgfqpoint{6.269418in}{2.116830in}}%
\pgfpathlineto{\pgfqpoint{6.275669in}{2.095935in}}%
\pgfpathlineto{\pgfqpoint{6.284286in}{2.075040in}}%
\pgfpathlineto{\pgfqpoint{6.295396in}{2.054145in}}%
\pgfpathlineto{\pgfqpoint{6.308662in}{2.033250in}}%
\pgfpathlineto{\pgfqpoint{6.323194in}{2.012354in}}%
\pgfpathlineto{\pgfqpoint{6.337581in}{1.991459in}}%
\pgfpathlineto{\pgfqpoint{6.350139in}{1.970564in}}%
\pgfpathlineto{\pgfqpoint{6.359355in}{1.949669in}}%
\pgfpathlineto{\pgfqpoint{6.364444in}{1.928774in}}%
\pgfpathlineto{\pgfqpoint{6.365810in}{1.907879in}}%
\pgfpathlineto{\pgfqpoint{6.365172in}{1.886984in}}%
\pgfpathlineto{\pgfqpoint{6.365279in}{1.866089in}}%
\pgfpathlineto{\pgfqpoint{6.369230in}{1.845194in}}%
\pgfpathlineto{\pgfqpoint{6.379600in}{1.824299in}}%
\pgfpathlineto{\pgfqpoint{6.397599in}{1.803403in}}%
\pgfpathlineto{\pgfqpoint{6.422527in}{1.782508in}}%
\pgfpathlineto{\pgfqpoint{6.451695in}{1.761613in}}%
\pgfpathlineto{\pgfqpoint{6.480932in}{1.740718in}}%
\pgfpathlineto{\pgfqpoint{6.505631in}{1.719823in}}%
\pgfpathlineto{\pgfqpoint{6.522103in}{1.698928in}}%
\pgfpathlineto{\pgfqpoint{6.528846in}{1.678033in}}%
\pgfpathlineto{\pgfqpoint{6.527267in}{1.657138in}}%
\pgfpathlineto{\pgfqpoint{6.521555in}{1.636243in}}%
\pgfpathlineto{\pgfqpoint{6.517671in}{1.615348in}}%
\pgfpathlineto{\pgfqpoint{6.521675in}{1.594452in}}%
\pgfpathlineto{\pgfqpoint{6.537886in}{1.573557in}}%
\pgfpathlineto{\pgfqpoint{6.567371in}{1.552662in}}%
\pgfpathlineto{\pgfqpoint{6.607248in}{1.531767in}}%
\pgfpathlineto{\pgfqpoint{6.651109in}{1.510872in}}%
\pgfpathlineto{\pgfqpoint{6.690561in}{1.489977in}}%
\pgfpathlineto{\pgfqpoint{6.717579in}{1.469082in}}%
\pgfpathlineto{\pgfqpoint{6.726976in}{1.448187in}}%
\pgfpathlineto{\pgfqpoint{6.718174in}{1.427292in}}%
\pgfpathlineto{\pgfqpoint{6.695582in}{1.406397in}}%
\pgfpathlineto{\pgfqpoint{6.667355in}{1.385501in}}%
\pgfpathlineto{\pgfqpoint{6.642845in}{1.364606in}}%
\pgfpathlineto{\pgfqpoint{6.629613in}{1.343711in}}%
\pgfpathlineto{\pgfqpoint{6.631020in}{1.322816in}}%
\pgfpathlineto{\pgfqpoint{6.645239in}{1.301921in}}%
\pgfpathlineto{\pgfqpoint{6.666004in}{1.281026in}}%
\pgfpathlineto{\pgfqpoint{6.684740in}{1.260131in}}%
\pgfpathlineto{\pgfqpoint{6.693251in}{1.239236in}}%
\pgfpathlineto{\pgfqpoint{6.686019in}{1.218341in}}%
\pgfpathlineto{\pgfqpoint{6.661446in}{1.197446in}}%
\pgfpathlineto{\pgfqpoint{6.621871in}{1.176551in}}%
\pgfpathlineto{\pgfqpoint{6.572588in}{1.155655in}}%
\pgfpathlineto{\pgfqpoint{6.520287in}{1.134760in}}%
\pgfpathlineto{\pgfqpoint{6.471358in}{1.113865in}}%
\pgfpathlineto{\pgfqpoint{6.430433in}{1.092970in}}%
\pgfpathlineto{\pgfqpoint{6.399509in}{1.072075in}}%
\pgfpathlineto{\pgfqpoint{6.377847in}{1.051180in}}%
\pgfpathlineto{\pgfqpoint{6.362664in}{1.030285in}}%
\pgfpathlineto{\pgfqpoint{6.350324in}{1.009390in}}%
\pgfpathlineto{\pgfqpoint{6.337568in}{0.988495in}}%
\pgfpathlineto{\pgfqpoint{6.322357in}{0.967600in}}%
\pgfpathlineto{\pgfqpoint{6.304132in}{0.946704in}}%
\pgfpathlineto{\pgfqpoint{6.283567in}{0.925809in}}%
\pgfpathlineto{\pgfqpoint{6.262066in}{0.904914in}}%
\pgfpathclose%
\pgfusepath{stroke,fill}%
}%
\begin{pgfscope}%
\pgfsys@transformshift{0.000000in}{0.000000in}%
\pgfsys@useobject{currentmarker}{}%
\end{pgfscope}%
\end{pgfscope}%
\begin{pgfscope}%
\pgfpathrectangle{\pgfqpoint{4.907609in}{0.550000in}}{\pgfqpoint{2.527174in}{3.850000in}}%
\pgfusepath{clip}%
\pgfsetbuttcap%
\pgfsetmiterjoin%
\definecolor{currentfill}{rgb}{0.347059,0.458824,0.641176}%
\pgfsetfillcolor{currentfill}%
\pgfsetlinewidth{0.752812pt}%
\definecolor{currentstroke}{rgb}{0.298039,0.298039,0.298039}%
\pgfsetstrokecolor{currentstroke}%
\pgfsetdash{}{0pt}%
\pgfpathmoveto{\pgfqpoint{6.171196in}{0.681818in}}%
\pgfpathlineto{\pgfqpoint{6.171196in}{0.681818in}}%
\pgfpathlineto{\pgfqpoint{6.171196in}{0.681818in}}%
\pgfpathlineto{\pgfqpoint{6.171196in}{0.681818in}}%
\pgfpathclose%
\pgfusepath{stroke,fill}%
\end{pgfscope}%
\begin{pgfscope}%
\pgfpathrectangle{\pgfqpoint{4.907609in}{0.550000in}}{\pgfqpoint{2.527174in}{3.850000in}}%
\pgfusepath{clip}%
\pgfsetbuttcap%
\pgfsetmiterjoin%
\definecolor{currentfill}{rgb}{0.798529,0.536765,0.389706}%
\pgfsetfillcolor{currentfill}%
\pgfsetlinewidth{0.752812pt}%
\definecolor{currentstroke}{rgb}{0.298039,0.298039,0.298039}%
\pgfsetstrokecolor{currentstroke}%
\pgfsetdash{}{0pt}%
\pgfpathmoveto{\pgfqpoint{6.171196in}{0.681818in}}%
\pgfpathlineto{\pgfqpoint{6.171196in}{0.681818in}}%
\pgfpathlineto{\pgfqpoint{6.171196in}{0.681818in}}%
\pgfpathlineto{\pgfqpoint{6.171196in}{0.681818in}}%
\pgfpathclose%
\pgfusepath{stroke,fill}%
\end{pgfscope}%
\begin{pgfscope}%
\pgfpathrectangle{\pgfqpoint{4.907609in}{0.550000in}}{\pgfqpoint{2.527174in}{3.850000in}}%
\pgfusepath{clip}%
\pgfsetroundcap%
\pgfsetroundjoin%
\pgfsetlinewidth{1.505625pt}%
\definecolor{currentstroke}{rgb}{0.298039,0.298039,0.298039}%
\pgfsetstrokecolor{currentstroke}%
\pgfsetdash{}{0pt}%
\pgfpathmoveto{\pgfqpoint{6.171196in}{0.904602in}}%
\pgfpathlineto{\pgfqpoint{6.171196in}{1.623207in}}%
\pgfusepath{stroke}%
\end{pgfscope}%
\begin{pgfscope}%
\pgfpathrectangle{\pgfqpoint{4.907609in}{0.550000in}}{\pgfqpoint{2.527174in}{3.850000in}}%
\pgfusepath{clip}%
\pgfsetroundcap%
\pgfsetroundjoin%
\pgfsetlinewidth{4.516875pt}%
\definecolor{currentstroke}{rgb}{0.298039,0.298039,0.298039}%
\pgfsetstrokecolor{currentstroke}%
\pgfsetdash{}{0pt}%
\pgfpathmoveto{\pgfqpoint{6.171196in}{1.013445in}}%
\pgfpathlineto{\pgfqpoint{6.171196in}{1.257907in}}%
\pgfusepath{stroke}%
\end{pgfscope}%
\begin{pgfscope}%
\pgfsetrectcap%
\pgfsetmiterjoin%
\pgfsetlinewidth{1.254687pt}%
\definecolor{currentstroke}{rgb}{0.800000,0.800000,0.800000}%
\pgfsetstrokecolor{currentstroke}%
\pgfsetdash{}{0pt}%
\pgfpathmoveto{\pgfqpoint{4.907609in}{0.550000in}}%
\pgfpathlineto{\pgfqpoint{4.907609in}{4.400000in}}%
\pgfusepath{stroke}%
\end{pgfscope}%
\begin{pgfscope}%
\pgfsetrectcap%
\pgfsetmiterjoin%
\pgfsetlinewidth{1.254687pt}%
\definecolor{currentstroke}{rgb}{0.800000,0.800000,0.800000}%
\pgfsetstrokecolor{currentstroke}%
\pgfsetdash{}{0pt}%
\pgfpathmoveto{\pgfqpoint{7.434783in}{0.550000in}}%
\pgfpathlineto{\pgfqpoint{7.434783in}{4.400000in}}%
\pgfusepath{stroke}%
\end{pgfscope}%
\begin{pgfscope}%
\pgfsetrectcap%
\pgfsetmiterjoin%
\pgfsetlinewidth{1.254687pt}%
\definecolor{currentstroke}{rgb}{0.800000,0.800000,0.800000}%
\pgfsetstrokecolor{currentstroke}%
\pgfsetdash{}{0pt}%
\pgfpathmoveto{\pgfqpoint{4.907609in}{0.550000in}}%
\pgfpathlineto{\pgfqpoint{7.434783in}{0.550000in}}%
\pgfusepath{stroke}%
\end{pgfscope}%
\begin{pgfscope}%
\pgfsetrectcap%
\pgfsetmiterjoin%
\pgfsetlinewidth{1.254687pt}%
\definecolor{currentstroke}{rgb}{0.800000,0.800000,0.800000}%
\pgfsetstrokecolor{currentstroke}%
\pgfsetdash{}{0pt}%
\pgfpathmoveto{\pgfqpoint{4.907609in}{4.400000in}}%
\pgfpathlineto{\pgfqpoint{7.434783in}{4.400000in}}%
\pgfusepath{stroke}%
\end{pgfscope}%
\begin{pgfscope}%
\pgfpathrectangle{\pgfqpoint{4.907609in}{0.550000in}}{\pgfqpoint{2.527174in}{3.850000in}}%
\pgfusepath{clip}%
\pgfsetbuttcap%
\pgfsetroundjoin%
\definecolor{currentfill}{rgb}{1.000000,1.000000,1.000000}%
\pgfsetfillcolor{currentfill}%
\pgfsetlinewidth{1.003750pt}%
\definecolor{currentstroke}{rgb}{0.298039,0.298039,0.298039}%
\pgfsetstrokecolor{currentstroke}%
\pgfsetdash{}{0pt}%
\pgfsys@defobject{currentmarker}{\pgfqpoint{-0.020833in}{-0.020833in}}{\pgfqpoint{0.020833in}{0.020833in}}{%
\pgfpathmoveto{\pgfqpoint{0.000000in}{-0.020833in}}%
\pgfpathcurveto{\pgfqpoint{0.005525in}{-0.020833in}}{\pgfqpoint{0.010825in}{-0.018638in}}{\pgfqpoint{0.014731in}{-0.014731in}}%
\pgfpathcurveto{\pgfqpoint{0.018638in}{-0.010825in}}{\pgfqpoint{0.020833in}{-0.005525in}}{\pgfqpoint{0.020833in}{0.000000in}}%
\pgfpathcurveto{\pgfqpoint{0.020833in}{0.005525in}}{\pgfqpoint{0.018638in}{0.010825in}}{\pgfqpoint{0.014731in}{0.014731in}}%
\pgfpathcurveto{\pgfqpoint{0.010825in}{0.018638in}}{\pgfqpoint{0.005525in}{0.020833in}}{\pgfqpoint{0.000000in}{0.020833in}}%
\pgfpathcurveto{\pgfqpoint{-0.005525in}{0.020833in}}{\pgfqpoint{-0.010825in}{0.018638in}}{\pgfqpoint{-0.014731in}{0.014731in}}%
\pgfpathcurveto{\pgfqpoint{-0.018638in}{0.010825in}}{\pgfqpoint{-0.020833in}{0.005525in}}{\pgfqpoint{-0.020833in}{0.000000in}}%
\pgfpathcurveto{\pgfqpoint{-0.020833in}{-0.005525in}}{\pgfqpoint{-0.018638in}{-0.010825in}}{\pgfqpoint{-0.014731in}{-0.014731in}}%
\pgfpathcurveto{\pgfqpoint{-0.010825in}{-0.018638in}}{\pgfqpoint{-0.005525in}{-0.020833in}}{\pgfqpoint{0.000000in}{-0.020833in}}%
\pgfpathclose%
\pgfusepath{stroke,fill}%
}%
\begin{pgfscope}%
\pgfsys@transformshift{6.171196in}{1.014628in}%
\pgfsys@useobject{currentmarker}{}%
\end{pgfscope}%
\end{pgfscope}%
\begin{pgfscope}%
\pgfsetbuttcap%
\pgfsetmiterjoin%
\definecolor{currentfill}{rgb}{1.000000,1.000000,1.000000}%
\pgfsetfillcolor{currentfill}%
\pgfsetlinewidth{0.000000pt}%
\definecolor{currentstroke}{rgb}{0.000000,0.000000,0.000000}%
\pgfsetstrokecolor{currentstroke}%
\pgfsetstrokeopacity{0.000000}%
\pgfsetdash{}{0pt}%
\pgfpathmoveto{\pgfqpoint{7.940217in}{0.550000in}}%
\pgfpathlineto{\pgfqpoint{10.467391in}{0.550000in}}%
\pgfpathlineto{\pgfqpoint{10.467391in}{4.400000in}}%
\pgfpathlineto{\pgfqpoint{7.940217in}{4.400000in}}%
\pgfpathclose%
\pgfusepath{fill}%
\end{pgfscope}%
\begin{pgfscope}%
\definecolor{textcolor}{rgb}{0.150000,0.150000,0.150000}%
\pgfsetstrokecolor{textcolor}%
\pgfsetfillcolor{textcolor}%
\pgftext[x=9.203804in,y=0.418056in,,top]{\color{textcolor}\sffamily\fontsize{11.000000}{13.200000}\selectfont From t2}%
\end{pgfscope}%
\begin{pgfscope}%
\pgfpathrectangle{\pgfqpoint{7.940217in}{0.550000in}}{\pgfqpoint{2.527174in}{3.850000in}}%
\pgfusepath{clip}%
\pgfsetroundcap%
\pgfsetroundjoin%
\pgfsetlinewidth{1.003750pt}%
\definecolor{currentstroke}{rgb}{0.800000,0.800000,0.800000}%
\pgfsetstrokecolor{currentstroke}%
\pgfsetdash{}{0pt}%
\pgfpathmoveto{\pgfqpoint{7.940217in}{0.681818in}}%
\pgfpathlineto{\pgfqpoint{10.467391in}{0.681818in}}%
\pgfusepath{stroke}%
\end{pgfscope}%
\begin{pgfscope}%
\pgfpathrectangle{\pgfqpoint{7.940217in}{0.550000in}}{\pgfqpoint{2.527174in}{3.850000in}}%
\pgfusepath{clip}%
\pgfsetroundcap%
\pgfsetroundjoin%
\pgfsetlinewidth{1.003750pt}%
\definecolor{currentstroke}{rgb}{0.800000,0.800000,0.800000}%
\pgfsetstrokecolor{currentstroke}%
\pgfsetdash{}{0pt}%
\pgfpathmoveto{\pgfqpoint{7.940217in}{1.285136in}}%
\pgfpathlineto{\pgfqpoint{10.467391in}{1.285136in}}%
\pgfusepath{stroke}%
\end{pgfscope}%
\begin{pgfscope}%
\pgfpathrectangle{\pgfqpoint{7.940217in}{0.550000in}}{\pgfqpoint{2.527174in}{3.850000in}}%
\pgfusepath{clip}%
\pgfsetroundcap%
\pgfsetroundjoin%
\pgfsetlinewidth{1.003750pt}%
\definecolor{currentstroke}{rgb}{0.800000,0.800000,0.800000}%
\pgfsetstrokecolor{currentstroke}%
\pgfsetdash{}{0pt}%
\pgfpathmoveto{\pgfqpoint{7.940217in}{1.888455in}}%
\pgfpathlineto{\pgfqpoint{10.467391in}{1.888455in}}%
\pgfusepath{stroke}%
\end{pgfscope}%
\begin{pgfscope}%
\pgfpathrectangle{\pgfqpoint{7.940217in}{0.550000in}}{\pgfqpoint{2.527174in}{3.850000in}}%
\pgfusepath{clip}%
\pgfsetroundcap%
\pgfsetroundjoin%
\pgfsetlinewidth{1.003750pt}%
\definecolor{currentstroke}{rgb}{0.800000,0.800000,0.800000}%
\pgfsetstrokecolor{currentstroke}%
\pgfsetdash{}{0pt}%
\pgfpathmoveto{\pgfqpoint{7.940217in}{2.491773in}}%
\pgfpathlineto{\pgfqpoint{10.467391in}{2.491773in}}%
\pgfusepath{stroke}%
\end{pgfscope}%
\begin{pgfscope}%
\pgfpathrectangle{\pgfqpoint{7.940217in}{0.550000in}}{\pgfqpoint{2.527174in}{3.850000in}}%
\pgfusepath{clip}%
\pgfsetroundcap%
\pgfsetroundjoin%
\pgfsetlinewidth{1.003750pt}%
\definecolor{currentstroke}{rgb}{0.800000,0.800000,0.800000}%
\pgfsetstrokecolor{currentstroke}%
\pgfsetdash{}{0pt}%
\pgfpathmoveto{\pgfqpoint{7.940217in}{3.095092in}}%
\pgfpathlineto{\pgfqpoint{10.467391in}{3.095092in}}%
\pgfusepath{stroke}%
\end{pgfscope}%
\begin{pgfscope}%
\pgfpathrectangle{\pgfqpoint{7.940217in}{0.550000in}}{\pgfqpoint{2.527174in}{3.850000in}}%
\pgfusepath{clip}%
\pgfsetroundcap%
\pgfsetroundjoin%
\pgfsetlinewidth{1.003750pt}%
\definecolor{currentstroke}{rgb}{0.800000,0.800000,0.800000}%
\pgfsetstrokecolor{currentstroke}%
\pgfsetdash{}{0pt}%
\pgfpathmoveto{\pgfqpoint{7.940217in}{3.698410in}}%
\pgfpathlineto{\pgfqpoint{10.467391in}{3.698410in}}%
\pgfusepath{stroke}%
\end{pgfscope}%
\begin{pgfscope}%
\pgfpathrectangle{\pgfqpoint{7.940217in}{0.550000in}}{\pgfqpoint{2.527174in}{3.850000in}}%
\pgfusepath{clip}%
\pgfsetroundcap%
\pgfsetroundjoin%
\pgfsetlinewidth{1.003750pt}%
\definecolor{currentstroke}{rgb}{0.800000,0.800000,0.800000}%
\pgfsetstrokecolor{currentstroke}%
\pgfsetdash{}{0pt}%
\pgfpathmoveto{\pgfqpoint{7.940217in}{4.301729in}}%
\pgfpathlineto{\pgfqpoint{10.467391in}{4.301729in}}%
\pgfusepath{stroke}%
\end{pgfscope}%
\begin{pgfscope}%
\definecolor{textcolor}{rgb}{0.150000,0.150000,0.150000}%
\pgfsetstrokecolor{textcolor}%
\pgfsetfillcolor{textcolor}%
\pgftext[x=7.884662in,y=2.475000in,,bottom,rotate=90.000000]{\color{textcolor}\sffamily\fontsize{12.000000}{14.400000}\selectfont \ }%
\end{pgfscope}%
\begin{pgfscope}%
\pgfpathrectangle{\pgfqpoint{7.940217in}{0.550000in}}{\pgfqpoint{2.527174in}{3.850000in}}%
\pgfusepath{clip}%
\pgfsetbuttcap%
\pgfsetroundjoin%
\definecolor{currentfill}{rgb}{0.347059,0.458824,0.641176}%
\pgfsetfillcolor{currentfill}%
\pgfsetlinewidth{1.505625pt}%
\definecolor{currentstroke}{rgb}{0.298039,0.298039,0.298039}%
\pgfsetstrokecolor{currentstroke}%
\pgfsetdash{}{0pt}%
\pgfsys@defobject{currentmarker}{\pgfqpoint{8.192935in}{0.796166in}}{\pgfqpoint{9.203804in}{1.053685in}}{%
\pgfpathmoveto{\pgfqpoint{9.203804in}{0.796166in}}%
\pgfpathlineto{\pgfqpoint{9.201523in}{0.796166in}}%
\pgfpathlineto{\pgfqpoint{9.200052in}{0.798767in}}%
\pgfpathlineto{\pgfqpoint{9.201951in}{0.801368in}}%
\pgfpathlineto{\pgfqpoint{9.202154in}{0.803969in}}%
\pgfpathlineto{\pgfqpoint{9.201062in}{0.806571in}}%
\pgfpathlineto{\pgfqpoint{9.177270in}{0.809172in}}%
\pgfpathlineto{\pgfqpoint{9.015703in}{0.811773in}}%
\pgfpathlineto{\pgfqpoint{8.580702in}{0.814374in}}%
\pgfpathlineto{\pgfqpoint{8.192935in}{0.816975in}}%
\pgfpathlineto{\pgfqpoint{8.384049in}{0.819577in}}%
\pgfpathlineto{\pgfqpoint{8.862071in}{0.822178in}}%
\pgfpathlineto{\pgfqpoint{9.106300in}{0.824779in}}%
\pgfpathlineto{\pgfqpoint{9.176256in}{0.827380in}}%
\pgfpathlineto{\pgfqpoint{9.196147in}{0.829981in}}%
\pgfpathlineto{\pgfqpoint{9.203238in}{0.832583in}}%
\pgfpathlineto{\pgfqpoint{9.203804in}{0.835184in}}%
\pgfpathlineto{\pgfqpoint{9.203804in}{0.837785in}}%
\pgfpathlineto{\pgfqpoint{9.203804in}{0.840386in}}%
\pgfpathlineto{\pgfqpoint{9.203804in}{0.842987in}}%
\pgfpathlineto{\pgfqpoint{9.203804in}{0.845589in}}%
\pgfpathlineto{\pgfqpoint{9.203804in}{0.848190in}}%
\pgfpathlineto{\pgfqpoint{9.203804in}{0.850791in}}%
\pgfpathlineto{\pgfqpoint{9.203804in}{0.853392in}}%
\pgfpathlineto{\pgfqpoint{9.203804in}{0.855993in}}%
\pgfpathlineto{\pgfqpoint{9.203804in}{0.858595in}}%
\pgfpathlineto{\pgfqpoint{9.203804in}{0.861196in}}%
\pgfpathlineto{\pgfqpoint{9.203804in}{0.863797in}}%
\pgfpathlineto{\pgfqpoint{9.203804in}{0.866398in}}%
\pgfpathlineto{\pgfqpoint{9.203804in}{0.868999in}}%
\pgfpathlineto{\pgfqpoint{9.203804in}{0.871601in}}%
\pgfpathlineto{\pgfqpoint{9.203804in}{0.874202in}}%
\pgfpathlineto{\pgfqpoint{9.203804in}{0.876803in}}%
\pgfpathlineto{\pgfqpoint{9.203804in}{0.879404in}}%
\pgfpathlineto{\pgfqpoint{9.203804in}{0.882005in}}%
\pgfpathlineto{\pgfqpoint{9.203804in}{0.884607in}}%
\pgfpathlineto{\pgfqpoint{9.203804in}{0.887208in}}%
\pgfpathlineto{\pgfqpoint{9.203804in}{0.889809in}}%
\pgfpathlineto{\pgfqpoint{9.203804in}{0.892410in}}%
\pgfpathlineto{\pgfqpoint{9.203804in}{0.895011in}}%
\pgfpathlineto{\pgfqpoint{9.203804in}{0.897613in}}%
\pgfpathlineto{\pgfqpoint{9.203804in}{0.900214in}}%
\pgfpathlineto{\pgfqpoint{9.203804in}{0.902815in}}%
\pgfpathlineto{\pgfqpoint{9.203804in}{0.905416in}}%
\pgfpathlineto{\pgfqpoint{9.203804in}{0.908018in}}%
\pgfpathlineto{\pgfqpoint{9.203804in}{0.910619in}}%
\pgfpathlineto{\pgfqpoint{9.203804in}{0.913220in}}%
\pgfpathlineto{\pgfqpoint{9.203804in}{0.915821in}}%
\pgfpathlineto{\pgfqpoint{9.203804in}{0.918422in}}%
\pgfpathlineto{\pgfqpoint{9.203804in}{0.921024in}}%
\pgfpathlineto{\pgfqpoint{9.203804in}{0.923625in}}%
\pgfpathlineto{\pgfqpoint{9.203804in}{0.926226in}}%
\pgfpathlineto{\pgfqpoint{9.203804in}{0.928827in}}%
\pgfpathlineto{\pgfqpoint{9.203804in}{0.931428in}}%
\pgfpathlineto{\pgfqpoint{9.203804in}{0.934030in}}%
\pgfpathlineto{\pgfqpoint{9.203804in}{0.936631in}}%
\pgfpathlineto{\pgfqpoint{9.203804in}{0.939232in}}%
\pgfpathlineto{\pgfqpoint{9.203804in}{0.941833in}}%
\pgfpathlineto{\pgfqpoint{9.203804in}{0.944434in}}%
\pgfpathlineto{\pgfqpoint{9.203804in}{0.947036in}}%
\pgfpathlineto{\pgfqpoint{9.203804in}{0.949637in}}%
\pgfpathlineto{\pgfqpoint{9.203804in}{0.952238in}}%
\pgfpathlineto{\pgfqpoint{9.203804in}{0.954839in}}%
\pgfpathlineto{\pgfqpoint{9.203804in}{0.957440in}}%
\pgfpathlineto{\pgfqpoint{9.203804in}{0.960042in}}%
\pgfpathlineto{\pgfqpoint{9.203804in}{0.962643in}}%
\pgfpathlineto{\pgfqpoint{9.203804in}{0.965244in}}%
\pgfpathlineto{\pgfqpoint{9.203804in}{0.967845in}}%
\pgfpathlineto{\pgfqpoint{9.203804in}{0.970446in}}%
\pgfpathlineto{\pgfqpoint{9.203804in}{0.973048in}}%
\pgfpathlineto{\pgfqpoint{9.203804in}{0.975649in}}%
\pgfpathlineto{\pgfqpoint{9.203804in}{0.978250in}}%
\pgfpathlineto{\pgfqpoint{9.203804in}{0.980851in}}%
\pgfpathlineto{\pgfqpoint{9.203804in}{0.983452in}}%
\pgfpathlineto{\pgfqpoint{9.203804in}{0.986054in}}%
\pgfpathlineto{\pgfqpoint{9.203804in}{0.988655in}}%
\pgfpathlineto{\pgfqpoint{9.203804in}{0.991256in}}%
\pgfpathlineto{\pgfqpoint{9.203804in}{0.993857in}}%
\pgfpathlineto{\pgfqpoint{9.203804in}{0.996458in}}%
\pgfpathlineto{\pgfqpoint{9.203804in}{0.999060in}}%
\pgfpathlineto{\pgfqpoint{9.203804in}{1.001661in}}%
\pgfpathlineto{\pgfqpoint{9.203804in}{1.004262in}}%
\pgfpathlineto{\pgfqpoint{9.203804in}{1.006863in}}%
\pgfpathlineto{\pgfqpoint{9.203804in}{1.009464in}}%
\pgfpathlineto{\pgfqpoint{9.203804in}{1.012066in}}%
\pgfpathlineto{\pgfqpoint{9.203804in}{1.014667in}}%
\pgfpathlineto{\pgfqpoint{9.203804in}{1.017268in}}%
\pgfpathlineto{\pgfqpoint{9.203804in}{1.019869in}}%
\pgfpathlineto{\pgfqpoint{9.203804in}{1.022470in}}%
\pgfpathlineto{\pgfqpoint{9.203804in}{1.025072in}}%
\pgfpathlineto{\pgfqpoint{9.203804in}{1.027673in}}%
\pgfpathlineto{\pgfqpoint{9.203804in}{1.030274in}}%
\pgfpathlineto{\pgfqpoint{9.203804in}{1.032875in}}%
\pgfpathlineto{\pgfqpoint{9.203804in}{1.035476in}}%
\pgfpathlineto{\pgfqpoint{9.203804in}{1.038078in}}%
\pgfpathlineto{\pgfqpoint{9.203804in}{1.040679in}}%
\pgfpathlineto{\pgfqpoint{9.203804in}{1.043280in}}%
\pgfpathlineto{\pgfqpoint{9.203804in}{1.045881in}}%
\pgfpathlineto{\pgfqpoint{9.203804in}{1.048483in}}%
\pgfpathlineto{\pgfqpoint{9.203785in}{1.051084in}}%
\pgfpathlineto{\pgfqpoint{9.202955in}{1.053685in}}%
\pgfpathlineto{\pgfqpoint{9.203804in}{1.053685in}}%
\pgfpathlineto{\pgfqpoint{9.203804in}{1.053685in}}%
\pgfpathlineto{\pgfqpoint{9.203804in}{1.051084in}}%
\pgfpathlineto{\pgfqpoint{9.203804in}{1.048483in}}%
\pgfpathlineto{\pgfqpoint{9.203804in}{1.045881in}}%
\pgfpathlineto{\pgfqpoint{9.203804in}{1.043280in}}%
\pgfpathlineto{\pgfqpoint{9.203804in}{1.040679in}}%
\pgfpathlineto{\pgfqpoint{9.203804in}{1.038078in}}%
\pgfpathlineto{\pgfqpoint{9.203804in}{1.035476in}}%
\pgfpathlineto{\pgfqpoint{9.203804in}{1.032875in}}%
\pgfpathlineto{\pgfqpoint{9.203804in}{1.030274in}}%
\pgfpathlineto{\pgfqpoint{9.203804in}{1.027673in}}%
\pgfpathlineto{\pgfqpoint{9.203804in}{1.025072in}}%
\pgfpathlineto{\pgfqpoint{9.203804in}{1.022470in}}%
\pgfpathlineto{\pgfqpoint{9.203804in}{1.019869in}}%
\pgfpathlineto{\pgfqpoint{9.203804in}{1.017268in}}%
\pgfpathlineto{\pgfqpoint{9.203804in}{1.014667in}}%
\pgfpathlineto{\pgfqpoint{9.203804in}{1.012066in}}%
\pgfpathlineto{\pgfqpoint{9.203804in}{1.009464in}}%
\pgfpathlineto{\pgfqpoint{9.203804in}{1.006863in}}%
\pgfpathlineto{\pgfqpoint{9.203804in}{1.004262in}}%
\pgfpathlineto{\pgfqpoint{9.203804in}{1.001661in}}%
\pgfpathlineto{\pgfqpoint{9.203804in}{0.999060in}}%
\pgfpathlineto{\pgfqpoint{9.203804in}{0.996458in}}%
\pgfpathlineto{\pgfqpoint{9.203804in}{0.993857in}}%
\pgfpathlineto{\pgfqpoint{9.203804in}{0.991256in}}%
\pgfpathlineto{\pgfqpoint{9.203804in}{0.988655in}}%
\pgfpathlineto{\pgfqpoint{9.203804in}{0.986054in}}%
\pgfpathlineto{\pgfqpoint{9.203804in}{0.983452in}}%
\pgfpathlineto{\pgfqpoint{9.203804in}{0.980851in}}%
\pgfpathlineto{\pgfqpoint{9.203804in}{0.978250in}}%
\pgfpathlineto{\pgfqpoint{9.203804in}{0.975649in}}%
\pgfpathlineto{\pgfqpoint{9.203804in}{0.973048in}}%
\pgfpathlineto{\pgfqpoint{9.203804in}{0.970446in}}%
\pgfpathlineto{\pgfqpoint{9.203804in}{0.967845in}}%
\pgfpathlineto{\pgfqpoint{9.203804in}{0.965244in}}%
\pgfpathlineto{\pgfqpoint{9.203804in}{0.962643in}}%
\pgfpathlineto{\pgfqpoint{9.203804in}{0.960042in}}%
\pgfpathlineto{\pgfqpoint{9.203804in}{0.957440in}}%
\pgfpathlineto{\pgfqpoint{9.203804in}{0.954839in}}%
\pgfpathlineto{\pgfqpoint{9.203804in}{0.952238in}}%
\pgfpathlineto{\pgfqpoint{9.203804in}{0.949637in}}%
\pgfpathlineto{\pgfqpoint{9.203804in}{0.947036in}}%
\pgfpathlineto{\pgfqpoint{9.203804in}{0.944434in}}%
\pgfpathlineto{\pgfqpoint{9.203804in}{0.941833in}}%
\pgfpathlineto{\pgfqpoint{9.203804in}{0.939232in}}%
\pgfpathlineto{\pgfqpoint{9.203804in}{0.936631in}}%
\pgfpathlineto{\pgfqpoint{9.203804in}{0.934030in}}%
\pgfpathlineto{\pgfqpoint{9.203804in}{0.931428in}}%
\pgfpathlineto{\pgfqpoint{9.203804in}{0.928827in}}%
\pgfpathlineto{\pgfqpoint{9.203804in}{0.926226in}}%
\pgfpathlineto{\pgfqpoint{9.203804in}{0.923625in}}%
\pgfpathlineto{\pgfqpoint{9.203804in}{0.921024in}}%
\pgfpathlineto{\pgfqpoint{9.203804in}{0.918422in}}%
\pgfpathlineto{\pgfqpoint{9.203804in}{0.915821in}}%
\pgfpathlineto{\pgfqpoint{9.203804in}{0.913220in}}%
\pgfpathlineto{\pgfqpoint{9.203804in}{0.910619in}}%
\pgfpathlineto{\pgfqpoint{9.203804in}{0.908018in}}%
\pgfpathlineto{\pgfqpoint{9.203804in}{0.905416in}}%
\pgfpathlineto{\pgfqpoint{9.203804in}{0.902815in}}%
\pgfpathlineto{\pgfqpoint{9.203804in}{0.900214in}}%
\pgfpathlineto{\pgfqpoint{9.203804in}{0.897613in}}%
\pgfpathlineto{\pgfqpoint{9.203804in}{0.895011in}}%
\pgfpathlineto{\pgfqpoint{9.203804in}{0.892410in}}%
\pgfpathlineto{\pgfqpoint{9.203804in}{0.889809in}}%
\pgfpathlineto{\pgfqpoint{9.203804in}{0.887208in}}%
\pgfpathlineto{\pgfqpoint{9.203804in}{0.884607in}}%
\pgfpathlineto{\pgfqpoint{9.203804in}{0.882005in}}%
\pgfpathlineto{\pgfqpoint{9.203804in}{0.879404in}}%
\pgfpathlineto{\pgfqpoint{9.203804in}{0.876803in}}%
\pgfpathlineto{\pgfqpoint{9.203804in}{0.874202in}}%
\pgfpathlineto{\pgfqpoint{9.203804in}{0.871601in}}%
\pgfpathlineto{\pgfqpoint{9.203804in}{0.868999in}}%
\pgfpathlineto{\pgfqpoint{9.203804in}{0.866398in}}%
\pgfpathlineto{\pgfqpoint{9.203804in}{0.863797in}}%
\pgfpathlineto{\pgfqpoint{9.203804in}{0.861196in}}%
\pgfpathlineto{\pgfqpoint{9.203804in}{0.858595in}}%
\pgfpathlineto{\pgfqpoint{9.203804in}{0.855993in}}%
\pgfpathlineto{\pgfqpoint{9.203804in}{0.853392in}}%
\pgfpathlineto{\pgfqpoint{9.203804in}{0.850791in}}%
\pgfpathlineto{\pgfqpoint{9.203804in}{0.848190in}}%
\pgfpathlineto{\pgfqpoint{9.203804in}{0.845589in}}%
\pgfpathlineto{\pgfqpoint{9.203804in}{0.842987in}}%
\pgfpathlineto{\pgfqpoint{9.203804in}{0.840386in}}%
\pgfpathlineto{\pgfqpoint{9.203804in}{0.837785in}}%
\pgfpathlineto{\pgfqpoint{9.203804in}{0.835184in}}%
\pgfpathlineto{\pgfqpoint{9.203804in}{0.832583in}}%
\pgfpathlineto{\pgfqpoint{9.203804in}{0.829981in}}%
\pgfpathlineto{\pgfqpoint{9.203804in}{0.827380in}}%
\pgfpathlineto{\pgfqpoint{9.203804in}{0.824779in}}%
\pgfpathlineto{\pgfqpoint{9.203804in}{0.822178in}}%
\pgfpathlineto{\pgfqpoint{9.203804in}{0.819577in}}%
\pgfpathlineto{\pgfqpoint{9.203804in}{0.816975in}}%
\pgfpathlineto{\pgfqpoint{9.203804in}{0.814374in}}%
\pgfpathlineto{\pgfqpoint{9.203804in}{0.811773in}}%
\pgfpathlineto{\pgfqpoint{9.203804in}{0.809172in}}%
\pgfpathlineto{\pgfqpoint{9.203804in}{0.806571in}}%
\pgfpathlineto{\pgfqpoint{9.203804in}{0.803969in}}%
\pgfpathlineto{\pgfqpoint{9.203804in}{0.801368in}}%
\pgfpathlineto{\pgfqpoint{9.203804in}{0.798767in}}%
\pgfpathlineto{\pgfqpoint{9.203804in}{0.796166in}}%
\pgfpathclose%
\pgfusepath{stroke,fill}%
}%
\begin{pgfscope}%
\pgfsys@transformshift{0.000000in}{0.000000in}%
\pgfsys@useobject{currentmarker}{}%
\end{pgfscope}%
\end{pgfscope}%
\begin{pgfscope}%
\pgfpathrectangle{\pgfqpoint{7.940217in}{0.550000in}}{\pgfqpoint{2.527174in}{3.850000in}}%
\pgfusepath{clip}%
\pgfsetbuttcap%
\pgfsetroundjoin%
\definecolor{currentfill}{rgb}{0.798529,0.536765,0.389706}%
\pgfsetfillcolor{currentfill}%
\pgfsetlinewidth{1.505625pt}%
\definecolor{currentstroke}{rgb}{0.298039,0.298039,0.298039}%
\pgfsetstrokecolor{currentstroke}%
\pgfsetdash{}{0pt}%
\pgfsys@defobject{currentmarker}{\pgfqpoint{9.203804in}{0.772821in}}{\pgfqpoint{9.759585in}{1.992236in}}{%
\pgfpathmoveto{\pgfqpoint{9.347051in}{0.772821in}}%
\pgfpathlineto{\pgfqpoint{9.203804in}{0.772821in}}%
\pgfpathlineto{\pgfqpoint{9.203804in}{0.785138in}}%
\pgfpathlineto{\pgfqpoint{9.203804in}{0.797455in}}%
\pgfpathlineto{\pgfqpoint{9.203804in}{0.809773in}}%
\pgfpathlineto{\pgfqpoint{9.203804in}{0.822090in}}%
\pgfpathlineto{\pgfqpoint{9.203804in}{0.834407in}}%
\pgfpathlineto{\pgfqpoint{9.203804in}{0.846725in}}%
\pgfpathlineto{\pgfqpoint{9.203804in}{0.859042in}}%
\pgfpathlineto{\pgfqpoint{9.203804in}{0.871359in}}%
\pgfpathlineto{\pgfqpoint{9.203804in}{0.883677in}}%
\pgfpathlineto{\pgfqpoint{9.203804in}{0.895994in}}%
\pgfpathlineto{\pgfqpoint{9.203804in}{0.908311in}}%
\pgfpathlineto{\pgfqpoint{9.203804in}{0.920628in}}%
\pgfpathlineto{\pgfqpoint{9.203804in}{0.932946in}}%
\pgfpathlineto{\pgfqpoint{9.203804in}{0.945263in}}%
\pgfpathlineto{\pgfqpoint{9.203804in}{0.957580in}}%
\pgfpathlineto{\pgfqpoint{9.203804in}{0.969898in}}%
\pgfpathlineto{\pgfqpoint{9.203804in}{0.982215in}}%
\pgfpathlineto{\pgfqpoint{9.203804in}{0.994532in}}%
\pgfpathlineto{\pgfqpoint{9.203804in}{1.006850in}}%
\pgfpathlineto{\pgfqpoint{9.203804in}{1.019167in}}%
\pgfpathlineto{\pgfqpoint{9.203804in}{1.031484in}}%
\pgfpathlineto{\pgfqpoint{9.203804in}{1.043802in}}%
\pgfpathlineto{\pgfqpoint{9.203804in}{1.056119in}}%
\pgfpathlineto{\pgfqpoint{9.203804in}{1.068436in}}%
\pgfpathlineto{\pgfqpoint{9.203804in}{1.080754in}}%
\pgfpathlineto{\pgfqpoint{9.203804in}{1.093071in}}%
\pgfpathlineto{\pgfqpoint{9.203804in}{1.105388in}}%
\pgfpathlineto{\pgfqpoint{9.203804in}{1.117706in}}%
\pgfpathlineto{\pgfqpoint{9.203804in}{1.130023in}}%
\pgfpathlineto{\pgfqpoint{9.203804in}{1.142340in}}%
\pgfpathlineto{\pgfqpoint{9.203804in}{1.154658in}}%
\pgfpathlineto{\pgfqpoint{9.203804in}{1.166975in}}%
\pgfpathlineto{\pgfqpoint{9.203804in}{1.179292in}}%
\pgfpathlineto{\pgfqpoint{9.203804in}{1.191610in}}%
\pgfpathlineto{\pgfqpoint{9.203804in}{1.203927in}}%
\pgfpathlineto{\pgfqpoint{9.203804in}{1.216244in}}%
\pgfpathlineto{\pgfqpoint{9.203804in}{1.228562in}}%
\pgfpathlineto{\pgfqpoint{9.203804in}{1.240879in}}%
\pgfpathlineto{\pgfqpoint{9.203804in}{1.253196in}}%
\pgfpathlineto{\pgfqpoint{9.203804in}{1.265514in}}%
\pgfpathlineto{\pgfqpoint{9.203804in}{1.277831in}}%
\pgfpathlineto{\pgfqpoint{9.203804in}{1.290148in}}%
\pgfpathlineto{\pgfqpoint{9.203804in}{1.302465in}}%
\pgfpathlineto{\pgfqpoint{9.203804in}{1.314783in}}%
\pgfpathlineto{\pgfqpoint{9.203804in}{1.327100in}}%
\pgfpathlineto{\pgfqpoint{9.203804in}{1.339417in}}%
\pgfpathlineto{\pgfqpoint{9.203804in}{1.351735in}}%
\pgfpathlineto{\pgfqpoint{9.203804in}{1.364052in}}%
\pgfpathlineto{\pgfqpoint{9.203804in}{1.376369in}}%
\pgfpathlineto{\pgfqpoint{9.203804in}{1.388687in}}%
\pgfpathlineto{\pgfqpoint{9.203804in}{1.401004in}}%
\pgfpathlineto{\pgfqpoint{9.203804in}{1.413321in}}%
\pgfpathlineto{\pgfqpoint{9.203804in}{1.425639in}}%
\pgfpathlineto{\pgfqpoint{9.203804in}{1.437956in}}%
\pgfpathlineto{\pgfqpoint{9.203804in}{1.450273in}}%
\pgfpathlineto{\pgfqpoint{9.203804in}{1.462591in}}%
\pgfpathlineto{\pgfqpoint{9.203804in}{1.474908in}}%
\pgfpathlineto{\pgfqpoint{9.203804in}{1.487225in}}%
\pgfpathlineto{\pgfqpoint{9.203804in}{1.499543in}}%
\pgfpathlineto{\pgfqpoint{9.203804in}{1.511860in}}%
\pgfpathlineto{\pgfqpoint{9.203804in}{1.524177in}}%
\pgfpathlineto{\pgfqpoint{9.203804in}{1.536495in}}%
\pgfpathlineto{\pgfqpoint{9.203804in}{1.548812in}}%
\pgfpathlineto{\pgfqpoint{9.203804in}{1.561129in}}%
\pgfpathlineto{\pgfqpoint{9.203804in}{1.573447in}}%
\pgfpathlineto{\pgfqpoint{9.203804in}{1.585764in}}%
\pgfpathlineto{\pgfqpoint{9.203804in}{1.598081in}}%
\pgfpathlineto{\pgfqpoint{9.203804in}{1.610399in}}%
\pgfpathlineto{\pgfqpoint{9.203804in}{1.622716in}}%
\pgfpathlineto{\pgfqpoint{9.203804in}{1.635033in}}%
\pgfpathlineto{\pgfqpoint{9.203804in}{1.647351in}}%
\pgfpathlineto{\pgfqpoint{9.203804in}{1.659668in}}%
\pgfpathlineto{\pgfqpoint{9.203804in}{1.671985in}}%
\pgfpathlineto{\pgfqpoint{9.203804in}{1.684302in}}%
\pgfpathlineto{\pgfqpoint{9.203804in}{1.696620in}}%
\pgfpathlineto{\pgfqpoint{9.203804in}{1.708937in}}%
\pgfpathlineto{\pgfqpoint{9.203804in}{1.721254in}}%
\pgfpathlineto{\pgfqpoint{9.203804in}{1.733572in}}%
\pgfpathlineto{\pgfqpoint{9.203804in}{1.745889in}}%
\pgfpathlineto{\pgfqpoint{9.203804in}{1.758206in}}%
\pgfpathlineto{\pgfqpoint{9.203804in}{1.770524in}}%
\pgfpathlineto{\pgfqpoint{9.203804in}{1.782841in}}%
\pgfpathlineto{\pgfqpoint{9.203804in}{1.795158in}}%
\pgfpathlineto{\pgfqpoint{9.203804in}{1.807476in}}%
\pgfpathlineto{\pgfqpoint{9.203804in}{1.819793in}}%
\pgfpathlineto{\pgfqpoint{9.203804in}{1.832110in}}%
\pgfpathlineto{\pgfqpoint{9.203804in}{1.844428in}}%
\pgfpathlineto{\pgfqpoint{9.203804in}{1.856745in}}%
\pgfpathlineto{\pgfqpoint{9.203804in}{1.869062in}}%
\pgfpathlineto{\pgfqpoint{9.203804in}{1.881380in}}%
\pgfpathlineto{\pgfqpoint{9.203804in}{1.893697in}}%
\pgfpathlineto{\pgfqpoint{9.203804in}{1.906014in}}%
\pgfpathlineto{\pgfqpoint{9.203804in}{1.918332in}}%
\pgfpathlineto{\pgfqpoint{9.203804in}{1.930649in}}%
\pgfpathlineto{\pgfqpoint{9.203804in}{1.942966in}}%
\pgfpathlineto{\pgfqpoint{9.203804in}{1.955284in}}%
\pgfpathlineto{\pgfqpoint{9.203804in}{1.967601in}}%
\pgfpathlineto{\pgfqpoint{9.203804in}{1.979918in}}%
\pgfpathlineto{\pgfqpoint{9.203804in}{1.992236in}}%
\pgfpathlineto{\pgfqpoint{9.206559in}{1.992236in}}%
\pgfpathlineto{\pgfqpoint{9.206559in}{1.992236in}}%
\pgfpathlineto{\pgfqpoint{9.206815in}{1.979918in}}%
\pgfpathlineto{\pgfqpoint{9.206970in}{1.967601in}}%
\pgfpathlineto{\pgfqpoint{9.207068in}{1.955284in}}%
\pgfpathlineto{\pgfqpoint{9.207165in}{1.942966in}}%
\pgfpathlineto{\pgfqpoint{9.207316in}{1.930649in}}%
\pgfpathlineto{\pgfqpoint{9.207563in}{1.918332in}}%
\pgfpathlineto{\pgfqpoint{9.207937in}{1.906014in}}%
\pgfpathlineto{\pgfqpoint{9.208449in}{1.893697in}}%
\pgfpathlineto{\pgfqpoint{9.209089in}{1.881380in}}%
\pgfpathlineto{\pgfqpoint{9.209809in}{1.869062in}}%
\pgfpathlineto{\pgfqpoint{9.210528in}{1.856745in}}%
\pgfpathlineto{\pgfqpoint{9.211131in}{1.844428in}}%
\pgfpathlineto{\pgfqpoint{9.211511in}{1.832110in}}%
\pgfpathlineto{\pgfqpoint{9.211598in}{1.819793in}}%
\pgfpathlineto{\pgfqpoint{9.211400in}{1.807476in}}%
\pgfpathlineto{\pgfqpoint{9.211008in}{1.795158in}}%
\pgfpathlineto{\pgfqpoint{9.210575in}{1.782841in}}%
\pgfpathlineto{\pgfqpoint{9.210271in}{1.770524in}}%
\pgfpathlineto{\pgfqpoint{9.210232in}{1.758206in}}%
\pgfpathlineto{\pgfqpoint{9.210528in}{1.745889in}}%
\pgfpathlineto{\pgfqpoint{9.211150in}{1.733572in}}%
\pgfpathlineto{\pgfqpoint{9.212036in}{1.721254in}}%
\pgfpathlineto{\pgfqpoint{9.213101in}{1.708937in}}%
\pgfpathlineto{\pgfqpoint{9.214278in}{1.696620in}}%
\pgfpathlineto{\pgfqpoint{9.215537in}{1.684302in}}%
\pgfpathlineto{\pgfqpoint{9.216883in}{1.671985in}}%
\pgfpathlineto{\pgfqpoint{9.218330in}{1.659668in}}%
\pgfpathlineto{\pgfqpoint{9.219868in}{1.647351in}}%
\pgfpathlineto{\pgfqpoint{9.221435in}{1.635033in}}%
\pgfpathlineto{\pgfqpoint{9.222930in}{1.622716in}}%
\pgfpathlineto{\pgfqpoint{9.224265in}{1.610399in}}%
\pgfpathlineto{\pgfqpoint{9.225422in}{1.598081in}}%
\pgfpathlineto{\pgfqpoint{9.226490in}{1.585764in}}%
\pgfpathlineto{\pgfqpoint{9.227651in}{1.573447in}}%
\pgfpathlineto{\pgfqpoint{9.229096in}{1.561129in}}%
\pgfpathlineto{\pgfqpoint{9.230920in}{1.548812in}}%
\pgfpathlineto{\pgfqpoint{9.233054in}{1.536495in}}%
\pgfpathlineto{\pgfqpoint{9.235264in}{1.524177in}}%
\pgfpathlineto{\pgfqpoint{9.237233in}{1.511860in}}%
\pgfpathlineto{\pgfqpoint{9.238679in}{1.499543in}}%
\pgfpathlineto{\pgfqpoint{9.239461in}{1.487225in}}%
\pgfpathlineto{\pgfqpoint{9.239622in}{1.474908in}}%
\pgfpathlineto{\pgfqpoint{9.239381in}{1.462591in}}%
\pgfpathlineto{\pgfqpoint{9.239084in}{1.450273in}}%
\pgfpathlineto{\pgfqpoint{9.239131in}{1.437956in}}%
\pgfpathlineto{\pgfqpoint{9.239918in}{1.425639in}}%
\pgfpathlineto{\pgfqpoint{9.241764in}{1.413321in}}%
\pgfpathlineto{\pgfqpoint{9.244852in}{1.401004in}}%
\pgfpathlineto{\pgfqpoint{9.249188in}{1.388687in}}%
\pgfpathlineto{\pgfqpoint{9.254588in}{1.376369in}}%
\pgfpathlineto{\pgfqpoint{9.260715in}{1.364052in}}%
\pgfpathlineto{\pgfqpoint{9.267131in}{1.351735in}}%
\pgfpathlineto{\pgfqpoint{9.273371in}{1.339417in}}%
\pgfpathlineto{\pgfqpoint{9.279003in}{1.327100in}}%
\pgfpathlineto{\pgfqpoint{9.283700in}{1.314783in}}%
\pgfpathlineto{\pgfqpoint{9.287305in}{1.302465in}}%
\pgfpathlineto{\pgfqpoint{9.289887in}{1.290148in}}%
\pgfpathlineto{\pgfqpoint{9.291795in}{1.277831in}}%
\pgfpathlineto{\pgfqpoint{9.293671in}{1.265514in}}%
\pgfpathlineto{\pgfqpoint{9.296463in}{1.253196in}}%
\pgfpathlineto{\pgfqpoint{9.301398in}{1.240879in}}%
\pgfpathlineto{\pgfqpoint{9.309917in}{1.228562in}}%
\pgfpathlineto{\pgfqpoint{9.323533in}{1.216244in}}%
\pgfpathlineto{\pgfqpoint{9.343611in}{1.203927in}}%
\pgfpathlineto{\pgfqpoint{9.371090in}{1.191610in}}%
\pgfpathlineto{\pgfqpoint{9.406235in}{1.179292in}}%
\pgfpathlineto{\pgfqpoint{9.448484in}{1.166975in}}%
\pgfpathlineto{\pgfqpoint{9.496448in}{1.154658in}}%
\pgfpathlineto{\pgfqpoint{9.548031in}{1.142340in}}%
\pgfpathlineto{\pgfqpoint{9.600593in}{1.130023in}}%
\pgfpathlineto{\pgfqpoint{9.651051in}{1.117706in}}%
\pgfpathlineto{\pgfqpoint{9.695927in}{1.105388in}}%
\pgfpathlineto{\pgfqpoint{9.731429in}{1.093071in}}%
\pgfpathlineto{\pgfqpoint{9.753736in}{1.080754in}}%
\pgfpathlineto{\pgfqpoint{9.759585in}{1.068436in}}%
\pgfpathlineto{\pgfqpoint{9.747060in}{1.056119in}}%
\pgfpathlineto{\pgfqpoint{9.716315in}{1.043802in}}%
\pgfpathlineto{\pgfqpoint{9.669910in}{1.031484in}}%
\pgfpathlineto{\pgfqpoint{9.612591in}{1.019167in}}%
\pgfpathlineto{\pgfqpoint{9.550606in}{1.006850in}}%
\pgfpathlineto{\pgfqpoint{9.490769in}{0.994532in}}%
\pgfpathlineto{\pgfqpoint{9.439521in}{0.982215in}}%
\pgfpathlineto{\pgfqpoint{9.402098in}{0.969898in}}%
\pgfpathlineto{\pgfqpoint{9.381857in}{0.957580in}}%
\pgfpathlineto{\pgfqpoint{9.379807in}{0.945263in}}%
\pgfpathlineto{\pgfqpoint{9.394452in}{0.932946in}}%
\pgfpathlineto{\pgfqpoint{9.422063in}{0.920628in}}%
\pgfpathlineto{\pgfqpoint{9.457400in}{0.908311in}}%
\pgfpathlineto{\pgfqpoint{9.494727in}{0.895994in}}%
\pgfpathlineto{\pgfqpoint{9.528789in}{0.883677in}}%
\pgfpathlineto{\pgfqpoint{9.555386in}{0.871359in}}%
\pgfpathlineto{\pgfqpoint{9.571433in}{0.859042in}}%
\pgfpathlineto{\pgfqpoint{9.574718in}{0.846725in}}%
\pgfpathlineto{\pgfqpoint{9.563796in}{0.834407in}}%
\pgfpathlineto{\pgfqpoint{9.538321in}{0.822090in}}%
\pgfpathlineto{\pgfqpoint{9.499662in}{0.809773in}}%
\pgfpathlineto{\pgfqpoint{9.451300in}{0.797455in}}%
\pgfpathlineto{\pgfqpoint{9.398478in}{0.785138in}}%
\pgfpathlineto{\pgfqpoint{9.347051in}{0.772821in}}%
\pgfpathclose%
\pgfusepath{stroke,fill}%
}%
\begin{pgfscope}%
\pgfsys@transformshift{0.000000in}{0.000000in}%
\pgfsys@useobject{currentmarker}{}%
\end{pgfscope}%
\end{pgfscope}%
\begin{pgfscope}%
\pgfpathrectangle{\pgfqpoint{7.940217in}{0.550000in}}{\pgfqpoint{2.527174in}{3.850000in}}%
\pgfusepath{clip}%
\pgfsetbuttcap%
\pgfsetmiterjoin%
\definecolor{currentfill}{rgb}{0.347059,0.458824,0.641176}%
\pgfsetfillcolor{currentfill}%
\pgfsetlinewidth{0.752812pt}%
\definecolor{currentstroke}{rgb}{0.298039,0.298039,0.298039}%
\pgfsetstrokecolor{currentstroke}%
\pgfsetdash{}{0pt}%
\pgfpathmoveto{\pgfqpoint{9.203804in}{0.681818in}}%
\pgfpathlineto{\pgfqpoint{9.203804in}{0.681818in}}%
\pgfpathlineto{\pgfqpoint{9.203804in}{0.681818in}}%
\pgfpathlineto{\pgfqpoint{9.203804in}{0.681818in}}%
\pgfpathclose%
\pgfusepath{stroke,fill}%
\end{pgfscope}%
\begin{pgfscope}%
\pgfpathrectangle{\pgfqpoint{7.940217in}{0.550000in}}{\pgfqpoint{2.527174in}{3.850000in}}%
\pgfusepath{clip}%
\pgfsetbuttcap%
\pgfsetmiterjoin%
\definecolor{currentfill}{rgb}{0.798529,0.536765,0.389706}%
\pgfsetfillcolor{currentfill}%
\pgfsetlinewidth{0.752812pt}%
\definecolor{currentstroke}{rgb}{0.298039,0.298039,0.298039}%
\pgfsetstrokecolor{currentstroke}%
\pgfsetdash{}{0pt}%
\pgfpathmoveto{\pgfqpoint{9.203804in}{0.681818in}}%
\pgfpathlineto{\pgfqpoint{9.203804in}{0.681818in}}%
\pgfpathlineto{\pgfqpoint{9.203804in}{0.681818in}}%
\pgfpathlineto{\pgfqpoint{9.203804in}{0.681818in}}%
\pgfpathclose%
\pgfusepath{stroke,fill}%
\end{pgfscope}%
\begin{pgfscope}%
\pgfpathrectangle{\pgfqpoint{7.940217in}{0.550000in}}{\pgfqpoint{2.527174in}{3.850000in}}%
\pgfusepath{clip}%
\pgfsetroundcap%
\pgfsetroundjoin%
\pgfsetlinewidth{1.505625pt}%
\definecolor{currentstroke}{rgb}{0.298039,0.298039,0.298039}%
\pgfsetstrokecolor{currentstroke}%
\pgfsetdash{}{0pt}%
\pgfpathmoveto{\pgfqpoint{9.203804in}{0.772821in}}%
\pgfpathlineto{\pgfqpoint{9.203804in}{1.052409in}}%
\pgfusepath{stroke}%
\end{pgfscope}%
\begin{pgfscope}%
\pgfpathrectangle{\pgfqpoint{7.940217in}{0.550000in}}{\pgfqpoint{2.527174in}{3.850000in}}%
\pgfusepath{clip}%
\pgfsetroundcap%
\pgfsetroundjoin%
\pgfsetlinewidth{4.516875pt}%
\definecolor{currentstroke}{rgb}{0.298039,0.298039,0.298039}%
\pgfsetstrokecolor{currentstroke}%
\pgfsetdash{}{0pt}%
\pgfpathmoveto{\pgfqpoint{9.203804in}{0.816404in}}%
\pgfpathlineto{\pgfqpoint{9.203804in}{0.910816in}}%
\pgfusepath{stroke}%
\end{pgfscope}%
\begin{pgfscope}%
\pgfsetrectcap%
\pgfsetmiterjoin%
\pgfsetlinewidth{1.254687pt}%
\definecolor{currentstroke}{rgb}{0.800000,0.800000,0.800000}%
\pgfsetstrokecolor{currentstroke}%
\pgfsetdash{}{0pt}%
\pgfpathmoveto{\pgfqpoint{7.940217in}{0.550000in}}%
\pgfpathlineto{\pgfqpoint{7.940217in}{4.400000in}}%
\pgfusepath{stroke}%
\end{pgfscope}%
\begin{pgfscope}%
\pgfsetrectcap%
\pgfsetmiterjoin%
\pgfsetlinewidth{1.254687pt}%
\definecolor{currentstroke}{rgb}{0.800000,0.800000,0.800000}%
\pgfsetstrokecolor{currentstroke}%
\pgfsetdash{}{0pt}%
\pgfpathmoveto{\pgfqpoint{10.467391in}{0.550000in}}%
\pgfpathlineto{\pgfqpoint{10.467391in}{4.400000in}}%
\pgfusepath{stroke}%
\end{pgfscope}%
\begin{pgfscope}%
\pgfsetrectcap%
\pgfsetmiterjoin%
\pgfsetlinewidth{1.254687pt}%
\definecolor{currentstroke}{rgb}{0.800000,0.800000,0.800000}%
\pgfsetstrokecolor{currentstroke}%
\pgfsetdash{}{0pt}%
\pgfpathmoveto{\pgfqpoint{7.940217in}{0.550000in}}%
\pgfpathlineto{\pgfqpoint{10.467391in}{0.550000in}}%
\pgfusepath{stroke}%
\end{pgfscope}%
\begin{pgfscope}%
\pgfsetrectcap%
\pgfsetmiterjoin%
\pgfsetlinewidth{1.254687pt}%
\definecolor{currentstroke}{rgb}{0.800000,0.800000,0.800000}%
\pgfsetstrokecolor{currentstroke}%
\pgfsetdash{}{0pt}%
\pgfpathmoveto{\pgfqpoint{7.940217in}{4.400000in}}%
\pgfpathlineto{\pgfqpoint{10.467391in}{4.400000in}}%
\pgfusepath{stroke}%
\end{pgfscope}%
\begin{pgfscope}%
\pgfpathrectangle{\pgfqpoint{7.940217in}{0.550000in}}{\pgfqpoint{2.527174in}{3.850000in}}%
\pgfusepath{clip}%
\pgfsetbuttcap%
\pgfsetroundjoin%
\definecolor{currentfill}{rgb}{1.000000,1.000000,1.000000}%
\pgfsetfillcolor{currentfill}%
\pgfsetlinewidth{1.003750pt}%
\definecolor{currentstroke}{rgb}{0.298039,0.298039,0.298039}%
\pgfsetstrokecolor{currentstroke}%
\pgfsetdash{}{0pt}%
\pgfsys@defobject{currentmarker}{\pgfqpoint{-0.020833in}{-0.020833in}}{\pgfqpoint{0.020833in}{0.020833in}}{%
\pgfpathmoveto{\pgfqpoint{0.000000in}{-0.020833in}}%
\pgfpathcurveto{\pgfqpoint{0.005525in}{-0.020833in}}{\pgfqpoint{0.010825in}{-0.018638in}}{\pgfqpoint{0.014731in}{-0.014731in}}%
\pgfpathcurveto{\pgfqpoint{0.018638in}{-0.010825in}}{\pgfqpoint{0.020833in}{-0.005525in}}{\pgfqpoint{0.020833in}{0.000000in}}%
\pgfpathcurveto{\pgfqpoint{0.020833in}{0.005525in}}{\pgfqpoint{0.018638in}{0.010825in}}{\pgfqpoint{0.014731in}{0.014731in}}%
\pgfpathcurveto{\pgfqpoint{0.010825in}{0.018638in}}{\pgfqpoint{0.005525in}{0.020833in}}{\pgfqpoint{0.000000in}{0.020833in}}%
\pgfpathcurveto{\pgfqpoint{-0.005525in}{0.020833in}}{\pgfqpoint{-0.010825in}{0.018638in}}{\pgfqpoint{-0.014731in}{0.014731in}}%
\pgfpathcurveto{\pgfqpoint{-0.018638in}{0.010825in}}{\pgfqpoint{-0.020833in}{0.005525in}}{\pgfqpoint{-0.020833in}{0.000000in}}%
\pgfpathcurveto{\pgfqpoint{-0.020833in}{-0.005525in}}{\pgfqpoint{-0.018638in}{-0.010825in}}{\pgfqpoint{-0.014731in}{-0.014731in}}%
\pgfpathcurveto{\pgfqpoint{-0.010825in}{-0.018638in}}{\pgfqpoint{-0.005525in}{-0.020833in}}{\pgfqpoint{0.000000in}{-0.020833in}}%
\pgfpathclose%
\pgfusepath{stroke,fill}%
}%
\begin{pgfscope}%
\pgfsys@transformshift{9.203804in}{0.819447in}%
\pgfsys@useobject{currentmarker}{}%
\end{pgfscope}%
\end{pgfscope}%
\begin{pgfscope}%
\pgfsetbuttcap%
\pgfsetmiterjoin%
\definecolor{currentfill}{rgb}{1.000000,1.000000,1.000000}%
\pgfsetfillcolor{currentfill}%
\pgfsetlinewidth{0.000000pt}%
\definecolor{currentstroke}{rgb}{0.000000,0.000000,0.000000}%
\pgfsetstrokecolor{currentstroke}%
\pgfsetstrokeopacity{0.000000}%
\pgfsetdash{}{0pt}%
\pgfpathmoveto{\pgfqpoint{10.972826in}{0.550000in}}%
\pgfpathlineto{\pgfqpoint{13.500000in}{0.550000in}}%
\pgfpathlineto{\pgfqpoint{13.500000in}{4.400000in}}%
\pgfpathlineto{\pgfqpoint{10.972826in}{4.400000in}}%
\pgfpathclose%
\pgfusepath{fill}%
\end{pgfscope}%
\begin{pgfscope}%
\definecolor{textcolor}{rgb}{0.150000,0.150000,0.150000}%
\pgfsetstrokecolor{textcolor}%
\pgfsetfillcolor{textcolor}%
\pgftext[x=12.236413in,y=0.418056in,,top]{\color{textcolor}\sffamily\fontsize{11.000000}{13.200000}\selectfont From t3}%
\end{pgfscope}%
\begin{pgfscope}%
\pgfpathrectangle{\pgfqpoint{10.972826in}{0.550000in}}{\pgfqpoint{2.527174in}{3.850000in}}%
\pgfusepath{clip}%
\pgfsetroundcap%
\pgfsetroundjoin%
\pgfsetlinewidth{1.003750pt}%
\definecolor{currentstroke}{rgb}{0.800000,0.800000,0.800000}%
\pgfsetstrokecolor{currentstroke}%
\pgfsetdash{}{0pt}%
\pgfpathmoveto{\pgfqpoint{10.972826in}{0.681818in}}%
\pgfpathlineto{\pgfqpoint{13.500000in}{0.681818in}}%
\pgfusepath{stroke}%
\end{pgfscope}%
\begin{pgfscope}%
\pgfpathrectangle{\pgfqpoint{10.972826in}{0.550000in}}{\pgfqpoint{2.527174in}{3.850000in}}%
\pgfusepath{clip}%
\pgfsetroundcap%
\pgfsetroundjoin%
\pgfsetlinewidth{1.003750pt}%
\definecolor{currentstroke}{rgb}{0.800000,0.800000,0.800000}%
\pgfsetstrokecolor{currentstroke}%
\pgfsetdash{}{0pt}%
\pgfpathmoveto{\pgfqpoint{10.972826in}{1.285136in}}%
\pgfpathlineto{\pgfqpoint{13.500000in}{1.285136in}}%
\pgfusepath{stroke}%
\end{pgfscope}%
\begin{pgfscope}%
\pgfpathrectangle{\pgfqpoint{10.972826in}{0.550000in}}{\pgfqpoint{2.527174in}{3.850000in}}%
\pgfusepath{clip}%
\pgfsetroundcap%
\pgfsetroundjoin%
\pgfsetlinewidth{1.003750pt}%
\definecolor{currentstroke}{rgb}{0.800000,0.800000,0.800000}%
\pgfsetstrokecolor{currentstroke}%
\pgfsetdash{}{0pt}%
\pgfpathmoveto{\pgfqpoint{10.972826in}{1.888455in}}%
\pgfpathlineto{\pgfqpoint{13.500000in}{1.888455in}}%
\pgfusepath{stroke}%
\end{pgfscope}%
\begin{pgfscope}%
\pgfpathrectangle{\pgfqpoint{10.972826in}{0.550000in}}{\pgfqpoint{2.527174in}{3.850000in}}%
\pgfusepath{clip}%
\pgfsetroundcap%
\pgfsetroundjoin%
\pgfsetlinewidth{1.003750pt}%
\definecolor{currentstroke}{rgb}{0.800000,0.800000,0.800000}%
\pgfsetstrokecolor{currentstroke}%
\pgfsetdash{}{0pt}%
\pgfpathmoveto{\pgfqpoint{10.972826in}{2.491773in}}%
\pgfpathlineto{\pgfqpoint{13.500000in}{2.491773in}}%
\pgfusepath{stroke}%
\end{pgfscope}%
\begin{pgfscope}%
\pgfpathrectangle{\pgfqpoint{10.972826in}{0.550000in}}{\pgfqpoint{2.527174in}{3.850000in}}%
\pgfusepath{clip}%
\pgfsetroundcap%
\pgfsetroundjoin%
\pgfsetlinewidth{1.003750pt}%
\definecolor{currentstroke}{rgb}{0.800000,0.800000,0.800000}%
\pgfsetstrokecolor{currentstroke}%
\pgfsetdash{}{0pt}%
\pgfpathmoveto{\pgfqpoint{10.972826in}{3.095092in}}%
\pgfpathlineto{\pgfqpoint{13.500000in}{3.095092in}}%
\pgfusepath{stroke}%
\end{pgfscope}%
\begin{pgfscope}%
\pgfpathrectangle{\pgfqpoint{10.972826in}{0.550000in}}{\pgfqpoint{2.527174in}{3.850000in}}%
\pgfusepath{clip}%
\pgfsetroundcap%
\pgfsetroundjoin%
\pgfsetlinewidth{1.003750pt}%
\definecolor{currentstroke}{rgb}{0.800000,0.800000,0.800000}%
\pgfsetstrokecolor{currentstroke}%
\pgfsetdash{}{0pt}%
\pgfpathmoveto{\pgfqpoint{10.972826in}{3.698410in}}%
\pgfpathlineto{\pgfqpoint{13.500000in}{3.698410in}}%
\pgfusepath{stroke}%
\end{pgfscope}%
\begin{pgfscope}%
\pgfpathrectangle{\pgfqpoint{10.972826in}{0.550000in}}{\pgfqpoint{2.527174in}{3.850000in}}%
\pgfusepath{clip}%
\pgfsetroundcap%
\pgfsetroundjoin%
\pgfsetlinewidth{1.003750pt}%
\definecolor{currentstroke}{rgb}{0.800000,0.800000,0.800000}%
\pgfsetstrokecolor{currentstroke}%
\pgfsetdash{}{0pt}%
\pgfpathmoveto{\pgfqpoint{10.972826in}{4.301729in}}%
\pgfpathlineto{\pgfqpoint{13.500000in}{4.301729in}}%
\pgfusepath{stroke}%
\end{pgfscope}%
\begin{pgfscope}%
\definecolor{textcolor}{rgb}{0.150000,0.150000,0.150000}%
\pgfsetstrokecolor{textcolor}%
\pgfsetfillcolor{textcolor}%
\pgftext[x=10.917271in,y=2.475000in,,bottom,rotate=90.000000]{\color{textcolor}\sffamily\fontsize{12.000000}{14.400000}\selectfont \ }%
\end{pgfscope}%
\begin{pgfscope}%
\pgfpathrectangle{\pgfqpoint{10.972826in}{0.550000in}}{\pgfqpoint{2.527174in}{3.850000in}}%
\pgfusepath{clip}%
\pgfsetbuttcap%
\pgfsetroundjoin%
\definecolor{currentfill}{rgb}{0.347059,0.458824,0.641176}%
\pgfsetfillcolor{currentfill}%
\pgfsetlinewidth{1.505625pt}%
\definecolor{currentstroke}{rgb}{0.298039,0.298039,0.298039}%
\pgfsetstrokecolor{currentstroke}%
\pgfsetdash{}{0pt}%
\pgfsys@defobject{currentmarker}{\pgfqpoint{11.225543in}{0.725000in}}{\pgfqpoint{12.236413in}{0.746524in}}{%
\pgfpathmoveto{\pgfqpoint{12.236413in}{0.725000in}}%
\pgfpathlineto{\pgfqpoint{12.140951in}{0.725000in}}%
\pgfpathlineto{\pgfqpoint{11.225543in}{0.725217in}}%
\pgfpathlineto{\pgfqpoint{11.887035in}{0.725435in}}%
\pgfpathlineto{\pgfqpoint{12.220259in}{0.725652in}}%
\pgfpathlineto{\pgfqpoint{12.234036in}{0.725870in}}%
\pgfpathlineto{\pgfqpoint{12.234693in}{0.726087in}}%
\pgfpathlineto{\pgfqpoint{12.235742in}{0.726305in}}%
\pgfpathlineto{\pgfqpoint{12.236408in}{0.726522in}}%
\pgfpathlineto{\pgfqpoint{12.236315in}{0.726739in}}%
\pgfpathlineto{\pgfqpoint{12.235804in}{0.726957in}}%
\pgfpathlineto{\pgfqpoint{12.236388in}{0.727174in}}%
\pgfpathlineto{\pgfqpoint{12.236413in}{0.727392in}}%
\pgfpathlineto{\pgfqpoint{12.236413in}{0.727609in}}%
\pgfpathlineto{\pgfqpoint{12.236413in}{0.727826in}}%
\pgfpathlineto{\pgfqpoint{12.236354in}{0.728044in}}%
\pgfpathlineto{\pgfqpoint{12.234370in}{0.728261in}}%
\pgfpathlineto{\pgfqpoint{12.235532in}{0.728479in}}%
\pgfpathlineto{\pgfqpoint{12.236400in}{0.728696in}}%
\pgfpathlineto{\pgfqpoint{12.236413in}{0.728914in}}%
\pgfpathlineto{\pgfqpoint{12.236413in}{0.729131in}}%
\pgfpathlineto{\pgfqpoint{12.236413in}{0.729348in}}%
\pgfpathlineto{\pgfqpoint{12.236413in}{0.729566in}}%
\pgfpathlineto{\pgfqpoint{12.236413in}{0.729783in}}%
\pgfpathlineto{\pgfqpoint{12.236413in}{0.730001in}}%
\pgfpathlineto{\pgfqpoint{12.236413in}{0.730218in}}%
\pgfpathlineto{\pgfqpoint{12.236413in}{0.730435in}}%
\pgfpathlineto{\pgfqpoint{12.236413in}{0.730653in}}%
\pgfpathlineto{\pgfqpoint{12.236413in}{0.730870in}}%
\pgfpathlineto{\pgfqpoint{12.236413in}{0.731088in}}%
\pgfpathlineto{\pgfqpoint{12.236413in}{0.731305in}}%
\pgfpathlineto{\pgfqpoint{12.236413in}{0.731523in}}%
\pgfpathlineto{\pgfqpoint{12.236413in}{0.731740in}}%
\pgfpathlineto{\pgfqpoint{12.236413in}{0.731957in}}%
\pgfpathlineto{\pgfqpoint{12.236413in}{0.732175in}}%
\pgfpathlineto{\pgfqpoint{12.236413in}{0.732392in}}%
\pgfpathlineto{\pgfqpoint{12.236413in}{0.732610in}}%
\pgfpathlineto{\pgfqpoint{12.236413in}{0.732827in}}%
\pgfpathlineto{\pgfqpoint{12.236413in}{0.733044in}}%
\pgfpathlineto{\pgfqpoint{12.236413in}{0.733262in}}%
\pgfpathlineto{\pgfqpoint{12.236413in}{0.733479in}}%
\pgfpathlineto{\pgfqpoint{12.236413in}{0.733697in}}%
\pgfpathlineto{\pgfqpoint{12.236413in}{0.733914in}}%
\pgfpathlineto{\pgfqpoint{12.236413in}{0.734132in}}%
\pgfpathlineto{\pgfqpoint{12.236413in}{0.734349in}}%
\pgfpathlineto{\pgfqpoint{12.236413in}{0.734566in}}%
\pgfpathlineto{\pgfqpoint{12.236413in}{0.734784in}}%
\pgfpathlineto{\pgfqpoint{12.236413in}{0.735001in}}%
\pgfpathlineto{\pgfqpoint{12.236413in}{0.735219in}}%
\pgfpathlineto{\pgfqpoint{12.236413in}{0.735436in}}%
\pgfpathlineto{\pgfqpoint{12.236413in}{0.735654in}}%
\pgfpathlineto{\pgfqpoint{12.236413in}{0.735871in}}%
\pgfpathlineto{\pgfqpoint{12.236413in}{0.736088in}}%
\pgfpathlineto{\pgfqpoint{12.236413in}{0.736306in}}%
\pgfpathlineto{\pgfqpoint{12.236413in}{0.736523in}}%
\pgfpathlineto{\pgfqpoint{12.236413in}{0.736741in}}%
\pgfpathlineto{\pgfqpoint{12.236413in}{0.736958in}}%
\pgfpathlineto{\pgfqpoint{12.236413in}{0.737175in}}%
\pgfpathlineto{\pgfqpoint{12.236413in}{0.737393in}}%
\pgfpathlineto{\pgfqpoint{12.236413in}{0.737610in}}%
\pgfpathlineto{\pgfqpoint{12.236413in}{0.737828in}}%
\pgfpathlineto{\pgfqpoint{12.236413in}{0.738045in}}%
\pgfpathlineto{\pgfqpoint{12.236413in}{0.738263in}}%
\pgfpathlineto{\pgfqpoint{12.236413in}{0.738480in}}%
\pgfpathlineto{\pgfqpoint{12.236413in}{0.738697in}}%
\pgfpathlineto{\pgfqpoint{12.236413in}{0.738915in}}%
\pgfpathlineto{\pgfqpoint{12.236413in}{0.739132in}}%
\pgfpathlineto{\pgfqpoint{12.236413in}{0.739350in}}%
\pgfpathlineto{\pgfqpoint{12.236413in}{0.739567in}}%
\pgfpathlineto{\pgfqpoint{12.236413in}{0.739784in}}%
\pgfpathlineto{\pgfqpoint{12.236413in}{0.740002in}}%
\pgfpathlineto{\pgfqpoint{12.236413in}{0.740219in}}%
\pgfpathlineto{\pgfqpoint{12.236413in}{0.740437in}}%
\pgfpathlineto{\pgfqpoint{12.236413in}{0.740654in}}%
\pgfpathlineto{\pgfqpoint{12.236413in}{0.740872in}}%
\pgfpathlineto{\pgfqpoint{12.236413in}{0.741089in}}%
\pgfpathlineto{\pgfqpoint{12.236413in}{0.741306in}}%
\pgfpathlineto{\pgfqpoint{12.236413in}{0.741524in}}%
\pgfpathlineto{\pgfqpoint{12.236413in}{0.741741in}}%
\pgfpathlineto{\pgfqpoint{12.236413in}{0.741959in}}%
\pgfpathlineto{\pgfqpoint{12.236413in}{0.742176in}}%
\pgfpathlineto{\pgfqpoint{12.236413in}{0.742394in}}%
\pgfpathlineto{\pgfqpoint{12.236413in}{0.742611in}}%
\pgfpathlineto{\pgfqpoint{12.236413in}{0.742828in}}%
\pgfpathlineto{\pgfqpoint{12.236397in}{0.743046in}}%
\pgfpathlineto{\pgfqpoint{12.236062in}{0.743263in}}%
\pgfpathlineto{\pgfqpoint{12.236395in}{0.743481in}}%
\pgfpathlineto{\pgfqpoint{12.236413in}{0.743698in}}%
\pgfpathlineto{\pgfqpoint{12.236413in}{0.743915in}}%
\pgfpathlineto{\pgfqpoint{12.236413in}{0.744133in}}%
\pgfpathlineto{\pgfqpoint{12.236413in}{0.744350in}}%
\pgfpathlineto{\pgfqpoint{12.236413in}{0.744568in}}%
\pgfpathlineto{\pgfqpoint{12.236413in}{0.744785in}}%
\pgfpathlineto{\pgfqpoint{12.236413in}{0.745003in}}%
\pgfpathlineto{\pgfqpoint{12.236413in}{0.745220in}}%
\pgfpathlineto{\pgfqpoint{12.236413in}{0.745437in}}%
\pgfpathlineto{\pgfqpoint{12.236413in}{0.745655in}}%
\pgfpathlineto{\pgfqpoint{12.236413in}{0.745872in}}%
\pgfpathlineto{\pgfqpoint{12.236413in}{0.746090in}}%
\pgfpathlineto{\pgfqpoint{12.236396in}{0.746307in}}%
\pgfpathlineto{\pgfqpoint{12.236062in}{0.746524in}}%
\pgfpathlineto{\pgfqpoint{12.236413in}{0.746524in}}%
\pgfpathlineto{\pgfqpoint{12.236413in}{0.746524in}}%
\pgfpathlineto{\pgfqpoint{12.236413in}{0.746307in}}%
\pgfpathlineto{\pgfqpoint{12.236413in}{0.746090in}}%
\pgfpathlineto{\pgfqpoint{12.236413in}{0.745872in}}%
\pgfpathlineto{\pgfqpoint{12.236413in}{0.745655in}}%
\pgfpathlineto{\pgfqpoint{12.236413in}{0.745437in}}%
\pgfpathlineto{\pgfqpoint{12.236413in}{0.745220in}}%
\pgfpathlineto{\pgfqpoint{12.236413in}{0.745003in}}%
\pgfpathlineto{\pgfqpoint{12.236413in}{0.744785in}}%
\pgfpathlineto{\pgfqpoint{12.236413in}{0.744568in}}%
\pgfpathlineto{\pgfqpoint{12.236413in}{0.744350in}}%
\pgfpathlineto{\pgfqpoint{12.236413in}{0.744133in}}%
\pgfpathlineto{\pgfqpoint{12.236413in}{0.743915in}}%
\pgfpathlineto{\pgfqpoint{12.236413in}{0.743698in}}%
\pgfpathlineto{\pgfqpoint{12.236413in}{0.743481in}}%
\pgfpathlineto{\pgfqpoint{12.236413in}{0.743263in}}%
\pgfpathlineto{\pgfqpoint{12.236413in}{0.743046in}}%
\pgfpathlineto{\pgfqpoint{12.236413in}{0.742828in}}%
\pgfpathlineto{\pgfqpoint{12.236413in}{0.742611in}}%
\pgfpathlineto{\pgfqpoint{12.236413in}{0.742394in}}%
\pgfpathlineto{\pgfqpoint{12.236413in}{0.742176in}}%
\pgfpathlineto{\pgfqpoint{12.236413in}{0.741959in}}%
\pgfpathlineto{\pgfqpoint{12.236413in}{0.741741in}}%
\pgfpathlineto{\pgfqpoint{12.236413in}{0.741524in}}%
\pgfpathlineto{\pgfqpoint{12.236413in}{0.741306in}}%
\pgfpathlineto{\pgfqpoint{12.236413in}{0.741089in}}%
\pgfpathlineto{\pgfqpoint{12.236413in}{0.740872in}}%
\pgfpathlineto{\pgfqpoint{12.236413in}{0.740654in}}%
\pgfpathlineto{\pgfqpoint{12.236413in}{0.740437in}}%
\pgfpathlineto{\pgfqpoint{12.236413in}{0.740219in}}%
\pgfpathlineto{\pgfqpoint{12.236413in}{0.740002in}}%
\pgfpathlineto{\pgfqpoint{12.236413in}{0.739784in}}%
\pgfpathlineto{\pgfqpoint{12.236413in}{0.739567in}}%
\pgfpathlineto{\pgfqpoint{12.236413in}{0.739350in}}%
\pgfpathlineto{\pgfqpoint{12.236413in}{0.739132in}}%
\pgfpathlineto{\pgfqpoint{12.236413in}{0.738915in}}%
\pgfpathlineto{\pgfqpoint{12.236413in}{0.738697in}}%
\pgfpathlineto{\pgfqpoint{12.236413in}{0.738480in}}%
\pgfpathlineto{\pgfqpoint{12.236413in}{0.738263in}}%
\pgfpathlineto{\pgfqpoint{12.236413in}{0.738045in}}%
\pgfpathlineto{\pgfqpoint{12.236413in}{0.737828in}}%
\pgfpathlineto{\pgfqpoint{12.236413in}{0.737610in}}%
\pgfpathlineto{\pgfqpoint{12.236413in}{0.737393in}}%
\pgfpathlineto{\pgfqpoint{12.236413in}{0.737175in}}%
\pgfpathlineto{\pgfqpoint{12.236413in}{0.736958in}}%
\pgfpathlineto{\pgfqpoint{12.236413in}{0.736741in}}%
\pgfpathlineto{\pgfqpoint{12.236413in}{0.736523in}}%
\pgfpathlineto{\pgfqpoint{12.236413in}{0.736306in}}%
\pgfpathlineto{\pgfqpoint{12.236413in}{0.736088in}}%
\pgfpathlineto{\pgfqpoint{12.236413in}{0.735871in}}%
\pgfpathlineto{\pgfqpoint{12.236413in}{0.735654in}}%
\pgfpathlineto{\pgfqpoint{12.236413in}{0.735436in}}%
\pgfpathlineto{\pgfqpoint{12.236413in}{0.735219in}}%
\pgfpathlineto{\pgfqpoint{12.236413in}{0.735001in}}%
\pgfpathlineto{\pgfqpoint{12.236413in}{0.734784in}}%
\pgfpathlineto{\pgfqpoint{12.236413in}{0.734566in}}%
\pgfpathlineto{\pgfqpoint{12.236413in}{0.734349in}}%
\pgfpathlineto{\pgfqpoint{12.236413in}{0.734132in}}%
\pgfpathlineto{\pgfqpoint{12.236413in}{0.733914in}}%
\pgfpathlineto{\pgfqpoint{12.236413in}{0.733697in}}%
\pgfpathlineto{\pgfqpoint{12.236413in}{0.733479in}}%
\pgfpathlineto{\pgfqpoint{12.236413in}{0.733262in}}%
\pgfpathlineto{\pgfqpoint{12.236413in}{0.733044in}}%
\pgfpathlineto{\pgfqpoint{12.236413in}{0.732827in}}%
\pgfpathlineto{\pgfqpoint{12.236413in}{0.732610in}}%
\pgfpathlineto{\pgfqpoint{12.236413in}{0.732392in}}%
\pgfpathlineto{\pgfqpoint{12.236413in}{0.732175in}}%
\pgfpathlineto{\pgfqpoint{12.236413in}{0.731957in}}%
\pgfpathlineto{\pgfqpoint{12.236413in}{0.731740in}}%
\pgfpathlineto{\pgfqpoint{12.236413in}{0.731523in}}%
\pgfpathlineto{\pgfqpoint{12.236413in}{0.731305in}}%
\pgfpathlineto{\pgfqpoint{12.236413in}{0.731088in}}%
\pgfpathlineto{\pgfqpoint{12.236413in}{0.730870in}}%
\pgfpathlineto{\pgfqpoint{12.236413in}{0.730653in}}%
\pgfpathlineto{\pgfqpoint{12.236413in}{0.730435in}}%
\pgfpathlineto{\pgfqpoint{12.236413in}{0.730218in}}%
\pgfpathlineto{\pgfqpoint{12.236413in}{0.730001in}}%
\pgfpathlineto{\pgfqpoint{12.236413in}{0.729783in}}%
\pgfpathlineto{\pgfqpoint{12.236413in}{0.729566in}}%
\pgfpathlineto{\pgfqpoint{12.236413in}{0.729348in}}%
\pgfpathlineto{\pgfqpoint{12.236413in}{0.729131in}}%
\pgfpathlineto{\pgfqpoint{12.236413in}{0.728914in}}%
\pgfpathlineto{\pgfqpoint{12.236413in}{0.728696in}}%
\pgfpathlineto{\pgfqpoint{12.236413in}{0.728479in}}%
\pgfpathlineto{\pgfqpoint{12.236413in}{0.728261in}}%
\pgfpathlineto{\pgfqpoint{12.236413in}{0.728044in}}%
\pgfpathlineto{\pgfqpoint{12.236413in}{0.727826in}}%
\pgfpathlineto{\pgfqpoint{12.236413in}{0.727609in}}%
\pgfpathlineto{\pgfqpoint{12.236413in}{0.727392in}}%
\pgfpathlineto{\pgfqpoint{12.236413in}{0.727174in}}%
\pgfpathlineto{\pgfqpoint{12.236413in}{0.726957in}}%
\pgfpathlineto{\pgfqpoint{12.236413in}{0.726739in}}%
\pgfpathlineto{\pgfqpoint{12.236413in}{0.726522in}}%
\pgfpathlineto{\pgfqpoint{12.236413in}{0.726305in}}%
\pgfpathlineto{\pgfqpoint{12.236413in}{0.726087in}}%
\pgfpathlineto{\pgfqpoint{12.236413in}{0.725870in}}%
\pgfpathlineto{\pgfqpoint{12.236413in}{0.725652in}}%
\pgfpathlineto{\pgfqpoint{12.236413in}{0.725435in}}%
\pgfpathlineto{\pgfqpoint{12.236413in}{0.725217in}}%
\pgfpathlineto{\pgfqpoint{12.236413in}{0.725000in}}%
\pgfpathclose%
\pgfusepath{stroke,fill}%
}%
\begin{pgfscope}%
\pgfsys@transformshift{0.000000in}{0.000000in}%
\pgfsys@useobject{currentmarker}{}%
\end{pgfscope}%
\end{pgfscope}%
\begin{pgfscope}%
\pgfpathrectangle{\pgfqpoint{10.972826in}{0.550000in}}{\pgfqpoint{2.527174in}{3.850000in}}%
\pgfusepath{clip}%
\pgfsetbuttcap%
\pgfsetroundjoin%
\definecolor{currentfill}{rgb}{0.798529,0.536765,0.389706}%
\pgfsetfillcolor{currentfill}%
\pgfsetlinewidth{1.505625pt}%
\definecolor{currentstroke}{rgb}{0.298039,0.298039,0.298039}%
\pgfsetstrokecolor{currentstroke}%
\pgfsetdash{}{0pt}%
\pgfsys@defobject{currentmarker}{\pgfqpoint{12.236413in}{0.725139in}}{\pgfqpoint{12.792194in}{1.871889in}}{%
\pgfpathmoveto{\pgfqpoint{12.792194in}{0.725139in}}%
\pgfpathlineto{\pgfqpoint{12.236413in}{0.725139in}}%
\pgfpathlineto{\pgfqpoint{12.236413in}{0.736722in}}%
\pgfpathlineto{\pgfqpoint{12.236413in}{0.748306in}}%
\pgfpathlineto{\pgfqpoint{12.236413in}{0.759889in}}%
\pgfpathlineto{\pgfqpoint{12.236413in}{0.771472in}}%
\pgfpathlineto{\pgfqpoint{12.236413in}{0.783056in}}%
\pgfpathlineto{\pgfqpoint{12.236413in}{0.794639in}}%
\pgfpathlineto{\pgfqpoint{12.236413in}{0.806222in}}%
\pgfpathlineto{\pgfqpoint{12.236413in}{0.817806in}}%
\pgfpathlineto{\pgfqpoint{12.236413in}{0.829389in}}%
\pgfpathlineto{\pgfqpoint{12.236413in}{0.840972in}}%
\pgfpathlineto{\pgfqpoint{12.236413in}{0.852556in}}%
\pgfpathlineto{\pgfqpoint{12.236413in}{0.864139in}}%
\pgfpathlineto{\pgfqpoint{12.236413in}{0.875722in}}%
\pgfpathlineto{\pgfqpoint{12.236413in}{0.887306in}}%
\pgfpathlineto{\pgfqpoint{12.236413in}{0.898889in}}%
\pgfpathlineto{\pgfqpoint{12.236413in}{0.910472in}}%
\pgfpathlineto{\pgfqpoint{12.236413in}{0.922056in}}%
\pgfpathlineto{\pgfqpoint{12.236413in}{0.933639in}}%
\pgfpathlineto{\pgfqpoint{12.236413in}{0.945222in}}%
\pgfpathlineto{\pgfqpoint{12.236413in}{0.956806in}}%
\pgfpathlineto{\pgfqpoint{12.236413in}{0.968389in}}%
\pgfpathlineto{\pgfqpoint{12.236413in}{0.979972in}}%
\pgfpathlineto{\pgfqpoint{12.236413in}{0.991556in}}%
\pgfpathlineto{\pgfqpoint{12.236413in}{1.003139in}}%
\pgfpathlineto{\pgfqpoint{12.236413in}{1.014722in}}%
\pgfpathlineto{\pgfqpoint{12.236413in}{1.026306in}}%
\pgfpathlineto{\pgfqpoint{12.236413in}{1.037889in}}%
\pgfpathlineto{\pgfqpoint{12.236413in}{1.049472in}}%
\pgfpathlineto{\pgfqpoint{12.236413in}{1.061056in}}%
\pgfpathlineto{\pgfqpoint{12.236413in}{1.072639in}}%
\pgfpathlineto{\pgfqpoint{12.236413in}{1.084222in}}%
\pgfpathlineto{\pgfqpoint{12.236413in}{1.095806in}}%
\pgfpathlineto{\pgfqpoint{12.236413in}{1.107389in}}%
\pgfpathlineto{\pgfqpoint{12.236413in}{1.118972in}}%
\pgfpathlineto{\pgfqpoint{12.236413in}{1.130556in}}%
\pgfpathlineto{\pgfqpoint{12.236413in}{1.142139in}}%
\pgfpathlineto{\pgfqpoint{12.236413in}{1.153722in}}%
\pgfpathlineto{\pgfqpoint{12.236413in}{1.165306in}}%
\pgfpathlineto{\pgfqpoint{12.236413in}{1.176889in}}%
\pgfpathlineto{\pgfqpoint{12.236413in}{1.188472in}}%
\pgfpathlineto{\pgfqpoint{12.236413in}{1.200056in}}%
\pgfpathlineto{\pgfqpoint{12.236413in}{1.211639in}}%
\pgfpathlineto{\pgfqpoint{12.236413in}{1.223222in}}%
\pgfpathlineto{\pgfqpoint{12.236413in}{1.234806in}}%
\pgfpathlineto{\pgfqpoint{12.236413in}{1.246389in}}%
\pgfpathlineto{\pgfqpoint{12.236413in}{1.257972in}}%
\pgfpathlineto{\pgfqpoint{12.236413in}{1.269556in}}%
\pgfpathlineto{\pgfqpoint{12.236413in}{1.281139in}}%
\pgfpathlineto{\pgfqpoint{12.236413in}{1.292722in}}%
\pgfpathlineto{\pgfqpoint{12.236413in}{1.304306in}}%
\pgfpathlineto{\pgfqpoint{12.236413in}{1.315889in}}%
\pgfpathlineto{\pgfqpoint{12.236413in}{1.327472in}}%
\pgfpathlineto{\pgfqpoint{12.236413in}{1.339056in}}%
\pgfpathlineto{\pgfqpoint{12.236413in}{1.350639in}}%
\pgfpathlineto{\pgfqpoint{12.236413in}{1.362222in}}%
\pgfpathlineto{\pgfqpoint{12.236413in}{1.373806in}}%
\pgfpathlineto{\pgfqpoint{12.236413in}{1.385389in}}%
\pgfpathlineto{\pgfqpoint{12.236413in}{1.396972in}}%
\pgfpathlineto{\pgfqpoint{12.236413in}{1.408556in}}%
\pgfpathlineto{\pgfqpoint{12.236413in}{1.420139in}}%
\pgfpathlineto{\pgfqpoint{12.236413in}{1.431723in}}%
\pgfpathlineto{\pgfqpoint{12.236413in}{1.443306in}}%
\pgfpathlineto{\pgfqpoint{12.236413in}{1.454889in}}%
\pgfpathlineto{\pgfqpoint{12.236413in}{1.466473in}}%
\pgfpathlineto{\pgfqpoint{12.236413in}{1.478056in}}%
\pgfpathlineto{\pgfqpoint{12.236413in}{1.489639in}}%
\pgfpathlineto{\pgfqpoint{12.236413in}{1.501223in}}%
\pgfpathlineto{\pgfqpoint{12.236413in}{1.512806in}}%
\pgfpathlineto{\pgfqpoint{12.236413in}{1.524389in}}%
\pgfpathlineto{\pgfqpoint{12.236413in}{1.535973in}}%
\pgfpathlineto{\pgfqpoint{12.236413in}{1.547556in}}%
\pgfpathlineto{\pgfqpoint{12.236413in}{1.559139in}}%
\pgfpathlineto{\pgfqpoint{12.236413in}{1.570723in}}%
\pgfpathlineto{\pgfqpoint{12.236413in}{1.582306in}}%
\pgfpathlineto{\pgfqpoint{12.236413in}{1.593889in}}%
\pgfpathlineto{\pgfqpoint{12.236413in}{1.605473in}}%
\pgfpathlineto{\pgfqpoint{12.236413in}{1.617056in}}%
\pgfpathlineto{\pgfqpoint{12.236413in}{1.628639in}}%
\pgfpathlineto{\pgfqpoint{12.236413in}{1.640223in}}%
\pgfpathlineto{\pgfqpoint{12.236413in}{1.651806in}}%
\pgfpathlineto{\pgfqpoint{12.236413in}{1.663389in}}%
\pgfpathlineto{\pgfqpoint{12.236413in}{1.674973in}}%
\pgfpathlineto{\pgfqpoint{12.236413in}{1.686556in}}%
\pgfpathlineto{\pgfqpoint{12.236413in}{1.698139in}}%
\pgfpathlineto{\pgfqpoint{12.236413in}{1.709723in}}%
\pgfpathlineto{\pgfqpoint{12.236413in}{1.721306in}}%
\pgfpathlineto{\pgfqpoint{12.236413in}{1.732889in}}%
\pgfpathlineto{\pgfqpoint{12.236413in}{1.744473in}}%
\pgfpathlineto{\pgfqpoint{12.236413in}{1.756056in}}%
\pgfpathlineto{\pgfqpoint{12.236413in}{1.767639in}}%
\pgfpathlineto{\pgfqpoint{12.236413in}{1.779223in}}%
\pgfpathlineto{\pgfqpoint{12.236413in}{1.790806in}}%
\pgfpathlineto{\pgfqpoint{12.236413in}{1.802389in}}%
\pgfpathlineto{\pgfqpoint{12.236413in}{1.813973in}}%
\pgfpathlineto{\pgfqpoint{12.236413in}{1.825556in}}%
\pgfpathlineto{\pgfqpoint{12.236413in}{1.837139in}}%
\pgfpathlineto{\pgfqpoint{12.236413in}{1.848723in}}%
\pgfpathlineto{\pgfqpoint{12.236413in}{1.860306in}}%
\pgfpathlineto{\pgfqpoint{12.236413in}{1.871889in}}%
\pgfpathlineto{\pgfqpoint{12.237168in}{1.871889in}}%
\pgfpathlineto{\pgfqpoint{12.237168in}{1.871889in}}%
\pgfpathlineto{\pgfqpoint{12.237103in}{1.860306in}}%
\pgfpathlineto{\pgfqpoint{12.236939in}{1.848723in}}%
\pgfpathlineto{\pgfqpoint{12.236750in}{1.837139in}}%
\pgfpathlineto{\pgfqpoint{12.236600in}{1.825556in}}%
\pgfpathlineto{\pgfqpoint{12.236521in}{1.813973in}}%
\pgfpathlineto{\pgfqpoint{12.236523in}{1.802389in}}%
\pgfpathlineto{\pgfqpoint{12.236607in}{1.790806in}}%
\pgfpathlineto{\pgfqpoint{12.236775in}{1.779223in}}%
\pgfpathlineto{\pgfqpoint{12.237014in}{1.767639in}}%
\pgfpathlineto{\pgfqpoint{12.237294in}{1.756056in}}%
\pgfpathlineto{\pgfqpoint{12.237584in}{1.744473in}}%
\pgfpathlineto{\pgfqpoint{12.237873in}{1.732889in}}%
\pgfpathlineto{\pgfqpoint{12.238168in}{1.721306in}}%
\pgfpathlineto{\pgfqpoint{12.238465in}{1.709723in}}%
\pgfpathlineto{\pgfqpoint{12.238748in}{1.698139in}}%
\pgfpathlineto{\pgfqpoint{12.239000in}{1.686556in}}%
\pgfpathlineto{\pgfqpoint{12.239188in}{1.674973in}}%
\pgfpathlineto{\pgfqpoint{12.239242in}{1.663389in}}%
\pgfpathlineto{\pgfqpoint{12.239094in}{1.651806in}}%
\pgfpathlineto{\pgfqpoint{12.238766in}{1.640223in}}%
\pgfpathlineto{\pgfqpoint{12.238398in}{1.628639in}}%
\pgfpathlineto{\pgfqpoint{12.238151in}{1.617056in}}%
\pgfpathlineto{\pgfqpoint{12.238093in}{1.605473in}}%
\pgfpathlineto{\pgfqpoint{12.238178in}{1.593889in}}%
\pgfpathlineto{\pgfqpoint{12.238326in}{1.582306in}}%
\pgfpathlineto{\pgfqpoint{12.238498in}{1.570723in}}%
\pgfpathlineto{\pgfqpoint{12.238676in}{1.559139in}}%
\pgfpathlineto{\pgfqpoint{12.238812in}{1.547556in}}%
\pgfpathlineto{\pgfqpoint{12.238808in}{1.535973in}}%
\pgfpathlineto{\pgfqpoint{12.238588in}{1.524389in}}%
\pgfpathlineto{\pgfqpoint{12.238163in}{1.512806in}}%
\pgfpathlineto{\pgfqpoint{12.237643in}{1.501223in}}%
\pgfpathlineto{\pgfqpoint{12.237164in}{1.489639in}}%
\pgfpathlineto{\pgfqpoint{12.236817in}{1.478056in}}%
\pgfpathlineto{\pgfqpoint{12.236630in}{1.466473in}}%
\pgfpathlineto{\pgfqpoint{12.236595in}{1.454889in}}%
\pgfpathlineto{\pgfqpoint{12.236703in}{1.443306in}}%
\pgfpathlineto{\pgfqpoint{12.236947in}{1.431723in}}%
\pgfpathlineto{\pgfqpoint{12.237298in}{1.420139in}}%
\pgfpathlineto{\pgfqpoint{12.237677in}{1.408556in}}%
\pgfpathlineto{\pgfqpoint{12.237984in}{1.396972in}}%
\pgfpathlineto{\pgfqpoint{12.238160in}{1.385389in}}%
\pgfpathlineto{\pgfqpoint{12.238239in}{1.373806in}}%
\pgfpathlineto{\pgfqpoint{12.238332in}{1.362222in}}%
\pgfpathlineto{\pgfqpoint{12.238540in}{1.350639in}}%
\pgfpathlineto{\pgfqpoint{12.238887in}{1.339056in}}%
\pgfpathlineto{\pgfqpoint{12.239300in}{1.327472in}}%
\pgfpathlineto{\pgfqpoint{12.239647in}{1.315889in}}%
\pgfpathlineto{\pgfqpoint{12.239813in}{1.304306in}}%
\pgfpathlineto{\pgfqpoint{12.239778in}{1.292722in}}%
\pgfpathlineto{\pgfqpoint{12.239660in}{1.281139in}}%
\pgfpathlineto{\pgfqpoint{12.239660in}{1.269556in}}%
\pgfpathlineto{\pgfqpoint{12.239991in}{1.257972in}}%
\pgfpathlineto{\pgfqpoint{12.240842in}{1.246389in}}%
\pgfpathlineto{\pgfqpoint{12.242367in}{1.234806in}}%
\pgfpathlineto{\pgfqpoint{12.244644in}{1.223222in}}%
\pgfpathlineto{\pgfqpoint{12.247614in}{1.211639in}}%
\pgfpathlineto{\pgfqpoint{12.251071in}{1.200056in}}%
\pgfpathlineto{\pgfqpoint{12.254718in}{1.188472in}}%
\pgfpathlineto{\pgfqpoint{12.258272in}{1.176889in}}%
\pgfpathlineto{\pgfqpoint{12.261587in}{1.165306in}}%
\pgfpathlineto{\pgfqpoint{12.264721in}{1.153722in}}%
\pgfpathlineto{\pgfqpoint{12.267899in}{1.142139in}}%
\pgfpathlineto{\pgfqpoint{12.271372in}{1.130556in}}%
\pgfpathlineto{\pgfqpoint{12.275282in}{1.118972in}}%
\pgfpathlineto{\pgfqpoint{12.279588in}{1.107389in}}%
\pgfpathlineto{\pgfqpoint{12.284015in}{1.095806in}}%
\pgfpathlineto{\pgfqpoint{12.288174in}{1.084222in}}%
\pgfpathlineto{\pgfqpoint{12.291904in}{1.072639in}}%
\pgfpathlineto{\pgfqpoint{12.295560in}{1.061056in}}%
\pgfpathlineto{\pgfqpoint{12.299877in}{1.049472in}}%
\pgfpathlineto{\pgfqpoint{12.305524in}{1.037889in}}%
\pgfpathlineto{\pgfqpoint{12.312937in}{1.026306in}}%
\pgfpathlineto{\pgfqpoint{12.322747in}{1.014722in}}%
\pgfpathlineto{\pgfqpoint{12.336569in}{1.003139in}}%
\pgfpathlineto{\pgfqpoint{12.357596in}{0.991556in}}%
\pgfpathlineto{\pgfqpoint{12.390348in}{0.979972in}}%
\pgfpathlineto{\pgfqpoint{12.438832in}{0.968389in}}%
\pgfpathlineto{\pgfqpoint{12.502876in}{0.956806in}}%
\pgfpathlineto{\pgfqpoint{12.574772in}{0.945222in}}%
\pgfpathlineto{\pgfqpoint{12.640501in}{0.933639in}}%
\pgfpathlineto{\pgfqpoint{12.687107in}{0.922056in}}%
\pgfpathlineto{\pgfqpoint{12.710207in}{0.910472in}}%
\pgfpathlineto{\pgfqpoint{12.713183in}{0.898889in}}%
\pgfpathlineto{\pgfqpoint{12.699005in}{0.887306in}}%
\pgfpathlineto{\pgfqpoint{12.665778in}{0.875722in}}%
\pgfpathlineto{\pgfqpoint{12.612312in}{0.864139in}}%
\pgfpathlineto{\pgfqpoint{12.545841in}{0.852556in}}%
\pgfpathlineto{\pgfqpoint{12.481071in}{0.840972in}}%
\pgfpathlineto{\pgfqpoint{12.431476in}{0.829389in}}%
\pgfpathlineto{\pgfqpoint{12.402727in}{0.817806in}}%
\pgfpathlineto{\pgfqpoint{12.394024in}{0.806222in}}%
\pgfpathlineto{\pgfqpoint{12.404474in}{0.794639in}}%
\pgfpathlineto{\pgfqpoint{12.438188in}{0.783056in}}%
\pgfpathlineto{\pgfqpoint{12.502731in}{0.771472in}}%
\pgfpathlineto{\pgfqpoint{12.598726in}{0.759889in}}%
\pgfpathlineto{\pgfqpoint{12.706493in}{0.748306in}}%
\pgfpathlineto{\pgfqpoint{12.784845in}{0.736722in}}%
\pgfpathlineto{\pgfqpoint{12.792194in}{0.725139in}}%
\pgfpathclose%
\pgfusepath{stroke,fill}%
}%
\begin{pgfscope}%
\pgfsys@transformshift{0.000000in}{0.000000in}%
\pgfsys@useobject{currentmarker}{}%
\end{pgfscope}%
\end{pgfscope}%
\begin{pgfscope}%
\pgfpathrectangle{\pgfqpoint{10.972826in}{0.550000in}}{\pgfqpoint{2.527174in}{3.850000in}}%
\pgfusepath{clip}%
\pgfsetbuttcap%
\pgfsetmiterjoin%
\definecolor{currentfill}{rgb}{0.347059,0.458824,0.641176}%
\pgfsetfillcolor{currentfill}%
\pgfsetlinewidth{0.752812pt}%
\definecolor{currentstroke}{rgb}{0.298039,0.298039,0.298039}%
\pgfsetstrokecolor{currentstroke}%
\pgfsetdash{}{0pt}%
\pgfpathmoveto{\pgfqpoint{12.236413in}{0.681818in}}%
\pgfpathlineto{\pgfqpoint{12.236413in}{0.681818in}}%
\pgfpathlineto{\pgfqpoint{12.236413in}{0.681818in}}%
\pgfpathlineto{\pgfqpoint{12.236413in}{0.681818in}}%
\pgfpathclose%
\pgfusepath{stroke,fill}%
\end{pgfscope}%
\begin{pgfscope}%
\pgfpathrectangle{\pgfqpoint{10.972826in}{0.550000in}}{\pgfqpoint{2.527174in}{3.850000in}}%
\pgfusepath{clip}%
\pgfsetbuttcap%
\pgfsetmiterjoin%
\definecolor{currentfill}{rgb}{0.798529,0.536765,0.389706}%
\pgfsetfillcolor{currentfill}%
\pgfsetlinewidth{0.752812pt}%
\definecolor{currentstroke}{rgb}{0.298039,0.298039,0.298039}%
\pgfsetstrokecolor{currentstroke}%
\pgfsetdash{}{0pt}%
\pgfpathmoveto{\pgfqpoint{12.236413in}{0.681818in}}%
\pgfpathlineto{\pgfqpoint{12.236413in}{0.681818in}}%
\pgfpathlineto{\pgfqpoint{12.236413in}{0.681818in}}%
\pgfpathlineto{\pgfqpoint{12.236413in}{0.681818in}}%
\pgfpathclose%
\pgfusepath{stroke,fill}%
\end{pgfscope}%
\begin{pgfscope}%
\pgfpathrectangle{\pgfqpoint{10.972826in}{0.550000in}}{\pgfqpoint{2.527174in}{3.850000in}}%
\pgfusepath{clip}%
\pgfsetroundcap%
\pgfsetroundjoin%
\pgfsetlinewidth{1.505625pt}%
\definecolor{currentstroke}{rgb}{0.298039,0.298039,0.298039}%
\pgfsetstrokecolor{currentstroke}%
\pgfsetdash{}{0pt}%
\pgfpathmoveto{\pgfqpoint{12.236413in}{0.725000in}}%
\pgfpathlineto{\pgfqpoint{12.236413in}{0.753990in}}%
\pgfusepath{stroke}%
\end{pgfscope}%
\begin{pgfscope}%
\pgfpathrectangle{\pgfqpoint{10.972826in}{0.550000in}}{\pgfqpoint{2.527174in}{3.850000in}}%
\pgfusepath{clip}%
\pgfsetroundcap%
\pgfsetroundjoin%
\pgfsetlinewidth{4.516875pt}%
\definecolor{currentstroke}{rgb}{0.298039,0.298039,0.298039}%
\pgfsetstrokecolor{currentstroke}%
\pgfsetdash{}{0pt}%
\pgfpathmoveto{\pgfqpoint{12.236413in}{0.725227in}}%
\pgfpathlineto{\pgfqpoint{12.236413in}{0.736798in}}%
\pgfusepath{stroke}%
\end{pgfscope}%
\begin{pgfscope}%
\pgfsetrectcap%
\pgfsetmiterjoin%
\pgfsetlinewidth{1.254687pt}%
\definecolor{currentstroke}{rgb}{0.800000,0.800000,0.800000}%
\pgfsetstrokecolor{currentstroke}%
\pgfsetdash{}{0pt}%
\pgfpathmoveto{\pgfqpoint{10.972826in}{0.550000in}}%
\pgfpathlineto{\pgfqpoint{10.972826in}{4.400000in}}%
\pgfusepath{stroke}%
\end{pgfscope}%
\begin{pgfscope}%
\pgfsetrectcap%
\pgfsetmiterjoin%
\pgfsetlinewidth{1.254687pt}%
\definecolor{currentstroke}{rgb}{0.800000,0.800000,0.800000}%
\pgfsetstrokecolor{currentstroke}%
\pgfsetdash{}{0pt}%
\pgfpathmoveto{\pgfqpoint{13.500000in}{0.550000in}}%
\pgfpathlineto{\pgfqpoint{13.500000in}{4.400000in}}%
\pgfusepath{stroke}%
\end{pgfscope}%
\begin{pgfscope}%
\pgfsetrectcap%
\pgfsetmiterjoin%
\pgfsetlinewidth{1.254687pt}%
\definecolor{currentstroke}{rgb}{0.800000,0.800000,0.800000}%
\pgfsetstrokecolor{currentstroke}%
\pgfsetdash{}{0pt}%
\pgfpathmoveto{\pgfqpoint{10.972826in}{0.550000in}}%
\pgfpathlineto{\pgfqpoint{13.500000in}{0.550000in}}%
\pgfusepath{stroke}%
\end{pgfscope}%
\begin{pgfscope}%
\pgfsetrectcap%
\pgfsetmiterjoin%
\pgfsetlinewidth{1.254687pt}%
\definecolor{currentstroke}{rgb}{0.800000,0.800000,0.800000}%
\pgfsetstrokecolor{currentstroke}%
\pgfsetdash{}{0pt}%
\pgfpathmoveto{\pgfqpoint{10.972826in}{4.400000in}}%
\pgfpathlineto{\pgfqpoint{13.500000in}{4.400000in}}%
\pgfusepath{stroke}%
\end{pgfscope}%
\begin{pgfscope}%
\pgfpathrectangle{\pgfqpoint{10.972826in}{0.550000in}}{\pgfqpoint{2.527174in}{3.850000in}}%
\pgfusepath{clip}%
\pgfsetbuttcap%
\pgfsetroundjoin%
\definecolor{currentfill}{rgb}{1.000000,1.000000,1.000000}%
\pgfsetfillcolor{currentfill}%
\pgfsetlinewidth{1.003750pt}%
\definecolor{currentstroke}{rgb}{0.298039,0.298039,0.298039}%
\pgfsetstrokecolor{currentstroke}%
\pgfsetdash{}{0pt}%
\pgfsys@defobject{currentmarker}{\pgfqpoint{-0.020833in}{-0.020833in}}{\pgfqpoint{0.020833in}{0.020833in}}{%
\pgfpathmoveto{\pgfqpoint{0.000000in}{-0.020833in}}%
\pgfpathcurveto{\pgfqpoint{0.005525in}{-0.020833in}}{\pgfqpoint{0.010825in}{-0.018638in}}{\pgfqpoint{0.014731in}{-0.014731in}}%
\pgfpathcurveto{\pgfqpoint{0.018638in}{-0.010825in}}{\pgfqpoint{0.020833in}{-0.005525in}}{\pgfqpoint{0.020833in}{0.000000in}}%
\pgfpathcurveto{\pgfqpoint{0.020833in}{0.005525in}}{\pgfqpoint{0.018638in}{0.010825in}}{\pgfqpoint{0.014731in}{0.014731in}}%
\pgfpathcurveto{\pgfqpoint{0.010825in}{0.018638in}}{\pgfqpoint{0.005525in}{0.020833in}}{\pgfqpoint{0.000000in}{0.020833in}}%
\pgfpathcurveto{\pgfqpoint{-0.005525in}{0.020833in}}{\pgfqpoint{-0.010825in}{0.018638in}}{\pgfqpoint{-0.014731in}{0.014731in}}%
\pgfpathcurveto{\pgfqpoint{-0.018638in}{0.010825in}}{\pgfqpoint{-0.020833in}{0.005525in}}{\pgfqpoint{-0.020833in}{0.000000in}}%
\pgfpathcurveto{\pgfqpoint{-0.020833in}{-0.005525in}}{\pgfqpoint{-0.018638in}{-0.010825in}}{\pgfqpoint{-0.014731in}{-0.014731in}}%
\pgfpathcurveto{\pgfqpoint{-0.010825in}{-0.018638in}}{\pgfqpoint{-0.005525in}{-0.020833in}}{\pgfqpoint{0.000000in}{-0.020833in}}%
\pgfpathclose%
\pgfusepath{stroke,fill}%
}%
\begin{pgfscope}%
\pgfsys@transformshift{12.236413in}{0.725320in}%
\pgfsys@useobject{currentmarker}{}%
\end{pgfscope}%
\end{pgfscope}%
\begin{pgfscope}%
\definecolor{textcolor}{rgb}{0.150000,0.150000,0.150000}%
\pgfsetstrokecolor{textcolor}%
\pgfsetfillcolor{textcolor}%
\pgftext[x=7.500000in,y=0.200000in,,]{\color{textcolor}\sffamily\fontsize{12.000000}{14.400000}\selectfont Remaining Response Times}%
\end{pgfscope}%
\end{pgfpicture}%
\makeatother%
\endgroup%

                    
                        The right part (blue) of \autoref{fig:taskchain_with_real_tasks_as_perturbation} shows the task chain response time distribution profile with the full workload executed (i.e. LO-criticality tasks included). We see the perturbation due to the LO tasks on the critical task chain execution. Our workload is schedulable (no execution drops and deadline misses have reasonable overheads) and the task chain meets high response times compared to its average ``nominal" response time for $\approx10\%$ of the executions (above 200ms response time). We arbitrarily define the task chain deadline $D = 160$ms. 
                        
        \subsection{Phase de Calibration}
                    This phase is dedicated to configure the Core Control Component parameters  ($rWCRT_i(\tau_i)$, $t_{sw}$ and $W_{max}$) and run the reference experiments of the task chain behavior on a worst-case stress context (step~\circleTxt[4]).
     
            \subsubsection{Chaine de tâches avec stress forcé}
                        In this part we use \textit{Stress-ng} to simulate a worst case stress condition. The task chain potential worst case response time in this context raises at 300ms.
                        Such increase by 100\% of the max chain response time under this scenario indicates the pertinence of using a MCA. Regarding such result, our workload stresses the task chain in a significant magnitude.
                        \cmnt{
                        \begin{figure}[ht]
                        \begin{adjustbox}{clip,trim=0.5cm 0.3cm 1.5cm 1cm,max width=\linewidth}   % TRIM~: left, bottom, right, top    
                            %% Creator: Matplotlib, PGF backend
%%
%% To include the figure in your LaTeX document, write
%%   \input{<filename>.pgf}
%%
%% Make sure the required packages are loaded in your preamble
%%   \usepackage{pgf}
%%
%% and, on pdftex
%%   \usepackage[utf8]{inputenc}\DeclareUnicodeCharacter{2212}{-}
%%
%% or, on luatex and xetex
%%   \usepackage{unicode-math}
%%
%% Figures using additional raster images can only be included by \input if
%% they are in the same directory as the main LaTeX file. For loading figures
%% from other directories you can use the `import` package
%%   \usepackage{import}
%%
%% and then include the figures with
%%   \import{<path to file>}{<filename>.pgf}
%%
%% Matplotlib used the following preamble
%%
\begingroup%
\makeatletter%
\begin{pgfpicture}%
\pgfpathrectangle{\pgfpointorigin}{\pgfqpoint{6.400000in}{4.800000in}}%
\pgfusepath{use as bounding box, clip}%
\begin{pgfscope}%
\pgfsetbuttcap%
\pgfsetmiterjoin%
\definecolor{currentfill}{rgb}{1.000000,1.000000,1.000000}%
\pgfsetfillcolor{currentfill}%
\pgfsetlinewidth{0.000000pt}%
\definecolor{currentstroke}{rgb}{1.000000,1.000000,1.000000}%
\pgfsetstrokecolor{currentstroke}%
\pgfsetdash{}{0pt}%
\pgfpathmoveto{\pgfqpoint{0.000000in}{0.000000in}}%
\pgfpathlineto{\pgfqpoint{6.400000in}{0.000000in}}%
\pgfpathlineto{\pgfqpoint{6.400000in}{4.800000in}}%
\pgfpathlineto{\pgfqpoint{0.000000in}{4.800000in}}%
\pgfpathclose%
\pgfusepath{fill}%
\end{pgfscope}%
\begin{pgfscope}%
\pgfsetbuttcap%
\pgfsetmiterjoin%
\definecolor{currentfill}{rgb}{1.000000,1.000000,1.000000}%
\pgfsetfillcolor{currentfill}%
\pgfsetlinewidth{0.000000pt}%
\definecolor{currentstroke}{rgb}{0.000000,0.000000,0.000000}%
\pgfsetstrokecolor{currentstroke}%
\pgfsetstrokeopacity{0.000000}%
\pgfsetdash{}{0pt}%
\pgfpathmoveto{\pgfqpoint{0.800000in}{0.528000in}}%
\pgfpathlineto{\pgfqpoint{5.760000in}{0.528000in}}%
\pgfpathlineto{\pgfqpoint{5.760000in}{4.224000in}}%
\pgfpathlineto{\pgfqpoint{0.800000in}{4.224000in}}%
\pgfpathclose%
\pgfusepath{fill}%
\end{pgfscope}%
\begin{pgfscope}%
\definecolor{textcolor}{rgb}{0.150000,0.150000,0.150000}%
\pgfsetstrokecolor{textcolor}%
\pgfsetfillcolor{textcolor}%
\pgftext[x=3.280000in,y=0.396056in,,top]{\color{textcolor}\sffamily\fontsize{11.000000}{13.200000}\selectfont Isolated profile     |     Stressed profile}%
\end{pgfscope}%
\begin{pgfscope}%
\pgfpathrectangle{\pgfqpoint{0.800000in}{0.528000in}}{\pgfqpoint{4.960000in}{3.696000in}}%
\pgfusepath{clip}%
\pgfsetroundcap%
\pgfsetroundjoin%
\pgfsetlinewidth{1.003750pt}%
\definecolor{currentstroke}{rgb}{0.800000,0.800000,0.800000}%
\pgfsetstrokecolor{currentstroke}%
\pgfsetdash{}{0pt}%
\pgfpathmoveto{\pgfqpoint{0.800000in}{0.594639in}}%
\pgfpathlineto{\pgfqpoint{5.760000in}{0.594639in}}%
\pgfusepath{stroke}%
\end{pgfscope}%
\begin{pgfscope}%
\definecolor{textcolor}{rgb}{0.150000,0.150000,0.150000}%
\pgfsetstrokecolor{textcolor}%
\pgfsetfillcolor{textcolor}%
\pgftext[x=0.515972in, y=0.541833in, left, base]{\color{textcolor}\sffamily\fontsize{11.000000}{13.200000}\selectfont \(\displaystyle {80}\)}%
\end{pgfscope}%
\begin{pgfscope}%
\pgfpathrectangle{\pgfqpoint{0.800000in}{0.528000in}}{\pgfqpoint{4.960000in}{3.696000in}}%
\pgfusepath{clip}%
\pgfsetroundcap%
\pgfsetroundjoin%
\pgfsetlinewidth{1.003750pt}%
\definecolor{currentstroke}{rgb}{0.800000,0.800000,0.800000}%
\pgfsetstrokecolor{currentstroke}%
\pgfsetdash{}{0pt}%
\pgfpathmoveto{\pgfqpoint{0.800000in}{1.058046in}}%
\pgfpathlineto{\pgfqpoint{5.760000in}{1.058046in}}%
\pgfusepath{stroke}%
\end{pgfscope}%
\begin{pgfscope}%
\definecolor{textcolor}{rgb}{0.150000,0.150000,0.150000}%
\pgfsetstrokecolor{textcolor}%
\pgfsetfillcolor{textcolor}%
\pgftext[x=0.439930in, y=1.005240in, left, base]{\color{textcolor}\sffamily\fontsize{11.000000}{13.200000}\selectfont \(\displaystyle {100}\)}%
\end{pgfscope}%
\begin{pgfscope}%
\pgfpathrectangle{\pgfqpoint{0.800000in}{0.528000in}}{\pgfqpoint{4.960000in}{3.696000in}}%
\pgfusepath{clip}%
\pgfsetroundcap%
\pgfsetroundjoin%
\pgfsetlinewidth{1.003750pt}%
\definecolor{currentstroke}{rgb}{0.800000,0.800000,0.800000}%
\pgfsetstrokecolor{currentstroke}%
\pgfsetdash{}{0pt}%
\pgfpathmoveto{\pgfqpoint{0.800000in}{1.521454in}}%
\pgfpathlineto{\pgfqpoint{5.760000in}{1.521454in}}%
\pgfusepath{stroke}%
\end{pgfscope}%
\begin{pgfscope}%
\definecolor{textcolor}{rgb}{0.150000,0.150000,0.150000}%
\pgfsetstrokecolor{textcolor}%
\pgfsetfillcolor{textcolor}%
\pgftext[x=0.439930in, y=1.468647in, left, base]{\color{textcolor}\sffamily\fontsize{11.000000}{13.200000}\selectfont \(\displaystyle {120}\)}%
\end{pgfscope}%
\begin{pgfscope}%
\pgfpathrectangle{\pgfqpoint{0.800000in}{0.528000in}}{\pgfqpoint{4.960000in}{3.696000in}}%
\pgfusepath{clip}%
\pgfsetroundcap%
\pgfsetroundjoin%
\pgfsetlinewidth{1.003750pt}%
\definecolor{currentstroke}{rgb}{0.800000,0.800000,0.800000}%
\pgfsetstrokecolor{currentstroke}%
\pgfsetdash{}{0pt}%
\pgfpathmoveto{\pgfqpoint{0.800000in}{1.984861in}}%
\pgfpathlineto{\pgfqpoint{5.760000in}{1.984861in}}%
\pgfusepath{stroke}%
\end{pgfscope}%
\begin{pgfscope}%
\definecolor{textcolor}{rgb}{0.150000,0.150000,0.150000}%
\pgfsetstrokecolor{textcolor}%
\pgfsetfillcolor{textcolor}%
\pgftext[x=0.439930in, y=1.932054in, left, base]{\color{textcolor}\sffamily\fontsize{11.000000}{13.200000}\selectfont \(\displaystyle {140}\)}%
\end{pgfscope}%
\begin{pgfscope}%
\pgfpathrectangle{\pgfqpoint{0.800000in}{0.528000in}}{\pgfqpoint{4.960000in}{3.696000in}}%
\pgfusepath{clip}%
\pgfsetroundcap%
\pgfsetroundjoin%
\pgfsetlinewidth{1.003750pt}%
\definecolor{currentstroke}{rgb}{0.800000,0.800000,0.800000}%
\pgfsetstrokecolor{currentstroke}%
\pgfsetdash{}{0pt}%
\pgfpathmoveto{\pgfqpoint{0.800000in}{2.448268in}}%
\pgfpathlineto{\pgfqpoint{5.760000in}{2.448268in}}%
\pgfusepath{stroke}%
\end{pgfscope}%
\begin{pgfscope}%
\definecolor{textcolor}{rgb}{0.150000,0.150000,0.150000}%
\pgfsetstrokecolor{textcolor}%
\pgfsetfillcolor{textcolor}%
\pgftext[x=0.439930in, y=2.395461in, left, base]{\color{textcolor}\sffamily\fontsize{11.000000}{13.200000}\selectfont \(\displaystyle {160}\)}%
\end{pgfscope}%
\begin{pgfscope}%
\pgfpathrectangle{\pgfqpoint{0.800000in}{0.528000in}}{\pgfqpoint{4.960000in}{3.696000in}}%
\pgfusepath{clip}%
\pgfsetroundcap%
\pgfsetroundjoin%
\pgfsetlinewidth{1.003750pt}%
\definecolor{currentstroke}{rgb}{0.800000,0.800000,0.800000}%
\pgfsetstrokecolor{currentstroke}%
\pgfsetdash{}{0pt}%
\pgfpathmoveto{\pgfqpoint{0.800000in}{2.911675in}}%
\pgfpathlineto{\pgfqpoint{5.760000in}{2.911675in}}%
\pgfusepath{stroke}%
\end{pgfscope}%
\begin{pgfscope}%
\definecolor{textcolor}{rgb}{0.150000,0.150000,0.150000}%
\pgfsetstrokecolor{textcolor}%
\pgfsetfillcolor{textcolor}%
\pgftext[x=0.439930in, y=2.858869in, left, base]{\color{textcolor}\sffamily\fontsize{11.000000}{13.200000}\selectfont \(\displaystyle {180}\)}%
\end{pgfscope}%
\begin{pgfscope}%
\pgfpathrectangle{\pgfqpoint{0.800000in}{0.528000in}}{\pgfqpoint{4.960000in}{3.696000in}}%
\pgfusepath{clip}%
\pgfsetroundcap%
\pgfsetroundjoin%
\pgfsetlinewidth{1.003750pt}%
\definecolor{currentstroke}{rgb}{0.800000,0.800000,0.800000}%
\pgfsetstrokecolor{currentstroke}%
\pgfsetdash{}{0pt}%
\pgfpathmoveto{\pgfqpoint{0.800000in}{3.375082in}}%
\pgfpathlineto{\pgfqpoint{5.760000in}{3.375082in}}%
\pgfusepath{stroke}%
\end{pgfscope}%
\begin{pgfscope}%
\definecolor{textcolor}{rgb}{0.150000,0.150000,0.150000}%
\pgfsetstrokecolor{textcolor}%
\pgfsetfillcolor{textcolor}%
\pgftext[x=0.439930in, y=3.322276in, left, base]{\color{textcolor}\sffamily\fontsize{11.000000}{13.200000}\selectfont \(\displaystyle {200}\)}%
\end{pgfscope}%
\begin{pgfscope}%
\pgfpathrectangle{\pgfqpoint{0.800000in}{0.528000in}}{\pgfqpoint{4.960000in}{3.696000in}}%
\pgfusepath{clip}%
\pgfsetroundcap%
\pgfsetroundjoin%
\pgfsetlinewidth{1.003750pt}%
\definecolor{currentstroke}{rgb}{0.800000,0.800000,0.800000}%
\pgfsetstrokecolor{currentstroke}%
\pgfsetdash{}{0pt}%
\pgfpathmoveto{\pgfqpoint{0.800000in}{3.838490in}}%
\pgfpathlineto{\pgfqpoint{5.760000in}{3.838490in}}%
\pgfusepath{stroke}%
\end{pgfscope}%
\begin{pgfscope}%
\definecolor{textcolor}{rgb}{0.150000,0.150000,0.150000}%
\pgfsetstrokecolor{textcolor}%
\pgfsetfillcolor{textcolor}%
\pgftext[x=0.439930in, y=3.785683in, left, base]{\color{textcolor}\sffamily\fontsize{11.000000}{13.200000}\selectfont \(\displaystyle {220}\)}%
\end{pgfscope}%
\begin{pgfscope}%
\definecolor{textcolor}{rgb}{0.150000,0.150000,0.150000}%
\pgfsetstrokecolor{textcolor}%
\pgfsetfillcolor{textcolor}%
\pgftext[x=0.384375in,y=2.376000in,,bottom,rotate=90.000000]{\color{textcolor}\sffamily\fontsize{12.000000}{14.400000}\selectfont Time (ms)}%
\end{pgfscope}%
\begin{pgfscope}%
\pgfpathrectangle{\pgfqpoint{0.800000in}{0.528000in}}{\pgfqpoint{4.960000in}{3.696000in}}%
\pgfusepath{clip}%
\pgfsetbuttcap%
\pgfsetroundjoin%
\definecolor{currentfill}{rgb}{0.347059,0.458824,0.641176}%
\pgfsetfillcolor{currentfill}%
\pgfsetlinewidth{1.505625pt}%
\definecolor{currentstroke}{rgb}{0.298039,0.298039,0.298039}%
\pgfsetstrokecolor{currentstroke}%
\pgfsetdash{}{0pt}%
\pgfsys@defobject{currentmarker}{\pgfqpoint{1.296000in}{0.696000in}}{\pgfqpoint{3.280000in}{1.875520in}}{%
\pgfpathmoveto{\pgfqpoint{3.280000in}{0.696000in}}%
\pgfpathlineto{\pgfqpoint{3.260196in}{0.696000in}}%
\pgfpathlineto{\pgfqpoint{3.252184in}{0.707914in}}%
\pgfpathlineto{\pgfqpoint{3.242119in}{0.719829in}}%
\pgfpathlineto{\pgfqpoint{3.229982in}{0.731743in}}%
\pgfpathlineto{\pgfqpoint{3.215961in}{0.743657in}}%
\pgfpathlineto{\pgfqpoint{3.200492in}{0.755572in}}%
\pgfpathlineto{\pgfqpoint{3.184266in}{0.767486in}}%
\pgfpathlineto{\pgfqpoint{3.168195in}{0.779400in}}%
\pgfpathlineto{\pgfqpoint{3.153332in}{0.791315in}}%
\pgfpathlineto{\pgfqpoint{3.140759in}{0.803229in}}%
\pgfpathlineto{\pgfqpoint{3.131451in}{0.815143in}}%
\pgfpathlineto{\pgfqpoint{3.126146in}{0.827058in}}%
\pgfpathlineto{\pgfqpoint{3.125241in}{0.838972in}}%
\pgfpathlineto{\pgfqpoint{3.128740in}{0.850886in}}%
\pgfpathlineto{\pgfqpoint{3.136255in}{0.862801in}}%
\pgfpathlineto{\pgfqpoint{3.147077in}{0.874715in}}%
\pgfpathlineto{\pgfqpoint{3.160275in}{0.886629in}}%
\pgfpathlineto{\pgfqpoint{3.174828in}{0.898544in}}%
\pgfpathlineto{\pgfqpoint{3.189752in}{0.910458in}}%
\pgfpathlineto{\pgfqpoint{3.204200in}{0.922373in}}%
\pgfpathlineto{\pgfqpoint{3.217529in}{0.934287in}}%
\pgfpathlineto{\pgfqpoint{3.229328in}{0.946201in}}%
\pgfpathlineto{\pgfqpoint{3.239401in}{0.958116in}}%
\pgfpathlineto{\pgfqpoint{3.247738in}{0.970030in}}%
\pgfpathlineto{\pgfqpoint{3.254461in}{0.981944in}}%
\pgfpathlineto{\pgfqpoint{3.259775in}{0.993859in}}%
\pgfpathlineto{\pgfqpoint{3.263918in}{1.005773in}}%
\pgfpathlineto{\pgfqpoint{3.267132in}{1.017687in}}%
\pgfpathlineto{\pgfqpoint{3.269634in}{1.029602in}}%
\pgfpathlineto{\pgfqpoint{3.271605in}{1.041516in}}%
\pgfpathlineto{\pgfqpoint{3.273186in}{1.053430in}}%
\pgfpathlineto{\pgfqpoint{3.274479in}{1.065345in}}%
\pgfpathlineto{\pgfqpoint{3.275556in}{1.077259in}}%
\pgfpathlineto{\pgfqpoint{3.276462in}{1.089173in}}%
\pgfpathlineto{\pgfqpoint{3.277225in}{1.101088in}}%
\pgfpathlineto{\pgfqpoint{3.277864in}{1.113002in}}%
\pgfpathlineto{\pgfqpoint{3.278390in}{1.124916in}}%
\pgfpathlineto{\pgfqpoint{3.278815in}{1.136831in}}%
\pgfpathlineto{\pgfqpoint{3.279150in}{1.148745in}}%
\pgfpathlineto{\pgfqpoint{3.279406in}{1.160659in}}%
\pgfpathlineto{\pgfqpoint{3.279596in}{1.172574in}}%
\pgfpathlineto{\pgfqpoint{3.279733in}{1.184488in}}%
\pgfpathlineto{\pgfqpoint{3.279829in}{1.196402in}}%
\pgfpathlineto{\pgfqpoint{3.279893in}{1.208317in}}%
\pgfpathlineto{\pgfqpoint{3.279936in}{1.220231in}}%
\pgfpathlineto{\pgfqpoint{3.279962in}{1.232145in}}%
\pgfpathlineto{\pgfqpoint{3.279979in}{1.244060in}}%
\pgfpathlineto{\pgfqpoint{3.279988in}{1.255974in}}%
\pgfpathlineto{\pgfqpoint{3.279994in}{1.267888in}}%
\pgfpathlineto{\pgfqpoint{3.279997in}{1.279803in}}%
\pgfpathlineto{\pgfqpoint{3.279998in}{1.291717in}}%
\pgfpathlineto{\pgfqpoint{3.279999in}{1.303631in}}%
\pgfpathlineto{\pgfqpoint{3.279999in}{1.315546in}}%
\pgfpathlineto{\pgfqpoint{3.279999in}{1.327460in}}%
\pgfpathlineto{\pgfqpoint{3.279999in}{1.339374in}}%
\pgfpathlineto{\pgfqpoint{3.279997in}{1.351289in}}%
\pgfpathlineto{\pgfqpoint{3.279993in}{1.363203in}}%
\pgfpathlineto{\pgfqpoint{3.279984in}{1.375118in}}%
\pgfpathlineto{\pgfqpoint{3.279964in}{1.387032in}}%
\pgfpathlineto{\pgfqpoint{3.279925in}{1.398946in}}%
\pgfpathlineto{\pgfqpoint{3.279845in}{1.410861in}}%
\pgfpathlineto{\pgfqpoint{3.279692in}{1.422775in}}%
\pgfpathlineto{\pgfqpoint{3.279402in}{1.434689in}}%
\pgfpathlineto{\pgfqpoint{3.278873in}{1.446604in}}%
\pgfpathlineto{\pgfqpoint{3.277933in}{1.458518in}}%
\pgfpathlineto{\pgfqpoint{3.276311in}{1.470432in}}%
\pgfpathlineto{\pgfqpoint{3.273586in}{1.482347in}}%
\pgfpathlineto{\pgfqpoint{3.269140in}{1.494261in}}%
\pgfpathlineto{\pgfqpoint{3.262082in}{1.506175in}}%
\pgfpathlineto{\pgfqpoint{3.251181in}{1.518090in}}%
\pgfpathlineto{\pgfqpoint{3.234805in}{1.530004in}}%
\pgfpathlineto{\pgfqpoint{3.210870in}{1.541918in}}%
\pgfpathlineto{\pgfqpoint{3.176846in}{1.553833in}}%
\pgfpathlineto{\pgfqpoint{3.129823in}{1.565747in}}%
\pgfpathlineto{\pgfqpoint{3.066674in}{1.577661in}}%
\pgfpathlineto{\pgfqpoint{2.984338in}{1.589576in}}%
\pgfpathlineto{\pgfqpoint{2.880229in}{1.601490in}}%
\pgfpathlineto{\pgfqpoint{2.752757in}{1.613404in}}%
\pgfpathlineto{\pgfqpoint{2.601915in}{1.625319in}}%
\pgfpathlineto{\pgfqpoint{2.429843in}{1.637233in}}%
\pgfpathlineto{\pgfqpoint{2.241267in}{1.649147in}}%
\pgfpathlineto{\pgfqpoint{2.043669in}{1.661062in}}%
\pgfpathlineto{\pgfqpoint{1.847087in}{1.672976in}}%
\pgfpathlineto{\pgfqpoint{1.663461in}{1.684890in}}%
\pgfpathlineto{\pgfqpoint{1.505556in}{1.696805in}}%
\pgfpathlineto{\pgfqpoint{1.385549in}{1.708719in}}%
\pgfpathlineto{\pgfqpoint{1.313509in}{1.720633in}}%
\pgfpathlineto{\pgfqpoint{1.296000in}{1.732548in}}%
\pgfpathlineto{\pgfqpoint{1.335101in}{1.744462in}}%
\pgfpathlineto{\pgfqpoint{1.428026in}{1.756376in}}%
\pgfpathlineto{\pgfqpoint{1.567450in}{1.768291in}}%
\pgfpathlineto{\pgfqpoint{1.742475in}{1.780205in}}%
\pgfpathlineto{\pgfqpoint{1.940068in}{1.792119in}}%
\pgfpathlineto{\pgfqpoint{2.146697in}{1.804034in}}%
\pgfpathlineto{\pgfqpoint{2.349872in}{1.815948in}}%
\pgfpathlineto{\pgfqpoint{2.539352in}{1.827863in}}%
\pgfpathlineto{\pgfqpoint{2.707862in}{1.839777in}}%
\pgfpathlineto{\pgfqpoint{2.851291in}{1.851691in}}%
\pgfpathlineto{\pgfqpoint{2.968429in}{1.863606in}}%
\pgfpathlineto{\pgfqpoint{3.060391in}{1.875520in}}%
\pgfpathlineto{\pgfqpoint{3.280000in}{1.875520in}}%
\pgfpathlineto{\pgfqpoint{3.280000in}{1.875520in}}%
\pgfpathlineto{\pgfqpoint{3.280000in}{1.863606in}}%
\pgfpathlineto{\pgfqpoint{3.280000in}{1.851691in}}%
\pgfpathlineto{\pgfqpoint{3.280000in}{1.839777in}}%
\pgfpathlineto{\pgfqpoint{3.280000in}{1.827863in}}%
\pgfpathlineto{\pgfqpoint{3.280000in}{1.815948in}}%
\pgfpathlineto{\pgfqpoint{3.280000in}{1.804034in}}%
\pgfpathlineto{\pgfqpoint{3.280000in}{1.792119in}}%
\pgfpathlineto{\pgfqpoint{3.280000in}{1.780205in}}%
\pgfpathlineto{\pgfqpoint{3.280000in}{1.768291in}}%
\pgfpathlineto{\pgfqpoint{3.280000in}{1.756376in}}%
\pgfpathlineto{\pgfqpoint{3.280000in}{1.744462in}}%
\pgfpathlineto{\pgfqpoint{3.280000in}{1.732548in}}%
\pgfpathlineto{\pgfqpoint{3.280000in}{1.720633in}}%
\pgfpathlineto{\pgfqpoint{3.280000in}{1.708719in}}%
\pgfpathlineto{\pgfqpoint{3.280000in}{1.696805in}}%
\pgfpathlineto{\pgfqpoint{3.280000in}{1.684890in}}%
\pgfpathlineto{\pgfqpoint{3.280000in}{1.672976in}}%
\pgfpathlineto{\pgfqpoint{3.280000in}{1.661062in}}%
\pgfpathlineto{\pgfqpoint{3.280000in}{1.649147in}}%
\pgfpathlineto{\pgfqpoint{3.280000in}{1.637233in}}%
\pgfpathlineto{\pgfqpoint{3.280000in}{1.625319in}}%
\pgfpathlineto{\pgfqpoint{3.280000in}{1.613404in}}%
\pgfpathlineto{\pgfqpoint{3.280000in}{1.601490in}}%
\pgfpathlineto{\pgfqpoint{3.280000in}{1.589576in}}%
\pgfpathlineto{\pgfqpoint{3.280000in}{1.577661in}}%
\pgfpathlineto{\pgfqpoint{3.280000in}{1.565747in}}%
\pgfpathlineto{\pgfqpoint{3.280000in}{1.553833in}}%
\pgfpathlineto{\pgfqpoint{3.280000in}{1.541918in}}%
\pgfpathlineto{\pgfqpoint{3.280000in}{1.530004in}}%
\pgfpathlineto{\pgfqpoint{3.280000in}{1.518090in}}%
\pgfpathlineto{\pgfqpoint{3.280000in}{1.506175in}}%
\pgfpathlineto{\pgfqpoint{3.280000in}{1.494261in}}%
\pgfpathlineto{\pgfqpoint{3.280000in}{1.482347in}}%
\pgfpathlineto{\pgfqpoint{3.280000in}{1.470432in}}%
\pgfpathlineto{\pgfqpoint{3.280000in}{1.458518in}}%
\pgfpathlineto{\pgfqpoint{3.280000in}{1.446604in}}%
\pgfpathlineto{\pgfqpoint{3.280000in}{1.434689in}}%
\pgfpathlineto{\pgfqpoint{3.280000in}{1.422775in}}%
\pgfpathlineto{\pgfqpoint{3.280000in}{1.410861in}}%
\pgfpathlineto{\pgfqpoint{3.280000in}{1.398946in}}%
\pgfpathlineto{\pgfqpoint{3.280000in}{1.387032in}}%
\pgfpathlineto{\pgfqpoint{3.280000in}{1.375118in}}%
\pgfpathlineto{\pgfqpoint{3.280000in}{1.363203in}}%
\pgfpathlineto{\pgfqpoint{3.280000in}{1.351289in}}%
\pgfpathlineto{\pgfqpoint{3.280000in}{1.339374in}}%
\pgfpathlineto{\pgfqpoint{3.280000in}{1.327460in}}%
\pgfpathlineto{\pgfqpoint{3.280000in}{1.315546in}}%
\pgfpathlineto{\pgfqpoint{3.280000in}{1.303631in}}%
\pgfpathlineto{\pgfqpoint{3.280000in}{1.291717in}}%
\pgfpathlineto{\pgfqpoint{3.280000in}{1.279803in}}%
\pgfpathlineto{\pgfqpoint{3.280000in}{1.267888in}}%
\pgfpathlineto{\pgfqpoint{3.280000in}{1.255974in}}%
\pgfpathlineto{\pgfqpoint{3.280000in}{1.244060in}}%
\pgfpathlineto{\pgfqpoint{3.280000in}{1.232145in}}%
\pgfpathlineto{\pgfqpoint{3.280000in}{1.220231in}}%
\pgfpathlineto{\pgfqpoint{3.280000in}{1.208317in}}%
\pgfpathlineto{\pgfqpoint{3.280000in}{1.196402in}}%
\pgfpathlineto{\pgfqpoint{3.280000in}{1.184488in}}%
\pgfpathlineto{\pgfqpoint{3.280000in}{1.172574in}}%
\pgfpathlineto{\pgfqpoint{3.280000in}{1.160659in}}%
\pgfpathlineto{\pgfqpoint{3.280000in}{1.148745in}}%
\pgfpathlineto{\pgfqpoint{3.280000in}{1.136831in}}%
\pgfpathlineto{\pgfqpoint{3.280000in}{1.124916in}}%
\pgfpathlineto{\pgfqpoint{3.280000in}{1.113002in}}%
\pgfpathlineto{\pgfqpoint{3.280000in}{1.101088in}}%
\pgfpathlineto{\pgfqpoint{3.280000in}{1.089173in}}%
\pgfpathlineto{\pgfqpoint{3.280000in}{1.077259in}}%
\pgfpathlineto{\pgfqpoint{3.280000in}{1.065345in}}%
\pgfpathlineto{\pgfqpoint{3.280000in}{1.053430in}}%
\pgfpathlineto{\pgfqpoint{3.280000in}{1.041516in}}%
\pgfpathlineto{\pgfqpoint{3.280000in}{1.029602in}}%
\pgfpathlineto{\pgfqpoint{3.280000in}{1.017687in}}%
\pgfpathlineto{\pgfqpoint{3.280000in}{1.005773in}}%
\pgfpathlineto{\pgfqpoint{3.280000in}{0.993859in}}%
\pgfpathlineto{\pgfqpoint{3.280000in}{0.981944in}}%
\pgfpathlineto{\pgfqpoint{3.280000in}{0.970030in}}%
\pgfpathlineto{\pgfqpoint{3.280000in}{0.958116in}}%
\pgfpathlineto{\pgfqpoint{3.280000in}{0.946201in}}%
\pgfpathlineto{\pgfqpoint{3.280000in}{0.934287in}}%
\pgfpathlineto{\pgfqpoint{3.280000in}{0.922373in}}%
\pgfpathlineto{\pgfqpoint{3.280000in}{0.910458in}}%
\pgfpathlineto{\pgfqpoint{3.280000in}{0.898544in}}%
\pgfpathlineto{\pgfqpoint{3.280000in}{0.886629in}}%
\pgfpathlineto{\pgfqpoint{3.280000in}{0.874715in}}%
\pgfpathlineto{\pgfqpoint{3.280000in}{0.862801in}}%
\pgfpathlineto{\pgfqpoint{3.280000in}{0.850886in}}%
\pgfpathlineto{\pgfqpoint{3.280000in}{0.838972in}}%
\pgfpathlineto{\pgfqpoint{3.280000in}{0.827058in}}%
\pgfpathlineto{\pgfqpoint{3.280000in}{0.815143in}}%
\pgfpathlineto{\pgfqpoint{3.280000in}{0.803229in}}%
\pgfpathlineto{\pgfqpoint{3.280000in}{0.791315in}}%
\pgfpathlineto{\pgfqpoint{3.280000in}{0.779400in}}%
\pgfpathlineto{\pgfqpoint{3.280000in}{0.767486in}}%
\pgfpathlineto{\pgfqpoint{3.280000in}{0.755572in}}%
\pgfpathlineto{\pgfqpoint{3.280000in}{0.743657in}}%
\pgfpathlineto{\pgfqpoint{3.280000in}{0.731743in}}%
\pgfpathlineto{\pgfqpoint{3.280000in}{0.719829in}}%
\pgfpathlineto{\pgfqpoint{3.280000in}{0.707914in}}%
\pgfpathlineto{\pgfqpoint{3.280000in}{0.696000in}}%
\pgfpathclose%
\pgfusepath{stroke,fill}%
}%
\begin{pgfscope}%
\pgfsys@transformshift{0.000000in}{0.000000in}%
\pgfsys@useobject{currentmarker}{}%
\end{pgfscope}%
\end{pgfscope}%
\begin{pgfscope}%
\pgfpathrectangle{\pgfqpoint{0.800000in}{0.528000in}}{\pgfqpoint{4.960000in}{3.696000in}}%
\pgfusepath{clip}%
\pgfsetbuttcap%
\pgfsetroundjoin%
\definecolor{currentfill}{rgb}{0.798529,0.536765,0.389706}%
\pgfsetfillcolor{currentfill}%
\pgfsetlinewidth{1.505625pt}%
\definecolor{currentstroke}{rgb}{0.298039,0.298039,0.298039}%
\pgfsetstrokecolor{currentstroke}%
\pgfsetdash{}{0pt}%
\pgfsys@defobject{currentmarker}{\pgfqpoint{3.280000in}{1.315596in}}{\pgfqpoint{4.744000in}{4.056000in}}{%
\pgfpathmoveto{\pgfqpoint{3.323362in}{1.315596in}}%
\pgfpathlineto{\pgfqpoint{3.280000in}{1.315596in}}%
\pgfpathlineto{\pgfqpoint{3.280000in}{1.343277in}}%
\pgfpathlineto{\pgfqpoint{3.280000in}{1.370958in}}%
\pgfpathlineto{\pgfqpoint{3.280000in}{1.398638in}}%
\pgfpathlineto{\pgfqpoint{3.280000in}{1.426319in}}%
\pgfpathlineto{\pgfqpoint{3.280000in}{1.454000in}}%
\pgfpathlineto{\pgfqpoint{3.280000in}{1.481681in}}%
\pgfpathlineto{\pgfqpoint{3.280000in}{1.509362in}}%
\pgfpathlineto{\pgfqpoint{3.280000in}{1.537043in}}%
\pgfpathlineto{\pgfqpoint{3.280000in}{1.564723in}}%
\pgfpathlineto{\pgfqpoint{3.280000in}{1.592404in}}%
\pgfpathlineto{\pgfqpoint{3.280000in}{1.620085in}}%
\pgfpathlineto{\pgfqpoint{3.280000in}{1.647766in}}%
\pgfpathlineto{\pgfqpoint{3.280000in}{1.675447in}}%
\pgfpathlineto{\pgfqpoint{3.280000in}{1.703128in}}%
\pgfpathlineto{\pgfqpoint{3.280000in}{1.730809in}}%
\pgfpathlineto{\pgfqpoint{3.280000in}{1.758489in}}%
\pgfpathlineto{\pgfqpoint{3.280000in}{1.786170in}}%
\pgfpathlineto{\pgfqpoint{3.280000in}{1.813851in}}%
\pgfpathlineto{\pgfqpoint{3.280000in}{1.841532in}}%
\pgfpathlineto{\pgfqpoint{3.280000in}{1.869213in}}%
\pgfpathlineto{\pgfqpoint{3.280000in}{1.896894in}}%
\pgfpathlineto{\pgfqpoint{3.280000in}{1.924575in}}%
\pgfpathlineto{\pgfqpoint{3.280000in}{1.952255in}}%
\pgfpathlineto{\pgfqpoint{3.280000in}{1.979936in}}%
\pgfpathlineto{\pgfqpoint{3.280000in}{2.007617in}}%
\pgfpathlineto{\pgfqpoint{3.280000in}{2.035298in}}%
\pgfpathlineto{\pgfqpoint{3.280000in}{2.062979in}}%
\pgfpathlineto{\pgfqpoint{3.280000in}{2.090660in}}%
\pgfpathlineto{\pgfqpoint{3.280000in}{2.118340in}}%
\pgfpathlineto{\pgfqpoint{3.280000in}{2.146021in}}%
\pgfpathlineto{\pgfqpoint{3.280000in}{2.173702in}}%
\pgfpathlineto{\pgfqpoint{3.280000in}{2.201383in}}%
\pgfpathlineto{\pgfqpoint{3.280000in}{2.229064in}}%
\pgfpathlineto{\pgfqpoint{3.280000in}{2.256745in}}%
\pgfpathlineto{\pgfqpoint{3.280000in}{2.284426in}}%
\pgfpathlineto{\pgfqpoint{3.280000in}{2.312106in}}%
\pgfpathlineto{\pgfqpoint{3.280000in}{2.339787in}}%
\pgfpathlineto{\pgfqpoint{3.280000in}{2.367468in}}%
\pgfpathlineto{\pgfqpoint{3.280000in}{2.395149in}}%
\pgfpathlineto{\pgfqpoint{3.280000in}{2.422830in}}%
\pgfpathlineto{\pgfqpoint{3.280000in}{2.450511in}}%
\pgfpathlineto{\pgfqpoint{3.280000in}{2.478192in}}%
\pgfpathlineto{\pgfqpoint{3.280000in}{2.505872in}}%
\pgfpathlineto{\pgfqpoint{3.280000in}{2.533553in}}%
\pgfpathlineto{\pgfqpoint{3.280000in}{2.561234in}}%
\pgfpathlineto{\pgfqpoint{3.280000in}{2.588915in}}%
\pgfpathlineto{\pgfqpoint{3.280000in}{2.616596in}}%
\pgfpathlineto{\pgfqpoint{3.280000in}{2.644277in}}%
\pgfpathlineto{\pgfqpoint{3.280000in}{2.671957in}}%
\pgfpathlineto{\pgfqpoint{3.280000in}{2.699638in}}%
\pgfpathlineto{\pgfqpoint{3.280000in}{2.727319in}}%
\pgfpathlineto{\pgfqpoint{3.280000in}{2.755000in}}%
\pgfpathlineto{\pgfqpoint{3.280000in}{2.782681in}}%
\pgfpathlineto{\pgfqpoint{3.280000in}{2.810362in}}%
\pgfpathlineto{\pgfqpoint{3.280000in}{2.838043in}}%
\pgfpathlineto{\pgfqpoint{3.280000in}{2.865723in}}%
\pgfpathlineto{\pgfqpoint{3.280000in}{2.893404in}}%
\pgfpathlineto{\pgfqpoint{3.280000in}{2.921085in}}%
\pgfpathlineto{\pgfqpoint{3.280000in}{2.948766in}}%
\pgfpathlineto{\pgfqpoint{3.280000in}{2.976447in}}%
\pgfpathlineto{\pgfqpoint{3.280000in}{3.004128in}}%
\pgfpathlineto{\pgfqpoint{3.280000in}{3.031809in}}%
\pgfpathlineto{\pgfqpoint{3.280000in}{3.059489in}}%
\pgfpathlineto{\pgfqpoint{3.280000in}{3.087170in}}%
\pgfpathlineto{\pgfqpoint{3.280000in}{3.114851in}}%
\pgfpathlineto{\pgfqpoint{3.280000in}{3.142532in}}%
\pgfpathlineto{\pgfqpoint{3.280000in}{3.170213in}}%
\pgfpathlineto{\pgfqpoint{3.280000in}{3.197894in}}%
\pgfpathlineto{\pgfqpoint{3.280000in}{3.225574in}}%
\pgfpathlineto{\pgfqpoint{3.280000in}{3.253255in}}%
\pgfpathlineto{\pgfqpoint{3.280000in}{3.280936in}}%
\pgfpathlineto{\pgfqpoint{3.280000in}{3.308617in}}%
\pgfpathlineto{\pgfqpoint{3.280000in}{3.336298in}}%
\pgfpathlineto{\pgfqpoint{3.280000in}{3.363979in}}%
\pgfpathlineto{\pgfqpoint{3.280000in}{3.391660in}}%
\pgfpathlineto{\pgfqpoint{3.280000in}{3.419340in}}%
\pgfpathlineto{\pgfqpoint{3.280000in}{3.447021in}}%
\pgfpathlineto{\pgfqpoint{3.280000in}{3.474702in}}%
\pgfpathlineto{\pgfqpoint{3.280000in}{3.502383in}}%
\pgfpathlineto{\pgfqpoint{3.280000in}{3.530064in}}%
\pgfpathlineto{\pgfqpoint{3.280000in}{3.557745in}}%
\pgfpathlineto{\pgfqpoint{3.280000in}{3.585426in}}%
\pgfpathlineto{\pgfqpoint{3.280000in}{3.613106in}}%
\pgfpathlineto{\pgfqpoint{3.280000in}{3.640787in}}%
\pgfpathlineto{\pgfqpoint{3.280000in}{3.668468in}}%
\pgfpathlineto{\pgfqpoint{3.280000in}{3.696149in}}%
\pgfpathlineto{\pgfqpoint{3.280000in}{3.723830in}}%
\pgfpathlineto{\pgfqpoint{3.280000in}{3.751511in}}%
\pgfpathlineto{\pgfqpoint{3.280000in}{3.779191in}}%
\pgfpathlineto{\pgfqpoint{3.280000in}{3.806872in}}%
\pgfpathlineto{\pgfqpoint{3.280000in}{3.834553in}}%
\pgfpathlineto{\pgfqpoint{3.280000in}{3.862234in}}%
\pgfpathlineto{\pgfqpoint{3.280000in}{3.889915in}}%
\pgfpathlineto{\pgfqpoint{3.280000in}{3.917596in}}%
\pgfpathlineto{\pgfqpoint{3.280000in}{3.945277in}}%
\pgfpathlineto{\pgfqpoint{3.280000in}{3.972957in}}%
\pgfpathlineto{\pgfqpoint{3.280000in}{4.000638in}}%
\pgfpathlineto{\pgfqpoint{3.280000in}{4.028319in}}%
\pgfpathlineto{\pgfqpoint{3.280000in}{4.056000in}}%
\pgfpathlineto{\pgfqpoint{3.287836in}{4.056000in}}%
\pgfpathlineto{\pgfqpoint{3.287836in}{4.056000in}}%
\pgfpathlineto{\pgfqpoint{3.291495in}{4.028319in}}%
\pgfpathlineto{\pgfqpoint{3.296545in}{4.000638in}}%
\pgfpathlineto{\pgfqpoint{3.303376in}{3.972957in}}%
\pgfpathlineto{\pgfqpoint{3.312431in}{3.945277in}}%
\pgfpathlineto{\pgfqpoint{3.324188in}{3.917596in}}%
\pgfpathlineto{\pgfqpoint{3.339129in}{3.889915in}}%
\pgfpathlineto{\pgfqpoint{3.357688in}{3.862234in}}%
\pgfpathlineto{\pgfqpoint{3.380191in}{3.834553in}}%
\pgfpathlineto{\pgfqpoint{3.406766in}{3.806872in}}%
\pgfpathlineto{\pgfqpoint{3.437265in}{3.779191in}}%
\pgfpathlineto{\pgfqpoint{3.471173in}{3.751511in}}%
\pgfpathlineto{\pgfqpoint{3.507558in}{3.723830in}}%
\pgfpathlineto{\pgfqpoint{3.545050in}{3.696149in}}%
\pgfpathlineto{\pgfqpoint{3.581883in}{3.668468in}}%
\pgfpathlineto{\pgfqpoint{3.616007in}{3.640787in}}%
\pgfpathlineto{\pgfqpoint{3.645258in}{3.613106in}}%
\pgfpathlineto{\pgfqpoint{3.667580in}{3.585426in}}%
\pgfpathlineto{\pgfqpoint{3.681260in}{3.557745in}}%
\pgfpathlineto{\pgfqpoint{3.685146in}{3.530064in}}%
\pgfpathlineto{\pgfqpoint{3.678809in}{3.502383in}}%
\pgfpathlineto{\pgfqpoint{3.662610in}{3.474702in}}%
\pgfpathlineto{\pgfqpoint{3.637677in}{3.447021in}}%
\pgfpathlineto{\pgfqpoint{3.605772in}{3.419340in}}%
\pgfpathlineto{\pgfqpoint{3.569090in}{3.391660in}}%
\pgfpathlineto{\pgfqpoint{3.530026in}{3.363979in}}%
\pgfpathlineto{\pgfqpoint{3.490930in}{3.336298in}}%
\pgfpathlineto{\pgfqpoint{3.453922in}{3.308617in}}%
\pgfpathlineto{\pgfqpoint{3.420764in}{3.280936in}}%
\pgfpathlineto{\pgfqpoint{3.392815in}{3.253255in}}%
\pgfpathlineto{\pgfqpoint{3.371064in}{3.225574in}}%
\pgfpathlineto{\pgfqpoint{3.356222in}{3.197894in}}%
\pgfpathlineto{\pgfqpoint{3.348838in}{3.170213in}}%
\pgfpathlineto{\pgfqpoint{3.349428in}{3.142532in}}%
\pgfpathlineto{\pgfqpoint{3.358573in}{3.114851in}}%
\pgfpathlineto{\pgfqpoint{3.376982in}{3.087170in}}%
\pgfpathlineto{\pgfqpoint{3.405496in}{3.059489in}}%
\pgfpathlineto{\pgfqpoint{3.445037in}{3.031809in}}%
\pgfpathlineto{\pgfqpoint{3.496507in}{3.004128in}}%
\pgfpathlineto{\pgfqpoint{3.560646in}{2.976447in}}%
\pgfpathlineto{\pgfqpoint{3.637862in}{2.948766in}}%
\pgfpathlineto{\pgfqpoint{3.728048in}{2.921085in}}%
\pgfpathlineto{\pgfqpoint{3.830409in}{2.893404in}}%
\pgfpathlineto{\pgfqpoint{3.943299in}{2.865723in}}%
\pgfpathlineto{\pgfqpoint{4.064115in}{2.838043in}}%
\pgfpathlineto{\pgfqpoint{4.189247in}{2.810362in}}%
\pgfpathlineto{\pgfqpoint{4.314117in}{2.782681in}}%
\pgfpathlineto{\pgfqpoint{4.433332in}{2.755000in}}%
\pgfpathlineto{\pgfqpoint{4.540970in}{2.727319in}}%
\pgfpathlineto{\pgfqpoint{4.630983in}{2.699638in}}%
\pgfpathlineto{\pgfqpoint{4.697714in}{2.671957in}}%
\pgfpathlineto{\pgfqpoint{4.736456in}{2.644277in}}%
\pgfpathlineto{\pgfqpoint{4.744000in}{2.616596in}}%
\pgfpathlineto{\pgfqpoint{4.719067in}{2.588915in}}%
\pgfpathlineto{\pgfqpoint{4.662556in}{2.561234in}}%
\pgfpathlineto{\pgfqpoint{4.577540in}{2.533553in}}%
\pgfpathlineto{\pgfqpoint{4.469008in}{2.505872in}}%
\pgfpathlineto{\pgfqpoint{4.343361in}{2.478192in}}%
\pgfpathlineto{\pgfqpoint{4.207754in}{2.450511in}}%
\pgfpathlineto{\pgfqpoint{4.069379in}{2.422830in}}%
\pgfpathlineto{\pgfqpoint{3.934794in}{2.395149in}}%
\pgfpathlineto{\pgfqpoint{3.809398in}{2.367468in}}%
\pgfpathlineto{\pgfqpoint{3.697107in}{2.339787in}}%
\pgfpathlineto{\pgfqpoint{3.600255in}{2.312106in}}%
\pgfpathlineto{\pgfqpoint{3.519696in}{2.284426in}}%
\pgfpathlineto{\pgfqpoint{3.455061in}{2.256745in}}%
\pgfpathlineto{\pgfqpoint{3.405099in}{2.229064in}}%
\pgfpathlineto{\pgfqpoint{3.368047in}{2.201383in}}%
\pgfpathlineto{\pgfqpoint{3.341961in}{2.173702in}}%
\pgfpathlineto{\pgfqpoint{3.324989in}{2.146021in}}%
\pgfpathlineto{\pgfqpoint{3.315553in}{2.118340in}}%
\pgfpathlineto{\pgfqpoint{3.312454in}{2.090660in}}%
\pgfpathlineto{\pgfqpoint{3.314886in}{2.062979in}}%
\pgfpathlineto{\pgfqpoint{3.322400in}{2.035298in}}%
\pgfpathlineto{\pgfqpoint{3.334811in}{2.007617in}}%
\pgfpathlineto{\pgfqpoint{3.352067in}{1.979936in}}%
\pgfpathlineto{\pgfqpoint{3.374116in}{1.952255in}}%
\pgfpathlineto{\pgfqpoint{3.400748in}{1.924575in}}%
\pgfpathlineto{\pgfqpoint{3.431473in}{1.896894in}}%
\pgfpathlineto{\pgfqpoint{3.465422in}{1.869213in}}%
\pgfpathlineto{\pgfqpoint{3.501301in}{1.841532in}}%
\pgfpathlineto{\pgfqpoint{3.537416in}{1.813851in}}%
\pgfpathlineto{\pgfqpoint{3.571766in}{1.786170in}}%
\pgfpathlineto{\pgfqpoint{3.602205in}{1.758489in}}%
\pgfpathlineto{\pgfqpoint{3.626650in}{1.730809in}}%
\pgfpathlineto{\pgfqpoint{3.643312in}{1.703128in}}%
\pgfpathlineto{\pgfqpoint{3.650913in}{1.675447in}}%
\pgfpathlineto{\pgfqpoint{3.648839in}{1.647766in}}%
\pgfpathlineto{\pgfqpoint{3.637231in}{1.620085in}}%
\pgfpathlineto{\pgfqpoint{3.616963in}{1.592404in}}%
\pgfpathlineto{\pgfqpoint{3.589534in}{1.564723in}}%
\pgfpathlineto{\pgfqpoint{3.556887in}{1.537043in}}%
\pgfpathlineto{\pgfqpoint{3.521180in}{1.509362in}}%
\pgfpathlineto{\pgfqpoint{3.484549in}{1.481681in}}%
\pgfpathlineto{\pgfqpoint{3.448908in}{1.454000in}}%
\pgfpathlineto{\pgfqpoint{3.415793in}{1.426319in}}%
\pgfpathlineto{\pgfqpoint{3.386281in}{1.398638in}}%
\pgfpathlineto{\pgfqpoint{3.360978in}{1.370958in}}%
\pgfpathlineto{\pgfqpoint{3.340061in}{1.343277in}}%
\pgfpathlineto{\pgfqpoint{3.323362in}{1.315596in}}%
\pgfpathclose%
\pgfusepath{stroke,fill}%
}%
\begin{pgfscope}%
\pgfsys@transformshift{0.000000in}{0.000000in}%
\pgfsys@useobject{currentmarker}{}%
\end{pgfscope}%
\end{pgfscope}%
\begin{pgfscope}%
\pgfpathrectangle{\pgfqpoint{0.800000in}{0.528000in}}{\pgfqpoint{4.960000in}{3.696000in}}%
\pgfusepath{clip}%
\pgfsetbuttcap%
\pgfsetmiterjoin%
\definecolor{currentfill}{rgb}{0.347059,0.458824,0.641176}%
\pgfsetfillcolor{currentfill}%
\pgfsetlinewidth{0.752812pt}%
\definecolor{currentstroke}{rgb}{0.298039,0.298039,0.298039}%
\pgfsetstrokecolor{currentstroke}%
\pgfsetdash{}{0pt}%
\pgfpathmoveto{\pgfqpoint{3.280000in}{-1.258990in}}%
\pgfpathlineto{\pgfqpoint{3.280000in}{-1.258990in}}%
\pgfpathlineto{\pgfqpoint{3.280000in}{-1.258990in}}%
\pgfpathlineto{\pgfqpoint{3.280000in}{-1.258990in}}%
\pgfpathclose%
\pgfusepath{stroke,fill}%
\end{pgfscope}%
\begin{pgfscope}%
\pgfpathrectangle{\pgfqpoint{0.800000in}{0.528000in}}{\pgfqpoint{4.960000in}{3.696000in}}%
\pgfusepath{clip}%
\pgfsetbuttcap%
\pgfsetmiterjoin%
\definecolor{currentfill}{rgb}{0.798529,0.536765,0.389706}%
\pgfsetfillcolor{currentfill}%
\pgfsetlinewidth{0.752812pt}%
\definecolor{currentstroke}{rgb}{0.298039,0.298039,0.298039}%
\pgfsetstrokecolor{currentstroke}%
\pgfsetdash{}{0pt}%
\pgfpathmoveto{\pgfqpoint{3.280000in}{-1.258990in}}%
\pgfpathlineto{\pgfqpoint{3.280000in}{-1.258990in}}%
\pgfpathlineto{\pgfqpoint{3.280000in}{-1.258990in}}%
\pgfpathlineto{\pgfqpoint{3.280000in}{-1.258990in}}%
\pgfpathclose%
\pgfusepath{stroke,fill}%
\end{pgfscope}%
\begin{pgfscope}%
\pgfpathrectangle{\pgfqpoint{0.800000in}{0.528000in}}{\pgfqpoint{4.960000in}{3.696000in}}%
\pgfusepath{clip}%
\pgfsetroundcap%
\pgfsetroundjoin%
\pgfsetlinewidth{1.505625pt}%
\definecolor{currentstroke}{rgb}{0.298039,0.298039,0.298039}%
\pgfsetstrokecolor{currentstroke}%
\pgfsetdash{}{0pt}%
\pgfpathmoveto{\pgfqpoint{3.280000in}{0.831023in}}%
\pgfpathlineto{\pgfqpoint{3.280000in}{3.722306in}}%
\pgfusepath{stroke}%
\end{pgfscope}%
\begin{pgfscope}%
\pgfpathrectangle{\pgfqpoint{0.800000in}{0.528000in}}{\pgfqpoint{4.960000in}{3.696000in}}%
\pgfusepath{clip}%
\pgfsetroundcap%
\pgfsetroundjoin%
\pgfsetlinewidth{4.516875pt}%
\definecolor{currentstroke}{rgb}{0.298039,0.298039,0.298039}%
\pgfsetstrokecolor{currentstroke}%
\pgfsetdash{}{0pt}%
\pgfpathmoveto{\pgfqpoint{3.280000in}{1.736905in}}%
\pgfpathlineto{\pgfqpoint{3.280000in}{2.590660in}}%
\pgfusepath{stroke}%
\end{pgfscope}%
\begin{pgfscope}%
\pgfsetrectcap%
\pgfsetmiterjoin%
\pgfsetlinewidth{1.254687pt}%
\definecolor{currentstroke}{rgb}{0.800000,0.800000,0.800000}%
\pgfsetstrokecolor{currentstroke}%
\pgfsetdash{}{0pt}%
\pgfpathmoveto{\pgfqpoint{0.800000in}{0.528000in}}%
\pgfpathlineto{\pgfqpoint{0.800000in}{4.224000in}}%
\pgfusepath{stroke}%
\end{pgfscope}%
\begin{pgfscope}%
\pgfsetrectcap%
\pgfsetmiterjoin%
\pgfsetlinewidth{1.254687pt}%
\definecolor{currentstroke}{rgb}{0.800000,0.800000,0.800000}%
\pgfsetstrokecolor{currentstroke}%
\pgfsetdash{}{0pt}%
\pgfpathmoveto{\pgfqpoint{5.760000in}{0.528000in}}%
\pgfpathlineto{\pgfqpoint{5.760000in}{4.224000in}}%
\pgfusepath{stroke}%
\end{pgfscope}%
\begin{pgfscope}%
\pgfsetrectcap%
\pgfsetmiterjoin%
\pgfsetlinewidth{1.254687pt}%
\definecolor{currentstroke}{rgb}{0.800000,0.800000,0.800000}%
\pgfsetstrokecolor{currentstroke}%
\pgfsetdash{}{0pt}%
\pgfpathmoveto{\pgfqpoint{0.800000in}{0.528000in}}%
\pgfpathlineto{\pgfqpoint{5.760000in}{0.528000in}}%
\pgfusepath{stroke}%
\end{pgfscope}%
\begin{pgfscope}%
\pgfsetrectcap%
\pgfsetmiterjoin%
\pgfsetlinewidth{1.254687pt}%
\definecolor{currentstroke}{rgb}{0.800000,0.800000,0.800000}%
\pgfsetstrokecolor{currentstroke}%
\pgfsetdash{}{0pt}%
\pgfpathmoveto{\pgfqpoint{0.800000in}{4.224000in}}%
\pgfpathlineto{\pgfqpoint{5.760000in}{4.224000in}}%
\pgfusepath{stroke}%
\end{pgfscope}%
\begin{pgfscope}%
\pgfpathrectangle{\pgfqpoint{0.800000in}{0.528000in}}{\pgfqpoint{4.960000in}{3.696000in}}%
\pgfusepath{clip}%
\pgfsetbuttcap%
\pgfsetroundjoin%
\definecolor{currentfill}{rgb}{1.000000,1.000000,1.000000}%
\pgfsetfillcolor{currentfill}%
\pgfsetlinewidth{1.003750pt}%
\definecolor{currentstroke}{rgb}{0.298039,0.298039,0.298039}%
\pgfsetstrokecolor{currentstroke}%
\pgfsetdash{}{0pt}%
\pgfsys@defobject{currentmarker}{\pgfqpoint{-0.020833in}{-0.020833in}}{\pgfqpoint{0.020833in}{0.020833in}}{%
\pgfpathmoveto{\pgfqpoint{0.000000in}{-0.020833in}}%
\pgfpathcurveto{\pgfqpoint{0.005525in}{-0.020833in}}{\pgfqpoint{0.010825in}{-0.018638in}}{\pgfqpoint{0.014731in}{-0.014731in}}%
\pgfpathcurveto{\pgfqpoint{0.018638in}{-0.010825in}}{\pgfqpoint{0.020833in}{-0.005525in}}{\pgfqpoint{0.020833in}{0.000000in}}%
\pgfpathcurveto{\pgfqpoint{0.020833in}{0.005525in}}{\pgfqpoint{0.018638in}{0.010825in}}{\pgfqpoint{0.014731in}{0.014731in}}%
\pgfpathcurveto{\pgfqpoint{0.010825in}{0.018638in}}{\pgfqpoint{0.005525in}{0.020833in}}{\pgfqpoint{0.000000in}{0.020833in}}%
\pgfpathcurveto{\pgfqpoint{-0.005525in}{0.020833in}}{\pgfqpoint{-0.010825in}{0.018638in}}{\pgfqpoint{-0.014731in}{0.014731in}}%
\pgfpathcurveto{\pgfqpoint{-0.018638in}{0.010825in}}{\pgfqpoint{-0.020833in}{0.005525in}}{\pgfqpoint{-0.020833in}{0.000000in}}%
\pgfpathcurveto{\pgfqpoint{-0.020833in}{-0.005525in}}{\pgfqpoint{-0.018638in}{-0.010825in}}{\pgfqpoint{-0.014731in}{-0.014731in}}%
\pgfpathcurveto{\pgfqpoint{-0.010825in}{-0.018638in}}{\pgfqpoint{-0.005525in}{-0.020833in}}{\pgfqpoint{0.000000in}{-0.020833in}}%
\pgfpathclose%
\pgfusepath{stroke,fill}%
}%
\begin{pgfscope}%
\pgfsys@transformshift{3.280000in}{1.737203in}%
\pgfsys@useobject{currentmarker}{}%
\end{pgfscope}%
\end{pgfscope}%
\end{pgfpicture}%
\makeatother%
\endgroup%

                        \end{adjustbox}
                            \caption{Tasks chain end-to-end rWCRT distribution with stress-ng}
                            \label{fig:Taskchain_stress_phase4}
                        \end{figure}
                        }
                        
            \subsubsection{Chaine de tâche en isolation}
                        For step~\circleTxt[5], we execute the task chain in isolation (i.e. degraded mode). Execution time profile is on the left part (blue) of \autoref{fig:taskchain_with_real_tasks_as_perturbation}.
                        We calibrate the Monitor \& Control mechanism parameters. We need the different $rWCRT$s for each value of $\tau_i$ as defined in \autoref{def:chainState}. For such linear 5-task chain we logically have $i \in \{1,5\}$. At run-time, the remaining response times are logged in degraded mode, i.e. the task chain in isolation, and we keep an upper value of the worst measured remaining response time for each $\tau_i$ as its $rWCRT(\tau_i)$ in \autoref{tab:rWCRT_i}.         Finally, regarding previous results from step~\circleTxt[3], we set $W_{max} = 1$ms, and $t_{sw} = 500\mu$s for our platform.
                
                        \begin{table}[ht]
                            \renewcommand{\arraystretch}{1}
                            \centering
                            \caption{Task Chain $rWCRT(\tau_i)$ values in degraded mode}
                            \label{tab:rWCRT_i}
                            \begin{tabular}{@{}lrrrrr@{}}
                            \toprule
                             $ rWCRT$ & $\tau_0$ & $\tau_1$ & $\tau_2$ & $\tau_3$  & $\tau_4$\\
                             \midrule
                                time (ms)         & 129     &    93   &     68  &  49.5    &   25 \\
                             \bottomrule
                            \end{tabular}
                        \end{table}
        
            \subsubsection{Chaine de tâche avec mécanisme de Contrôle}
                    \begin{figure}[ht]
                        \begin{adjustbox}{clip,trim=0.3cm 0.18cm 1.2cm 1cm,max width=\linewidth}   % TRIM~: left, bottom, right, top    
                        \input{graphiques/Final_legende.pgf}
                        \end{adjustbox}
                        \caption{Task Chain response time profile from steps \circleTxt[3], \circleTxt[6], \circleTxt[7]}
                        \label{fig:chainResponseProfiles}
                    \end{figure}
                        With the previous calibration, we can execute the task chain alone with the Control mechanism enabled. In this isolation case, we should see almost no switch to degraded mode (and on a perfect case, no switches at all) as they must be false-positive. 
                        This experiment allows to validate the parameters set on the previous step.
                        On our tests, we measured 0.3\% of false positive triggers to degraded mode. The task chain in degraded mode response time distribution profile is illustrated in \autoref{fig:chainResponseProfiles}.
            
        \subsection{Phase de Validation en exécution}
            \subsubsection{Chaine de tâches avec système complet et mécanisme de Contrôle}
                    As a final experiment, we test the complete workload (HI and LO tasks) with the Monitoring \& Control Agent enabled and configured from previous step. 
                    First we observe the MCA CPU use, that is inferior to 1\%. For a 120s long experiment, it ran for 1.3s overall (including setup time). We were not able to find any difference regarding CPU percentage use with and without our mechanism, either with a big task sets (small tasks only, CPU usage around 80\% displayed) and with smaller task sets (e.g. only the task chain described above). Such footprint is low enough to include easily such mechanism. 
     
                    In term of \textbf{efficiency}, our MCA prevented every task chain execution over a 170ms response time. Only 6 occurrences (0.1\%) missed the deadline set at 160ms. \cmnt{while without the Control enabled, we measured in step \circleTxt[3] 84\% of deadline misses.} The MCA brought down the average response time of the chain from 168ms (no Control enabled) to 129ms. Such value is way closer to the average task response time profile in isolation (125ms). The few missed deadlines can be explained by the implementation framework we used, with a workload (MiBench tasks) not fully compliant with real-time programming constructs recommendations that causes uncontrolled linux system calls for instance. In conjunction with the exacting deadline we arbitrarily set at 160ms while the general workload is demanding (generating 84\% deadline misses without the MCA in step \circleTxt[3]), this explains this non-perfect result. We could use more pessimistic $rWCRT(\tau_i)$ values to achieve no deadline misses, at the expense of a worse result on the quality criteria. By the end it is a question of compromise, depending on the specific needs.
                    %%%% On peut jouer sur les réglages du système pour améliorer/limiter le nombre de deadline misses mais ce sera au prix de l'exécution des LO tasks. 
                    %%%% Ces deadline misses s'expliquent par l'environnement sur lequel les tests ont été faits et le nvieau d'exigence de la deadline fixée.
                 
                    The \textbf{quality} of our calibration seems promising as there were less switches to degraded mode with the Control enabled than the number of deadline misses with no Control at all. This implies that preventing a deadline miss had a more general impact reducing the overall number of timing faults.
                 
                    In term of \textbf{performance}, the system maintained LO-criticality mode for 82s / 120s total, i.e. a performance factor of 0.69 for a loss of 31\% of the time in degraded mode.
                 
                    All those metrics are promising for the use of a Monitoring and Control Agent in order to change a chain response time at an optimum value to avoid the great majority of the deadline misses and on the same time still take few compromises on the LO-criticality tasks execution.
                 %No control~: 310/368 deadline misses. Average~: 168ms
                 %Isolation~: 3 false-positives. Average~: 126ms
                 %Control~: 6/215/493. Average~: 129ms
            
               %• deadline misses for efficiency;
               %• the CPU time given to LO-criticality tasks, compared to the CPU time given with no guarantee on the task chain from step3©, for performance;
               %• switches  to  degraded  mode,  compared  to  the  actual amount of deadline misses with no guarantees from the MCA, obtained in step 3, for quality
               
    \section{Conclusions expérimentales}  
\ifdefined\included
\else
\bibliographystyle{StyleThese}
\bibliography{these}
\end{document}
\fi