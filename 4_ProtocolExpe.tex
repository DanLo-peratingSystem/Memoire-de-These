% !TeX root = These.tex
\ifdefined\included
\else
\documentclass[french, a4paper, 11pt, twoside, pdftex]{StyleThese}
\usepackage{iflang}
\usepackage{bibentry}



%\usepackage[sectionbib]{chapterbib}          % Cross-reference package (Natural BiB)
%\usepackage{natbib}                  % Put References at the end of each chapter
%\usepackage{bibunits}
% Do not put 'sectionbib' option here.
% Sectionbib option in 'natbib' will do.


\usepackage{fancyhdr}                    % Fancy Header and Footer

\usepackage[utf8]{inputenc}
\usepackage[T1]{fontenc}
\usepackage[french]{babel} %
\usepackage{lmodern} \normalfont %to load T1lmr.fd 
\DeclareFontShape{T1}{cmr}{b}{sc} { <-> ssub * cmr/bx/sc }{}
%\hyphenation{gar}

\usepackage{amsmath,amssymb}             % AMS Math
\usepackage{nicefrac}
\usepackage{siunitx}					%% Unites Math SI

\usepackage{blindtext}

\usepackage{datetime}

\usepackage{lipsum} 

\usepackage[inline]{enumitem}

\usepackage{hhline}
%\usepackage[left=1.5in,right=1.3in,top=1.1in,bottom=1.1in]{geometry}
\usepackage[left=1.5in,right=1.3in,top=1.1in,bottom=1.1in,includefoot,includehead,headheight=13.6pt]{geometry}

%%\renewcommand{\baselinestretch}{1.05}

%%%%%%%% Multi-figures avec sub-captions
\usepackage{caption}
\usepackage{subcaption}

% Table of contents for each chapter

\usepackage[nottoc, notlof, notlot]{tocbibind}
\usepackage[nohints]{minitoc}
\setcounter{minitocdepth}{2}
\mtcindent=15pt
% Use \minitoc where to put a table of contents

\usepackage{aecompl}

%% Package cosmetic meilleur layout du texte en jouant sur le spacing par caractères
\usepackage[activate={true,nocompatibility},final,tracking=true,kerning=true,factor=1100,stretch=10,shrink=10]{microtype}
\usepackage[absolute,overlay]{textpos} 
\setlength{\TPHorizModule}{\paperwidth}\setlength{\TPVertModule}{\paperheight}
\sloppy

%%%%%%%%%%% JOLIS TABLEAUX
\usepackage{tabularx}		%\usepackage{tabular}
\usepackage{multirow}
\newcommand{\mc}{\multicolumn} 
\newcommand{\mr}[2]{\multirow{#1}{*}{#2}} 	\newcommand{\mrQ}{\multirow{-4}{*}}
\usepackage{booktabs}

\usepackage[usenames,dvipsnames]{xcolor} 

\makeatletter
\newcommand{\ccolor}[3][]{%
	\kern-\fboxsep
	\if\relax\detokenize{#1}\relax
	\expandafter\@firstoftwo
	\else
	\expandafter\@secondoftwo
	\fi
	{\colorbox{#2}}%
	{\colorbox[#1]{#2}}%
	{#3}\kern-\fboxsep
}
\makeatother

%%%%% Insertion graphiques format PGF
\usepackage{pgfplots}
\pgfplotsset{width=\linewidth, compat=1.16}%, compat=1.17}
\usepackage{adjustbox}          %%% PERMET DE LES RECADRER + FACILEMENT


%%%%%%%%%% Bullets de listes sans saut de ligne %%%%%%%%%%
\usepackage{xparse}

\ExplSyntaxOn%
\seq_new:N \l_local_enum_seq

\newcommand{\storethestuff}[1]{%
  \seq_set_from_clist:Nn \l_local_enum_seq {#1}%
}

\newcommand{\dotheenumstuff}{%
\int_zero:N \l_tmpa_int
\seq_map_inline:Nn \l_local_enum_seq {%
    \int_incr:N \l_tmpa_int% Increase the counter
    \item ##1
    % Check whether the list has reached the end -- if so, use '.' instead of ','
    %\int_compare:nNnTF 
    % { \l_tmpa_int } < {\seq_count:N \l_local_enum_seq} 
    % {,} {.}
  }
}
\ExplSyntaxOff

\NewDocumentCommand{\linebullets}{+m}{%
  \storethestuff{#1}%
  \begin{enumerate*}[label={\alph*)},font={\bfseries},itemjoin={{, }}]
    \dotheenumstuff%
  \end{enumerate*}
}

\newcommand{\cmnt}[1]{}  %%%%% AJOUT DE COMMENTAIRE MULTILIGNES


%%%%%%%%%% ECRITURE CARACTERES DANS UN CERCLE %%%%%%%%%%
%\def\circleTxt[#1]{\raisebox{.5pt}{\textcircled{\raisebox{-1pt}{#1}}}}
\newcommand{\ctxt}[1]{\raisebox{.5pt}{\textcircled{\raisebox{-1.2pt}{#1}}}}
% Glossary / list of abbreviations

\usepackage[intoc]{nomencl}
\IfLanguageName{english}{%
\renewcommand{\nomname}{Glossary}
}{ %
\renewcommand{\nomname}{Liste des Abréviations}
}

\makenomenclature

% My pdf code

\usepackage{ifpdf}

\ifpdf
  \usepackage[pdftex]{graphicx}
  \DeclareGraphicsExtensions{.pdf,PDF,.png,PNG,.jpg,JPG}
  \usepackage[pagebackref,hyperindex=true]{hyperref} %% use \autoref{} instead of Table~\ref{}.
  \usepackage{tikz}
  \usetikzlibrary{arrows,shapes,calc}
\else
  \usepackage{graphicx}
  \DeclareGraphicsExtensions{.ps,.eps}
  \usepackage[a4paper,dvipdfm,pagebackref,hyperindex=true]{hyperref}
\fi

\graphicspath{{.}{schemas/}{graphiques/}{tables/}}

%% nicer backref links. NOTE: The flag ThesisInEnglish is used to define the
% language in the back references. Read more about it in These.tex

\IfLanguageName{english}{
\renewcommand*{\backref}[1]{}
\renewcommand*{\backrefalt}[4]{%
\ifcase #1 %
(Not cited.)%
\or
(Cited in page~#2.)%
\else
(Cited in pages~#2.)%
\fi}
\renewcommand*{\backrefsep}{, }
\renewcommand*{\backreftwosep}{ and~}
\renewcommand*{\backreflastsep}{ and~}
}{
\renewcommand*{\backref}[1]{}
\renewcommand*{\backrefalt}[4]{%
\ifcase #1 %
(Non cité.)%
\or
(Cité en page~#2.)%
\else
(Cité en pages~#2.)%
\fi}
\renewcommand*{\backrefsep}{, }
\renewcommand*{\backreftwosep}{ et~}
\renewcommand*{\backreflastsep}{ et~}
}

% Links in pdf
\usepackage{color}
\definecolor{linkcol}{rgb}{0,0,0.4} 
\definecolor{citecol}{rgb}{0.5,0,0} 
\definecolor{linkcol}{rgb}{0,0,0} 
\definecolor{citecol}{rgb}{0,0,0}
% Change this to change the informations included in the pdf file

\hypersetup
{
bookmarksopen=true,
pdftitle="Prévention des fautes temporelles sur architectures multicœur pour les systèmes à criticité mixte",
pdfauthor="Daniel LOCHE", %auteur du document
pdfsubject="Thèse", %sujet du document
%pdftoolbar=false, %barre d'outils non visible
pdfmenubar=true, %barre de menu visible
pdfhighlight=/O, %effet d'un clic sur un lien hypertexte
colorlinks=true, %couleurs sur les liens hypertextes
pdfpagemode=UseNone, %aucun mode de page
%pdfpagelayout=DoublePage, %ouverture en simple page
pdffitwindow=true, %pages ouvertes entierement dans toute la fenetre
linkcolor=linkcol, %couleur des liens hypertextes internes
citecolor=citecol, %couleur des liens pour les citations
urlcolor=linkcol %couleur des liens pour les url
}

% definitions.
% -------------------

\setcounter{secnumdepth}{3}
\setcounter{tocdepth}{2}

% Some useful commands and shortcut for maths:  partial derivative and stuff

\newcommand{\pd}[2]{\frac{\partial #1}{\partial #2}}
\def\abs{\operatorname{abs}}
\def\argmax{\operatornamewithlimits{arg\,max}}
\def\argmin{\operatornamewithlimits{arg\,min}}
\def\diag{\operatorname{Diag}}
\newcommand{\eqRef}[1]{(\ref{#1})}
\newcommand{\nline}{\smallbreak\noindent}

\usepackage{rotating}                    % Sideways of figures & tables

% \usepackage{txfonts}                     % Public Times New Roman text & math font
  
%%% Fancy Header %%%%%%%%%%%%%%%%%%%%%%%%%%%%%%%%%%%%%%%%%%%%%%%%%%%%%%%%%%%%%%%%%%
% Fancy Header Style Options

\pagestyle{fancy}                       % Sets fancy header and footer
\fancyfoot{}                            % Delete current footer settings

%\renewcommand{\chaptermark}[1]{         % Lower Case Chapter marker style
%  \markboth{\chaptername\ \thechapter.\ #1}}{}} %

%\renewcommand{\sectionmark}[1]{         % Lower case Section marker style
%  \markright{\thesection.\ #1}}         %

\fancyhead[LE,RO]{\bfseries\thepage}    % Page number (boldface) in left on even
% pages and right on odd pages
\fancyhead[RE]{\bfseries\nouppercase{\leftmark}}      % Chapter in the right on even pages
\fancyhead[LO]{\bfseries\nouppercase{\rightmark}}     % Section in the left on odd pages

\let\headruleORIG\headrule
\renewcommand{\headrule}{\color{black} \headruleORIG}
\renewcommand{\headrulewidth}{1.0pt}
\usepackage{colortbl}
\arrayrulecolor{black}

\fancypagestyle{plain}{
  \fancyhead{}
  \fancyfoot{}
  \renewcommand{\headrulewidth}{0pt} %%%%%%%%%%%%%%%%%%%%%%%%%%%%%%%%%%%%%%%%%%%%%%%%%%%%%%%%%%%%%%%%%%%%%%%%%%%%%%%%%%%%%
}

%\usepackage{MyAlgorithm}
%\usepackage[noend]{MyAlgorithmic}
%\usepackage[ED=EDSYS-SystEmb, Ets=INP]{tlsflyleaf}

%%% Clear Header %%%%%%%%%%%%%%%%%%%%%%%%%%%%%%%%%%%%%%%%%%%%%%%%%%%%%%%%%%%%%%%%%%
% Clear Header Style on the Last Empty Odd pages
\makeatletter

\def\cleardoublepage{\clearpage\if@twoside \ifodd\c@page\else%
  \hbox{}%
  \thispagestyle{empty}%              % Empty header styles
  \newpage%
  \if@twocolumn\hbox{}\newpage\fi\fi\fi}

\makeatother
 
%%%%%%%%%%%%%%%%%%%%%%%%%%%%%%%%%%%%%%%%%%%%%%%%%%%%%%%%%%%%%%%%%%%%%%%%%%%%%%% 
% Prints your review date and 'Draft Version' (From Josullvn, CS, CMU)
\newcommand{\reviewtimetoday}[2]{\special{!userdict begin
    /bop-hook{gsave 20 710 translate 45 rotate 0.8 setgray
      /Times-Roman findfont 12 scalefont setfont 0 0   moveto (#1) show
      0 -12 moveto (#2) show grestore}def end}}
% You can turn on or off this option.
% \reviewtimetoday{\today}{Draft Version}
%%%%%%%%%%%%%%%%%%%%%%%%%%%%%%%%%%%%%%%%%%%%%%%%%%%%%%%%%%%%%%%%%%%%%%%%%%%%%%% 

\newenvironment{maxime}[1]
{
	\def\Arg{#1}
\vspace*{0cm}
\hfill
\begin{minipage}{0.6\textwidth}%
%\rule[0.5ex]{\textwidth}{0.1mm}\\%
\hrulefill $\:$ \\%$\:$ {\bf #1}\\
%\vspace*{-0.25cm}
\it 
}%
{%
	
\hrulefill $\:$ {\bf \Arg}
\vspace*{0.5cm}%
\end{minipage}
}

\let\minitocORIG\minitoc
\renewcommand{\minitoc}{\minitocORIG \vspace{1.5em}}

%\usepackage{slashbox}

\newenvironment{bulletList}%
{ \begin{list}%
	{$\bullet$}%
	{\setlength{\labelwidth}{25pt}%
	 \setlength{\leftmargin}{30pt}%
	 \setlength{\itemsep}{\parsep}}}%
{ \end{list} }


%%%%%%% Outils pour \comment \alert \add %%%%%
\usepackage{easyReview}
\usepackage{soulutf8} % for accented letters

\let\newalert\alert
\renewcommand{\alert}[1]{\textit{\newalert{#1}}}

%\usepackage[commandnameprefix=ifneeded]{changes} %% \chhighlight and \chcomment to avoid collision with easyReview
\renewcommand{\epsilon}{\varepsilon}

% centered page environment

\newenvironment{vcenterpage}
{\newpage\vspace*{\fill}\thispagestyle{empty}\renewcommand{\headrulewidth}{0pt}}
{\vspace*{\fill}}

\usepackage{tablefootnote}

%%%%%% MISE EN FORME CADRES DEFINITIONS/THEOREMES/LEMES %%%%%%%%%%
\usepackage{amsthm}  % for theoremstyle

\theoremstyle{plain} 
\newtheorem{theorem}{Théorème}[section]
\newtheorem{corollary}{Corolaire}[theorem]

%\theoremstyle{lemma}
%\newtheorem{lemma}[theorem]{Lemme}


\theoremstyle{definition}
\newtheorem{definition}[theorem]{Définition}


\cmnt{
	\usepackage{ntheorem} %\usepackage{amsthm}  % for theoremstyle
	%\usepackage{mdframed}
	\usepackage[most]{tcolorbox}
	
	\theoremstyle{plain} 
	\theoremindent20pt
	\theoremheaderfont{\normalfont\bfseries\hspace{-\theoremindent}}
	\newtheorem{theorem}{Théorème}[section]
	\newtheorem{corollary}{Corolaire}[theorem]
	
	\theoremstyle{plain}
	\newtheorem{lemma}[theorem]{Lemme}
	
	
	\tcolorboxenvironment{theorem}{
		blanker,
		breakable,
		before skip=\topsep,
		after skip=\topsep,
		borderline west={1pt}{10pt}{double, shorten <=12pt}
	}
	
	\theorembodyfont{\normalfont}
	\theoremindent20pt
	\theoremheaderfont{\normalfont\bfseries\hspace{-\theoremindent}}
	\newtheorem{definition}[theorem]{Définition}
	
	
	\tcolorboxenvironment{definition}{
		blanker,
		breakable,
		before skip=\topsep,
		after skip=\topsep,
		borderline west={1pt}{10pt}{shorten <=12pt}
	}
}

\cmnt{ 
	\begin{theorem}
		Ceci est un Théorème.
	\end{theorem} 
	
	\begin{corollary}
		Ceci est un Corollaire.
	\end{corollary}
	
	\begin{definition}
		Ceci est une Définition.
	\end{definition}
	
	\begin{lemma}
		Ceci est un Lemme.
	\end{lemma}
}

\def\UrlBigBreaks{\do\/\do-\do:}
\usepackage{url}

\sloppy
\begin{document}
\setcounter{chapter}{4} %% Numéro du chapitre
\dominitoc
\faketableofcontents
\fi

\setlength{\belowcaptionskip}{-6pt}  %% COMMANDE POSSIBLE POUR POUVOIR BIDOUILLER L'ESPACEMENT SOUS LA LÉGENDE D'UNE FIGURE
\chapter{CHAPITRE 4} \label{chap:4_ProtocolExpe}
\minitoc

%%%% BON JE PENSE QUE VOUS AVEZ COMPRIS A CE MOMENT.
\section{section name}

	\subsection{subsection name}
	\subsubsection{subsubsection name}



Juste pour le plaisir, ceci est un exemple de tableau, qui dépasse sur 2 pages différentes mais rend bien quand même : l'entête du tableau se répète sur la nouvelle page.
Notez qu'il s'écrit pas tout à fait comme un tableau classique.

\setlength{\tabcolsep}{4pt}
\begin{longtable}{@{}llll@{}}
%	\centering
	\caption{Tâches MiBench}\label{tab:mibench_tasks} \\
%	\begin{tabular}{@{}llll@{}}
		\toprule
		Nom          & Description                                  & Type d'entrée      & Type de sortie         \\ \midrule
	\endfirsthead
		\multicolumn{4}{c}{\tablename\ \thetable\ -- Tâches MiBench (suite)}\\[1ex]
		\toprule
		Nom          & Description                                  & Type d'entrée      & Type de sortie         \\ \midrule
	\endhead
		\bottomrule
	\endfoot
		basicmath	 & calculs scientifiques				& /					 & données (dec.)	\\
		bitcount     & comptage de bits	vers entiers        & texte ASCII        & texte ASCII      \\
		qsort        & algorithme de tri                    & texte ASCII        & texte ASCII      \\
		susan c      & reconnaissance de coins              & image (.pgm)       & image (.pgm)     \\
		susan e      & reconnaissance de bords              & image (.pgm)       & image (.pgm)     \\
		susan s      & lissage d'image (réduction de bruit) & image (.pgm)       & image (.pgm)     \\
		jpeg c       & encodeur JPEG                        & image (.ppm)       & image (.jpeg)    \\
		jpeg d       & décodeur JPEG                        & image (.jpeg)      & image (.ppm)     \\
		lame         & encodeur MP3                         & audio (.wave)      & audio (.mp3)     \\
		mad          & décodeur audio MP3                   & audio (.mp3)       & audio (.wave)    \\
		tiff2bw      & conversion en noir et blanc          & image (.tiff)      & image (.tiff)    \\
		tiff2rgba    & conversion couleurs RGB en TIFF		& image (.tiff)      & image (.tiff)    \\
		tiffdither   & tramage noir et blanc (dithering)    & image (.tiff)      & image (.tiff)    \\
		tiffmedian   & réduction de plage de couleur      	& image (.tiff)      & image (.tiff)    \\
		dijkstra     & recherche de plus court chemin		& texte ASCII        & texte ASCII      \\
		patricia     & recherche sur arbre PATRICIA			& texte ASCII        & texte ASCII      \\
		ghostscript  & interpréteur PostScript              & PostScript		 & image (.ppm)     \\
		ispell       & Vérificateur orthographique          & texte              & texte ASCII		\\
		rsynth       & synthèse vocale de texte             & texte              & audio (.AU)      \\
		stringsearch & recherche de mot dans un texte		& texte              & texte            \\
		blowfish d   & déchiffrement blowfish               & données (bin.)  	 & données (bin.)   \\
		blowfish e   & chiffrement blowfish                 & texte              & données (bin.)   \\
		pgp d        & déchiffrement asymétrique      		& données (bin.)  	 & texte            \\
		pgp e        & chiffrement asymétrique				& texte              & données (bin.)   \\
		rijndael d   & déchiffrement AES                    & données (bin.)  	 & texte            \\
		rijndael e   & chiffrement AES                      & texte              & données (bin.)   \\
		sha          & algorithme de calcul de hash			& texte              & texte ASCII      \\
		adpcm c      & encodeur de PWM                      & audio (.wave)      & données (bin.)   \\
		susan e      & reconnaissance de bords              & image (.pgm)       & image (.pgm)     \\
		susan s      & lissage d'image (réduction de bruit) & image (.pgm)       & image (.pgm)     \\
		jpeg c       & encodeur JPEG                        & image (.ppm)       & image (.jpeg)    \\
		jpeg d       & décodeur JPEG                        & image (.jpeg)      & image (.ppm)     \\
		lame         & encodeur MP3                         & audio (.wave)      & audio (.mp3)     \\
		mad          & décodeur audio MP3                   & audio (.mp3)       & audio (.wave)    \\
		tiff2bw      & conversion en noir et blanc          & image (.tiff)      & image (.tiff)    \\
		tiff2rgba    & conversion couleurs RGB en TIFF		& image (.tiff)      & image (.tiff)    \\
		adpcm d      & décodeur de PWM                      & données (bin.)  	 & données (bin.)   \\
		CRC32        & somme de contrôle 32 bits     		& audio (.wave)      & texte ASCII      \\
  	fft (fft\up{-1}) & transformée de fourrier (/inverse)	& signal (sinus)	 & signal (sinus)	\\
		gsm toast    & encodage GSM                         & audio (.AU)        & données (bin.)   \\
		gsm untoast  & decodage GSM                         & données (bin.)     & audio (.AU)	    \\
		tiffmedian   & réduction de plage de couleur      	& image (.tiff)      & image (.tiff)    \\
		dijkstra     & recherche de plus court chemin		& texte ASCII        & texte ASCII      \\
		patricia     & recherche sur arbre PATRICIA			& texte ASCII        & texte ASCII      \\
		ghostscript  & interpréteur PostScript              & PostScript		 & image (.ppm)     \\
		ispell       & Vérificateur orthographique          & texte              & texte ASCII		\\
		rsynth       & synthèse vocale de texte             & texte              & audio (.AU)      \\
		stringsearch & recherche de mot dans un texte		& texte              & texte            \\
		blowfish d   & déchiffrement blowfish               & données (bin.)  	 & données (bin.)   \\
		blowfish e   & chiffrement blowfish                 & texte              & données (bin.)   \\
		pgp d        & déchiffrement asymétrique      		& données (bin.)  	 & texte            \\
		pgp e        & chiffrement asymétrique				& texte              & données (bin.)   \\
		rijndael d   & déchiffrement AES                    & données (bin.)  	 & texte            \\
		rijndael e   & chiffrement AES                      & texte              & données (bin.)   \\
%		\bottomrule
%	\end{tabular}
\end{longtable}

	Pour finir, ceci est un exemple d'extrait de code, c'est sympa pour avoir de la coloration et mise en valeur !
	Voir configuration ici utilisée, qui est dans formatAndDefs.tex !
	\begin{lstlisting}[caption={Fichier d'entrée pour une chaine de tâche}, label={code:task.in}]
ID name         rWCRT    T   P  A prec FUNC     ARGS
1  fft_S       129000   40  10  1   0  fft      8 2048 [...]
2  bitcount_S   93000   60  10  1   1  bitcnts  75000 [...]
3  basicmath_S  68000   40  10  1   2  bmath_S  > ./out/[...]
4  sha_S        49000   60  10  1   3  sha      < ./dat/[...]
5  fft_inv_S    25000   40  10  1   4  fft_i    8 4096 >[...]
6  gsmUToast_S  94000  100  10  1   5  utoast   -dfps -c [...]
7  fft_S        67500  100  10  1   6  fft      8 16384 >[...]
8  patricia_L       0  100  10  1   7  patricia < ./dat/[...]
// T=Period | P=Priority | A=Affinity | prec=Precedency
// rWCRT in us ; T in ms ; A in hexadecimal
\end{lstlisting} %//  prec = 0 : Tache entrante | rWCRT = 0 : Tache de sortie.


	Exemple d'insertion d'un graphique au format .pgf (un peu l'équivalent du vectoriel/pdf, mais dans un format où toutes les données du graph sont stockées sans perte)
\begin{figure}[ht]
	\centering
	\scalebox{0.9}{%% Creator: Matplotlib, PGF backend
%%
%% To include the figure in your LaTeX document, write
%%   \input{<filename>.pgf}
%%
%% Make sure the required packages are loaded in your preamble
%%   \usepackage{pgf}
%%
%% and, on pdftex
%%   \usepackage[utf8]{inputenc}\DeclareUnicodeCharacter{2212}{-}
%%
%% or, on luatex and xetex
%%   \usepackage{unicode-math}
%%
%% Figures using additional raster images can only be included by \input if
%% they are in the same directory as the main LaTeX file. For loading figures
%% from other directories you can use the `import` package
%%   \usepackage{import}
%%
%% and then include the figures with
%%   \import{<path to file>}{<filename>.pgf}
%%
%% Matplotlib used the following preamble
%%   \usepackage{fontspec}
%%   \setmainfont{DejaVuSerif.ttf}[Path=/usr/local/lib/python3.6/dist-packages/matplotlib/mpl-data/fonts/ttf/]
%%   \setsansfont{DejaVuSans.ttf}[Path=/usr/local/lib/python3.6/dist-packages/matplotlib/mpl-data/fonts/ttf/]
%%   \setmonofont{DejaVuSansMono.ttf}[Path=/usr/local/lib/python3.6/dist-packages/matplotlib/mpl-data/fonts/ttf/]
%%
\begingroup%
\makeatletter%
\begin{pgfpicture}%
\pgfpathrectangle{\pgfpointorigin}{\pgfqpoint{6.400000in}{4.800000in}}%
\pgfusepath{use as bounding box, clip}%
\begin{pgfscope}%
\pgfsetbuttcap%
\pgfsetmiterjoin%
\definecolor{currentfill}{rgb}{1.000000,1.000000,1.000000}%
\pgfsetfillcolor{currentfill}%
\pgfsetlinewidth{0.000000pt}%
\definecolor{currentstroke}{rgb}{1.000000,1.000000,1.000000}%
\pgfsetstrokecolor{currentstroke}%
\pgfsetdash{}{0pt}%
\pgfpathmoveto{\pgfqpoint{0.000000in}{0.000000in}}%
\pgfpathlineto{\pgfqpoint{6.400000in}{0.000000in}}%
\pgfpathlineto{\pgfqpoint{6.400000in}{4.800000in}}%
\pgfpathlineto{\pgfqpoint{0.000000in}{4.800000in}}%
\pgfpathclose%
\pgfusepath{fill}%
\end{pgfscope}%
\begin{pgfscope}%
\pgfsetbuttcap%
\pgfsetmiterjoin%
\definecolor{currentfill}{rgb}{1.000000,1.000000,1.000000}%
\pgfsetfillcolor{currentfill}%
\pgfsetlinewidth{0.000000pt}%
\definecolor{currentstroke}{rgb}{0.000000,0.000000,0.000000}%
\pgfsetstrokecolor{currentstroke}%
\pgfsetstrokeopacity{0.000000}%
\pgfsetdash{}{0pt}%
\pgfpathmoveto{\pgfqpoint{0.800000in}{0.528000in}}%
\pgfpathlineto{\pgfqpoint{5.760000in}{0.528000in}}%
\pgfpathlineto{\pgfqpoint{5.760000in}{4.224000in}}%
\pgfpathlineto{\pgfqpoint{0.800000in}{4.224000in}}%
\pgfpathclose%
\pgfusepath{fill}%
\end{pgfscope}%
\begin{pgfscope}%
\pgfpathrectangle{\pgfqpoint{0.800000in}{0.528000in}}{\pgfqpoint{4.960000in}{3.696000in}}%
\pgfusepath{clip}%
\pgfsetroundcap%
\pgfsetroundjoin%
\pgfsetlinewidth{1.003750pt}%
\definecolor{currentstroke}{rgb}{0.800000,0.800000,0.800000}%
\pgfsetstrokecolor{currentstroke}%
\pgfsetdash{}{0pt}%
\pgfpathmoveto{\pgfqpoint{0.982991in}{0.528000in}}%
\pgfpathlineto{\pgfqpoint{0.982991in}{4.224000in}}%
\pgfusepath{stroke}%
\end{pgfscope}%
\begin{pgfscope}%
\definecolor{textcolor}{rgb}{0.150000,0.150000,0.150000}%
\pgfsetstrokecolor{textcolor}%
\pgfsetfillcolor{textcolor}%
\pgftext[x=0.982991in,y=0.396056in,,top]{\color{textcolor}\sffamily\fontsize{11.000000}{13.200000}\selectfont 80}%
\end{pgfscope}%
\begin{pgfscope}%
\pgfpathrectangle{\pgfqpoint{0.800000in}{0.528000in}}{\pgfqpoint{4.960000in}{3.696000in}}%
\pgfusepath{clip}%
\pgfsetroundcap%
\pgfsetroundjoin%
\pgfsetlinewidth{1.003750pt}%
\definecolor{currentstroke}{rgb}{0.800000,0.800000,0.800000}%
\pgfsetstrokecolor{currentstroke}%
\pgfsetdash{}{0pt}%
\pgfpathmoveto{\pgfqpoint{1.564331in}{0.528000in}}%
\pgfpathlineto{\pgfqpoint{1.564331in}{4.224000in}}%
\pgfusepath{stroke}%
\end{pgfscope}%
\begin{pgfscope}%
\definecolor{textcolor}{rgb}{0.150000,0.150000,0.150000}%
\pgfsetstrokecolor{textcolor}%
\pgfsetfillcolor{textcolor}%
\pgftext[x=1.564331in,y=0.396056in,,top]{\color{textcolor}\sffamily\fontsize{11.000000}{13.200000}\selectfont 100}%
\end{pgfscope}%
\begin{pgfscope}%
\pgfpathrectangle{\pgfqpoint{0.800000in}{0.528000in}}{\pgfqpoint{4.960000in}{3.696000in}}%
\pgfusepath{clip}%
\pgfsetroundcap%
\pgfsetroundjoin%
\pgfsetlinewidth{1.003750pt}%
\definecolor{currentstroke}{rgb}{0.800000,0.800000,0.800000}%
\pgfsetstrokecolor{currentstroke}%
\pgfsetdash{}{0pt}%
\pgfpathmoveto{\pgfqpoint{2.145672in}{0.528000in}}%
\pgfpathlineto{\pgfqpoint{2.145672in}{4.224000in}}%
\pgfusepath{stroke}%
\end{pgfscope}%
\begin{pgfscope}%
\definecolor{textcolor}{rgb}{0.150000,0.150000,0.150000}%
\pgfsetstrokecolor{textcolor}%
\pgfsetfillcolor{textcolor}%
\pgftext[x=2.145672in,y=0.396056in,,top]{\color{textcolor}\sffamily\fontsize{11.000000}{13.200000}\selectfont 120}%
\end{pgfscope}%
\begin{pgfscope}%
\pgfpathrectangle{\pgfqpoint{0.800000in}{0.528000in}}{\pgfqpoint{4.960000in}{3.696000in}}%
\pgfusepath{clip}%
\pgfsetroundcap%
\pgfsetroundjoin%
\pgfsetlinewidth{1.003750pt}%
\definecolor{currentstroke}{rgb}{0.800000,0.800000,0.800000}%
\pgfsetstrokecolor{currentstroke}%
\pgfsetdash{}{0pt}%
\pgfpathmoveto{\pgfqpoint{2.727012in}{0.528000in}}%
\pgfpathlineto{\pgfqpoint{2.727012in}{4.224000in}}%
\pgfusepath{stroke}%
\end{pgfscope}%
\begin{pgfscope}%
\definecolor{textcolor}{rgb}{0.150000,0.150000,0.150000}%
\pgfsetstrokecolor{textcolor}%
\pgfsetfillcolor{textcolor}%
\pgftext[x=2.727012in,y=0.396056in,,top]{\color{textcolor}\sffamily\fontsize{11.000000}{13.200000}\selectfont 140}%
\end{pgfscope}%
\begin{pgfscope}%
\pgfpathrectangle{\pgfqpoint{0.800000in}{0.528000in}}{\pgfqpoint{4.960000in}{3.696000in}}%
\pgfusepath{clip}%
\pgfsetroundcap%
\pgfsetroundjoin%
\pgfsetlinewidth{1.003750pt}%
\definecolor{currentstroke}{rgb}{0.800000,0.800000,0.800000}%
\pgfsetstrokecolor{currentstroke}%
\pgfsetdash{}{0pt}%
\pgfpathmoveto{\pgfqpoint{3.308352in}{0.528000in}}%
\pgfpathlineto{\pgfqpoint{3.308352in}{4.224000in}}%
\pgfusepath{stroke}%
\end{pgfscope}%
\begin{pgfscope}%
\definecolor{textcolor}{rgb}{0.150000,0.150000,0.150000}%
\pgfsetstrokecolor{textcolor}%
\pgfsetfillcolor{textcolor}%
\pgftext[x=3.308352in,y=0.396056in,,top]{\color{textcolor}\sffamily\fontsize{11.000000}{13.200000}\selectfont 160}%
\end{pgfscope}%
\begin{pgfscope}%
\pgfpathrectangle{\pgfqpoint{0.800000in}{0.528000in}}{\pgfqpoint{4.960000in}{3.696000in}}%
\pgfusepath{clip}%
\pgfsetroundcap%
\pgfsetroundjoin%
\pgfsetlinewidth{1.003750pt}%
\definecolor{currentstroke}{rgb}{0.800000,0.800000,0.800000}%
\pgfsetstrokecolor{currentstroke}%
\pgfsetdash{}{0pt}%
\pgfpathmoveto{\pgfqpoint{3.889692in}{0.528000in}}%
\pgfpathlineto{\pgfqpoint{3.889692in}{4.224000in}}%
\pgfusepath{stroke}%
\end{pgfscope}%
\begin{pgfscope}%
\definecolor{textcolor}{rgb}{0.150000,0.150000,0.150000}%
\pgfsetstrokecolor{textcolor}%
\pgfsetfillcolor{textcolor}%
\pgftext[x=3.889692in,y=0.396056in,,top]{\color{textcolor}\sffamily\fontsize{11.000000}{13.200000}\selectfont 180}%
\end{pgfscope}%
\begin{pgfscope}%
\pgfpathrectangle{\pgfqpoint{0.800000in}{0.528000in}}{\pgfqpoint{4.960000in}{3.696000in}}%
\pgfusepath{clip}%
\pgfsetroundcap%
\pgfsetroundjoin%
\pgfsetlinewidth{1.003750pt}%
\definecolor{currentstroke}{rgb}{0.800000,0.800000,0.800000}%
\pgfsetstrokecolor{currentstroke}%
\pgfsetdash{}{0pt}%
\pgfpathmoveto{\pgfqpoint{4.471032in}{0.528000in}}%
\pgfpathlineto{\pgfqpoint{4.471032in}{4.224000in}}%
\pgfusepath{stroke}%
\end{pgfscope}%
\begin{pgfscope}%
\definecolor{textcolor}{rgb}{0.150000,0.150000,0.150000}%
\pgfsetstrokecolor{textcolor}%
\pgfsetfillcolor{textcolor}%
\pgftext[x=4.471032in,y=0.396056in,,top]{\color{textcolor}\sffamily\fontsize{11.000000}{13.200000}\selectfont 200}%
\end{pgfscope}%
\begin{pgfscope}%
\pgfpathrectangle{\pgfqpoint{0.800000in}{0.528000in}}{\pgfqpoint{4.960000in}{3.696000in}}%
\pgfusepath{clip}%
\pgfsetroundcap%
\pgfsetroundjoin%
\pgfsetlinewidth{1.003750pt}%
\definecolor{currentstroke}{rgb}{0.800000,0.800000,0.800000}%
\pgfsetstrokecolor{currentstroke}%
\pgfsetdash{}{0pt}%
\pgfpathmoveto{\pgfqpoint{5.052373in}{0.528000in}}%
\pgfpathlineto{\pgfqpoint{5.052373in}{4.224000in}}%
\pgfusepath{stroke}%
\end{pgfscope}%
\begin{pgfscope}%
\definecolor{textcolor}{rgb}{0.150000,0.150000,0.150000}%
\pgfsetstrokecolor{textcolor}%
\pgfsetfillcolor{textcolor}%
\pgftext[x=5.052373in,y=0.396056in,,top]{\color{textcolor}\sffamily\fontsize{11.000000}{13.200000}\selectfont 220}%
\end{pgfscope}%
\begin{pgfscope}%
\pgfpathrectangle{\pgfqpoint{0.800000in}{0.528000in}}{\pgfqpoint{4.960000in}{3.696000in}}%
\pgfusepath{clip}%
\pgfsetroundcap%
\pgfsetroundjoin%
\pgfsetlinewidth{1.003750pt}%
\definecolor{currentstroke}{rgb}{0.800000,0.800000,0.800000}%
\pgfsetstrokecolor{currentstroke}%
\pgfsetdash{}{0pt}%
\pgfpathmoveto{\pgfqpoint{5.633713in}{0.528000in}}%
\pgfpathlineto{\pgfqpoint{5.633713in}{4.224000in}}%
\pgfusepath{stroke}%
\end{pgfscope}%
\begin{pgfscope}%
\definecolor{textcolor}{rgb}{0.150000,0.150000,0.150000}%
\pgfsetstrokecolor{textcolor}%
\pgfsetfillcolor{textcolor}%
\pgftext[x=5.633713in,y=0.396056in,,top]{\color{textcolor}\sffamily\fontsize{11.000000}{13.200000}\selectfont 240}%
\end{pgfscope}%
\begin{pgfscope}%
\definecolor{textcolor}{rgb}{0.150000,0.150000,0.150000}%
\pgfsetstrokecolor{textcolor}%
\pgfsetfillcolor{textcolor}%
\pgftext[x=3.280000in,y=0.192646in,,top]{\color{textcolor}\sffamily\fontsize{12.000000}{14.400000}\selectfont Temps de réponse bout-en-bout (ms)}%
\end{pgfscope}%
\begin{pgfscope}%
\pgfpathrectangle{\pgfqpoint{0.800000in}{0.528000in}}{\pgfqpoint{4.960000in}{3.696000in}}%
\pgfusepath{clip}%
\pgfsetroundcap%
\pgfsetroundjoin%
\pgfsetlinewidth{1.003750pt}%
\definecolor{currentstroke}{rgb}{0.800000,0.800000,0.800000}%
\pgfsetstrokecolor{currentstroke}%
\pgfsetdash{}{0pt}%
\pgfpathmoveto{\pgfqpoint{0.800000in}{0.528000in}}%
\pgfpathlineto{\pgfqpoint{5.760000in}{0.528000in}}%
\pgfusepath{stroke}%
\end{pgfscope}%
\begin{pgfscope}%
\definecolor{textcolor}{rgb}{0.150000,0.150000,0.150000}%
\pgfsetstrokecolor{textcolor}%
\pgfsetfillcolor{textcolor}%
\pgftext[x=0.527886in, y=0.469962in, left, base]{\color{textcolor}\sffamily\fontsize{11.000000}{13.200000}\selectfont 0\%}%
\end{pgfscope}%
\begin{pgfscope}%
\pgfpathrectangle{\pgfqpoint{0.800000in}{0.528000in}}{\pgfqpoint{4.960000in}{3.696000in}}%
\pgfusepath{clip}%
\pgfsetroundcap%
\pgfsetroundjoin%
\pgfsetlinewidth{1.003750pt}%
\definecolor{currentstroke}{rgb}{0.800000,0.800000,0.800000}%
\pgfsetstrokecolor{currentstroke}%
\pgfsetdash{}{0pt}%
\pgfpathmoveto{\pgfqpoint{0.800000in}{1.274913in}}%
\pgfpathlineto{\pgfqpoint{5.760000in}{1.274913in}}%
\pgfusepath{stroke}%
\end{pgfscope}%
\begin{pgfscope}%
\definecolor{textcolor}{rgb}{0.150000,0.150000,0.150000}%
\pgfsetstrokecolor{textcolor}%
\pgfsetfillcolor{textcolor}%
\pgftext[x=0.527886in, y=1.216875in, left, base]{\color{textcolor}\sffamily\fontsize{11.000000}{13.200000}\selectfont 2\%}%
\end{pgfscope}%
\begin{pgfscope}%
\pgfpathrectangle{\pgfqpoint{0.800000in}{0.528000in}}{\pgfqpoint{4.960000in}{3.696000in}}%
\pgfusepath{clip}%
\pgfsetroundcap%
\pgfsetroundjoin%
\pgfsetlinewidth{1.003750pt}%
\definecolor{currentstroke}{rgb}{0.800000,0.800000,0.800000}%
\pgfsetstrokecolor{currentstroke}%
\pgfsetdash{}{0pt}%
\pgfpathmoveto{\pgfqpoint{0.800000in}{2.021825in}}%
\pgfpathlineto{\pgfqpoint{5.760000in}{2.021825in}}%
\pgfusepath{stroke}%
\end{pgfscope}%
\begin{pgfscope}%
\definecolor{textcolor}{rgb}{0.150000,0.150000,0.150000}%
\pgfsetstrokecolor{textcolor}%
\pgfsetfillcolor{textcolor}%
\pgftext[x=0.527886in, y=1.963788in, left, base]{\color{textcolor}\sffamily\fontsize{11.000000}{13.200000}\selectfont 4\%}%
\end{pgfscope}%
\begin{pgfscope}%
\pgfpathrectangle{\pgfqpoint{0.800000in}{0.528000in}}{\pgfqpoint{4.960000in}{3.696000in}}%
\pgfusepath{clip}%
\pgfsetroundcap%
\pgfsetroundjoin%
\pgfsetlinewidth{1.003750pt}%
\definecolor{currentstroke}{rgb}{0.800000,0.800000,0.800000}%
\pgfsetstrokecolor{currentstroke}%
\pgfsetdash{}{0pt}%
\pgfpathmoveto{\pgfqpoint{0.800000in}{2.768738in}}%
\pgfpathlineto{\pgfqpoint{5.760000in}{2.768738in}}%
\pgfusepath{stroke}%
\end{pgfscope}%
\begin{pgfscope}%
\definecolor{textcolor}{rgb}{0.150000,0.150000,0.150000}%
\pgfsetstrokecolor{textcolor}%
\pgfsetfillcolor{textcolor}%
\pgftext[x=0.527886in, y=2.710700in, left, base]{\color{textcolor}\sffamily\fontsize{11.000000}{13.200000}\selectfont 6\%}%
\end{pgfscope}%
\begin{pgfscope}%
\pgfpathrectangle{\pgfqpoint{0.800000in}{0.528000in}}{\pgfqpoint{4.960000in}{3.696000in}}%
\pgfusepath{clip}%
\pgfsetroundcap%
\pgfsetroundjoin%
\pgfsetlinewidth{1.003750pt}%
\definecolor{currentstroke}{rgb}{0.800000,0.800000,0.800000}%
\pgfsetstrokecolor{currentstroke}%
\pgfsetdash{}{0pt}%
\pgfpathmoveto{\pgfqpoint{0.800000in}{3.515650in}}%
\pgfpathlineto{\pgfqpoint{5.760000in}{3.515650in}}%
\pgfusepath{stroke}%
\end{pgfscope}%
\begin{pgfscope}%
\definecolor{textcolor}{rgb}{0.150000,0.150000,0.150000}%
\pgfsetstrokecolor{textcolor}%
\pgfsetfillcolor{textcolor}%
\pgftext[x=0.527886in, y=3.457613in, left, base]{\color{textcolor}\sffamily\fontsize{11.000000}{13.200000}\selectfont 8\%}%
\end{pgfscope}%
\begin{pgfscope}%
\definecolor{textcolor}{rgb}{0.150000,0.150000,0.150000}%
\pgfsetstrokecolor{textcolor}%
\pgfsetfillcolor{textcolor}%
\pgftext[x=0.272331in,y=2.376000in,,bottom,rotate=90.000000]{\color{textcolor}\sffamily\fontsize{12.000000}{14.400000}\selectfont Densité de distribution}%
\end{pgfscope}%
\begin{pgfscope}%
\pgfpathrectangle{\pgfqpoint{0.800000in}{0.528000in}}{\pgfqpoint{4.960000in}{3.696000in}}%
\pgfusepath{clip}%
\pgfsetbuttcap%
\pgfsetroundjoin%
\definecolor{currentfill}{rgb}{0.018039,0.067059,0.550196}%
\pgfsetfillcolor{currentfill}%
\pgfsetfillopacity{0.250000}%
\pgfsetlinewidth{1.003750pt}%
\definecolor{currentstroke}{rgb}{0.018039,0.067059,0.550196}%
\pgfsetstrokecolor{currentstroke}%
\pgfsetdash{}{0pt}%
\pgfsys@defobject{currentmarker}{\pgfqpoint{1.025455in}{0.528000in}}{\pgfqpoint{2.674537in}{2.186749in}}{%
\pgfpathmoveto{\pgfqpoint{1.025455in}{0.529344in}}%
\pgfpathlineto{\pgfqpoint{1.025455in}{0.528000in}}%
\pgfpathlineto{\pgfqpoint{1.033741in}{0.528000in}}%
\pgfpathlineto{\pgfqpoint{1.042028in}{0.528000in}}%
\pgfpathlineto{\pgfqpoint{1.050315in}{0.528000in}}%
\pgfpathlineto{\pgfqpoint{1.058602in}{0.528000in}}%
\pgfpathlineto{\pgfqpoint{1.066889in}{0.528000in}}%
\pgfpathlineto{\pgfqpoint{1.075176in}{0.528000in}}%
\pgfpathlineto{\pgfqpoint{1.083462in}{0.528000in}}%
\pgfpathlineto{\pgfqpoint{1.091749in}{0.528000in}}%
\pgfpathlineto{\pgfqpoint{1.100036in}{0.528000in}}%
\pgfpathlineto{\pgfqpoint{1.108323in}{0.528000in}}%
\pgfpathlineto{\pgfqpoint{1.116610in}{0.528000in}}%
\pgfpathlineto{\pgfqpoint{1.124897in}{0.528000in}}%
\pgfpathlineto{\pgfqpoint{1.133184in}{0.528000in}}%
\pgfpathlineto{\pgfqpoint{1.141470in}{0.528000in}}%
\pgfpathlineto{\pgfqpoint{1.149757in}{0.528000in}}%
\pgfpathlineto{\pgfqpoint{1.158044in}{0.528000in}}%
\pgfpathlineto{\pgfqpoint{1.166331in}{0.528000in}}%
\pgfpathlineto{\pgfqpoint{1.174618in}{0.528000in}}%
\pgfpathlineto{\pgfqpoint{1.182905in}{0.528000in}}%
\pgfpathlineto{\pgfqpoint{1.191191in}{0.528000in}}%
\pgfpathlineto{\pgfqpoint{1.199478in}{0.528000in}}%
\pgfpathlineto{\pgfqpoint{1.207765in}{0.528000in}}%
\pgfpathlineto{\pgfqpoint{1.216052in}{0.528000in}}%
\pgfpathlineto{\pgfqpoint{1.224339in}{0.528000in}}%
\pgfpathlineto{\pgfqpoint{1.232626in}{0.528000in}}%
\pgfpathlineto{\pgfqpoint{1.240913in}{0.528000in}}%
\pgfpathlineto{\pgfqpoint{1.249199in}{0.528000in}}%
\pgfpathlineto{\pgfqpoint{1.257486in}{0.528000in}}%
\pgfpathlineto{\pgfqpoint{1.265773in}{0.528000in}}%
\pgfpathlineto{\pgfqpoint{1.274060in}{0.528000in}}%
\pgfpathlineto{\pgfqpoint{1.282347in}{0.528000in}}%
\pgfpathlineto{\pgfqpoint{1.290634in}{0.528000in}}%
\pgfpathlineto{\pgfqpoint{1.298920in}{0.528000in}}%
\pgfpathlineto{\pgfqpoint{1.307207in}{0.528000in}}%
\pgfpathlineto{\pgfqpoint{1.315494in}{0.528000in}}%
\pgfpathlineto{\pgfqpoint{1.323781in}{0.528000in}}%
\pgfpathlineto{\pgfqpoint{1.332068in}{0.528000in}}%
\pgfpathlineto{\pgfqpoint{1.340355in}{0.528000in}}%
\pgfpathlineto{\pgfqpoint{1.348642in}{0.528000in}}%
\pgfpathlineto{\pgfqpoint{1.356928in}{0.528000in}}%
\pgfpathlineto{\pgfqpoint{1.365215in}{0.528000in}}%
\pgfpathlineto{\pgfqpoint{1.373502in}{0.528000in}}%
\pgfpathlineto{\pgfqpoint{1.381789in}{0.528000in}}%
\pgfpathlineto{\pgfqpoint{1.390076in}{0.528000in}}%
\pgfpathlineto{\pgfqpoint{1.398363in}{0.528000in}}%
\pgfpathlineto{\pgfqpoint{1.406650in}{0.528000in}}%
\pgfpathlineto{\pgfqpoint{1.414936in}{0.528000in}}%
\pgfpathlineto{\pgfqpoint{1.423223in}{0.528000in}}%
\pgfpathlineto{\pgfqpoint{1.431510in}{0.528000in}}%
\pgfpathlineto{\pgfqpoint{1.439797in}{0.528000in}}%
\pgfpathlineto{\pgfqpoint{1.448084in}{0.528000in}}%
\pgfpathlineto{\pgfqpoint{1.456371in}{0.528000in}}%
\pgfpathlineto{\pgfqpoint{1.464657in}{0.528000in}}%
\pgfpathlineto{\pgfqpoint{1.472944in}{0.528000in}}%
\pgfpathlineto{\pgfqpoint{1.481231in}{0.528000in}}%
\pgfpathlineto{\pgfqpoint{1.489518in}{0.528000in}}%
\pgfpathlineto{\pgfqpoint{1.497805in}{0.528000in}}%
\pgfpathlineto{\pgfqpoint{1.506092in}{0.528000in}}%
\pgfpathlineto{\pgfqpoint{1.514379in}{0.528000in}}%
\pgfpathlineto{\pgfqpoint{1.522665in}{0.528000in}}%
\pgfpathlineto{\pgfqpoint{1.530952in}{0.528000in}}%
\pgfpathlineto{\pgfqpoint{1.539239in}{0.528000in}}%
\pgfpathlineto{\pgfqpoint{1.547526in}{0.528000in}}%
\pgfpathlineto{\pgfqpoint{1.555813in}{0.528000in}}%
\pgfpathlineto{\pgfqpoint{1.564100in}{0.528000in}}%
\pgfpathlineto{\pgfqpoint{1.572386in}{0.528000in}}%
\pgfpathlineto{\pgfqpoint{1.580673in}{0.528000in}}%
\pgfpathlineto{\pgfqpoint{1.588960in}{0.528000in}}%
\pgfpathlineto{\pgfqpoint{1.597247in}{0.528000in}}%
\pgfpathlineto{\pgfqpoint{1.605534in}{0.528000in}}%
\pgfpathlineto{\pgfqpoint{1.613821in}{0.528000in}}%
\pgfpathlineto{\pgfqpoint{1.622108in}{0.528000in}}%
\pgfpathlineto{\pgfqpoint{1.630394in}{0.528000in}}%
\pgfpathlineto{\pgfqpoint{1.638681in}{0.528000in}}%
\pgfpathlineto{\pgfqpoint{1.646968in}{0.528000in}}%
\pgfpathlineto{\pgfqpoint{1.655255in}{0.528000in}}%
\pgfpathlineto{\pgfqpoint{1.663542in}{0.528000in}}%
\pgfpathlineto{\pgfqpoint{1.671829in}{0.528000in}}%
\pgfpathlineto{\pgfqpoint{1.680115in}{0.528000in}}%
\pgfpathlineto{\pgfqpoint{1.688402in}{0.528000in}}%
\pgfpathlineto{\pgfqpoint{1.696689in}{0.528000in}}%
\pgfpathlineto{\pgfqpoint{1.704976in}{0.528000in}}%
\pgfpathlineto{\pgfqpoint{1.713263in}{0.528000in}}%
\pgfpathlineto{\pgfqpoint{1.721550in}{0.528000in}}%
\pgfpathlineto{\pgfqpoint{1.729837in}{0.528000in}}%
\pgfpathlineto{\pgfqpoint{1.738123in}{0.528000in}}%
\pgfpathlineto{\pgfqpoint{1.746410in}{0.528000in}}%
\pgfpathlineto{\pgfqpoint{1.754697in}{0.528000in}}%
\pgfpathlineto{\pgfqpoint{1.762984in}{0.528000in}}%
\pgfpathlineto{\pgfqpoint{1.771271in}{0.528000in}}%
\pgfpathlineto{\pgfqpoint{1.779558in}{0.528000in}}%
\pgfpathlineto{\pgfqpoint{1.787844in}{0.528000in}}%
\pgfpathlineto{\pgfqpoint{1.796131in}{0.528000in}}%
\pgfpathlineto{\pgfqpoint{1.804418in}{0.528000in}}%
\pgfpathlineto{\pgfqpoint{1.812705in}{0.528000in}}%
\pgfpathlineto{\pgfqpoint{1.820992in}{0.528000in}}%
\pgfpathlineto{\pgfqpoint{1.829279in}{0.528000in}}%
\pgfpathlineto{\pgfqpoint{1.837566in}{0.528000in}}%
\pgfpathlineto{\pgfqpoint{1.845852in}{0.528000in}}%
\pgfpathlineto{\pgfqpoint{1.854139in}{0.528000in}}%
\pgfpathlineto{\pgfqpoint{1.862426in}{0.528000in}}%
\pgfpathlineto{\pgfqpoint{1.870713in}{0.528000in}}%
\pgfpathlineto{\pgfqpoint{1.879000in}{0.528000in}}%
\pgfpathlineto{\pgfqpoint{1.887287in}{0.528000in}}%
\pgfpathlineto{\pgfqpoint{1.895573in}{0.528000in}}%
\pgfpathlineto{\pgfqpoint{1.903860in}{0.528000in}}%
\pgfpathlineto{\pgfqpoint{1.912147in}{0.528000in}}%
\pgfpathlineto{\pgfqpoint{1.920434in}{0.528000in}}%
\pgfpathlineto{\pgfqpoint{1.928721in}{0.528000in}}%
\pgfpathlineto{\pgfqpoint{1.937008in}{0.528000in}}%
\pgfpathlineto{\pgfqpoint{1.945295in}{0.528000in}}%
\pgfpathlineto{\pgfqpoint{1.953581in}{0.528000in}}%
\pgfpathlineto{\pgfqpoint{1.961868in}{0.528000in}}%
\pgfpathlineto{\pgfqpoint{1.970155in}{0.528000in}}%
\pgfpathlineto{\pgfqpoint{1.978442in}{0.528000in}}%
\pgfpathlineto{\pgfqpoint{1.986729in}{0.528000in}}%
\pgfpathlineto{\pgfqpoint{1.995016in}{0.528000in}}%
\pgfpathlineto{\pgfqpoint{2.003302in}{0.528000in}}%
\pgfpathlineto{\pgfqpoint{2.011589in}{0.528000in}}%
\pgfpathlineto{\pgfqpoint{2.019876in}{0.528000in}}%
\pgfpathlineto{\pgfqpoint{2.028163in}{0.528000in}}%
\pgfpathlineto{\pgfqpoint{2.036450in}{0.528000in}}%
\pgfpathlineto{\pgfqpoint{2.044737in}{0.528000in}}%
\pgfpathlineto{\pgfqpoint{2.053024in}{0.528000in}}%
\pgfpathlineto{\pgfqpoint{2.061310in}{0.528000in}}%
\pgfpathlineto{\pgfqpoint{2.069597in}{0.528000in}}%
\pgfpathlineto{\pgfqpoint{2.077884in}{0.528000in}}%
\pgfpathlineto{\pgfqpoint{2.086171in}{0.528000in}}%
\pgfpathlineto{\pgfqpoint{2.094458in}{0.528000in}}%
\pgfpathlineto{\pgfqpoint{2.102745in}{0.528000in}}%
\pgfpathlineto{\pgfqpoint{2.111032in}{0.528000in}}%
\pgfpathlineto{\pgfqpoint{2.119318in}{0.528000in}}%
\pgfpathlineto{\pgfqpoint{2.127605in}{0.528000in}}%
\pgfpathlineto{\pgfqpoint{2.135892in}{0.528000in}}%
\pgfpathlineto{\pgfqpoint{2.144179in}{0.528000in}}%
\pgfpathlineto{\pgfqpoint{2.152466in}{0.528000in}}%
\pgfpathlineto{\pgfqpoint{2.160753in}{0.528000in}}%
\pgfpathlineto{\pgfqpoint{2.169039in}{0.528000in}}%
\pgfpathlineto{\pgfqpoint{2.177326in}{0.528000in}}%
\pgfpathlineto{\pgfqpoint{2.185613in}{0.528000in}}%
\pgfpathlineto{\pgfqpoint{2.193900in}{0.528000in}}%
\pgfpathlineto{\pgfqpoint{2.202187in}{0.528000in}}%
\pgfpathlineto{\pgfqpoint{2.210474in}{0.528000in}}%
\pgfpathlineto{\pgfqpoint{2.218761in}{0.528000in}}%
\pgfpathlineto{\pgfqpoint{2.227047in}{0.528000in}}%
\pgfpathlineto{\pgfqpoint{2.235334in}{0.528000in}}%
\pgfpathlineto{\pgfqpoint{2.243621in}{0.528000in}}%
\pgfpathlineto{\pgfqpoint{2.251908in}{0.528000in}}%
\pgfpathlineto{\pgfqpoint{2.260195in}{0.528000in}}%
\pgfpathlineto{\pgfqpoint{2.268482in}{0.528000in}}%
\pgfpathlineto{\pgfqpoint{2.276768in}{0.528000in}}%
\pgfpathlineto{\pgfqpoint{2.285055in}{0.528000in}}%
\pgfpathlineto{\pgfqpoint{2.293342in}{0.528000in}}%
\pgfpathlineto{\pgfqpoint{2.301629in}{0.528000in}}%
\pgfpathlineto{\pgfqpoint{2.309916in}{0.528000in}}%
\pgfpathlineto{\pgfqpoint{2.318203in}{0.528000in}}%
\pgfpathlineto{\pgfqpoint{2.326490in}{0.528000in}}%
\pgfpathlineto{\pgfqpoint{2.334776in}{0.528000in}}%
\pgfpathlineto{\pgfqpoint{2.343063in}{0.528000in}}%
\pgfpathlineto{\pgfqpoint{2.351350in}{0.528000in}}%
\pgfpathlineto{\pgfqpoint{2.359637in}{0.528000in}}%
\pgfpathlineto{\pgfqpoint{2.367924in}{0.528000in}}%
\pgfpathlineto{\pgfqpoint{2.376211in}{0.528000in}}%
\pgfpathlineto{\pgfqpoint{2.384497in}{0.528000in}}%
\pgfpathlineto{\pgfqpoint{2.392784in}{0.528000in}}%
\pgfpathlineto{\pgfqpoint{2.401071in}{0.528000in}}%
\pgfpathlineto{\pgfqpoint{2.409358in}{0.528000in}}%
\pgfpathlineto{\pgfqpoint{2.417645in}{0.528000in}}%
\pgfpathlineto{\pgfqpoint{2.425932in}{0.528000in}}%
\pgfpathlineto{\pgfqpoint{2.434219in}{0.528000in}}%
\pgfpathlineto{\pgfqpoint{2.442505in}{0.528000in}}%
\pgfpathlineto{\pgfqpoint{2.450792in}{0.528000in}}%
\pgfpathlineto{\pgfqpoint{2.459079in}{0.528000in}}%
\pgfpathlineto{\pgfqpoint{2.467366in}{0.528000in}}%
\pgfpathlineto{\pgfqpoint{2.475653in}{0.528000in}}%
\pgfpathlineto{\pgfqpoint{2.483940in}{0.528000in}}%
\pgfpathlineto{\pgfqpoint{2.492226in}{0.528000in}}%
\pgfpathlineto{\pgfqpoint{2.500513in}{0.528000in}}%
\pgfpathlineto{\pgfqpoint{2.508800in}{0.528000in}}%
\pgfpathlineto{\pgfqpoint{2.517087in}{0.528000in}}%
\pgfpathlineto{\pgfqpoint{2.525374in}{0.528000in}}%
\pgfpathlineto{\pgfqpoint{2.533661in}{0.528000in}}%
\pgfpathlineto{\pgfqpoint{2.541948in}{0.528000in}}%
\pgfpathlineto{\pgfqpoint{2.550234in}{0.528000in}}%
\pgfpathlineto{\pgfqpoint{2.558521in}{0.528000in}}%
\pgfpathlineto{\pgfqpoint{2.566808in}{0.528000in}}%
\pgfpathlineto{\pgfqpoint{2.575095in}{0.528000in}}%
\pgfpathlineto{\pgfqpoint{2.583382in}{0.528000in}}%
\pgfpathlineto{\pgfqpoint{2.591669in}{0.528000in}}%
\pgfpathlineto{\pgfqpoint{2.599955in}{0.528000in}}%
\pgfpathlineto{\pgfqpoint{2.608242in}{0.528000in}}%
\pgfpathlineto{\pgfqpoint{2.616529in}{0.528000in}}%
\pgfpathlineto{\pgfqpoint{2.624816in}{0.528000in}}%
\pgfpathlineto{\pgfqpoint{2.633103in}{0.528000in}}%
\pgfpathlineto{\pgfqpoint{2.641390in}{0.528000in}}%
\pgfpathlineto{\pgfqpoint{2.649677in}{0.528000in}}%
\pgfpathlineto{\pgfqpoint{2.657963in}{0.528000in}}%
\pgfpathlineto{\pgfqpoint{2.666250in}{0.528000in}}%
\pgfpathlineto{\pgfqpoint{2.674537in}{0.528000in}}%
\pgfpathlineto{\pgfqpoint{2.674537in}{0.542142in}}%
\pgfpathlineto{\pgfqpoint{2.674537in}{0.542142in}}%
\pgfpathlineto{\pgfqpoint{2.666250in}{0.546986in}}%
\pgfpathlineto{\pgfqpoint{2.657963in}{0.553248in}}%
\pgfpathlineto{\pgfqpoint{2.649677in}{0.561258in}}%
\pgfpathlineto{\pgfqpoint{2.641390in}{0.571396in}}%
\pgfpathlineto{\pgfqpoint{2.633103in}{0.584088in}}%
\pgfpathlineto{\pgfqpoint{2.624816in}{0.599809in}}%
\pgfpathlineto{\pgfqpoint{2.616529in}{0.619069in}}%
\pgfpathlineto{\pgfqpoint{2.608242in}{0.642407in}}%
\pgfpathlineto{\pgfqpoint{2.599955in}{0.670372in}}%
\pgfpathlineto{\pgfqpoint{2.591669in}{0.703507in}}%
\pgfpathlineto{\pgfqpoint{2.583382in}{0.742320in}}%
\pgfpathlineto{\pgfqpoint{2.575095in}{0.787261in}}%
\pgfpathlineto{\pgfqpoint{2.566808in}{0.838687in}}%
\pgfpathlineto{\pgfqpoint{2.558521in}{0.896830in}}%
\pgfpathlineto{\pgfqpoint{2.550234in}{0.961761in}}%
\pgfpathlineto{\pgfqpoint{2.541948in}{1.033365in}}%
\pgfpathlineto{\pgfqpoint{2.533661in}{1.111306in}}%
\pgfpathlineto{\pgfqpoint{2.525374in}{1.195010in}}%
\pgfpathlineto{\pgfqpoint{2.517087in}{1.283650in}}%
\pgfpathlineto{\pgfqpoint{2.508800in}{1.376149in}}%
\pgfpathlineto{\pgfqpoint{2.500513in}{1.471181in}}%
\pgfpathlineto{\pgfqpoint{2.492226in}{1.567204in}}%
\pgfpathlineto{\pgfqpoint{2.483940in}{1.662490in}}%
\pgfpathlineto{\pgfqpoint{2.475653in}{1.755177in}}%
\pgfpathlineto{\pgfqpoint{2.467366in}{1.843325in}}%
\pgfpathlineto{\pgfqpoint{2.459079in}{1.924989in}}%
\pgfpathlineto{\pgfqpoint{2.450792in}{1.998285in}}%
\pgfpathlineto{\pgfqpoint{2.442505in}{2.061470in}}%
\pgfpathlineto{\pgfqpoint{2.434219in}{2.113007in}}%
\pgfpathlineto{\pgfqpoint{2.425932in}{2.151632in}}%
\pgfpathlineto{\pgfqpoint{2.417645in}{2.176405in}}%
\pgfpathlineto{\pgfqpoint{2.409358in}{2.186749in}}%
\pgfpathlineto{\pgfqpoint{2.401071in}{2.182472in}}%
\pgfpathlineto{\pgfqpoint{2.392784in}{2.163773in}}%
\pgfpathlineto{\pgfqpoint{2.384497in}{2.131229in}}%
\pgfpathlineto{\pgfqpoint{2.376211in}{2.085764in}}%
\pgfpathlineto{\pgfqpoint{2.367924in}{2.028608in}}%
\pgfpathlineto{\pgfqpoint{2.359637in}{1.961242in}}%
\pgfpathlineto{\pgfqpoint{2.351350in}{1.885331in}}%
\pgfpathlineto{\pgfqpoint{2.343063in}{1.802662in}}%
\pgfpathlineto{\pgfqpoint{2.334776in}{1.715068in}}%
\pgfpathlineto{\pgfqpoint{2.326490in}{1.624371in}}%
\pgfpathlineto{\pgfqpoint{2.318203in}{1.532315in}}%
\pgfpathlineto{\pgfqpoint{2.309916in}{1.440520in}}%
\pgfpathlineto{\pgfqpoint{2.301629in}{1.350441in}}%
\pgfpathlineto{\pgfqpoint{2.293342in}{1.263335in}}%
\pgfpathlineto{\pgfqpoint{2.285055in}{1.180247in}}%
\pgfpathlineto{\pgfqpoint{2.276768in}{1.102000in}}%
\pgfpathlineto{\pgfqpoint{2.268482in}{1.029199in}}%
\pgfpathlineto{\pgfqpoint{2.260195in}{0.962241in}}%
\pgfpathlineto{\pgfqpoint{2.251908in}{0.901332in}}%
\pgfpathlineto{\pgfqpoint{2.243621in}{0.846511in}}%
\pgfpathlineto{\pgfqpoint{2.235334in}{0.797670in}}%
\pgfpathlineto{\pgfqpoint{2.227047in}{0.754587in}}%
\pgfpathlineto{\pgfqpoint{2.218761in}{0.716948in}}%
\pgfpathlineto{\pgfqpoint{2.210474in}{0.684373in}}%
\pgfpathlineto{\pgfqpoint{2.202187in}{0.656440in}}%
\pgfpathlineto{\pgfqpoint{2.193900in}{0.632703in}}%
\pgfpathlineto{\pgfqpoint{2.185613in}{0.612709in}}%
\pgfpathlineto{\pgfqpoint{2.177326in}{0.596017in}}%
\pgfpathlineto{\pgfqpoint{2.169039in}{0.582201in}}%
\pgfpathlineto{\pgfqpoint{2.160753in}{0.570863in}}%
\pgfpathlineto{\pgfqpoint{2.152466in}{0.561638in}}%
\pgfpathlineto{\pgfqpoint{2.144179in}{0.554197in}}%
\pgfpathlineto{\pgfqpoint{2.135892in}{0.548244in}}%
\pgfpathlineto{\pgfqpoint{2.127605in}{0.543522in}}%
\pgfpathlineto{\pgfqpoint{2.119318in}{0.539809in}}%
\pgfpathlineto{\pgfqpoint{2.111032in}{0.536913in}}%
\pgfpathlineto{\pgfqpoint{2.102745in}{0.534674in}}%
\pgfpathlineto{\pgfqpoint{2.094458in}{0.532958in}}%
\pgfpathlineto{\pgfqpoint{2.086171in}{0.531653in}}%
\pgfpathlineto{\pgfqpoint{2.077884in}{0.530670in}}%
\pgfpathlineto{\pgfqpoint{2.069597in}{0.529935in}}%
\pgfpathlineto{\pgfqpoint{2.061310in}{0.529391in}}%
\pgfpathlineto{\pgfqpoint{2.053024in}{0.528992in}}%
\pgfpathlineto{\pgfqpoint{2.044737in}{0.528701in}}%
\pgfpathlineto{\pgfqpoint{2.036450in}{0.528491in}}%
\pgfpathlineto{\pgfqpoint{2.028163in}{0.528341in}}%
\pgfpathlineto{\pgfqpoint{2.019876in}{0.528235in}}%
\pgfpathlineto{\pgfqpoint{2.011589in}{0.528161in}}%
\pgfpathlineto{\pgfqpoint{2.003302in}{0.528109in}}%
\pgfpathlineto{\pgfqpoint{1.995016in}{0.528073in}}%
\pgfpathlineto{\pgfqpoint{1.986729in}{0.528049in}}%
\pgfpathlineto{\pgfqpoint{1.978442in}{0.528032in}}%
\pgfpathlineto{\pgfqpoint{1.970155in}{0.528021in}}%
\pgfpathlineto{\pgfqpoint{1.961868in}{0.528014in}}%
\pgfpathlineto{\pgfqpoint{1.953581in}{0.528009in}}%
\pgfpathlineto{\pgfqpoint{1.945295in}{0.528006in}}%
\pgfpathlineto{\pgfqpoint{1.937008in}{0.528004in}}%
\pgfpathlineto{\pgfqpoint{1.928721in}{0.528002in}}%
\pgfpathlineto{\pgfqpoint{1.920434in}{0.528001in}}%
\pgfpathlineto{\pgfqpoint{1.912147in}{0.528001in}}%
\pgfpathlineto{\pgfqpoint{1.903860in}{0.528001in}}%
\pgfpathlineto{\pgfqpoint{1.895573in}{0.528001in}}%
\pgfpathlineto{\pgfqpoint{1.887287in}{0.528001in}}%
\pgfpathlineto{\pgfqpoint{1.879000in}{0.528001in}}%
\pgfpathlineto{\pgfqpoint{1.870713in}{0.528001in}}%
\pgfpathlineto{\pgfqpoint{1.862426in}{0.528001in}}%
\pgfpathlineto{\pgfqpoint{1.854139in}{0.528002in}}%
\pgfpathlineto{\pgfqpoint{1.845852in}{0.528002in}}%
\pgfpathlineto{\pgfqpoint{1.837566in}{0.528003in}}%
\pgfpathlineto{\pgfqpoint{1.829279in}{0.528005in}}%
\pgfpathlineto{\pgfqpoint{1.820992in}{0.528007in}}%
\pgfpathlineto{\pgfqpoint{1.812705in}{0.528010in}}%
\pgfpathlineto{\pgfqpoint{1.804418in}{0.528014in}}%
\pgfpathlineto{\pgfqpoint{1.796131in}{0.528019in}}%
\pgfpathlineto{\pgfqpoint{1.787844in}{0.528026in}}%
\pgfpathlineto{\pgfqpoint{1.779558in}{0.528035in}}%
\pgfpathlineto{\pgfqpoint{1.771271in}{0.528048in}}%
\pgfpathlineto{\pgfqpoint{1.762984in}{0.528063in}}%
\pgfpathlineto{\pgfqpoint{1.754697in}{0.528084in}}%
\pgfpathlineto{\pgfqpoint{1.746410in}{0.528110in}}%
\pgfpathlineto{\pgfqpoint{1.738123in}{0.528142in}}%
\pgfpathlineto{\pgfqpoint{1.729837in}{0.528182in}}%
\pgfpathlineto{\pgfqpoint{1.721550in}{0.528232in}}%
\pgfpathlineto{\pgfqpoint{1.713263in}{0.528293in}}%
\pgfpathlineto{\pgfqpoint{1.704976in}{0.528366in}}%
\pgfpathlineto{\pgfqpoint{1.696689in}{0.528453in}}%
\pgfpathlineto{\pgfqpoint{1.688402in}{0.528557in}}%
\pgfpathlineto{\pgfqpoint{1.680115in}{0.528678in}}%
\pgfpathlineto{\pgfqpoint{1.671829in}{0.528820in}}%
\pgfpathlineto{\pgfqpoint{1.663542in}{0.528982in}}%
\pgfpathlineto{\pgfqpoint{1.655255in}{0.529168in}}%
\pgfpathlineto{\pgfqpoint{1.646968in}{0.529379in}}%
\pgfpathlineto{\pgfqpoint{1.638681in}{0.529616in}}%
\pgfpathlineto{\pgfqpoint{1.630394in}{0.529880in}}%
\pgfpathlineto{\pgfqpoint{1.622108in}{0.530175in}}%
\pgfpathlineto{\pgfqpoint{1.613821in}{0.530500in}}%
\pgfpathlineto{\pgfqpoint{1.605534in}{0.530858in}}%
\pgfpathlineto{\pgfqpoint{1.597247in}{0.531252in}}%
\pgfpathlineto{\pgfqpoint{1.588960in}{0.531685in}}%
\pgfpathlineto{\pgfqpoint{1.580673in}{0.532162in}}%
\pgfpathlineto{\pgfqpoint{1.572386in}{0.532687in}}%
\pgfpathlineto{\pgfqpoint{1.564100in}{0.533269in}}%
\pgfpathlineto{\pgfqpoint{1.555813in}{0.533916in}}%
\pgfpathlineto{\pgfqpoint{1.547526in}{0.534641in}}%
\pgfpathlineto{\pgfqpoint{1.539239in}{0.535458in}}%
\pgfpathlineto{\pgfqpoint{1.530952in}{0.536384in}}%
\pgfpathlineto{\pgfqpoint{1.522665in}{0.537440in}}%
\pgfpathlineto{\pgfqpoint{1.514379in}{0.538648in}}%
\pgfpathlineto{\pgfqpoint{1.506092in}{0.540037in}}%
\pgfpathlineto{\pgfqpoint{1.497805in}{0.541636in}}%
\pgfpathlineto{\pgfqpoint{1.489518in}{0.543477in}}%
\pgfpathlineto{\pgfqpoint{1.481231in}{0.545594in}}%
\pgfpathlineto{\pgfqpoint{1.472944in}{0.548023in}}%
\pgfpathlineto{\pgfqpoint{1.464657in}{0.550798in}}%
\pgfpathlineto{\pgfqpoint{1.456371in}{0.553952in}}%
\pgfpathlineto{\pgfqpoint{1.448084in}{0.557515in}}%
\pgfpathlineto{\pgfqpoint{1.439797in}{0.561508in}}%
\pgfpathlineto{\pgfqpoint{1.431510in}{0.565949in}}%
\pgfpathlineto{\pgfqpoint{1.423223in}{0.570843in}}%
\pgfpathlineto{\pgfqpoint{1.414936in}{0.576180in}}%
\pgfpathlineto{\pgfqpoint{1.406650in}{0.581940in}}%
\pgfpathlineto{\pgfqpoint{1.398363in}{0.588084in}}%
\pgfpathlineto{\pgfqpoint{1.390076in}{0.594555in}}%
\pgfpathlineto{\pgfqpoint{1.381789in}{0.601277in}}%
\pgfpathlineto{\pgfqpoint{1.373502in}{0.608157in}}%
\pgfpathlineto{\pgfqpoint{1.365215in}{0.615086in}}%
\pgfpathlineto{\pgfqpoint{1.356928in}{0.621937in}}%
\pgfpathlineto{\pgfqpoint{1.348642in}{0.628573in}}%
\pgfpathlineto{\pgfqpoint{1.340355in}{0.634850in}}%
\pgfpathlineto{\pgfqpoint{1.332068in}{0.640621in}}%
\pgfpathlineto{\pgfqpoint{1.323781in}{0.645741in}}%
\pgfpathlineto{\pgfqpoint{1.315494in}{0.650075in}}%
\pgfpathlineto{\pgfqpoint{1.307207in}{0.653501in}}%
\pgfpathlineto{\pgfqpoint{1.298920in}{0.655918in}}%
\pgfpathlineto{\pgfqpoint{1.290634in}{0.657251in}}%
\pgfpathlineto{\pgfqpoint{1.282347in}{0.657450in}}%
\pgfpathlineto{\pgfqpoint{1.274060in}{0.656500in}}%
\pgfpathlineto{\pgfqpoint{1.265773in}{0.654414in}}%
\pgfpathlineto{\pgfqpoint{1.257486in}{0.651242in}}%
\pgfpathlineto{\pgfqpoint{1.249199in}{0.647057in}}%
\pgfpathlineto{\pgfqpoint{1.240913in}{0.641965in}}%
\pgfpathlineto{\pgfqpoint{1.232626in}{0.636089in}}%
\pgfpathlineto{\pgfqpoint{1.224339in}{0.629570in}}%
\pgfpathlineto{\pgfqpoint{1.216052in}{0.622561in}}%
\pgfpathlineto{\pgfqpoint{1.207765in}{0.615218in}}%
\pgfpathlineto{\pgfqpoint{1.199478in}{0.607696in}}%
\pgfpathlineto{\pgfqpoint{1.191191in}{0.600142in}}%
\pgfpathlineto{\pgfqpoint{1.182905in}{0.592692in}}%
\pgfpathlineto{\pgfqpoint{1.174618in}{0.585468in}}%
\pgfpathlineto{\pgfqpoint{1.166331in}{0.578571in}}%
\pgfpathlineto{\pgfqpoint{1.158044in}{0.572082in}}%
\pgfpathlineto{\pgfqpoint{1.149757in}{0.566064in}}%
\pgfpathlineto{\pgfqpoint{1.141470in}{0.560557in}}%
\pgfpathlineto{\pgfqpoint{1.133184in}{0.555584in}}%
\pgfpathlineto{\pgfqpoint{1.124897in}{0.551149in}}%
\pgfpathlineto{\pgfqpoint{1.116610in}{0.547244in}}%
\pgfpathlineto{\pgfqpoint{1.108323in}{0.543846in}}%
\pgfpathlineto{\pgfqpoint{1.100036in}{0.540924in}}%
\pgfpathlineto{\pgfqpoint{1.091749in}{0.538441in}}%
\pgfpathlineto{\pgfqpoint{1.083462in}{0.536355in}}%
\pgfpathlineto{\pgfqpoint{1.075176in}{0.534622in}}%
\pgfpathlineto{\pgfqpoint{1.066889in}{0.533199in}}%
\pgfpathlineto{\pgfqpoint{1.058602in}{0.532043in}}%
\pgfpathlineto{\pgfqpoint{1.050315in}{0.531114in}}%
\pgfpathlineto{\pgfqpoint{1.042028in}{0.530376in}}%
\pgfpathlineto{\pgfqpoint{1.033741in}{0.529795in}}%
\pgfpathlineto{\pgfqpoint{1.025455in}{0.529344in}}%
\pgfpathclose%
\pgfusepath{stroke,fill}%
}%
\begin{pgfscope}%
\pgfsys@transformshift{0.000000in}{0.000000in}%
\pgfsys@useobject{currentmarker}{}%
\end{pgfscope}%
\end{pgfscope}%
\begin{pgfscope}%
\pgfsetrectcap%
\pgfsetmiterjoin%
\pgfsetlinewidth{1.254687pt}%
\definecolor{currentstroke}{rgb}{0.800000,0.800000,0.800000}%
\pgfsetstrokecolor{currentstroke}%
\pgfsetdash{}{0pt}%
\pgfpathmoveto{\pgfqpoint{0.800000in}{0.528000in}}%
\pgfpathlineto{\pgfqpoint{0.800000in}{4.224000in}}%
\pgfusepath{stroke}%
\end{pgfscope}%
\begin{pgfscope}%
\pgfsetrectcap%
\pgfsetmiterjoin%
\pgfsetlinewidth{1.254687pt}%
\definecolor{currentstroke}{rgb}{0.800000,0.800000,0.800000}%
\pgfsetstrokecolor{currentstroke}%
\pgfsetdash{}{0pt}%
\pgfpathmoveto{\pgfqpoint{5.760000in}{0.528000in}}%
\pgfpathlineto{\pgfqpoint{5.760000in}{4.224000in}}%
\pgfusepath{stroke}%
\end{pgfscope}%
\begin{pgfscope}%
\pgfsetrectcap%
\pgfsetmiterjoin%
\pgfsetlinewidth{1.254687pt}%
\definecolor{currentstroke}{rgb}{0.800000,0.800000,0.800000}%
\pgfsetstrokecolor{currentstroke}%
\pgfsetdash{}{0pt}%
\pgfpathmoveto{\pgfqpoint{0.800000in}{0.528000in}}%
\pgfpathlineto{\pgfqpoint{5.760000in}{0.528000in}}%
\pgfusepath{stroke}%
\end{pgfscope}%
\begin{pgfscope}%
\pgfsetrectcap%
\pgfsetmiterjoin%
\pgfsetlinewidth{1.254687pt}%
\definecolor{currentstroke}{rgb}{0.800000,0.800000,0.800000}%
\pgfsetstrokecolor{currentstroke}%
\pgfsetdash{}{0pt}%
\pgfpathmoveto{\pgfqpoint{0.800000in}{4.224000in}}%
\pgfpathlineto{\pgfqpoint{5.760000in}{4.224000in}}%
\pgfusepath{stroke}%
\end{pgfscope}%
\begin{pgfscope}%
\pgfsetbuttcap%
\pgfsetmiterjoin%
\definecolor{currentfill}{rgb}{1.000000,1.000000,1.000000}%
\pgfsetfillcolor{currentfill}%
\pgfsetfillopacity{0.800000}%
\pgfsetlinewidth{1.003750pt}%
\definecolor{currentstroke}{rgb}{0.800000,0.800000,0.800000}%
\pgfsetstrokecolor{currentstroke}%
\pgfsetstrokeopacity{0.800000}%
\pgfsetdash{}{0pt}%
\pgfpathmoveto{\pgfqpoint{3.380740in}{3.879049in}}%
\pgfpathlineto{\pgfqpoint{5.673056in}{3.879049in}}%
\pgfpathquadraticcurveto{\pgfqpoint{5.703611in}{3.879049in}}{\pgfqpoint{5.703611in}{3.959605in}}%
\pgfpathlineto{\pgfqpoint{5.703611in}{4.117056in}}%
\pgfpathquadraticcurveto{\pgfqpoint{5.703611in}{4.147611in}}{\pgfqpoint{5.653056in}{4.147611in}}%
\pgfpathlineto{\pgfqpoint{3.380740in}{4.147611in}}%
\pgfpathquadraticcurveto{\pgfqpoint{3.350184in}{4.147611in}}{\pgfqpoint{3.350184in}{4.117056in}}%
\pgfpathlineto{\pgfqpoint{3.350184in}{3.919605in}}%
\pgfpathquadraticcurveto{\pgfqpoint{3.350184in}{3.919605in}}{\pgfqpoint{3.380740in}{3.879049in}}%
\pgfpathclose%
\pgfusepath{stroke,fill}%
\end{pgfscope}%
\begin{pgfscope}%
\pgfsetbuttcap%
\pgfsetmiterjoin%
\definecolor{currentfill}{rgb}{0.018039,0.067059,0.550196}%
\pgfsetfillcolor{currentfill}%
\pgfsetfillopacity{0.250000}%
\pgfsetlinewidth{1.003750pt}%
\definecolor{currentstroke}{rgb}{0.018039,0.067059,0.550196}%
\pgfsetstrokecolor{currentstroke}%
\pgfsetdash{}{0pt}%
\pgfpathmoveto{\pgfqpoint{3.411295in}{3.970425in}}%
\pgfpathlineto{\pgfqpoint{3.716851in}{3.970425in}}%
\pgfpathlineto{\pgfqpoint{3.716851in}{4.077369in}}%
\pgfpathlineto{\pgfqpoint{3.411295in}{4.077369in}}%
\pgfpathclose%
\pgfusepath{stroke,fill}%
\end{pgfscope}%
\begin{pgfscope}%
\definecolor{textcolor}{rgb}{0.150000,0.150000,0.150000}%
\pgfsetstrokecolor{textcolor}%
\pgfsetfillcolor{textcolor}%
\pgftext[x=3.839073in,y=3.970425in,left,base]{\color{textcolor}\sffamily\fontsize{11.000000}{13.200000}\selectfont Cas sans interférences}%
\end{pgfscope}%
\end{pgfpicture}%
\makeatother%
\endgroup%
}
	\captionsetup{justification=centering}
	\caption{Profil de la chaîne de tâche en fonctionnement nominal \\ sans mécanisme de contrôle}
	\label{graph:taskchainsansinterference}
\end{figure}


\ifdefined\included
\else
\bibliographystyle{StyleThese}
\bibliography{these}
\end{document}
\fi
