\usepackage{iflang}
\usepackage{bibentry}



%\usepackage[sectionbib]{chapterbib}          % Cross-reference package (Natural BiB)
%\usepackage{natbib}                  % Put References at the end of each chapter
%\usepackage{bibunits}
% Do not put 'sectionbib' option here.
% Sectionbib option in 'natbib' will do.


\usepackage{fancyhdr}                    % Fancy Header and Footer

\usepackage[utf8]{inputenc}
\usepackage[T1]{fontenc}
\usepackage[french]{babel} %
\usepackage{lmodern} \normalfont %to load T1lmr.fd 
\DeclareFontShape{T1}{cmr}{b}{sc} { <-> ssub * cmr/bx/sc }{}
\DeclareUnicodeCharacter{202F}{\,}
\DeclareUnicodeCharacter{00A0}{\nobreakspace}
%\hyphenation{gar}

\usepackage{amsmath,amssymb}             % AMS Math
\usepackage{nicefrac}
\usepackage{siunitx}					%% Unites Math SI

\usepackage{blindtext}

\usepackage{datetime}

\usepackage[inline]{enumitem}

\usepackage{hhline}
%\usepackage[left=1.5in,right=1.3in,top=1.1in,bottom=1.1in]{geometry}
\usepackage[left=1.5in,right=1.3in,top=1.1in,bottom=1.1in,includefoot,includehead,headheight=13.6pt]{geometry}

%%\renewcommand{\baselinestretch}{1.05}

%%%%%%%% Multi-figures avec sub-captions
\usepackage{caption}
\usepackage{subcaption}

\usepackage{aecompl}

%% Package cosmetic meilleur layout du texte en jouant sur le spacing par caractères
\usepackage[activate={true,nocompatibility},final,tracking=true,kerning=true,factor=1100,stretch=10,shrink=10]{microtype}
\usepackage[absolute,overlay]{textpos}
\setlength{\TPHorizModule}{\paperwidth}\setlength{\TPVertModule}{\paperheight}
\sloppy

%%%%%%%%%%% JOLIS TABLEAUX
\usepackage{tabularx}		%\usepackage{tabular}
\usepackage{longtable}
\usepackage{multirow}
\newcommand{\mc}{\multicolumn}
\newcommand{\mr}[2]{\multirow{#1}{*}{#2}}
\newcommand{\mrQ}{\multirow{-4}{*}}
\usepackage{booktabs}
\usepackage{changepage}

\usepackage[usenames,dvipsnames]{xcolor}
%\usepackage{color}
\definecolor{linkcol}{rgb}{0,0,0.4}
\definecolor{citecol}{rgb}{0.5,0,0}
\definecolor{linkcol}{rgb}{0,0,0}
\definecolor{citecol}{rgb}{0,0,0}

\makeatletter
\newcommand{\ccolor}[3][]{%
	\kern-\fboxsep
	\if\relax\detokenize{#1}\relax
	\expandafter\@firstoftwo
	\else
	\expandafter\@secondoftwo
	\fi
	{\colorbox{#2}}%
	{\colorbox[#1]{#2}}%
	{#3}\kern-\fboxsep
}
\makeatother

%%%%% Insertion graphiques format PGF
\usepackage{pgfplots}
\pgfplotsset{width=\linewidth, compat=1.16}%, compat=1.17}
\usepackage{adjustbox}          %%% PERMET DE LES RECADRER + FACILEMENT


%%%%%%%%%% Bullets de listes sans saut de ligne %%%%%%%%%%
\usepackage{xparse}

\ExplSyntaxOn%
\seq_new:N \l_local_enum_seq

\newcommand{\storethestuff}[1]{%
  \seq_set_from_clist:Nn \l_local_enum_seq {#1}%
}

\newcommand{\dotheenumstuff}{%
\int_zero:N \l_tmpa_int
\seq_map_inline:Nn \l_local_enum_seq {%
    \int_incr:N \l_tmpa_int% Increase the counter
    \item ##1
    % Check whether the list has reached the end -- if so, use '.' instead of ','
    %\int_compare:nNnTF 
    % { \l_tmpa_int } < {\seq_count:N \l_local_enum_seq} 
    % {,} {.}
  }
}
\ExplSyntaxOff

\NewDocumentCommand{\linebullets}{+m}{%
  \storethestuff{#1}%
  \begin{enumerate*}[label={\alph*)},font={\bfseries},itemjoin={{, }}]
    \dotheenumstuff%
  \end{enumerate*}
}

\newenvironment{bulletList}%
{ \begin{list}%
		{$\bullet$}%
		{\setlength{\labelwidth}{25pt}%
			\setlength{\leftmargin}{30pt}%
			\setlength{\itemsep}{\parsep}}}%
	{ \end{list} }

\newcommand{\cmnt}[1]{}  %%%%% AJOUT DE COMMENTAIRE MULTILIGNES


%%%%%%%%%% ECRITURE CARACTERES DANS UN CERCLE %%%%%%%%%%
%\def\circleTxt[#1]{\raisebox{.5pt}{\textcircled{\raisebox{-1pt}{#1}}}}
\newcommand{\ctxt}[1]{\raisebox{.5pt}{\textcircled{\raisebox{-1.2pt}{#1}}}}
% Glossary / list of abbreviations

\usepackage[intoc]{nomencl}
\IfLanguageName{english}{%
\renewcommand{\nomname}{Glossary}
}{ %
\renewcommand{\nomname}{Liste des Abréviations}
}

\makenomenclature

%\usepackage{etoolbox}

\graphicspath{{.}{schemas/}{graphiques/}{tables/}}

% definitions.
% -------------------

\setcounter{secnumdepth}{3}
\setcounter{tocdepth}{2}

% Some useful commands and shortcut for maths:  partial derivative and stuff

\newcommand{\pd}[2]{\frac{\partial #1}{\partial #2}}
\def\abs{\operatorname{abs}}
\def\argmax{\operatornamewithlimits{arg\,max}}
\def\argmin{\operatornamewithlimits{arg\,min}}
\def\diag{\operatorname{Diag}}
\newcommand{\eqRef}[1]{(\ref{#1})}
\newcommand{\nline}{\smallbreak\noindent}

\usepackage{rotating}                    % Sideways of figures & tables

% \usepackage{txfonts}                     % Public Times New Roman text & math font

%%% Fancy Header %%%%%%%%%%%%%%%%%%%%%%%%%%%%%%%%%%%%%%%%%%%%%%%%%%%%%%%%%%%%%%%%%%
% Fancy Header Style Options

\pagestyle{fancy}                       % Sets fancy header and footer
\fancyfoot{}                            % Delete current footer settings

%\renewcommand{\chaptermark}[1]{         % Lower Case Chapter marker style
%  \markboth{\chaptername\ \thechapter.\ #1}}{}} %

%\renewcommand{\sectionmark}[1]{         % Lower case Section marker style
%  \markright{\thesection.\ #1}}         %

\fancyhead[LE,RO]{\bfseries\thepage}    % Page number (boldface) in left on even
% pages and right on odd pages
\fancyhead[RE]{\bfseries\nouppercase{\leftmark}}      % Chapter in the right on even pages
\fancyhead[LO]{\bfseries\nouppercase{\rightmark}}     % Section in the left on odd pages

\let\headruleORIG\headrule
\renewcommand{\headrule}{\color{black} \headruleORIG}
\renewcommand{\headrulewidth}{1.0pt}
\usepackage{colortbl}
\arrayrulecolor{black}

\fancypagestyle{plain}{
  \fancyhead{}
  \fancyfoot{}
  \renewcommand{\headrulewidth}{0pt} %%%%%%%%%%%%%%%%%%%%%%%%%%%%%%%%%%%%%%%%%%%%%%%%%%%%%%%%%%%%%%%%%%%%%%%%%%%%%%%%%%%%%
}

%\usepackage{MyAlgorithm}
%\usepackage[noend]{MyAlgorithmic}
%\usepackage[ED=EDSYS-SystEmb, Ets=INP]{tlsflyleaf}

%%% Clear Header %%%%%%%%%%%%%%%%%%%%%%%%%%%%%%%%%%%%%%%%%%%%%%%%%%%%%%%%%%%%%%%%%%
% Clear Header Style on the Last Empty Odd pages
\makeatletter

\def\cleardoublepage{\clearpage\if@twoside \ifodd\c@page\else%
  \hbox{}%
  \thispagestyle{empty}%              % Empty header styles
  \newpage%
  \if@twocolumn\hbox{}\newpage\fi\fi\fi}

\makeatother
 
%%%%%%%%%%%%%%%%%%%%%%%%%%%%%%%%%%%%%%%%%%%%%%%%%%%%%%%%%%%%%%%%%%%%%%%%%%%%%%% 
% Prints your review date and 'Draft Version' (From Josullvn, CS, CMU)
\newcommand{\reviewtimetoday}[2]{\special{!userdict begin
    /bop-hook{gsave 20 710 translate 45 rotate 0.8 setgray
      /Times-Roman findfont 12 scalefont setfont 0 0   moveto (#1) show
      0 -12 moveto (#2) show grestore}def end}}
% You can turn on or off this option.
% \reviewtimetoday{\today}{Draft Version}
%%%%%%%%%%%%%%%%%%%%%%%%%%%%%%%%%%%%%%%%%%%%%%%%%%%%%%%%%%%%%%%%%%%%%%%%%%%%%%% 

\newenvironment{maxime}[1]
{
	\def\Arg{#1}
\vspace*{0cm}
\hfill
\begin{minipage}{0.6\textwidth}%
%\rule[0.5ex]{\textwidth}{0.1mm}\\%
\hrulefill $\:$ \\%$\:$ {\bf #1}\\
%\vspace*{-0.25cm}
\it 
}%
{%
	
\hrulefill $\:$ {\bf \Arg}
\vspace*{0.5cm}%
\end{minipage}
}

\usepackage{epigraph}

%\usepackage{slashbox}

%%%%%%% Outils pour \comment \alert \add %%%%%
\usepackage{easyReview}
\usepackage{soulutf8} % for accented letters in easyReview

\let\newalert\alert
\renewcommand{\alert}[1]{\textit{\newalert{#1}}}

%\usepackage[commandnameprefix=ifneeded]{changes} %% \chhighlight and \chcomment to avoid collision with easyReview
\renewcommand{\epsilon}{\varepsilon}

% centered page environment

\newenvironment{vcenterpage}
{\newpage\vspace*{\fill}\thispagestyle{empty}\renewcommand{\headrulewidth}{0pt}}
{\vspace*{\fill}}

%\usepackage{tablefootnote}
\interfootnotelinepenalty=10000

%%%%%% MISE EN FORME CADRES DEFINITIONS/THEOREMES/LEMES %%%%%%%%%%
\usepackage{amsthm}  % for theoremstyle
\usepackage[framemethod=tikz]{mdframed}

\theoremstyle{plain}
\newtheorem{theorem}{Théorème}[section]
\surroundwithmdframed[outerlinewidth=1.5pt,
innerlinewidth=1.5pt,
middlelinewidth=2pt,
middlelinecolor=white,
bottomline=false,topline=false,rightline=false]{theorem}
\newtheorem{corollary}{Corolaire}[theorem]

%\theoremstyle{lemma}
%\newtheorem{lemma}[theorem]{Lemme}


\theoremstyle{definition}
\newtheorem{definition}[theorem]{Définition}
\surroundwithmdframed[%outerlinewidth=1.5pt,
%innerlinewidth=1.5pt,
middlelinewidth=2pt,
bottomline=false,topline=false,rightline=false]{definition}

\cmnt{
	\usepackage{ntheorem} %\usepackage{amsthm}  % for theoremstyle
	%\usepackage{mdframed}
	\usepackage[most]{tcolorbox}

	\theoremstyle{plain}
	\theoremindent20pt
	\theoremheaderfont{\normalfont\bfseries\hspace{-\theoremindent}}
	\newtheorem{theorem}{Théorème}[section]
	\newtheorem{corollary}{Corolaire}[theorem]

	\theoremstyle{plain}
	\newtheorem{lemma}[theorem]{Lemme}


	\tcolorboxenvironment{theorem}{
		blanker,
		breakable,
		bme skip=\topsep,
		after skip=\topsep,
		borderline west={1pt}{10pt}{double, shorten <=12pt}
	}

	\theorembodyfont{\normalfont}
	\theoremindent20pt
	\theoremheaderfont{\normalfont\bfseries\hspace{-\theoremindent}}
	\newtheorem{definition}[theorem]{Définition}


	\tcolorboxenvironment{definition}{
		blanker,
		breakable,
		before skip=\topsep,
		after skip=\topsep,
		borderline west={1pt}{10pt}{shorten <=12pt}
	}
}

\usepackage{listings}
\renewcommand{\lstlistingname}{Code}% Listing -> Algorithm

\definecolor{backcolour}{rgb}{0.95,0.95,0.92}
\definecolor{codegreen}{rgb}{0,0.6,0}
\lstdefinestyle{codeStyle}{
	language=C,
	backgroundcolor=\color{backcolour},
	commentstyle=\color{codegreen},
	basicstyle=\ttfamily\small,
	breaklines=true,
	captionpos=b,
	keepspaces=true,
	numbers=left,
	numbersep=5pt,
	showspaces=false,
	showstringspaces=false,
	showtabs=false,
	tabsize=2
}
\lstset{style=codeStyle}

\cmnt{
	\begin{theorem}
		Ceci est un Théorème.
	\end{theorem}

	\begin{corollary}
		Ceci est un Corollaire.
	\end{corollary}

	\begin{definition}
		Ceci est une Définition.
	\end{definition}

	\begin{lemma}
		Ceci est un Lemme.
	\end{lemma}
}

\def\UrlBigBreaks{\do\/\do-\do:}
\usepackage{url}

% Change this to change the informations included in the pdf file
% My pdf code
\usepackage{ifpdf}

\ifpdf
\usepackage[pdftex]{graphicx}
\DeclareGraphicsExtensions{.pdf,PDF,.png,PNG,.jpg,JPG}
\usepackage[pagebackref,hyperindex=true]{hyperref} %% use \autoref{} instead of Table~\ref{}.
\usepackage{tikz}
\usetikzlibrary{arrows,shapes,calc}
\else
\usepackage{graphicx}
\DeclareGraphicsExtensions{.ps,.eps}
\usepackage[a4paper,dvipdfm,pagebackref,hyperindex=true]{hyperref}
\fi

%% nicer backref links. NOTE: The flag ThesisInEnglish is used to define the
% language in the back references. Read more about it in These.tex
\IfLanguageName{english}{
	\renewcommand*{\backref}[1]{}
	\renewcommand*{\backrefalt}[4]{%
		\ifcase #1 %
		(Not cited.)%
		\or
		(Cited in page~#2.)%
		\else
		(Cited in pages~#2.)%
		\fi}
	\renewcommand*{\backrefsep}{, }
	\renewcommand*{\backreftwosep}{ and~}
	\renewcommand*{\backreflastsep}{ and~}
}{
	\renewcommand*{\backref}[1]{}
	\renewcommand*{\backrefalt}[4]{%
		\ifcase #1 %
		(Non cité.)%
		\or
		(Cité en page~#2.)%
		\else
		(Cité en pages~#2.)%
		\fi}
	\renewcommand*{\backrefsep}{, }
	\renewcommand*{\backreftwosep}{ et~}
	\renewcommand*{\backreflastsep}{ et~}
}

% Table of contents for each chapter
\usepackage[nottoc, notlof, notlot]{tocbibind}
\usepackage[nohints]{minitoc}
\setcounter{minitocdepth}{2}
\mtcindent=15pt
% Use \minitoc where to put a table of contents
\let\minitocORIG\minitoc
\renewcommand{\minitoc}{\minitocORIG \vspace{1.5em}}

\hypersetup
{
	bookmarksopen=true,
	pdftitle="Prévention des fautes temporelles sur architectures multicœur pour les systèmes à criticité mixte",
	pdfauthor="Daniel LOCHE", %auteur du document
	pdfsubject="Thèse", %sujet du document
	%pdftoolbar=false, %barre d'outils non visible
	pdfmenubar=true, %barre de menu visible
	pdfhighlight=/O, %effet d'un clic sur un lien hypertexte
	colorlinks=true, %couleurs sur les liens hypertextes
	pdfpagemode=UseNone, %aucun mode de page
	%pdfpagelayout=DoublePage, %ouverture en simple page
	pdffitwindow=false, %pages ouvertes entierement dans toute la fenetre
	linkcolor=linkcol, %couleur des liens hypertextes internes
	citecolor=citecol, %couleur des liens pour les citations
	urlcolor=linkcol %couleur des liens pour les url
}
