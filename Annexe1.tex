%\chapter*{Résultats Expérimentaux}\label{chap:annexe1}
\cleardoublepage

\newgeometry{top=3.3cm,bottom=3cm}      %% SI VOS ANNEXES SONT ÉNORMES ET QUE VOUS VOULEZ GRAPILLER SUR LA MARGE, C'EST ICI.
%% PENSER A FINIR LES ANNEXES PAR UN "/restoregeometry" DE FAÇON À CONSERVER LA MARGE D'ORIGINE SUR LE RESTE DU MANUSCRIPT

\section*{ANNEXE 1 -- Caractérisation des tâches MiBench} \label{Annexe1}
\markboth{ANNEXE}{Annexe}  %% A modifier : nom des entêtes pair/impair des pages de l'annexe.
\addstarredchapter{Annexe}
\setcounter{chapter}{0}  %% reset du compteur pour pas que les annexes se basent sur l'incrémentation des chapitres du manuscript.
\setcounter{table}{0}
\renewcommand{\thetable}{A\arabic{table}} %% Nouvelle numérotation pour les tableaux idem, pour pas continuer à la suite de numérotation du manuscrit.
 \begin{table}[ht!]
	\centering
	\caption{Caractérisation des tâches MiBench - 1200 runs par tâches} 	\label{tab:Phase1-2-results}
	\begin{adjustwidth}{-40pt}{0pt}
		\begin{tabular}{@{}lrrrrrrl@{}}
			\toprule
			         & \multicolumn{3}{c}{Tâche en isolation}  & \multicolumn{3}{c}{Tâche + interférences} &  \\
			\cmidrule(lr){2-4} \cmidrule(lr){5-7}
			Tâche	 & MIN (ms)  & AVG (ms)  & MAX (ms)  & MIN (ms)  & AVG (ms)  & MAX (ms)  & Classification     \\
			\midrule
			fft-S               & 1,7E+00   & 1,7E+00   & 1,7E+00   & 1,7E+00   & 1,7E+00   & 1,9E+00   & CLEAN    \\
			fft-inv-S           & 3,5E+00   & 3,5E+00   & 3,5E+00   & 3,5E+00   & 3,5E+00   & 3,7E+00   & CLEAN    \\
			basicmath-S         & 3,8E+00   & 3,8E+00   & 4,0E+00   & 3,8E+00   & 3,9E+00   & 4,3E+00   & CLEAN    \\
			fft-L               & 7,1E+00   & 7,1E+00   & 7,2E+00   & 6,7E+00   & 7,2E+00   & 7,7E+00   & CLEAN    \\
			rijndael-E-L        & 127,7E+00 & 152,3E+00 & 5,8E+03   & 1,2E+03   & 1,5E+03   & 1,7E+03   & REJECT   \\
			\midrule
			\textbf{Médianes} 	& \multicolumn{3}{c}{7,83 ms}	& \multicolumn{3}{c}{9,64 ms}	& 	\\
			\bottomrule
		\end{tabular}
	\end{adjustwidth}
\end{table}

\restoregeometry
